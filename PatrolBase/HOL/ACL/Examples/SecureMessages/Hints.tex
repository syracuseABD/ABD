\documentclass[10pt,twoside]{article}
\usepackage{./LaTeX/engineeringAssurance}

%Times for rm and math | Helvetica for ss | Courier for tt
\usepackage{mathptmx} % rm & math
\usepackage[scaled=0.90]{helvet} % ss
\usepackage{courier} % tt
\usepackage{amsmath}
\usepackage{enumerate}

\usepackage{alltt}
\normalfont
\usepackage[T1]{fontenc}
\newcommand{\action}[1]{\ensuremath{\langle #1 \rangle}}

\newcommand{\name}[1]{\ensuremath{\textit{#1}}}
\newcommand{\filename}[1]{\ensuremath{\mathtt{#1}}}
\newcommand{\rulespace}{\vspace*{2em}}


\newcommand{\pair}[1]{\ensuremath{\langle #1\rangle}}
%\newcommand{\annd}{\ensuremath{\ \&\ }}
\newcommand{\annd}{\ensuremath{\textrm{ and }}}
% \newcommand{\orr}{\ensuremath{\textrm{ or }}}
\newcommand{\pow}[1]{\ensuremath{\mathcal{P}(#1)}}
\newcommand{\set}[1]{\ensuremath{\{#1\}}}
\newcommand{\midset}[2]{\ensuremath{\set{#1 \mid #2}}}
\newcommand{\arrow}{\ensuremath{\rightarrow}}
\newcommand{\id}[1]{\ensuremath{\textsf{id}_{#1}}}
\newcommand{\subst}[3]{\ensuremath{{#1}\boldsymbol{[}#2\boldsymbol{/}#3\boldsymbol{]}}} 

%% For mathematical proofs with ``reasons'' on each step
\newenvironment{mathprf}
  {\begin{displaymath}\begin{array}{rcll}}{\end{array}\end{displaymath}}
\newcommand{\why}[1]{\ensuremath{\quad \text{#1}}}

%% for conventions (spelled out explicitly in text)
\newenvironment{convention}{\begin{description} \item[\textit{Convention:}
    ]}{\end{description}} 

\newcommand{\readernote}[2]{\noindent\shadowbox{\parbox{.95\textwidth}{%
      \begin{description} \item[\textit{#1:}] #2 \end{description}}}}



\newenvironment{indentedExample}{\begin{example}\ \begin{list}{}{}\item }{\end{list}\end{example}}

%\newenvironment{indentedExample}{\begin{example}}{\end{example}}
% for grammars

\newcommand{\isa}{\ensuremath{\; {:}{:}{=} \;}}
\newcommand{\goesto}{\ensuremath{\leadsto}}
%\newcommand{\ora}{\ensuremath{\;\mid\;}}
\newcommand{\ora}{\ensuremath{\;/\;}}
%\newcommand{\syncat}[1]{\hbox{\textcolor{red}{\sc$\langle$#1$\rangle$}}}
%\newcommand{\syncat}[1]{\hbox{{ \sc$\langle$#1$\rangle$}}}
%\newcommand{\syncat}[1]{\hbox{{ \bf #1 }}}
\newcommand{\syncat}[1]{\ensuremath{\textbf{#1}}\xspace}

% Miscellaneous

\newcommand{\defined}{\ensuremath{\quad \triangleq \quad}}
\newcommand{\defn}{\ensuremath{\stackrel{\mathrm{def}}{=}}}


\newcommand{\assign}{\ensuremath{:=}}

%%% for hiding self comments
%\renewcommand{\suebox}[1]{}
%\renewcommand{\chinbox}[1]{}

\newcommand{\key}[1]{\textbf{#1}}

% encryption

\newcommand{\encrypt}[2]{\ensuremath{\mathit{encrypt}(#2,#1)}}
\newcommand{\cat}[1]{\ensuremath{\langle\!\langle #1 \rangle \! \rangle}}

% Syntactic sets
\newcommand{\PName}{\ensuremath{\textbf{PName}}\xspace}
\newcommand{\PExp}{\ensuremath{\textbf{Princ}}\xspace}
\newcommand{\PropVar}{\ensuremath{\textbf{PropVar}}\xspace}
\newcommand{\LExp}{\ensuremath{\textbf{Form}}\xspace}
\newcommand{\LabelConst}{\ensuremath{\textbf{SecLabel}}\xspace}
\newcommand{\Level}{\ensuremath{\textbf{SecLevel}}\xspace}
\newcommand{\IntLabelConst}{\ensuremath{\textbf{IntLabel}}\xspace}
\newcommand{\IntLevel}{\ensuremath{\textbf{IntLevel}}\xspace}
% \newcommand{\PName}{\ensuremath{\textbf{\textsc{PName}}}\xspace}
% \newcommand{\PExp}{\ensuremath{\textbf{\textsc{Princ}}}\xspace}
% \newcommand{\PropVar}{\ensuremath{\textbf{\textsc{PropVar}}}\xspace}
% \newcommand{\LExp}{\ensuremath{\textbf{\textsc{Form}}}\xspace}
%\newcommand{\LExp}{\ensuremath{\textbf{\underline{Form} }}}
%\newcommand{\privs}{\ensuremath{\mathit{privs}}}
%\newcommand{\Targ}{\ensuremath{\mathcal{T}}}


% Principals
\newcommand{\with}{\ensuremath{\;\&\;}}
\newcommand{\quoting}{\ensuremath{\;|\;}}
\newcommand{\for}[1]{\ensuremath{\;\textsf{for}_{#1}\;}}

% Logical expressions
\newcommand{\has}{\ensuremath{\textsf{ has }}}
\newcommand{\says}{\ensuremath{\text{\footnotesize \textsf{ says }}}}
\newcommand{\controls}{\ensuremath{\text{\footnotesize \textsf{ controls }}}}
\newcommand{\serves}{\ensuremath{\textsf{ serves }}}
\newcommand{\speaksfor}{\ensuremath{\Rightarrow}}
\newcommand{\then}{\;\supset\;}
\newcommand{\phiplus}{\ensuremath{\varphi^{+}}}
% SKC - added syntactic sugar for ``represents''
\newcommand{\rreps}{\ensuremath{\text{\footnotesize \textsf{reps} }}}
\newcommand{\reps}[3]{\ensuremath{{#1} \text{\footnotesize \textsf{
          reps }}{#2}{\text{\footnotesize \textsf{ on }}}{#3}}}
\newcommand{\controlsandsays}{\ensuremath{\textsf{ controls+says }}}
\newcommand{\rp}[2]{\ensuremath{{#1} {\text{\footnotesize \textsf{
          reps }}}{#2}{\text{\footnotesize \textsf{ on }}}}}
\newcommand{\slv}[1]{\ensuremath{\text{\textsf{ slev}}(#1)}}
\newcommand{\ilv}[1]{\ensuremath{\text{\textsf{ ilev}}(#1)}}


% Semantics
\newcommand{\struct}[1]{\ensuremath{\langle {#1} \rangle}}
\newcommand{\krip}[1]{\ensuremath{\langle {#1} \rangle}}

\newcommand{\E}[1]{\ensuremath{\mathcal{E}_{\mathcal{M}}[\![#1]\!]}}
\newcommand{\Ee}{\ensuremath{\mathcal{E}_{\mathcal{M}}}}
\newcommand{\Em}[2]{\ensuremath{\mathcal{E}_{#2}[\![#1]\!]}}

%% for actions: 
%%   \actionit puts contents into textit mode
\newcommand{\action}[1]{\ensuremath{\langle #1 \rangle}}
\newcommand{\actionit}[1]{\ensuremath{\langle \textit{#1} \rangle}}

\newcommand{\sig}[1]{\ensuremath{\mathit{Signature}_{#1}}}

\newcommand{\mssmeet}{\ensuremath{\curlywedge}}
\newcommand{\mssbar}{\ensuremath{\|}}
\newcommand{\below}{\ensuremath{\leq}}

\newcommand{\eqmod}[1]{\ensuremath{\approx_{#1}}}

% Derivability

\newcommand{\infrule}[2]
   {\ensuremath{{\textstyle #1}\over{\textstyle #2}}}

\newcommand{\infname}[1]{\textit{#1}}

\newcommand{\irule}[3]
    {\ensuremath{\infname{#3}\quad {\displaystyle \frac{#1}{#2}}}}

\newcommand{\displaynewrule}[3]
{\begin{displaymath}
  \colorbox{LightGray}{\irule{#1}{#2}{#3}}
%  \colorbox{SpringGreen}{\irule{#1}{#2}{#3}}
\end{displaymath}}

\newcommand{\displaycorerule}[4]
{\begin{displaymath}
  \colorbox{lightgray}{\irule{#1}{#2}{#3}{#4}}
%  \colorbox{SkyBlue}{\irule{#1}{#2}{#3}{#4}}
\end{displaymath}}

\newcommand{\displaycoredef}[1]
{\begin{displaymath}
    \colorbox{lightgray}{\ensuremath{#1}}
%  \colorbox{SkyBlue}{\ensuremath{#1}}
\end{displaymath}}

%% Assorted notation for RBAC
\newcommand{\inherits}{\ensuremath{\,\succeq\,}}   % role inheritance
\newcommand{\Users}{\ensuremath{\mathit{Users}}}
\newcommand{\Perms}{\ensuremath{\mathit{Perms}}}
\newcommand{\Roles}{\ensuremath{\mathit{Roles}}}
\newcommand{\Sessions}{\ensuremath{\mathit{Sessions}}}
\newcommand{\SSD}{\ensuremath{\mathit{SSD}}}
\newcommand{\DSD}{\ensuremath{\mathit{DSD}}}
\newcommand{\ausers}{\ensuremath{\mathit{auth\_users}}}
\newcommand{\aperms}{\ensuremath{\mathit{auth\_perms}}}
\newcommand{\suser}{\ensuremath{\mathit{user}}}
\newcommand{\sroles}{\ensuremath{\mathit{roles}}}


% Generalized addresses
% \newcommand{\genAddr}[2]{\ensuremath{\langle\negmedspace\langle #1
%     \rangle\negmedspace\rangle}{\mid #2}}
\newcommand{\segName}[1]{\ensuremath{\langle\negmedspace\langle #1
    \rangle\negmedspace\rangle}}

%% address-descriptor location for named segment
\newcommand{\adLoc}[1]{\ensuremath{|\!| #1 |\!|}}
\newcommand{\segLoc}[1]
    {\ensuremath{\langle\!\!\langle #1 \rangle\!\!\rangle}}
\newcommand{\genAddr}[2]{\ensuremath{[ #1:#2 ]}}
\newcommand{\genAddrPair}[2]{\ensuremath{(#1,#2)}}

%% Formal proof environment

\newenvironment{formalProof}
{\begin{center}\small\begin{tabular}{ll}}{\end{tabular}\end{center}} 


%% Alternate Formal proof environment



\newenvironment{tabProof}{\begin{center}\footnotesize\begin{tabular}{r
        >{$}p{3.2in}<{$}p{1.0in}}}{\end{tabular}\end{center}}  
\newenvironment{tabProof2}{\begin{center}\footnotesize\begin{tabular}{r >{$}p{2.7in}<{$}p{1.5in}}}{\end{tabular}\end{center}} 

% if then else
\newcommand{\ite}[3]
   {\ensuremath{#1 \rightarrow #2 \mid #3}}

%%  Macro(s) for section summaries
\newcommand*{\summproblem}[1]{\def\fromsummproblem{#1}}
\newcommand*{\summanalysis}[1]{\def\fromsummanalysis{#1}}
\newcommand*{\summresults}[1]{\def\fromsummresults{#1}}

\newcommand{\missing}{\textsc{Missing Component}}
\summproblem{\missing}
\summanalysis{\missing}
\summresults{\missing}

\newcommand{\makesummary}
  {%\section{Summary} 
    \begin{enumerate}
    \item \textsc{What is the problem?}
%      \begin{quote}
%        \it What is the purpose of my thinking?  What precise
%        question am I trying to answer?  Within what point of view am I
%        thinking?
%      \end{quote}
%      \ \\ 

      \fromsummproblem
    \item  \textsc{How am I analyzing the problem?}
%      \begin{quote}
%        \it What am I taking for granted, and what assumptions am
%        I making? What information am I using?  How am I interpreting
%        that information?  What concepts or ideas are central to my
%        thinking? 
%      \end{quote}
%      \ \\ 

      \fromsummanalysis
    \item \textsc{What are the results?}
%      \begin{quote}
%        \it What conclusions am I coming to?  If I accept the
%        conclusions, what are the implications?  What would the
%        consequences be, if I put my thought into action?
%      \end{quote}
%      \ \\ 

      \fromsummresults
    \end{enumerate}

  \summproblem{\missing}
  \summanalysis{\missing}
  \summresults{\missing}
    
}


%----redefine implication as horseshoe instead of arrow--------

\renewcommand{\implies}{\supset}
\newcommand{\believes}{\ensuremath\textsf{ believes }}

%%% Local Variables: 
%%% mode: latex
%%% TeX-master: "book"
%%% End: 

\renewcommand{\infrule}[2]
   {\ensuremath{{\textstyle #1}\over{\textstyle #2}}}

\renewcommand{\infname}[1]{\textit{#1}}
% \renewcommand{\irule}[3]
%     {\ensuremath{\infname{#3}\quad {\displaystyle \frac{#1}{#2}}}}

\ifx\pdfoutput\undefined
\usepackage[dvips]{graphicx}
\else
\usepackage[pdftex]{graphicx}
\usepackage{epstopdf}
\pdfcompresslevel=9
\fi

%Notes in text
\newcommand{\chinbox}[1]{%
  \fbox{\parbox[t]{6.0in}{\textsc{Note to Self:} 
      \begin{center}
        #1
      \end{center}}}}

\newcommand{\problembox}[1]
{
  \fbox{\begin{minipage}{0.9\linewidth}
      \begin{center}
        \redtext{\underline{\textbf{\textsc{Assignment}}}}
      \end{center}
      #1
  \end{minipage}}
}


\usepackage{array}

% ---------------------------------------------------------------------
% Input defined macros and commands
% ---------------------------------------------------------------------
\input{./LaTeX/commands}
\input{./LaTeX/pic-macros}
\input{./LaTeX/ref-macros}

\usepackage{url}
\usepackage[line,arrow,frame,matrix]{xy}
\usepackage{./LaTeX/proof}
\usepackage{./LaTeX/holtex}
\usepackage{./LaTeX/holtexbasic}

% \usepackage[usenames,dvipsnames]{color}
\definecolor{orange}{rgb}{1,0.5,0}
\newcommand{\redtext}[1]{\textcolor{red}{#1}}
\newcommand{\bluetext}[1]{\textcolor{blue}{#1}}
\newcommand{\magtext}[1]{\textcolor{magenta}{#1}}
\newcommand{\greentext}[1]{\textcolor{green}{#1}}
\newcommand{\orangetext}[1]{\textcolor{orange}{#1}}
\newcommand{\standout}[1]{\textcolor{orange}{#1}}

\newcommand{\seq}[2]{\ensuremath{\set{#1} \vdash {#2}}}
\newcommand{\seqs}[2]{\ensuremath{#1 \vdash {#2}}}
\newcommand{\sq}[1]{\ensuremath{\vdash {#1}}}

\newcommand{\goal}[2]{\ensuremath{\set{#1}\;\text{ ?-- }\;{#2}}}
\newcommand{\goals}[2]{\ensuremath{{#1}\;\text{ ?-- }\;{#2}}}
\newcommand{\gls}[1]{\ensuremath{\text{ ?-- }\;{#1}}}

\renewcommand{\irule}[3]
    {\ensuremath{{\displaystyle \frac{#1}{#2}}\quad \infname{#3}}}
\newcommand{\tac}[3]{
  \ensuremath{\begin{tabular}{c}
    {#1}\\\hline\hline{#2}
  \end{tabular}\quad}{#3}}
% HOL theories

\title{\textsc{Hints for Structure, Interpretation, and Verification of
    Messages, Certificates, and Trust Assumptions for Assured Command
    and Control CONOPS}}

\author{Assigned: Monday 5 December 2011}

\date{Due: 1200 Wednesday 14 December 2011}

\makeindex

\begin{document}

% ---------------------------------------------------------------------
% Inputs for HOL reports
% ---------------------------------------------------------------------
\newcommand{\HOLcipherDate}{20 August 2016}
\newcommand{\HOLcipherTime}{12:36}
\begin{SaveVerbatim}{HOLcipherDatatypesasymMsg}
\HOLFreeVar{asymMsg} = \HOLConst{Ea} ('princ \HOLTyOp{pKey}) ('message \HOLTyOp{contents})
\end{SaveVerbatim}
\newcommand{\HOLcipherDatatypesasymMsg}{\UseVerbatim{HOLcipherDatatypesasymMsg}}
\begin{SaveVerbatim}{HOLcipherDatatypescontents}
\HOLFreeVar{contents} = \HOLConst{plain} 'message \HOLTokenBar{} \HOLConst{unknown}
\end{SaveVerbatim}
\newcommand{\HOLcipherDatatypescontents}{\UseVerbatim{HOLcipherDatatypescontents}}
\begin{SaveVerbatim}{HOLcipherDatatypesdigest}
\HOLFreeVar{digest} = \HOLConst{hash} ('message \HOLTyOp{contents})
\end{SaveVerbatim}
\newcommand{\HOLcipherDatatypesdigest}{\UseVerbatim{HOLcipherDatatypesdigest}}
\begin{SaveVerbatim}{HOLcipherDatatypespKey}
\HOLFreeVar{pKey} = \HOLConst{pubK} 'princ \HOLTokenBar{} \HOLConst{privK} 'princ
\end{SaveVerbatim}
\newcommand{\HOLcipherDatatypespKey}{\UseVerbatim{HOLcipherDatatypespKey}}
\begin{SaveVerbatim}{HOLcipherDatatypesSymKey}
\HOLFreeVar{SymKey} = \HOLConst{sym} \HOLTyOp{num}
\end{SaveVerbatim}
\newcommand{\HOLcipherDatatypesSymKey}{\UseVerbatim{HOLcipherDatatypesSymKey}}
\begin{SaveVerbatim}{HOLcipherDatatypessymMsg}
\HOLFreeVar{symMsg} = \HOLConst{Es} \HOLTyOp{SymKey} ('message \HOLTyOp{contents})
\end{SaveVerbatim}
\newcommand{\HOLcipherDatatypessymMsg}{\UseVerbatim{HOLcipherDatatypessymMsg}}
\newcommand{\HOLcipherDatatypes}{
\HOLcipherDatatypesasymMsg\HOLcipherDatatypescontents\HOLcipherDatatypesdigest\HOLcipherDatatypespKey\HOLcipherDatatypesSymKey\HOLcipherDatatypessymMsg}
\begin{SaveVerbatim}{HOLcipherDefinitionsgetMessageXXdef}
\HOLTokenTurnstile{} \HOLSymConst{\HOLTokenForall{}}\HOLBoundVar{msg}. \HOLConst{getMessage} (\HOLConst{plain} \HOLBoundVar{msg}) \HOLSymConst{=} \HOLBoundVar{msg}
\end{SaveVerbatim}
\newcommand{\HOLcipherDefinitionsgetMessageXXdef}{\UseVerbatim{HOLcipherDefinitionsgetMessageXXdef}}
\begin{SaveVerbatim}{HOLcipherDefinitionssignXXdef}
\HOLTokenTurnstile{} \HOLSymConst{\HOLTokenForall{}}\HOLBoundVar{P} \HOLBoundVar{dgst}. \HOLConst{sign} \HOLBoundVar{P} \HOLBoundVar{dgst} \HOLSymConst{=} \HOLConst{Ea} (\HOLConst{privK} \HOLBoundVar{P}) (\HOLConst{plain} \HOLBoundVar{dgst})
\end{SaveVerbatim}
\newcommand{\HOLcipherDefinitionssignXXdef}{\UseVerbatim{HOLcipherDefinitionssignXXdef}}
\begin{SaveVerbatim}{HOLcipherDefinitionssignVerifyXXdef}
\HOLTokenTurnstile{} \HOLSymConst{\HOLTokenForall{}}\HOLBoundVar{P} \HOLBoundVar{signature} \HOLBoundVar{msgContents}.
     \HOLConst{signVerify} \HOLBoundVar{P} \HOLBoundVar{signature} \HOLBoundVar{msgContents} \HOLSymConst{\HOLTokenEquiv{}}
     (\HOLConst{plain} (\HOLConst{hash} \HOLBoundVar{msgContents}) \HOLSymConst{=} \HOLConst{deciphP} (\HOLConst{pubK} \HOLBoundVar{P}) \HOLBoundVar{signature})
\end{SaveVerbatim}
\newcommand{\HOLcipherDefinitionssignVerifyXXdef}{\UseVerbatim{HOLcipherDefinitionssignVerifyXXdef}}
\newcommand{\HOLcipherDefinitions}{
\HOLDfnTag{cipher}{getMessage_def}\HOLcipherDefinitionsgetMessageXXdef
\HOLDfnTag{cipher}{sign_def}\HOLcipherDefinitionssignXXdef
\HOLDfnTag{cipher}{signVerify_def}\HOLcipherDefinitionssignVerifyXXdef
}
\begin{SaveVerbatim}{HOLcipherTheoremsdeciphPXXclauses}
\HOLTokenTurnstile{} (\HOLSymConst{\HOLTokenForall{}}\HOLBoundVar{P} \HOLBoundVar{text}.
      (\HOLConst{deciphP} (\HOLConst{pubK} \HOLBoundVar{P}) (\HOLConst{Ea} (\HOLConst{privK} \HOLBoundVar{P}) (\HOLConst{plain} \HOLBoundVar{text})) \HOLSymConst{=}
       \HOLConst{plain} \HOLBoundVar{text}) \HOLSymConst{\HOLTokenConj{}}
      (\HOLConst{deciphP} (\HOLConst{privK} \HOLBoundVar{P}) (\HOLConst{Ea} (\HOLConst{pubK} \HOLBoundVar{P}) (\HOLConst{plain} \HOLBoundVar{text})) \HOLSymConst{=}
       \HOLConst{plain} \HOLBoundVar{text})) \HOLSymConst{\HOLTokenConj{}}
   (\HOLSymConst{\HOLTokenForall{}}\HOLBoundVar{k} \HOLBoundVar{P} \HOLBoundVar{text}.
      (\HOLConst{deciphP} \HOLBoundVar{k} (\HOLConst{Ea} (\HOLConst{privK} \HOLBoundVar{P}) (\HOLConst{plain} \HOLBoundVar{text})) \HOLSymConst{=} \HOLConst{plain} \HOLBoundVar{text}) \HOLSymConst{\HOLTokenImp{}}
      (\HOLBoundVar{k} \HOLSymConst{=} \HOLConst{pubK} \HOLBoundVar{P})) \HOLSymConst{\HOLTokenConj{}}
   \HOLSymConst{\HOLTokenForall{}}\HOLBoundVar{k} \HOLBoundVar{P} \HOLBoundVar{text}.
     (\HOLConst{deciphP} \HOLBoundVar{k} (\HOLConst{Ea} (\HOLConst{pubK} \HOLBoundVar{P}) (\HOLConst{plain} \HOLBoundVar{text})) \HOLSymConst{=} \HOLConst{plain} \HOLBoundVar{text}) \HOLSymConst{\HOLTokenImp{}}
     (\HOLBoundVar{k} \HOLSymConst{=} \HOLConst{privK} \HOLBoundVar{P})
\end{SaveVerbatim}
\newcommand{\HOLcipherTheoremsdeciphPXXclauses}{\UseVerbatim{HOLcipherTheoremsdeciphPXXclauses}}
\begin{SaveVerbatim}{HOLcipherTheoremsdeciphPXXdef}
\HOLTokenTurnstile{} (\HOLConst{deciphP} \HOLFreeVar{key} (\HOLConst{Ea} (\HOLConst{privK} \HOLFreeVar{P}) (\HOLConst{plain} \HOLFreeVar{text})) \HOLSymConst{=}
    \HOLKeyword{if} \HOLFreeVar{key} \HOLSymConst{=} \HOLConst{pubK} \HOLFreeVar{P} \HOLKeyword{then} \HOLConst{plain} \HOLFreeVar{text} \HOLKeyword{else} \HOLConst{unknown}) \HOLSymConst{\HOLTokenConj{}}
   (\HOLConst{deciphP} \HOLFreeVar{key} (\HOLConst{Ea} (\HOLConst{pubK} \HOLFreeVar{P}) (\HOLConst{plain} \HOLFreeVar{text})) \HOLSymConst{=}
    \HOLKeyword{if} \HOLFreeVar{key} \HOLSymConst{=} \HOLConst{privK} \HOLFreeVar{P} \HOLKeyword{then} \HOLConst{plain} \HOLFreeVar{text} \HOLKeyword{else} \HOLConst{unknown})
\end{SaveVerbatim}
\newcommand{\HOLcipherTheoremsdeciphPXXdef}{\UseVerbatim{HOLcipherTheoremsdeciphPXXdef}}
\begin{SaveVerbatim}{HOLcipherTheoremsdeciphPXXind}
\HOLTokenTurnstile{} \HOLSymConst{\HOLTokenForall{}}\HOLBoundVar{P\sp{\prime}}.
     (\HOLSymConst{\HOLTokenForall{}}\HOLBoundVar{key} \HOLBoundVar{P} \HOLBoundVar{text}. \HOLBoundVar{P\sp{\prime}} \HOLBoundVar{key} (\HOLConst{Ea} (\HOLConst{privK} \HOLBoundVar{P}) (\HOLConst{plain} \HOLBoundVar{text}))) \HOLSymConst{\HOLTokenConj{}}
     (\HOLSymConst{\HOLTokenForall{}}\HOLBoundVar{key} \HOLBoundVar{P} \HOLBoundVar{text}. \HOLBoundVar{P\sp{\prime}} \HOLBoundVar{key} (\HOLConst{Ea} (\HOLConst{pubK} \HOLBoundVar{P}) (\HOLConst{plain} \HOLBoundVar{text}))) \HOLSymConst{\HOLTokenConj{}}
     (\HOLSymConst{\HOLTokenForall{}}\HOLBoundVar{key} \HOLBoundVar{v\sb{\mathrm{7}}}. \HOLBoundVar{P\sp{\prime}} \HOLBoundVar{key} (\HOLConst{Ea} \HOLBoundVar{v\sb{\mathrm{7}}} \HOLConst{unknown})) \HOLSymConst{\HOLTokenImp{}}
     \HOLSymConst{\HOLTokenForall{}}\HOLBoundVar{v} \HOLBoundVar{v\sb{\mathrm{1}}}. \HOLBoundVar{P\sp{\prime}} \HOLBoundVar{v} \HOLBoundVar{v\sb{\mathrm{1}}}
\end{SaveVerbatim}
\newcommand{\HOLcipherTheoremsdeciphPXXind}{\UseVerbatim{HOLcipherTheoremsdeciphPXXind}}
\begin{SaveVerbatim}{HOLcipherTheoremsdeciphSXXclauses}
\HOLTokenTurnstile{} (\HOLSymConst{\HOLTokenForall{}}\HOLBoundVar{k} \HOLBoundVar{text}. \HOLConst{deciphS} \HOLBoundVar{k} (\HOLConst{Es} \HOLBoundVar{k} (\HOLConst{plain} \HOLBoundVar{text})) \HOLSymConst{=} \HOLConst{plain} \HOLBoundVar{text}) \HOLSymConst{\HOLTokenConj{}}
   \HOLSymConst{\HOLTokenForall{}}\HOLBoundVar{k\sb{\mathrm{1}}} \HOLBoundVar{k\sb{\mathrm{2}}} \HOLBoundVar{text}.
     (\HOLConst{deciphS} \HOLBoundVar{k\sb{\mathrm{1}}} (\HOLConst{Es} \HOLBoundVar{k\sb{\mathrm{2}}} (\HOLConst{plain} \HOLBoundVar{text})) \HOLSymConst{=} \HOLConst{plain} \HOLBoundVar{text}) \HOLSymConst{\HOLTokenImp{}} (\HOLBoundVar{k\sb{\mathrm{1}}} \HOLSymConst{=} \HOLBoundVar{k\sb{\mathrm{2}}})
\end{SaveVerbatim}
\newcommand{\HOLcipherTheoremsdeciphSXXclauses}{\UseVerbatim{HOLcipherTheoremsdeciphSXXclauses}}
\begin{SaveVerbatim}{HOLcipherTheoremsdeciphSXXdef}
\HOLTokenTurnstile{} \HOLConst{deciphS} \HOLFreeVar{k\sb{\mathrm{1}}} (\HOLConst{Es} \HOLFreeVar{k\sb{\mathrm{2}}} (\HOLConst{plain} \HOLFreeVar{text})) \HOLSymConst{=}
   \HOLKeyword{if} \HOLFreeVar{k\sb{\mathrm{1}}} \HOLSymConst{=} \HOLFreeVar{k\sb{\mathrm{2}}} \HOLKeyword{then} \HOLConst{plain} \HOLFreeVar{text} \HOLKeyword{else} \HOLConst{unknown}
\end{SaveVerbatim}
\newcommand{\HOLcipherTheoremsdeciphSXXdef}{\UseVerbatim{HOLcipherTheoremsdeciphSXXdef}}
\begin{SaveVerbatim}{HOLcipherTheoremsdeciphSXXind}
\HOLTokenTurnstile{} \HOLSymConst{\HOLTokenForall{}}\HOLBoundVar{P}.
     (\HOLSymConst{\HOLTokenForall{}}\HOLBoundVar{k\sb{\mathrm{1}}} \HOLBoundVar{k\sb{\mathrm{2}}} \HOLBoundVar{text}. \HOLBoundVar{P} \HOLBoundVar{k\sb{\mathrm{1}}} (\HOLConst{Es} \HOLBoundVar{k\sb{\mathrm{2}}} (\HOLConst{plain} \HOLBoundVar{text}))) \HOLSymConst{\HOLTokenConj{}}
     (\HOLSymConst{\HOLTokenForall{}}\HOLBoundVar{v\sb{\mathrm{6}}} \HOLBoundVar{v\sb{\mathrm{5}}}. \HOLBoundVar{P} \HOLBoundVar{v\sb{\mathrm{6}}} (\HOLConst{Es} \HOLBoundVar{v\sb{\mathrm{5}}} \HOLConst{unknown})) \HOLSymConst{\HOLTokenImp{}}
     \HOLSymConst{\HOLTokenForall{}}\HOLBoundVar{v} \HOLBoundVar{v\sb{\mathrm{1}}}. \HOLBoundVar{P} \HOLBoundVar{v} \HOLBoundVar{v\sb{\mathrm{1}}}
\end{SaveVerbatim}
\newcommand{\HOLcipherTheoremsdeciphSXXind}{\UseVerbatim{HOLcipherTheoremsdeciphSXXind}}
\begin{SaveVerbatim}{HOLcipherTheoremssignVerifyXXOneOne}
\HOLTokenTurnstile{} \HOLSymConst{\HOLTokenForall{}}\HOLBoundVar{P} \HOLBoundVar{m\sb{\mathrm{1}}} \HOLBoundVar{m\sb{\mathrm{2}}}.
     \HOLConst{signVerify} \HOLBoundVar{P} (\HOLConst{Ea} (\HOLConst{privK} \HOLBoundVar{P}) (\HOLConst{plain} (\HOLConst{hash} (\HOLConst{plain} \HOLBoundVar{m\sb{\mathrm{1}}}))))
       (\HOLConst{plain} \HOLBoundVar{m\sb{\mathrm{2}}}) \HOLSymConst{\HOLTokenEquiv{}} (\HOLBoundVar{m\sb{\mathrm{1}}} \HOLSymConst{=} \HOLBoundVar{m\sb{\mathrm{2}}})
\end{SaveVerbatim}
\newcommand{\HOLcipherTheoremssignVerifyXXOneOne}{\UseVerbatim{HOLcipherTheoremssignVerifyXXOneOne}}
\begin{SaveVerbatim}{HOLcipherTheoremssignVerifyOK}
\HOLTokenTurnstile{} \HOLSymConst{\HOLTokenForall{}}\HOLBoundVar{P} \HOLBoundVar{msg}. \HOLConst{signVerify} \HOLBoundVar{P} (\HOLConst{sign} \HOLBoundVar{P} (\HOLConst{hash} (\HOLConst{plain} \HOLBoundVar{msg}))) (\HOLConst{plain} \HOLBoundVar{msg})
\end{SaveVerbatim}
\newcommand{\HOLcipherTheoremssignVerifyOK}{\UseVerbatim{HOLcipherTheoremssignVerifyOK}}
\newcommand{\HOLcipherTheorems}{
\HOLThmTag{cipher}{deciphP_clauses}\HOLcipherTheoremsdeciphPXXclauses
\HOLThmTag{cipher}{deciphP_def}\HOLcipherTheoremsdeciphPXXdef
\HOLThmTag{cipher}{deciphP_ind}\HOLcipherTheoremsdeciphPXXind
\HOLThmTag{cipher}{deciphS_clauses}\HOLcipherTheoremsdeciphSXXclauses
\HOLThmTag{cipher}{deciphS_def}\HOLcipherTheoremsdeciphSXXdef
\HOLThmTag{cipher}{deciphS_ind}\HOLcipherTheoremsdeciphSXXind
\HOLThmTag{cipher}{signVerify_11}\HOLcipherTheoremssignVerifyXXOneOne
\HOLThmTag{cipher}{signVerifyOK}\HOLcipherTheoremssignVerifyOK
}

\newcommand{\HOLcommandDate}{20 August 2016}
\newcommand{\HOLcommandTime}{11:38}
\begin{SaveVerbatim}{HOLcommandDatatypescommands}
\HOLFreeVar{commands} = \HOLConst{MC} \HOLTyOp{missionCommands} \HOLTokenBar{} \HOLConst{WC} \HOLTyOp{weaponCommands}
\end{SaveVerbatim}
\newcommand{\HOLcommandDatatypescommands}{\UseVerbatim{HOLcommandDatatypescommands}}
\begin{SaveVerbatim}{HOLcommandDatatypesmissionCommands}
\HOLFreeVar{missionCommands} = \HOLConst{go} \HOLTokenBar{} \HOLConst{nogo}
\end{SaveVerbatim}
\newcommand{\HOLcommandDatatypesmissionCommands}{\UseVerbatim{HOLcommandDatatypesmissionCommands}}
\begin{SaveVerbatim}{HOLcommandDatatypesweaponCommands}
\HOLFreeVar{weaponCommands} = \HOLConst{launch} \HOLTokenBar{} \HOLConst{abort}
\end{SaveVerbatim}
\newcommand{\HOLcommandDatatypesweaponCommands}{\UseVerbatim{HOLcommandDatatypesweaponCommands}}
\newcommand{\HOLcommandDatatypes}{
\HOLcommandDatatypescommands\HOLcommandDatatypesmissionCommands\HOLcommandDatatypesweaponCommands}

\newcommand{\HOLmissionRolesDate}{20 August 2016}
\newcommand{\HOLmissionRolesTime}{11:38}
\begin{SaveVerbatim}{HOLmissionRolesDatatypesmissionRoles}
\HOLFreeVar{missionRoles} = \HOLConst{BFC} \HOLTokenBar{} \HOLConst{GFC} \HOLTokenBar{} \HOLConst{BFO} \HOLTokenBar{} \HOLConst{GFO}
\end{SaveVerbatim}
\newcommand{\HOLmissionRolesDatatypesmissionRoles}{\UseVerbatim{HOLmissionRolesDatatypesmissionRoles}}
\newcommand{\HOLmissionRolesDatatypes}{
\HOLmissionRolesDatatypesmissionRoles}
\begin{SaveVerbatim}{HOLmissionRolesTheoremsAlternateControlsOneXXthm}
\HOLTokenTurnstile{} (\HOLFreeVar{M}\HOLSymConst{,}\HOLFreeVar{Oi}\HOLSymConst{,}\HOLFreeVar{Os}) \HOLConst{sat} \HOLFreeVar{P} \HOLConst{says} \HOLFreeVar{f} \HOLSymConst{\HOLTokenImp{}}
   (\HOLFreeVar{M}\HOLSymConst{,}\HOLFreeVar{Oi}\HOLSymConst{,}\HOLFreeVar{Os}) \HOLConst{sat} \HOLFreeVar{P} \HOLConst{controls} \HOLFreeVar{f} \HOLConst{andf} \HOLFreeVar{Q} \HOLConst{controls} \HOLFreeVar{f} \HOLSymConst{\HOLTokenImp{}}
   (\HOLFreeVar{M}\HOLSymConst{,}\HOLFreeVar{Oi}\HOLSymConst{,}\HOLFreeVar{Os}) \HOLConst{sat} \HOLFreeVar{f}
\end{SaveVerbatim}
\newcommand{\HOLmissionRolesTheoremsAlternateControlsOneXXthm}{\UseVerbatim{HOLmissionRolesTheoremsAlternateControlsOneXXthm}}
\begin{SaveVerbatim}{HOLmissionRolesTheoremsAlternateControlsTwoXXthm}
\HOLTokenTurnstile{} (\HOLFreeVar{M}\HOLSymConst{,}\HOLFreeVar{Oi}\HOLSymConst{,}\HOLFreeVar{Os}) \HOLConst{sat} \HOLFreeVar{Q} \HOLConst{says} \HOLFreeVar{f} \HOLSymConst{\HOLTokenImp{}}
   (\HOLFreeVar{M}\HOLSymConst{,}\HOLFreeVar{Oi}\HOLSymConst{,}\HOLFreeVar{Os}) \HOLConst{sat} \HOLFreeVar{P} \HOLConst{controls} \HOLFreeVar{f} \HOLConst{andf} \HOLFreeVar{Q} \HOLConst{controls} \HOLFreeVar{f} \HOLSymConst{\HOLTokenImp{}}
   (\HOLFreeVar{M}\HOLSymConst{,}\HOLFreeVar{Oi}\HOLSymConst{,}\HOLFreeVar{Os}) \HOLConst{sat} \HOLFreeVar{f}
\end{SaveVerbatim}
\newcommand{\HOLmissionRolesTheoremsAlternateControlsTwoXXthm}{\UseVerbatim{HOLmissionRolesTheoremsAlternateControlsTwoXXthm}}
\begin{SaveVerbatim}{HOLmissionRolesTheoremsDualControlXXthm}
\HOLTokenTurnstile{} (\HOLFreeVar{M}\HOLSymConst{,}\HOLFreeVar{Oi}\HOLSymConst{,}\HOLFreeVar{Os}) \HOLConst{sat} \HOLFreeVar{P} \HOLConst{says} \HOLFreeVar{s} \HOLSymConst{\HOLTokenImp{}}
   (\HOLFreeVar{M}\HOLSymConst{,}\HOLFreeVar{Oi}\HOLSymConst{,}\HOLFreeVar{Os}) \HOLConst{sat} \HOLFreeVar{Q} \HOLConst{says} \HOLFreeVar{s} \HOLSymConst{\HOLTokenImp{}}
   (\HOLFreeVar{M}\HOLSymConst{,}\HOLFreeVar{Oi}\HOLSymConst{,}\HOLFreeVar{Os}) \HOLConst{sat} \HOLFreeVar{P} \HOLConst{meet} \HOLFreeVar{Q} \HOLConst{controls} \HOLFreeVar{s} \HOLSymConst{\HOLTokenImp{}}
   (\HOLFreeVar{M}\HOLSymConst{,}\HOLFreeVar{Oi}\HOLSymConst{,}\HOLFreeVar{Os}) \HOLConst{sat} \HOLFreeVar{s}
\end{SaveVerbatim}
\newcommand{\HOLmissionRolesTheoremsDualControlXXthm}{\UseVerbatim{HOLmissionRolesTheoremsDualControlXXthm}}
\begin{SaveVerbatim}{HOLmissionRolesTheoremsImpliedControlsSaysXXthm}
\HOLTokenTurnstile{} \HOLSymConst{\HOLTokenForall{}}\HOLBoundVar{Q}.
     (\HOLFreeVar{M}\HOLSymConst{,}\HOLFreeVar{Oi}\HOLSymConst{,}\HOLFreeVar{Os}) \HOLConst{sat} \HOLFreeVar{P} \HOLConst{says} \HOLFreeVar{s\sb{\mathrm{1}}} \HOLSymConst{\HOLTokenImp{}}
     (\HOLFreeVar{M}\HOLSymConst{,}\HOLFreeVar{Oi}\HOLSymConst{,}\HOLFreeVar{Os}) \HOLConst{sat} \HOLFreeVar{P} \HOLConst{controls} \HOLFreeVar{s\sb{\mathrm{1}}} \HOLSymConst{\HOLTokenImp{}}
     (\HOLFreeVar{M}\HOLSymConst{,}\HOLFreeVar{Oi}\HOLSymConst{,}\HOLFreeVar{Os}) \HOLConst{sat} \HOLFreeVar{s\sb{\mathrm{1}}} \HOLConst{impf} \HOLFreeVar{s\sb{\mathrm{2}}} \HOLSymConst{\HOLTokenImp{}}
     (\HOLFreeVar{M}\HOLSymConst{,}\HOLFreeVar{Oi}\HOLSymConst{,}\HOLFreeVar{Os}) \HOLConst{sat} \HOLBoundVar{Q} \HOLConst{says} \HOLFreeVar{s\sb{\mathrm{2}}}
\end{SaveVerbatim}
\newcommand{\HOLmissionRolesTheoremsImpliedControlsSaysXXthm}{\UseVerbatim{HOLmissionRolesTheoremsImpliedControlsSaysXXthm}}
\newcommand{\HOLmissionRolesTheorems}{
\HOLThmTag{missionRoles}{AlternateControls1_thm}\HOLmissionRolesTheoremsAlternateControlsOneXXthm
\HOLThmTag{missionRoles}{AlternateControls2_thm}\HOLmissionRolesTheoremsAlternateControlsTwoXXthm
\HOLThmTag{missionRoles}{DualControl_thm}\HOLmissionRolesTheoremsDualControlXXthm
\HOLThmTag{missionRoles}{ImpliedControlsSays_thm}\HOLmissionRolesTheoremsImpliedControlsSaysXXthm
}

\newcommand{\HOLmissionStaffDate}{20 August 2016}
\newcommand{\HOLmissionStaffTime}{11:38}
\begin{SaveVerbatim}{HOLmissionStaffDatatypesmissionRoleStaff}
\HOLFreeVar{missionRoleStaff} = \HOLConst{Role} \HOLTyOp{missionRoles} \HOLTokenBar{} \HOLConst{Staff} \HOLTyOp{missionStaff}
\end{SaveVerbatim}
\newcommand{\HOLmissionStaffDatatypesmissionRoleStaff}{\UseVerbatim{HOLmissionStaffDatatypesmissionRoleStaff}}
\begin{SaveVerbatim}{HOLmissionStaffDatatypesmissionStaff}
\HOLFreeVar{missionStaff} = \HOLConst{Alice} \HOLTokenBar{} \HOLConst{Bob} \HOLTokenBar{} \HOLConst{Carol} \HOLTokenBar{} \HOLConst{Dan} \HOLTokenBar{} \HOLConst{Weapon}
\end{SaveVerbatim}
\newcommand{\HOLmissionStaffDatatypesmissionStaff}{\UseVerbatim{HOLmissionStaffDatatypesmissionStaff}}
\newcommand{\HOLmissionStaffDatatypes}{
\HOLmissionStaffDatatypesmissionRoleStaff\HOLmissionStaffDatatypesmissionStaff}
\begin{SaveVerbatim}{HOLmissionStaffTheoremsImpliedControlsDelegationXXthm}
\HOLTokenTurnstile{} \HOLSymConst{\HOLTokenForall{}}\HOLBoundVar{R}.
     (\HOLFreeVar{M}\HOLSymConst{,}\HOLFreeVar{Oi}\HOLSymConst{,}\HOLFreeVar{Os}) \HOLConst{sat} \HOLFreeVar{Q} \HOLConst{controls} \HOLFreeVar{f\sb{\mathrm{1}}} \HOLSymConst{\HOLTokenImp{}}
     (\HOLFreeVar{M}\HOLSymConst{,}\HOLFreeVar{Oi}\HOLSymConst{,}\HOLFreeVar{Os}) \HOLConst{sat} \HOLConst{reps} \HOLFreeVar{P} \HOLFreeVar{Q} \HOLFreeVar{f\sb{\mathrm{1}}} \HOLSymConst{\HOLTokenImp{}}
     (\HOLFreeVar{M}\HOLSymConst{,}\HOLFreeVar{Oi}\HOLSymConst{,}\HOLFreeVar{Os}) \HOLConst{sat} \HOLFreeVar{P} \HOLConst{quoting} \HOLFreeVar{Q} \HOLConst{says} \HOLFreeVar{f\sb{\mathrm{1}}} \HOLSymConst{\HOLTokenImp{}}
     (\HOLFreeVar{M}\HOLSymConst{,}\HOLFreeVar{Oi}\HOLSymConst{,}\HOLFreeVar{Os}) \HOLConst{sat} \HOLFreeVar{f\sb{\mathrm{1}}} \HOLConst{impf} \HOLFreeVar{f\sb{\mathrm{2}}} \HOLSymConst{\HOLTokenImp{}}
     (\HOLFreeVar{M}\HOLSymConst{,}\HOLFreeVar{Oi}\HOLSymConst{,}\HOLFreeVar{Os}) \HOLConst{sat} \HOLBoundVar{R} \HOLConst{says} \HOLFreeVar{f\sb{\mathrm{2}}}
\end{SaveVerbatim}
\newcommand{\HOLmissionStaffTheoremsImpliedControlsDelegationXXthm}{\UseVerbatim{HOLmissionStaffTheoremsImpliedControlsDelegationXXthm}}
\newcommand{\HOLmissionStaffTheorems}{
\HOLThmTag{missionStaff}{ImpliedControlsDelegation_thm}\HOLmissionStaffTheoremsImpliedControlsDelegationXXthm
}

\newcommand{\HOLrevisedMissionKeysDate}{20 August 2016}
\newcommand{\HOLrevisedMissionKeysTime}{12:36}
\begin{SaveVerbatim}{HOLrevisedMissionKeysDatatypesmissionCA}
\HOLFreeVar{missionCA} = \HOLConst{JFCA} \HOLTokenBar{} \HOLConst{BFCA} \HOLTokenBar{} \HOLConst{GFCA}
\end{SaveVerbatim}
\newcommand{\HOLrevisedMissionKeysDatatypesmissionCA}{\UseVerbatim{HOLrevisedMissionKeysDatatypesmissionCA}}
\begin{SaveVerbatim}{HOLrevisedMissionKeysDatatypesmissionPrincipals}
\HOLFreeVar{missionPrincipals} =
    \HOLConst{MRole} \HOLTyOp{missionRoles}
  \HOLTokenBar{} \HOLConst{MStaff} \HOLTyOp{missionStaff}
  \HOLTokenBar{} \HOLConst{MCA} \HOLTyOp{missionCA}
  \HOLTokenBar{} \HOLConst{MKey} (\HOLTyOp{missionStaffCA} \HOLTyOp{pKey})
\end{SaveVerbatim}
\newcommand{\HOLrevisedMissionKeysDatatypesmissionPrincipals}{\UseVerbatim{HOLrevisedMissionKeysDatatypesmissionPrincipals}}
\begin{SaveVerbatim}{HOLrevisedMissionKeysDatatypesmissionStaffCA}
\HOLFreeVar{missionStaffCA} = \HOLConst{KStaff} \HOLTyOp{missionStaff} \HOLTokenBar{} \HOLConst{KCA} \HOLTyOp{missionCA}
\end{SaveVerbatim}
\newcommand{\HOLrevisedMissionKeysDatatypesmissionStaffCA}{\UseVerbatim{HOLrevisedMissionKeysDatatypesmissionStaffCA}}
\newcommand{\HOLrevisedMissionKeysDatatypes}{
\HOLrevisedMissionKeysDatatypesmissionCA\HOLrevisedMissionKeysDatatypesmissionPrincipals\HOLrevisedMissionKeysDatatypesmissionStaffCA}

\newcommand{\HOLmessageCertificateDate}{20 August 2016}
\newcommand{\HOLmessageCertificateTime}{12:36}
\begin{SaveVerbatim}{HOLmessageCertificateDatatypesAuthority}
\HOLFreeVar{Authority} = \HOLConst{Auth} \HOLTyOp{missionStaff}
\end{SaveVerbatim}
\newcommand{\HOLmessageCertificateDatatypesAuthority}{\UseVerbatim{HOLmessageCertificateDatatypesAuthority}}
\begin{SaveVerbatim}{HOLmessageCertificateDatatypesDelegate}
\HOLFreeVar{Delegate} = \HOLConst{For} \HOLTyOp{missionStaff}
\end{SaveVerbatim}
\newcommand{\HOLmessageCertificateDatatypesDelegate}{\UseVerbatim{HOLmessageCertificateDatatypesDelegate}}
\begin{SaveVerbatim}{HOLmessageCertificateDatatypesDestination}
\HOLFreeVar{Destination} = \HOLConst{To} \HOLTyOp{missionStaff}
\end{SaveVerbatim}
\newcommand{\HOLmessageCertificateDatatypesDestination}{\UseVerbatim{HOLmessageCertificateDatatypesDestination}}
\begin{SaveVerbatim}{HOLmessageCertificateDatatypesIssuer}
\HOLFreeVar{Issuer} = \HOLConst{CA} \HOLTyOp{missionCA}
\end{SaveVerbatim}
\newcommand{\HOLmessageCertificateDatatypesIssuer}{\UseVerbatim{HOLmessageCertificateDatatypesIssuer}}
\begin{SaveVerbatim}{HOLmessageCertificateDatatypesKCertSignature}
\HOLFreeVar{KCertSignature} =
    \HOLConst{KCrtSig} (((\HOLTyOp{missionCA} \HOLTokenProd{} \HOLTyOp{missionStaffCA} \HOLTokenProd{} \HOLTyOp{missionStaffCA} \HOLTyOp{pKey})
              \HOLTyOp{digest}, \HOLTyOp{missionStaffCA}) \HOLTyOp{asymMsg})
\end{SaveVerbatim}
\newcommand{\HOLmessageCertificateDatatypesKCertSignature}{\UseVerbatim{HOLmessageCertificateDatatypesKCertSignature}}
\begin{SaveVerbatim}{HOLmessageCertificateDatatypesKeyCertificate}
\HOLFreeVar{KeyCertificate} = \HOLConst{KCert} \HOLTyOp{Issuer} \HOLTyOp{Subject} \HOLTyOp{SubPubKey} \HOLTyOp{KCertSignature}
\end{SaveVerbatim}
\newcommand{\HOLmessageCertificateDatatypesKeyCertificate}{\UseVerbatim{HOLmessageCertificateDatatypesKeyCertificate}}
\begin{SaveVerbatim}{HOLmessageCertificateDatatypesMessage}
\HOLFreeVar{Message} =
    \HOLConst{MSG} \HOLTyOp{Origin} \HOLTyOp{Role} \HOLTyOp{Destination} ((\HOLTyOp{SymKey}, \HOLTyOp{missionStaff}) \HOLTyOp{asymMsg})
        (\HOLTyOp{Orders} \HOLTyOp{symMsg}) ((\HOLTyOp{Orders} \HOLTyOp{digest}, \HOLTyOp{missionStaff}) \HOLTyOp{asymMsg})
\end{SaveVerbatim}
\newcommand{\HOLmessageCertificateDatatypesMessage}{\UseVerbatim{HOLmessageCertificateDatatypesMessage}}
\begin{SaveVerbatim}{HOLmessageCertificateDatatypesOrders}
\HOLFreeVar{Orders} = \HOLConst{CMD} \HOLTyOp{Origin} \HOLTyOp{Role} \HOLTyOp{Destination} \HOLTyOp{commands}
\end{SaveVerbatim}
\newcommand{\HOLmessageCertificateDatatypesOrders}{\UseVerbatim{HOLmessageCertificateDatatypesOrders}}
\begin{SaveVerbatim}{HOLmessageCertificateDatatypesOrigin}
\HOLFreeVar{Origin} = \HOLConst{From} \HOLTyOp{missionStaff}
\end{SaveVerbatim}
\newcommand{\HOLmessageCertificateDatatypesOrigin}{\UseVerbatim{HOLmessageCertificateDatatypesOrigin}}
\begin{SaveVerbatim}{HOLmessageCertificateDatatypesRCertSignature}
\HOLFreeVar{RCertSignature} =
    \HOLConst{RCrtSig} (((\HOLTyOp{missionStaff} \HOLTokenProd{}
               \HOLTyOp{missionRoles} \HOLTokenProd{}
               \HOLTyOp{missionStaff} \HOLTokenProd{} \HOLTyOp{missionRoles} \HOLTokenProd{} \HOLTyOp{commands}) \HOLTyOp{digest},
              \HOLTyOp{missionStaffCA}) \HOLTyOp{asymMsg})
\end{SaveVerbatim}
\newcommand{\HOLmessageCertificateDatatypesRCertSignature}{\UseVerbatim{HOLmessageCertificateDatatypesRCertSignature}}
\begin{SaveVerbatim}{HOLmessageCertificateDatatypesRole}
\HOLFreeVar{Role} = \HOLConst{As} \HOLTyOp{missionRoles}
\end{SaveVerbatim}
\newcommand{\HOLmessageCertificateDatatypesRole}{\UseVerbatim{HOLmessageCertificateDatatypesRole}}
\begin{SaveVerbatim}{HOLmessageCertificateDatatypesRoleCertificate}
\HOLFreeVar{RoleCertificate} =
    \HOLConst{RCert} \HOLTyOp{Authority} \HOLTyOp{Role} \HOLTyOp{Delegate} \HOLTyOp{Role} \HOLTyOp{commands} \HOLTyOp{RCertSignature}
\end{SaveVerbatim}
\newcommand{\HOLmessageCertificateDatatypesRoleCertificate}{\UseVerbatim{HOLmessageCertificateDatatypesRoleCertificate}}
\begin{SaveVerbatim}{HOLmessageCertificateDatatypesRootKeyCertificate}
\HOLFreeVar{RootKeyCertificate} = \HOLConst{RootKCert} \HOLTyOp{Subject} \HOLTyOp{SubPubKey}
\end{SaveVerbatim}
\newcommand{\HOLmessageCertificateDatatypesRootKeyCertificate}{\UseVerbatim{HOLmessageCertificateDatatypesRootKeyCertificate}}
\begin{SaveVerbatim}{HOLmessageCertificateDatatypesRootRoleCertificate}
\HOLFreeVar{RootRoleCertificate} = \HOLConst{RootRCert} \HOLTyOp{Delegate} \HOLTyOp{Role} \HOLTyOp{commands}
\end{SaveVerbatim}
\newcommand{\HOLmessageCertificateDatatypesRootRoleCertificate}{\UseVerbatim{HOLmessageCertificateDatatypesRootRoleCertificate}}
\begin{SaveVerbatim}{HOLmessageCertificateDatatypesSubject}
\HOLFreeVar{Subject} = \HOLConst{Entity} \HOLTyOp{missionStaffCA}
\end{SaveVerbatim}
\newcommand{\HOLmessageCertificateDatatypesSubject}{\UseVerbatim{HOLmessageCertificateDatatypesSubject}}
\begin{SaveVerbatim}{HOLmessageCertificateDatatypesSubPubKey}
\HOLFreeVar{SubPubKey} = \HOLConst{SPubKey} (\HOLTyOp{missionStaffCA} \HOLTyOp{pKey})
\end{SaveVerbatim}
\newcommand{\HOLmessageCertificateDatatypesSubPubKey}{\UseVerbatim{HOLmessageCertificateDatatypesSubPubKey}}
\newcommand{\HOLmessageCertificateDatatypes}{
\HOLmessageCertificateDatatypesAuthority\HOLmessageCertificateDatatypesDelegate\HOLmessageCertificateDatatypesDestination\HOLmessageCertificateDatatypesIssuer\HOLmessageCertificateDatatypesKCertSignature\HOLmessageCertificateDatatypesKeyCertificate\HOLmessageCertificateDatatypesMessage\HOLmessageCertificateDatatypesOrders\HOLmessageCertificateDatatypesOrigin\HOLmessageCertificateDatatypesRCertSignature\HOLmessageCertificateDatatypesRole\HOLmessageCertificateDatatypesRoleCertificate\HOLmessageCertificateDatatypesRootKeyCertificate\HOLmessageCertificateDatatypesRootRoleCertificate\HOLmessageCertificateDatatypesSubject\HOLmessageCertificateDatatypesSubPubKey}
\begin{SaveVerbatim}{HOLmessageCertificateDefinitionsksatXXdef}
\HOLTokenTurnstile{} \HOLSymConst{\HOLTokenForall{}}\HOLBoundVar{M} \HOLBoundVar{Oi} \HOLBoundVar{Os} \HOLBoundVar{kcert}.
     (\HOLBoundVar{M}\HOLSymConst{,}\HOLBoundVar{Oi}\HOLSymConst{,}\HOLBoundVar{Os}) \HOLConst{ksat} \HOLBoundVar{kcert} \HOLSymConst{\HOLTokenEquiv{}} (\HOLConst{Ekcrt} \HOLBoundVar{Oi} \HOLBoundVar{Os} \HOLBoundVar{M} \HOLBoundVar{kcert} \HOLSymConst{=} \ensuremath{\cal{U}}(:'world))
\end{SaveVerbatim}
\newcommand{\HOLmessageCertificateDefinitionsksatXXdef}{\UseVerbatim{HOLmessageCertificateDefinitionsksatXXdef}}
\begin{SaveVerbatim}{HOLmessageCertificateDefinitionsmsatXXdef}
\HOLTokenTurnstile{} \HOLSymConst{\HOLTokenForall{}}\HOLBoundVar{M} \HOLBoundVar{Oi} \HOLBoundVar{Os} \HOLBoundVar{msg}.
     (\HOLBoundVar{M}\HOLSymConst{,}\HOLBoundVar{Oi}\HOLSymConst{,}\HOLBoundVar{Os}) \HOLConst{msat} \HOLBoundVar{msg} \HOLSymConst{\HOLTokenEquiv{}} (\HOLConst{Emsg} \HOLBoundVar{Oi} \HOLBoundVar{Os} \HOLBoundVar{M} \HOLBoundVar{msg} \HOLSymConst{=} \ensuremath{\cal{U}}(:'world))
\end{SaveVerbatim}
\newcommand{\HOLmessageCertificateDefinitionsmsatXXdef}{\UseVerbatim{HOLmessageCertificateDefinitionsmsatXXdef}}
\begin{SaveVerbatim}{HOLmessageCertificateDefinitionsroleOriginXXdef}
\HOLTokenTurnstile{} \HOLSymConst{\HOLTokenForall{}}\HOLBoundVar{role}. \HOLConst{roleOrigin} (\HOLConst{As} \HOLBoundVar{role}) \HOLSymConst{=} \HOLBoundVar{role}
\end{SaveVerbatim}
\newcommand{\HOLmessageCertificateDefinitionsroleOriginXXdef}{\UseVerbatim{HOLmessageCertificateDefinitionsroleOriginXXdef}}
\begin{SaveVerbatim}{HOLmessageCertificateDefinitionsrootksatXXdef}
\HOLTokenTurnstile{} \HOLSymConst{\HOLTokenForall{}}\HOLBoundVar{M} \HOLBoundVar{Oi} \HOLBoundVar{Os} \HOLBoundVar{rootkcert}.
     (\HOLBoundVar{M}\HOLSymConst{,}\HOLBoundVar{Oi}\HOLSymConst{,}\HOLBoundVar{Os}) \HOLConst{rootksat} \HOLBoundVar{rootkcert} \HOLSymConst{\HOLTokenEquiv{}}
     (\HOLConst{Erootkcrt} \HOLBoundVar{Oi} \HOLBoundVar{Os} \HOLBoundVar{M} \HOLBoundVar{rootkcert} \HOLSymConst{=} \ensuremath{\cal{U}}(:'world))
\end{SaveVerbatim}
\newcommand{\HOLmessageCertificateDefinitionsrootksatXXdef}{\UseVerbatim{HOLmessageCertificateDefinitionsrootksatXXdef}}
\begin{SaveVerbatim}{HOLmessageCertificateDefinitionsrootrsatXXdef}
\HOLTokenTurnstile{} \HOLSymConst{\HOLTokenForall{}}\HOLBoundVar{M} \HOLBoundVar{Oi} \HOLBoundVar{Os} \HOLBoundVar{rootrcert}.
     (\HOLBoundVar{M}\HOLSymConst{,}\HOLBoundVar{Oi}\HOLSymConst{,}\HOLBoundVar{Os}) \HOLConst{rootrsat} \HOLBoundVar{rootrcert} \HOLSymConst{\HOLTokenEquiv{}}
     (\HOLConst{Erootrcrt} \HOLBoundVar{Oi} \HOLBoundVar{Os} \HOLBoundVar{M} \HOLBoundVar{rootrcert} \HOLSymConst{=} \ensuremath{\cal{U}}(:'world))
\end{SaveVerbatim}
\newcommand{\HOLmessageCertificateDefinitionsrootrsatXXdef}{\UseVerbatim{HOLmessageCertificateDefinitionsrootrsatXXdef}}
\begin{SaveVerbatim}{HOLmessageCertificateDefinitionsrsatXXdef}
\HOLTokenTurnstile{} \HOLSymConst{\HOLTokenForall{}}\HOLBoundVar{M} \HOLBoundVar{Oi} \HOLBoundVar{Os} \HOLBoundVar{rcert}.
     (\HOLBoundVar{M}\HOLSymConst{,}\HOLBoundVar{Oi}\HOLSymConst{,}\HOLBoundVar{Os}) \HOLConst{rsat} \HOLBoundVar{rcert} \HOLSymConst{\HOLTokenEquiv{}} (\HOLConst{Ercrt} \HOLBoundVar{Oi} \HOLBoundVar{Os} \HOLBoundVar{M} \HOLBoundVar{rcert} \HOLSymConst{=} \ensuremath{\cal{U}}(:'world))
\end{SaveVerbatim}
\newcommand{\HOLmessageCertificateDefinitionsrsatXXdef}{\UseVerbatim{HOLmessageCertificateDefinitionsrsatXXdef}}
\begin{SaveVerbatim}{HOLmessageCertificateDefinitionsstaffDestinationXXdef}
\HOLTokenTurnstile{} \HOLSymConst{\HOLTokenForall{}}\HOLBoundVar{staff}. \HOLConst{staffDestination} (\HOLConst{To} \HOLBoundVar{staff}) \HOLSymConst{=} \HOLBoundVar{staff}
\end{SaveVerbatim}
\newcommand{\HOLmessageCertificateDefinitionsstaffDestinationXXdef}{\UseVerbatim{HOLmessageCertificateDefinitionsstaffDestinationXXdef}}
\begin{SaveVerbatim}{HOLmessageCertificateDefinitionsstaffOriginXXdef}
\HOLTokenTurnstile{} \HOLSymConst{\HOLTokenForall{}}\HOLBoundVar{staff}. \HOLConst{staffOrigin} (\HOLConst{From} \HOLBoundVar{staff}) \HOLSymConst{=} \HOLBoundVar{staff}
\end{SaveVerbatim}
\newcommand{\HOLmessageCertificateDefinitionsstaffOriginXXdef}{\UseVerbatim{HOLmessageCertificateDefinitionsstaffOriginXXdef}}
\newcommand{\HOLmessageCertificateDefinitions}{
\HOLDfnTag{messageCertificate}{ksat_def}\HOLmessageCertificateDefinitionsksatXXdef
\HOLDfnTag{messageCertificate}{msat_def}\HOLmessageCertificateDefinitionsmsatXXdef
\HOLDfnTag{messageCertificate}{roleOrigin_def}\HOLmessageCertificateDefinitionsroleOriginXXdef
\HOLDfnTag{messageCertificate}{rootksat_def}\HOLmessageCertificateDefinitionsrootksatXXdef
\HOLDfnTag{messageCertificate}{rootrsat_def}\HOLmessageCertificateDefinitionsrootrsatXXdef
\HOLDfnTag{messageCertificate}{rsat_def}\HOLmessageCertificateDefinitionsrsatXXdef
\HOLDfnTag{messageCertificate}{staffDestination_def}\HOLmessageCertificateDefinitionsstaffDestinationXXdef
\HOLDfnTag{messageCertificate}{staffOrigin_def}\HOLmessageCertificateDefinitionsstaffOriginXXdef
}
\begin{SaveVerbatim}{HOLmessageCertificateTheoremscheckKCertXXdef}
\HOLTokenTurnstile{} \HOLConst{checkKCert}
     (\HOLConst{KCert} (\HOLConst{CA} \HOLFreeVar{mca}) (\HOLConst{Entity} \HOLFreeVar{mStaffCA}) (\HOLConst{SPubKey} \HOLFreeVar{pubKey})
        (\HOLConst{KCrtSig} \HOLFreeVar{kcertSig})) \HOLSymConst{\HOLTokenEquiv{}}
   \HOLConst{signVerify} (\HOLConst{KCA} \HOLFreeVar{mca}) \HOLFreeVar{kcertSig} (\HOLConst{plain} (\HOLFreeVar{mca}\HOLSymConst{,}\HOLFreeVar{mStaffCA}\HOLSymConst{,}\HOLFreeVar{pubKey}))
\end{SaveVerbatim}
\newcommand{\HOLmessageCertificateTheoremscheckKCertXXdef}{\UseVerbatim{HOLmessageCertificateTheoremscheckKCertXXdef}}
\begin{SaveVerbatim}{HOLmessageCertificateTheoremscheckKCertXXind}
\HOLTokenTurnstile{} \HOLSymConst{\HOLTokenForall{}}\HOLBoundVar{P}.
     (\HOLSymConst{\HOLTokenForall{}}\HOLBoundVar{mca} \HOLBoundVar{mStaffCA} \HOLBoundVar{pubKey} \HOLBoundVar{kcertSig}.
        \HOLBoundVar{P}
          (\HOLConst{KCert} (\HOLConst{CA} \HOLBoundVar{mca}) (\HOLConst{Entity} \HOLBoundVar{mStaffCA}) (\HOLConst{SPubKey} \HOLBoundVar{pubKey})
             (\HOLConst{KCrtSig} \HOLBoundVar{kcertSig}))) \HOLSymConst{\HOLTokenImp{}}
     \HOLSymConst{\HOLTokenForall{}}\HOLBoundVar{v}. \HOLBoundVar{P} \HOLBoundVar{v}
\end{SaveVerbatim}
\newcommand{\HOLmessageCertificateTheoremscheckKCertXXind}{\UseVerbatim{HOLmessageCertificateTheoremscheckKCertXXind}}
\begin{SaveVerbatim}{HOLmessageCertificateTheoremscheckKCertOK}
\HOLTokenTurnstile{} \HOLConst{checkKCert}
     (\HOLConst{KCert} (\HOLConst{CA} \HOLFreeVar{mca}) (\HOLConst{Entity} \HOLFreeVar{mStaffCA}) (\HOLConst{SPubKey} \HOLFreeVar{pubKey})
        (\HOLConst{KCrtSig}
           (\HOLConst{sign} (\HOLConst{KCA} \HOLFreeVar{mca})
              (\HOLConst{hash} (\HOLConst{plain} (\HOLFreeVar{mca}\HOLSymConst{,}\HOLFreeVar{mStaffCA}\HOLSymConst{,}\HOLFreeVar{pubKey}))))))
\end{SaveVerbatim}
\newcommand{\HOLmessageCertificateTheoremscheckKCertOK}{\UseVerbatim{HOLmessageCertificateTheoremscheckKCertOK}}
\begin{SaveVerbatim}{HOLmessageCertificateTheoremscheckMSGXXdef}
\HOLTokenTurnstile{} \HOLConst{checkMSG}
     (\HOLConst{MSG} (\HOLConst{From} \HOLFreeVar{sender}) (\HOLConst{As} \HOLFreeVar{role}) (\HOLConst{To} \HOLFreeVar{recipient}) \HOLFreeVar{enDEK} \HOLFreeVar{enCMDMsg}
        \HOLFreeVar{CMDMsgSig}) \HOLSymConst{\HOLTokenEquiv{}}
   (\HOLKeyword{let} \HOLBoundVar{dek} = \HOLConst{getMessage} (\HOLConst{deciphP} (\HOLConst{privK} \HOLFreeVar{recipient}) \HOLFreeVar{enDEK}) \HOLKeyword{in}
    \HOLKeyword{let} \HOLBoundVar{msgContents} = \HOLConst{deciphS} \HOLBoundVar{dek} \HOLFreeVar{enCMDMsg}
    \HOLKeyword{in}
      \HOLConst{signVerify} \HOLFreeVar{sender} \HOLFreeVar{CMDMsgSig} \HOLBoundVar{msgContents})
\end{SaveVerbatim}
\newcommand{\HOLmessageCertificateTheoremscheckMSGXXdef}{\UseVerbatim{HOLmessageCertificateTheoremscheckMSGXXdef}}
\begin{SaveVerbatim}{HOLmessageCertificateTheoremscheckMSGXXind}
\HOLTokenTurnstile{} \HOLSymConst{\HOLTokenForall{}}\HOLBoundVar{P}.
     (\HOLSymConst{\HOLTokenForall{}}\HOLBoundVar{sender} \HOLBoundVar{role} \HOLBoundVar{recipient} \HOLBoundVar{enDEK} \HOLBoundVar{enCMDMsg} \HOLBoundVar{CMDMsgSig}.
        \HOLBoundVar{P}
          (\HOLConst{MSG} (\HOLConst{From} \HOLBoundVar{sender}) (\HOLConst{As} \HOLBoundVar{role}) (\HOLConst{To} \HOLBoundVar{recipient}) \HOLBoundVar{enDEK}
             \HOLBoundVar{enCMDMsg} \HOLBoundVar{CMDMsgSig})) \HOLSymConst{\HOLTokenImp{}}
     \HOLSymConst{\HOLTokenForall{}}\HOLBoundVar{v}. \HOLBoundVar{P} \HOLBoundVar{v}
\end{SaveVerbatim}
\newcommand{\HOLmessageCertificateTheoremscheckMSGXXind}{\UseVerbatim{HOLmessageCertificateTheoremscheckMSGXXind}}
\begin{SaveVerbatim}{HOLmessageCertificateTheoremscheckMSGOK}
\HOLTokenTurnstile{} \HOLConst{checkMSG}
     (\HOLConst{MSG} (\HOLConst{From} \HOLFreeVar{sender}) (\HOLConst{As} \HOLFreeVar{role}) (\HOLConst{To} \HOLFreeVar{recipient})
        (\HOLConst{Ea} (\HOLConst{pubK} \HOLFreeVar{recipient}) (\HOLConst{plain} (\HOLConst{sym} \HOLFreeVar{dek})))
        (\HOLConst{Es} (\HOLConst{sym} \HOLFreeVar{dek}) (\HOLConst{plain} \HOLFreeVar{order}))
        (\HOLConst{sign} \HOLFreeVar{sender} (\HOLConst{hash} (\HOLConst{plain} \HOLFreeVar{order}))))
\end{SaveVerbatim}
\newcommand{\HOLmessageCertificateTheoremscheckMSGOK}{\UseVerbatim{HOLmessageCertificateTheoremscheckMSGOK}}
\begin{SaveVerbatim}{HOLmessageCertificateTheoremscheckRCertXXdef}
\HOLTokenTurnstile{} \HOLConst{checkRCert}
     (\HOLConst{RCert} (\HOLConst{Auth} \HOLFreeVar{commander}) (\HOLConst{As} \HOLFreeVar{cmdRole}) (\HOLConst{For} \HOLFreeVar{delegate})
        (\HOLConst{As} \HOLFreeVar{delegateRole}) \HOLFreeVar{order} (\HOLConst{RCrtSig} \HOLFreeVar{rcertSig})) \HOLSymConst{\HOLTokenEquiv{}}
   \HOLConst{signVerify} (\HOLConst{KStaff} \HOLFreeVar{commander}) \HOLFreeVar{rcertSig}
     (\HOLConst{plain} (\HOLFreeVar{commander}\HOLSymConst{,}\HOLFreeVar{cmdRole}\HOLSymConst{,}\HOLFreeVar{delegate}\HOLSymConst{,}\HOLFreeVar{delegateRole}\HOLSymConst{,}\HOLFreeVar{order}))
\end{SaveVerbatim}
\newcommand{\HOLmessageCertificateTheoremscheckRCertXXdef}{\UseVerbatim{HOLmessageCertificateTheoremscheckRCertXXdef}}
\begin{SaveVerbatim}{HOLmessageCertificateTheoremscheckRCertXXind}
\HOLTokenTurnstile{} \HOLSymConst{\HOLTokenForall{}}\HOLBoundVar{P}.
     (\HOLSymConst{\HOLTokenForall{}}\HOLBoundVar{commander} \HOLBoundVar{cmdRole} \HOLBoundVar{delegate} \HOLBoundVar{delegateRole} \HOLBoundVar{order} \HOLBoundVar{rcertSig}.
        \HOLBoundVar{P}
          (\HOLConst{RCert} (\HOLConst{Auth} \HOLBoundVar{commander}) (\HOLConst{As} \HOLBoundVar{cmdRole}) (\HOLConst{For} \HOLBoundVar{delegate})
             (\HOLConst{As} \HOLBoundVar{delegateRole}) \HOLBoundVar{order} (\HOLConst{RCrtSig} \HOLBoundVar{rcertSig}))) \HOLSymConst{\HOLTokenImp{}}
     \HOLSymConst{\HOLTokenForall{}}\HOLBoundVar{v}. \HOLBoundVar{P} \HOLBoundVar{v}
\end{SaveVerbatim}
\newcommand{\HOLmessageCertificateTheoremscheckRCertXXind}{\UseVerbatim{HOLmessageCertificateTheoremscheckRCertXXind}}
\begin{SaveVerbatim}{HOLmessageCertificateTheoremscheckRCertOK}
\HOLTokenTurnstile{} \HOLConst{checkRCert}
     (\HOLConst{RCert} (\HOLConst{Auth} \HOLFreeVar{commander}) (\HOLConst{As} \HOLFreeVar{cmdRole}) (\HOLConst{For} \HOLFreeVar{delegate})
        (\HOLConst{As} \HOLFreeVar{delegateRole}) \HOLFreeVar{order}
        (\HOLConst{RCrtSig}
           (\HOLConst{sign} (\HOLConst{KStaff} \HOLFreeVar{commander})
              (\HOLConst{hash}
                 (\HOLConst{plain}
                    (\HOLFreeVar{commander}\HOLSymConst{,}\HOLFreeVar{cmdRole}\HOLSymConst{,}\HOLFreeVar{delegate}\HOLSymConst{,}\HOLFreeVar{delegateRole}\HOLSymConst{,}
                     \HOLFreeVar{order}))))))
\end{SaveVerbatim}
\newcommand{\HOLmessageCertificateTheoremscheckRCertOK}{\UseVerbatim{HOLmessageCertificateTheoremscheckRCertOK}}
\begin{SaveVerbatim}{HOLmessageCertificateTheoremsEkcrtXXdef}
\HOLTokenTurnstile{} (\HOLConst{Ekcrt} \HOLFreeVar{Oi} \HOLFreeVar{Os} \HOLFreeVar{M}
      (\HOLConst{KCert} (\HOLConst{CA} \HOLFreeVar{mca}) (\HOLConst{Entity} (\HOLConst{KStaff} \HOLFreeVar{staff})) (\HOLConst{SPubKey} \HOLFreeVar{pubKey})
         (\HOLConst{KCrtSig} \HOLFreeVar{kcertSig})) \HOLSymConst{=}
    \HOLKeyword{if}
      \HOLConst{checkKCert}
        (\HOLConst{KCert} (\HOLConst{CA} \HOLFreeVar{mca}) (\HOLConst{Entity} (\HOLConst{KStaff} \HOLFreeVar{staff})) (\HOLConst{SPubKey} \HOLFreeVar{pubKey})
           (\HOLConst{KCrtSig} \HOLFreeVar{kcertSig}))
    \HOLKeyword{then}
      \HOLConst{Efn} \HOLFreeVar{Oi} \HOLFreeVar{Os} \HOLFreeVar{M}
        (\HOLConst{Name} (\HOLConst{MKey} (\HOLConst{pubK} (\HOLConst{KCA} \HOLFreeVar{mca}))) \HOLConst{says}
         \HOLConst{Name} (\HOLConst{MKey} \HOLFreeVar{pubKey}) \HOLConst{speaks_for} \HOLConst{Name} (\HOLConst{MStaff} \HOLFreeVar{staff}))
    \HOLKeyword{else} \HOLTokenLeftbrace{}\HOLTokenRightbrace{}) \HOLSymConst{\HOLTokenConj{}}
   (\HOLConst{Ekcrt} \HOLFreeVar{Oi} \HOLFreeVar{Os} \HOLFreeVar{M}
      (\HOLConst{KCert} (\HOLConst{CA} \HOLFreeVar{mca}) (\HOLConst{Entity} (\HOLConst{KCA} \HOLFreeVar{ca\sb{\mathrm{2}}})) (\HOLConst{SPubKey} \HOLFreeVar{pubKey})
         (\HOLConst{KCrtSig} \HOLFreeVar{kcertSig})) \HOLSymConst{=}
    \HOLKeyword{if}
      \HOLConst{checkKCert}
        (\HOLConst{KCert} (\HOLConst{CA} \HOLFreeVar{mca}) (\HOLConst{Entity} (\HOLConst{KCA} \HOLFreeVar{ca\sb{\mathrm{2}}})) (\HOLConst{SPubKey} \HOLFreeVar{pubKey})
           (\HOLConst{KCrtSig} \HOLFreeVar{kcertSig}))
    \HOLKeyword{then}
      \HOLConst{Efn} \HOLFreeVar{Oi} \HOLFreeVar{Os} \HOLFreeVar{M}
        (\HOLConst{Name} (\HOLConst{MKey} (\HOLConst{pubK} (\HOLConst{KCA} \HOLFreeVar{mca}))) \HOLConst{says}
         \HOLConst{Name} (\HOLConst{MKey} \HOLFreeVar{pubKey}) \HOLConst{speaks_for} \HOLConst{Name} (\HOLConst{MCA} \HOLFreeVar{ca\sb{\mathrm{2}}}))
    \HOLKeyword{else} \HOLTokenLeftbrace{}\HOLTokenRightbrace{})
\end{SaveVerbatim}
\newcommand{\HOLmessageCertificateTheoremsEkcrtXXdef}{\UseVerbatim{HOLmessageCertificateTheoremsEkcrtXXdef}}
\begin{SaveVerbatim}{HOLmessageCertificateTheoremsEkcrtXXind}
\HOLTokenTurnstile{} \HOLSymConst{\HOLTokenForall{}}\HOLBoundVar{P}.
     (\HOLSymConst{\HOLTokenForall{}}\HOLBoundVar{Oi} \HOLBoundVar{Os} \HOLBoundVar{M} \HOLBoundVar{mca} \HOLBoundVar{staff} \HOLBoundVar{pubKey} \HOLBoundVar{kcertSig}.
        \HOLBoundVar{P} \HOLBoundVar{Oi} \HOLBoundVar{Os} \HOLBoundVar{M}
          (\HOLConst{KCert} (\HOLConst{CA} \HOLBoundVar{mca}) (\HOLConst{Entity} (\HOLConst{KStaff} \HOLBoundVar{staff}))
             (\HOLConst{SPubKey} \HOLBoundVar{pubKey}) (\HOLConst{KCrtSig} \HOLBoundVar{kcertSig}))) \HOLSymConst{\HOLTokenConj{}}
     (\HOLSymConst{\HOLTokenForall{}}\HOLBoundVar{Oi} \HOLBoundVar{Os} \HOLBoundVar{M} \HOLBoundVar{mca} \HOLBoundVar{ca\sb{\mathrm{2}}} \HOLBoundVar{pubKey} \HOLBoundVar{kcertSig}.
        \HOLBoundVar{P} \HOLBoundVar{Oi} \HOLBoundVar{Os} \HOLBoundVar{M}
          (\HOLConst{KCert} (\HOLConst{CA} \HOLBoundVar{mca}) (\HOLConst{Entity} (\HOLConst{KCA} \HOLBoundVar{ca\sb{\mathrm{2}}})) (\HOLConst{SPubKey} \HOLBoundVar{pubKey})
             (\HOLConst{KCrtSig} \HOLBoundVar{kcertSig}))) \HOLSymConst{\HOLTokenImp{}}
     \HOLSymConst{\HOLTokenForall{}}\HOLBoundVar{v} \HOLBoundVar{v\sb{\mathrm{1}}} \HOLBoundVar{v\sb{\mathrm{2}}} \HOLBoundVar{v\sb{\mathrm{3}}}. \HOLBoundVar{P} \HOLBoundVar{v} \HOLBoundVar{v\sb{\mathrm{1}}} \HOLBoundVar{v\sb{\mathrm{2}}} \HOLBoundVar{v\sb{\mathrm{3}}}
\end{SaveVerbatim}
\newcommand{\HOLmessageCertificateTheoremsEkcrtXXind}{\UseVerbatim{HOLmessageCertificateTheoremsEkcrtXXind}}
\begin{SaveVerbatim}{HOLmessageCertificateTheoremsEmsgXXdef}
\HOLTokenTurnstile{} \HOLConst{Emsg} \HOLFreeVar{Oi} \HOLFreeVar{Os} \HOLFreeVar{M}
     (\HOLConst{MSG} (\HOLConst{From} \HOLFreeVar{sender}) (\HOLConst{As} \HOLFreeVar{role}) (\HOLConst{To} \HOLFreeVar{recipient}) \HOLFreeVar{enDEK} \HOLFreeVar{enCMD}
        \HOLFreeVar{CMDsig}) \HOLSymConst{=}
   \HOLKeyword{if}
     \HOLConst{checkMSG}
       (\HOLConst{MSG} (\HOLConst{From} \HOLFreeVar{sender}) (\HOLConst{As} \HOLFreeVar{role}) (\HOLConst{To} \HOLFreeVar{recipient}) \HOLFreeVar{enDEK} \HOLFreeVar{enCMD}
          \HOLFreeVar{CMDsig})
   \HOLKeyword{then}
     (\HOLKeyword{let} \HOLBoundVar{order} =
            \HOLConst{getCommand}
              (\HOLConst{MSG} (\HOLConst{From} \HOLFreeVar{sender}) (\HOLConst{As} \HOLFreeVar{role}) (\HOLConst{To} \HOLFreeVar{recipient}) \HOLFreeVar{enDEK}
                 \HOLFreeVar{enCMD} \HOLFreeVar{CMDsig})
      \HOLKeyword{in}
      \HOLKeyword{let} \HOLBoundVar{x} = \HOLConst{ordersParameters} \HOLBoundVar{order}
      \HOLKeyword{in}
        \HOLKeyword{if} \HOLBoundVar{x} \HOLSymConst{=} (\HOLFreeVar{sender}\HOLSymConst{,}\HOLFreeVar{role}\HOLSymConst{,}\HOLFreeVar{recipient}) \HOLKeyword{then}
          \HOLConst{Efn} \HOLFreeVar{Oi} \HOLFreeVar{Os} \HOLFreeVar{M}
            (\HOLConst{Name} (\HOLConst{MKey} (\HOLConst{pubK} (\HOLConst{KStaff} \HOLFreeVar{sender}))) \HOLConst{quoting}
             \HOLConst{Name} (\HOLConst{MRole} \HOLFreeVar{role}) \HOLConst{says} \HOLConst{prop} (\HOLConst{ordersCommand} \HOLBoundVar{order}))
        \HOLKeyword{else} \HOLTokenLeftbrace{}\HOLTokenRightbrace{})
   \HOLKeyword{else} \HOLTokenLeftbrace{}\HOLTokenRightbrace{}
\end{SaveVerbatim}
\newcommand{\HOLmessageCertificateTheoremsEmsgXXdef}{\UseVerbatim{HOLmessageCertificateTheoremsEmsgXXdef}}
\begin{SaveVerbatim}{HOLmessageCertificateTheoremsEmsgXXind}
\HOLTokenTurnstile{} \HOLSymConst{\HOLTokenForall{}}\HOLBoundVar{P}.
     (\HOLSymConst{\HOLTokenForall{}}\HOLBoundVar{Oi} \HOLBoundVar{Os} \HOLBoundVar{M} \HOLBoundVar{sender} \HOLBoundVar{role} \HOLBoundVar{recipient} \HOLBoundVar{enDEK} \HOLBoundVar{enCMD} \HOLBoundVar{CMDsig}.
        \HOLBoundVar{P} \HOLBoundVar{Oi} \HOLBoundVar{Os} \HOLBoundVar{M}
          (\HOLConst{MSG} (\HOLConst{From} \HOLBoundVar{sender}) (\HOLConst{As} \HOLBoundVar{role}) (\HOLConst{To} \HOLBoundVar{recipient}) \HOLBoundVar{enDEK}
             \HOLBoundVar{enCMD} \HOLBoundVar{CMDsig})) \HOLSymConst{\HOLTokenImp{}}
     \HOLSymConst{\HOLTokenForall{}}\HOLBoundVar{v} \HOLBoundVar{v\sb{\mathrm{1}}} \HOLBoundVar{v\sb{\mathrm{2}}} \HOLBoundVar{v\sb{\mathrm{3}}}. \HOLBoundVar{P} \HOLBoundVar{v} \HOLBoundVar{v\sb{\mathrm{1}}} \HOLBoundVar{v\sb{\mathrm{2}}} \HOLBoundVar{v\sb{\mathrm{3}}}
\end{SaveVerbatim}
\newcommand{\HOLmessageCertificateTheoremsEmsgXXind}{\UseVerbatim{HOLmessageCertificateTheoremsEmsgXXind}}
\begin{SaveVerbatim}{HOLmessageCertificateTheoremsErcrtXXdef}
\HOLTokenTurnstile{} \HOLConst{Ercrt} \HOLFreeVar{Oi} \HOLFreeVar{Os} \HOLFreeVar{M}
     (\HOLConst{RCert} (\HOLConst{Auth} \HOLFreeVar{commander}) (\HOLConst{As} \HOLFreeVar{cmdRole}) (\HOLConst{For} \HOLFreeVar{delegate})
        (\HOLConst{As} \HOLFreeVar{delegateRole}) \HOLFreeVar{order} (\HOLConst{RCrtSig} \HOLFreeVar{rcertSig})) \HOLSymConst{=}
   \HOLKeyword{if}
     \HOLConst{checkRCert}
       (\HOLConst{RCert} (\HOLConst{Auth} \HOLFreeVar{commander}) (\HOLConst{As} \HOLFreeVar{cmdRole}) (\HOLConst{For} \HOLFreeVar{delegate})
          (\HOLConst{As} \HOLFreeVar{delegateRole}) \HOLFreeVar{order} (\HOLConst{RCrtSig} \HOLFreeVar{rcertSig}))
   \HOLKeyword{then}
     \HOLConst{Efn} \HOLFreeVar{Oi} \HOLFreeVar{Os} \HOLFreeVar{M}
       (\HOLConst{Name} (\HOLConst{MKey} (\HOLConst{pubK} (\HOLConst{KStaff} \HOLFreeVar{commander}))) \HOLConst{says}
        \HOLConst{reps} (\HOLConst{Name} (\HOLConst{MStaff} \HOLFreeVar{delegate}))
          (\HOLConst{Name} (\HOLConst{MRole} \HOLFreeVar{delegateRole})) (\HOLConst{prop} \HOLFreeVar{order}))
   \HOLKeyword{else} \HOLTokenLeftbrace{}\HOLTokenRightbrace{}
\end{SaveVerbatim}
\newcommand{\HOLmessageCertificateTheoremsErcrtXXdef}{\UseVerbatim{HOLmessageCertificateTheoremsErcrtXXdef}}
\begin{SaveVerbatim}{HOLmessageCertificateTheoremsErcrtXXind}
\HOLTokenTurnstile{} \HOLSymConst{\HOLTokenForall{}}\HOLBoundVar{P}.
     (\HOLSymConst{\HOLTokenForall{}}\HOLBoundVar{Oi} \HOLBoundVar{Os} \HOLBoundVar{M} \HOLBoundVar{commander} \HOLBoundVar{cmdRole} \HOLBoundVar{delegate} \HOLBoundVar{delegateRole} \HOLBoundVar{order}
         \HOLBoundVar{rcertSig}.
        \HOLBoundVar{P} \HOLBoundVar{Oi} \HOLBoundVar{Os} \HOLBoundVar{M}
          (\HOLConst{RCert} (\HOLConst{Auth} \HOLBoundVar{commander}) (\HOLConst{As} \HOLBoundVar{cmdRole}) (\HOLConst{For} \HOLBoundVar{delegate})
             (\HOLConst{As} \HOLBoundVar{delegateRole}) \HOLBoundVar{order} (\HOLConst{RCrtSig} \HOLBoundVar{rcertSig}))) \HOLSymConst{\HOLTokenImp{}}
     \HOLSymConst{\HOLTokenForall{}}\HOLBoundVar{v} \HOLBoundVar{v\sb{\mathrm{1}}} \HOLBoundVar{v\sb{\mathrm{2}}} \HOLBoundVar{v\sb{\mathrm{3}}}. \HOLBoundVar{P} \HOLBoundVar{v} \HOLBoundVar{v\sb{\mathrm{1}}} \HOLBoundVar{v\sb{\mathrm{2}}} \HOLBoundVar{v\sb{\mathrm{3}}}
\end{SaveVerbatim}
\newcommand{\HOLmessageCertificateTheoremsErcrtXXind}{\UseVerbatim{HOLmessageCertificateTheoremsErcrtXXind}}
\begin{SaveVerbatim}{HOLmessageCertificateTheoremsErootkcrtXXdef}
\HOLTokenTurnstile{} (\HOLConst{Erootkcrt} \HOLFreeVar{Oi} \HOLFreeVar{Os} \HOLFreeVar{M}
      (\HOLConst{RootKCert} (\HOLConst{Entity} (\HOLConst{KStaff} \HOLFreeVar{staff})) (\HOLConst{SPubKey} \HOLFreeVar{pubKey})) \HOLSymConst{=}
    \HOLConst{Efn} \HOLFreeVar{Oi} \HOLFreeVar{Os} \HOLFreeVar{M}
      (\HOLConst{Name} (\HOLConst{MKey} \HOLFreeVar{pubKey}) \HOLConst{speaks_for} \HOLConst{Name} (\HOLConst{MStaff} \HOLFreeVar{staff}))) \HOLSymConst{\HOLTokenConj{}}
   (\HOLConst{Erootkcrt} \HOLFreeVar{Oi} \HOLFreeVar{Os} \HOLFreeVar{M}
      (\HOLConst{RootKCert} (\HOLConst{Entity} (\HOLConst{KCA} \HOLFreeVar{ca\sb{\mathrm{2}}})) (\HOLConst{SPubKey} \HOLFreeVar{pubKey})) \HOLSymConst{=}
    \HOLConst{Efn} \HOLFreeVar{Oi} \HOLFreeVar{Os} \HOLFreeVar{M} (\HOLConst{Name} (\HOLConst{MKey} \HOLFreeVar{pubKey}) \HOLConst{speaks_for} \HOLConst{Name} (\HOLConst{MCA} \HOLFreeVar{ca\sb{\mathrm{2}}})))
\end{SaveVerbatim}
\newcommand{\HOLmessageCertificateTheoremsErootkcrtXXdef}{\UseVerbatim{HOLmessageCertificateTheoremsErootkcrtXXdef}}
\begin{SaveVerbatim}{HOLmessageCertificateTheoremsErootkcrtXXind}
\HOLTokenTurnstile{} \HOLSymConst{\HOLTokenForall{}}\HOLBoundVar{P}.
     (\HOLSymConst{\HOLTokenForall{}}\HOLBoundVar{Oi} \HOLBoundVar{Os} \HOLBoundVar{M} \HOLBoundVar{staff} \HOLBoundVar{pubKey}.
        \HOLBoundVar{P} \HOLBoundVar{Oi} \HOLBoundVar{Os} \HOLBoundVar{M}
          (\HOLConst{RootKCert} (\HOLConst{Entity} (\HOLConst{KStaff} \HOLBoundVar{staff}))
             (\HOLConst{SPubKey} \HOLBoundVar{pubKey}))) \HOLSymConst{\HOLTokenConj{}}
     (\HOLSymConst{\HOLTokenForall{}}\HOLBoundVar{Oi} \HOLBoundVar{Os} \HOLBoundVar{M} \HOLBoundVar{ca\sb{\mathrm{2}}} \HOLBoundVar{pubKey}.
        \HOLBoundVar{P} \HOLBoundVar{Oi} \HOLBoundVar{Os} \HOLBoundVar{M}
          (\HOLConst{RootKCert} (\HOLConst{Entity} (\HOLConst{KCA} \HOLBoundVar{ca\sb{\mathrm{2}}})) (\HOLConst{SPubKey} \HOLBoundVar{pubKey}))) \HOLSymConst{\HOLTokenImp{}}
     \HOLSymConst{\HOLTokenForall{}}\HOLBoundVar{v} \HOLBoundVar{v\sb{\mathrm{1}}} \HOLBoundVar{v\sb{\mathrm{2}}} \HOLBoundVar{v\sb{\mathrm{3}}}. \HOLBoundVar{P} \HOLBoundVar{v} \HOLBoundVar{v\sb{\mathrm{1}}} \HOLBoundVar{v\sb{\mathrm{2}}} \HOLBoundVar{v\sb{\mathrm{3}}}
\end{SaveVerbatim}
\newcommand{\HOLmessageCertificateTheoremsErootkcrtXXind}{\UseVerbatim{HOLmessageCertificateTheoremsErootkcrtXXind}}
\begin{SaveVerbatim}{HOLmessageCertificateTheoremsErootrcrtXXdef}
\HOLTokenTurnstile{} \HOLConst{Erootrcrt} \HOLFreeVar{Oi} \HOLFreeVar{Os} \HOLFreeVar{M}
     (\HOLConst{RootRCert} (\HOLConst{For} \HOLFreeVar{delegate}) (\HOLConst{As} \HOLFreeVar{delegateRole}) \HOLFreeVar{order}) \HOLSymConst{=}
   \HOLConst{Efn} \HOLFreeVar{Oi} \HOLFreeVar{Os} \HOLFreeVar{M}
     (\HOLConst{reps} (\HOLConst{Name} (\HOLConst{MStaff} \HOLFreeVar{delegate})) (\HOLConst{Name} (\HOLConst{MRole} \HOLFreeVar{delegateRole}))
        (\HOLConst{prop} \HOLFreeVar{order}))
\end{SaveVerbatim}
\newcommand{\HOLmessageCertificateTheoremsErootrcrtXXdef}{\UseVerbatim{HOLmessageCertificateTheoremsErootrcrtXXdef}}
\begin{SaveVerbatim}{HOLmessageCertificateTheoremsErootrcrtXXind}
\HOLTokenTurnstile{} \HOLSymConst{\HOLTokenForall{}}\HOLBoundVar{P}.
     (\HOLSymConst{\HOLTokenForall{}}\HOLBoundVar{Oi} \HOLBoundVar{Os} \HOLBoundVar{M} \HOLBoundVar{delegate} \HOLBoundVar{delegateRole} \HOLBoundVar{order}.
        \HOLBoundVar{P} \HOLBoundVar{Oi} \HOLBoundVar{Os} \HOLBoundVar{M}
          (\HOLConst{RootRCert} (\HOLConst{For} \HOLBoundVar{delegate}) (\HOLConst{As} \HOLBoundVar{delegateRole}) \HOLBoundVar{order})) \HOLSymConst{\HOLTokenImp{}}
     \HOLSymConst{\HOLTokenForall{}}\HOLBoundVar{v} \HOLBoundVar{v\sb{\mathrm{1}}} \HOLBoundVar{v\sb{\mathrm{2}}} \HOLBoundVar{v\sb{\mathrm{3}}}. \HOLBoundVar{P} \HOLBoundVar{v} \HOLBoundVar{v\sb{\mathrm{1}}} \HOLBoundVar{v\sb{\mathrm{2}}} \HOLBoundVar{v\sb{\mathrm{3}}}
\end{SaveVerbatim}
\newcommand{\HOLmessageCertificateTheoremsErootrcrtXXind}{\UseVerbatim{HOLmessageCertificateTheoremsErootrcrtXXind}}
\begin{SaveVerbatim}{HOLmessageCertificateTheoremsgetCommandXXdef}
\HOLTokenTurnstile{} \HOLConst{getCommand}
     (\HOLConst{MSG} (\HOLConst{From} \HOLFreeVar{sender}) (\HOLConst{As} \HOLFreeVar{role}) (\HOLConst{To} \HOLFreeVar{recipient}) \HOLFreeVar{enDEK} \HOLFreeVar{enCMDMsg}
        \HOLFreeVar{CMDMsgSig}) \HOLSymConst{=}
   (\HOLKeyword{let} \HOLBoundVar{dek} = \HOLConst{getMessage} (\HOLConst{deciphP} (\HOLConst{privK} \HOLFreeVar{recipient}) \HOLFreeVar{enDEK})
    \HOLKeyword{in}
      \HOLConst{getMessage} (\HOLConst{deciphS} \HOLBoundVar{dek} \HOLFreeVar{enCMDMsg}))
\end{SaveVerbatim}
\newcommand{\HOLmessageCertificateTheoremsgetCommandXXdef}{\UseVerbatim{HOLmessageCertificateTheoremsgetCommandXXdef}}
\begin{SaveVerbatim}{HOLmessageCertificateTheoremsgetCommandXXind}
\HOLTokenTurnstile{} \HOLSymConst{\HOLTokenForall{}}\HOLBoundVar{P}.
     (\HOLSymConst{\HOLTokenForall{}}\HOLBoundVar{sender} \HOLBoundVar{role} \HOLBoundVar{recipient} \HOLBoundVar{enDEK} \HOLBoundVar{enCMDMsg} \HOLBoundVar{CMDMsgSig}.
        \HOLBoundVar{P}
          (\HOLConst{MSG} (\HOLConst{From} \HOLBoundVar{sender}) (\HOLConst{As} \HOLBoundVar{role}) (\HOLConst{To} \HOLBoundVar{recipient}) \HOLBoundVar{enDEK}
             \HOLBoundVar{enCMDMsg} \HOLBoundVar{CMDMsgSig})) \HOLSymConst{\HOLTokenImp{}}
     \HOLSymConst{\HOLTokenForall{}}\HOLBoundVar{v}. \HOLBoundVar{P} \HOLBoundVar{v}
\end{SaveVerbatim}
\newcommand{\HOLmessageCertificateTheoremsgetCommandXXind}{\UseVerbatim{HOLmessageCertificateTheoremsgetCommandXXind}}
\begin{SaveVerbatim}{HOLmessageCertificateTheoremsgetCommandXXthm}
\HOLTokenTurnstile{} \HOLConst{getCommand}
     (\HOLConst{MSG} (\HOLConst{From} \HOLFreeVar{sender}) (\HOLConst{As} \HOLFreeVar{role}) (\HOLConst{To} \HOLFreeVar{recipient})
        (\HOLConst{Ea} (\HOLConst{pubK} \HOLFreeVar{recipient}) (\HOLConst{plain} (\HOLConst{sym} \HOLFreeVar{dek})))
        (\HOLConst{Es} (\HOLConst{sym} \HOLFreeVar{dek}) (\HOLConst{plain} \HOLFreeVar{order}))
        (\HOLConst{sign} \HOLFreeVar{sender} (\HOLConst{hash} (\HOLConst{plain} \HOLFreeVar{order})))) \HOLSymConst{=}
   \HOLFreeVar{order}
\end{SaveVerbatim}
\newcommand{\HOLmessageCertificateTheoremsgetCommandXXthm}{\UseVerbatim{HOLmessageCertificateTheoremsgetCommandXXthm}}
\begin{SaveVerbatim}{HOLmessageCertificateTheoremskcertCASatXXthm}
\HOLTokenTurnstile{} \HOLSymConst{\HOLTokenForall{}}\HOLBoundVar{mca} \HOLBoundVar{ca\sb{\mathrm{2}}} \HOLBoundVar{Os} \HOLBoundVar{Oi} \HOLBoundVar{M}.
     (\HOLBoundVar{M}\HOLSymConst{,}\HOLBoundVar{Oi}\HOLSymConst{,}\HOLBoundVar{Os}) \HOLConst{sat}
     \HOLConst{Name} (\HOLConst{MKey} (\HOLConst{pubK} (\HOLConst{KCA} \HOLBoundVar{mca}))) \HOLConst{says}
     \HOLConst{Name} (\HOLConst{MKey} (\HOLConst{pubK} (\HOLConst{KCA} \HOLBoundVar{ca\sb{\mathrm{2}}}))) \HOLConst{speaks_for} \HOLConst{Name} (\HOLConst{MCA} \HOLBoundVar{ca\sb{\mathrm{2}}}) \HOLSymConst{\HOLTokenEquiv{}}
     (\HOLBoundVar{M}\HOLSymConst{,}\HOLBoundVar{Oi}\HOLSymConst{,}\HOLBoundVar{Os}) \HOLConst{ksat}
     \HOLConst{KCert} (\HOLConst{CA} \HOLBoundVar{mca}) (\HOLConst{Entity} (\HOLConst{KCA} \HOLBoundVar{ca\sb{\mathrm{2}}}))
       (\HOLConst{SPubKey} (\HOLConst{pubK} (\HOLConst{KCA} \HOLBoundVar{ca\sb{\mathrm{2}}})))
       (\HOLConst{KCrtSig}
          (\HOLConst{sign} (\HOLConst{KCA} \HOLBoundVar{mca})
             (\HOLConst{hash} (\HOLConst{plain} (\HOLBoundVar{mca}\HOLSymConst{,}\HOLConst{KCA} \HOLBoundVar{ca\sb{\mathrm{2}}}\HOLSymConst{,}\HOLConst{pubK} (\HOLConst{KCA} \HOLBoundVar{ca\sb{\mathrm{2}}}))))))
\end{SaveVerbatim}
\newcommand{\HOLmessageCertificateTheoremskcertCASatXXthm}{\UseVerbatim{HOLmessageCertificateTheoremskcertCASatXXthm}}
\begin{SaveVerbatim}{HOLmessageCertificateTheoremskcertStaffSatXXthm}
\HOLTokenTurnstile{} \HOLSymConst{\HOLTokenForall{}}\HOLBoundVar{staff} \HOLBoundVar{mca} \HOLBoundVar{Os} \HOLBoundVar{Oi} \HOLBoundVar{M}.
     (\HOLBoundVar{M}\HOLSymConst{,}\HOLBoundVar{Oi}\HOLSymConst{,}\HOLBoundVar{Os}) \HOLConst{sat}
     \HOLConst{Name} (\HOLConst{MKey} (\HOLConst{pubK} (\HOLConst{KCA} \HOLBoundVar{mca}))) \HOLConst{says}
     \HOLConst{Name} (\HOLConst{MKey} (\HOLConst{pubK} (\HOLConst{KStaff} \HOLBoundVar{staff}))) \HOLConst{speaks_for}
     \HOLConst{Name} (\HOLConst{MStaff} \HOLBoundVar{staff}) \HOLSymConst{\HOLTokenEquiv{}}
     (\HOLBoundVar{M}\HOLSymConst{,}\HOLBoundVar{Oi}\HOLSymConst{,}\HOLBoundVar{Os}) \HOLConst{ksat}
     \HOLConst{KCert} (\HOLConst{CA} \HOLBoundVar{mca}) (\HOLConst{Entity} (\HOLConst{KStaff} \HOLBoundVar{staff}))
       (\HOLConst{SPubKey} (\HOLConst{pubK} (\HOLConst{KStaff} \HOLBoundVar{staff})))
       (\HOLConst{KCrtSig}
          (\HOLConst{sign} (\HOLConst{KCA} \HOLBoundVar{mca})
             (\HOLConst{hash}
                (\HOLConst{plain}
                   (\HOLBoundVar{mca}\HOLSymConst{,}\HOLConst{KStaff} \HOLBoundVar{staff}\HOLSymConst{,}\HOLConst{pubK} (\HOLConst{KStaff} \HOLBoundVar{staff}))))))
\end{SaveVerbatim}
\newcommand{\HOLmessageCertificateTheoremskcertStaffSatXXthm}{\UseVerbatim{HOLmessageCertificateTheoremskcertStaffSatXXthm}}
\begin{SaveVerbatim}{HOLmessageCertificateTheoremskcrtCAInterpXXthm}
\HOLTokenTurnstile{} \HOLSymConst{\HOLTokenForall{}}\HOLBoundVar{pubKey} \HOLBoundVar{mca} \HOLBoundVar{kcertSig} \HOLBoundVar{ca\sb{\mathrm{2}}} \HOLBoundVar{Os} \HOLBoundVar{Oi} \HOLBoundVar{M}.
     \HOLConst{checkKCert}
       (\HOLConst{KCert} (\HOLConst{CA} \HOLBoundVar{mca}) (\HOLConst{Entity} (\HOLConst{KCA} \HOLBoundVar{ca\sb{\mathrm{2}}})) (\HOLConst{SPubKey} \HOLBoundVar{pubKey})
          (\HOLConst{KCrtSig} \HOLBoundVar{kcertSig})) \HOLSymConst{\HOLTokenImp{}}
     ((\HOLBoundVar{M}\HOLSymConst{,}\HOLBoundVar{Oi}\HOLSymConst{,}\HOLBoundVar{Os}) \HOLConst{ksat}
      \HOLConst{KCert} (\HOLConst{CA} \HOLBoundVar{mca}) (\HOLConst{Entity} (\HOLConst{KCA} \HOLBoundVar{ca\sb{\mathrm{2}}})) (\HOLConst{SPubKey} \HOLBoundVar{pubKey})
        (\HOLConst{KCrtSig} \HOLBoundVar{kcertSig}) \HOLSymConst{\HOLTokenEquiv{}}
      (\HOLBoundVar{M}\HOLSymConst{,}\HOLBoundVar{Oi}\HOLSymConst{,}\HOLBoundVar{Os}) \HOLConst{sat}
      \HOLConst{Name} (\HOLConst{MKey} (\HOLConst{pubK} (\HOLConst{KCA} \HOLBoundVar{mca}))) \HOLConst{says}
      \HOLConst{Name} (\HOLConst{MKey} \HOLBoundVar{pubKey}) \HOLConst{speaks_for} \HOLConst{Name} (\HOLConst{MCA} \HOLBoundVar{ca\sb{\mathrm{2}}}))
\end{SaveVerbatim}
\newcommand{\HOLmessageCertificateTheoremskcrtCAInterpXXthm}{\UseVerbatim{HOLmessageCertificateTheoremskcrtCAInterpXXthm}}
\begin{SaveVerbatim}{HOLmessageCertificateTheoremskcrtStaffInterpXXthm}
\HOLTokenTurnstile{} \HOLSymConst{\HOLTokenForall{}}\HOLBoundVar{staff} \HOLBoundVar{pubKey} \HOLBoundVar{mca} \HOLBoundVar{kcertSig} \HOLBoundVar{Os} \HOLBoundVar{Oi} \HOLBoundVar{M}.
     \HOLConst{checkKCert}
       (\HOLConst{KCert} (\HOLConst{CA} \HOLBoundVar{mca}) (\HOLConst{Entity} (\HOLConst{KStaff} \HOLBoundVar{staff})) (\HOLConst{SPubKey} \HOLBoundVar{pubKey})
          (\HOLConst{KCrtSig} \HOLBoundVar{kcertSig})) \HOLSymConst{\HOLTokenImp{}}
     ((\HOLBoundVar{M}\HOLSymConst{,}\HOLBoundVar{Oi}\HOLSymConst{,}\HOLBoundVar{Os}) \HOLConst{ksat}
      \HOLConst{KCert} (\HOLConst{CA} \HOLBoundVar{mca}) (\HOLConst{Entity} (\HOLConst{KStaff} \HOLBoundVar{staff})) (\HOLConst{SPubKey} \HOLBoundVar{pubKey})
        (\HOLConst{KCrtSig} \HOLBoundVar{kcertSig}) \HOLSymConst{\HOLTokenEquiv{}}
      (\HOLBoundVar{M}\HOLSymConst{,}\HOLBoundVar{Oi}\HOLSymConst{,}\HOLBoundVar{Os}) \HOLConst{sat}
      \HOLConst{Name} (\HOLConst{MKey} (\HOLConst{pubK} (\HOLConst{KCA} \HOLBoundVar{mca}))) \HOLConst{says}
      \HOLConst{Name} (\HOLConst{MKey} \HOLBoundVar{pubKey}) \HOLConst{speaks_for} \HOLConst{Name} (\HOLConst{MStaff} \HOLBoundVar{staff}))
\end{SaveVerbatim}
\newcommand{\HOLmessageCertificateTheoremskcrtStaffInterpXXthm}{\UseVerbatim{HOLmessageCertificateTheoremskcrtStaffInterpXXthm}}
\begin{SaveVerbatim}{HOLmessageCertificateTheoremsmsgInterpXXthm}
\HOLTokenTurnstile{} \HOLSymConst{\HOLTokenForall{}}\HOLBoundVar{sender} \HOLBoundVar{role} \HOLBoundVar{recipient} \HOLBoundVar{dek} \HOLBoundVar{action} \HOLBoundVar{Os} \HOLBoundVar{Oi} \HOLBoundVar{M}.
     \HOLConst{checkMSG}
       (\HOLConst{MSG} (\HOLConst{From} \HOLBoundVar{sender}) (\HOLConst{As} \HOLBoundVar{role}) (\HOLConst{To} \HOLBoundVar{recipient})
          (\HOLConst{Ea} (\HOLConst{pubK} \HOLBoundVar{recipient}) (\HOLConst{plain} (\HOLConst{sym} \HOLBoundVar{dek})))
          (\HOLConst{Es} (\HOLConst{sym} \HOLBoundVar{dek})
             (\HOLConst{plain}
                (\HOLConst{CMD} (\HOLConst{From} \HOLBoundVar{sender}) (\HOLConst{As} \HOLBoundVar{role}) (\HOLConst{To} \HOLBoundVar{recipient})
                   \HOLBoundVar{action})))
          (\HOLConst{sign} \HOLBoundVar{sender}
             (\HOLConst{hash}
                (\HOLConst{plain}
                   (\HOLConst{CMD} (\HOLConst{From} \HOLBoundVar{sender}) (\HOLConst{As} \HOLBoundVar{role}) (\HOLConst{To} \HOLBoundVar{recipient})
                      \HOLBoundVar{action}))))) \HOLSymConst{\HOLTokenImp{}}
     (\HOLBoundVar{M}\HOLSymConst{,}\HOLBoundVar{Oi}\HOLSymConst{,}\HOLBoundVar{Os}) \HOLConst{msat}
     \HOLConst{MSG} (\HOLConst{From} \HOLBoundVar{sender}) (\HOLConst{As} \HOLBoundVar{role}) (\HOLConst{To} \HOLBoundVar{recipient})
       (\HOLConst{Ea} (\HOLConst{pubK} \HOLBoundVar{recipient}) (\HOLConst{plain} (\HOLConst{sym} \HOLBoundVar{dek})))
       (\HOLConst{Es} (\HOLConst{sym} \HOLBoundVar{dek})
          (\HOLConst{plain}
             (\HOLConst{CMD} (\HOLConst{From} \HOLBoundVar{sender}) (\HOLConst{As} \HOLBoundVar{role}) (\HOLConst{To} \HOLBoundVar{recipient})
                \HOLBoundVar{action})))
       (\HOLConst{sign} \HOLBoundVar{sender}
          (\HOLConst{hash}
             (\HOLConst{plain}
                (\HOLConst{CMD} (\HOLConst{From} \HOLBoundVar{sender}) (\HOLConst{As} \HOLBoundVar{role}) (\HOLConst{To} \HOLBoundVar{recipient})
                   \HOLBoundVar{action})))) \HOLSymConst{\HOLTokenEquiv{}}
     (\HOLBoundVar{M}\HOLSymConst{,}\HOLBoundVar{Oi}\HOLSymConst{,}\HOLBoundVar{Os}) \HOLConst{sat}
     \HOLConst{Name} (\HOLConst{MKey} (\HOLConst{pubK} (\HOLConst{KStaff} \HOLBoundVar{sender}))) \HOLConst{quoting}
     \HOLConst{Name} (\HOLConst{MRole} \HOLBoundVar{role}) \HOLConst{says} \HOLConst{prop} \HOLBoundVar{action}
\end{SaveVerbatim}
\newcommand{\HOLmessageCertificateTheoremsmsgInterpXXthm}{\UseVerbatim{HOLmessageCertificateTheoremsmsgInterpXXthm}}
\begin{SaveVerbatim}{HOLmessageCertificateTheoremsmsgParametersXXdef}
\HOLTokenTurnstile{} \HOLConst{msgParameters}
     (\HOLConst{MSG} (\HOLConst{From} \HOLFreeVar{sender}) (\HOLConst{As} \HOLFreeVar{role}) (\HOLConst{To} \HOLFreeVar{recipient}) \HOLFreeVar{enDEK} \HOLFreeVar{enCMDMsg}
        \HOLFreeVar{CMDMsgSig}) \HOLSymConst{=}
   (\HOLFreeVar{sender}\HOLSymConst{,}\HOLFreeVar{role}\HOLSymConst{,}\HOLFreeVar{recipient}\HOLSymConst{,}\HOLFreeVar{enDEK}\HOLSymConst{,}\HOLFreeVar{enCMDMsg}\HOLSymConst{,}\HOLFreeVar{CMDMsgSig})
\end{SaveVerbatim}
\newcommand{\HOLmessageCertificateTheoremsmsgParametersXXdef}{\UseVerbatim{HOLmessageCertificateTheoremsmsgParametersXXdef}}
\begin{SaveVerbatim}{HOLmessageCertificateTheoremsmsgParametersXXind}
\HOLTokenTurnstile{} \HOLSymConst{\HOLTokenForall{}}\HOLBoundVar{P}.
     (\HOLSymConst{\HOLTokenForall{}}\HOLBoundVar{sender} \HOLBoundVar{role} \HOLBoundVar{recipient} \HOLBoundVar{enDEK} \HOLBoundVar{enCMDMsg} \HOLBoundVar{CMDMsgSig}.
        \HOLBoundVar{P}
          (\HOLConst{MSG} (\HOLConst{From} \HOLBoundVar{sender}) (\HOLConst{As} \HOLBoundVar{role}) (\HOLConst{To} \HOLBoundVar{recipient}) \HOLBoundVar{enDEK}
             \HOLBoundVar{enCMDMsg} \HOLBoundVar{CMDMsgSig})) \HOLSymConst{\HOLTokenImp{}}
     \HOLSymConst{\HOLTokenForall{}}\HOLBoundVar{v}. \HOLBoundVar{P} \HOLBoundVar{v}
\end{SaveVerbatim}
\newcommand{\HOLmessageCertificateTheoremsmsgParametersXXind}{\UseVerbatim{HOLmessageCertificateTheoremsmsgParametersXXind}}
\begin{SaveVerbatim}{HOLmessageCertificateTheoremsmsgSatXXthm}
\HOLTokenTurnstile{} \HOLSymConst{\HOLTokenForall{}}\HOLBoundVar{recipient} \HOLBoundVar{dek} \HOLBoundVar{sender} \HOLBoundVar{role} \HOLBoundVar{order} \HOLBoundVar{M} \HOLBoundVar{Oi} \HOLBoundVar{Os}.
     (\HOLBoundVar{M}\HOLSymConst{,}\HOLBoundVar{Oi}\HOLSymConst{,}\HOLBoundVar{Os}) \HOLConst{sat}
     \HOLConst{Name} (\HOLConst{MKey} (\HOLConst{pubK} (\HOLConst{KStaff} \HOLBoundVar{sender}))) \HOLConst{quoting}
     \HOLConst{Name} (\HOLConst{MRole} \HOLBoundVar{role}) \HOLConst{says} \HOLConst{prop} \HOLFreeVar{action} \HOLSymConst{\HOLTokenEquiv{}}
     (\HOLBoundVar{M}\HOLSymConst{,}\HOLBoundVar{Oi}\HOLSymConst{,}\HOLBoundVar{Os}) \HOLConst{msat}
     \HOLConst{MSG} (\HOLConst{From} \HOLBoundVar{sender}) (\HOLConst{As} \HOLBoundVar{role}) (\HOLConst{To} \HOLBoundVar{recipient})
       (\HOLConst{Ea} (\HOLConst{pubK} \HOLBoundVar{recipient}) (\HOLConst{plain} (\HOLConst{sym} \HOLBoundVar{dek})))
       (\HOLConst{Es} (\HOLConst{sym} \HOLBoundVar{dek})
          (\HOLConst{plain}
             (\HOLConst{CMD} (\HOLConst{From} \HOLBoundVar{sender}) (\HOLConst{As} \HOLBoundVar{role}) (\HOLConst{To} \HOLBoundVar{recipient})
                \HOLFreeVar{action})))
       (\HOLConst{sign} \HOLBoundVar{sender}
          (\HOLConst{hash}
             (\HOLConst{plain}
                (\HOLConst{CMD} (\HOLConst{From} \HOLBoundVar{sender}) (\HOLConst{As} \HOLBoundVar{role}) (\HOLConst{To} \HOLBoundVar{recipient})
                   \HOLFreeVar{action}))))
\end{SaveVerbatim}
\newcommand{\HOLmessageCertificateTheoremsmsgSatXXthm}{\UseVerbatim{HOLmessageCertificateTheoremsmsgSatXXthm}}
\begin{SaveVerbatim}{HOLmessageCertificateTheoremsordersCommandXXdef}
\HOLTokenTurnstile{} \HOLConst{ordersCommand}
     (\HOLConst{CMD} (\HOLConst{From} \HOLFreeVar{sender}) (\HOLConst{As} \HOLFreeVar{role}) (\HOLConst{To} \HOLFreeVar{recipient}) \HOLFreeVar{orders}) \HOLSymConst{=}
   \HOLFreeVar{orders}
\end{SaveVerbatim}
\newcommand{\HOLmessageCertificateTheoremsordersCommandXXdef}{\UseVerbatim{HOLmessageCertificateTheoremsordersCommandXXdef}}
\begin{SaveVerbatim}{HOLmessageCertificateTheoremsordersCommandXXind}
\HOLTokenTurnstile{} \HOLSymConst{\HOLTokenForall{}}\HOLBoundVar{P}.
     (\HOLSymConst{\HOLTokenForall{}}\HOLBoundVar{sender} \HOLBoundVar{role} \HOLBoundVar{recipient} \HOLBoundVar{orders}.
        \HOLBoundVar{P} (\HOLConst{CMD} (\HOLConst{From} \HOLBoundVar{sender}) (\HOLConst{As} \HOLBoundVar{role}) (\HOLConst{To} \HOLBoundVar{recipient}) \HOLBoundVar{orders})) \HOLSymConst{\HOLTokenImp{}}
     \HOLSymConst{\HOLTokenForall{}}\HOLBoundVar{v}. \HOLBoundVar{P} \HOLBoundVar{v}
\end{SaveVerbatim}
\newcommand{\HOLmessageCertificateTheoremsordersCommandXXind}{\UseVerbatim{HOLmessageCertificateTheoremsordersCommandXXind}}
\begin{SaveVerbatim}{HOLmessageCertificateTheoremsordersParametersXXdef}
\HOLTokenTurnstile{} \HOLConst{ordersParameters}
     (\HOLConst{CMD} (\HOLConst{From} \HOLFreeVar{sender}) (\HOLConst{As} \HOLFreeVar{role}) (\HOLConst{To} \HOLFreeVar{recipient}) \HOLFreeVar{orders}) \HOLSymConst{=}
   (\HOLFreeVar{sender}\HOLSymConst{,}\HOLFreeVar{role}\HOLSymConst{,}\HOLFreeVar{recipient})
\end{SaveVerbatim}
\newcommand{\HOLmessageCertificateTheoremsordersParametersXXdef}{\UseVerbatim{HOLmessageCertificateTheoremsordersParametersXXdef}}
\begin{SaveVerbatim}{HOLmessageCertificateTheoremsordersParametersXXind}
\HOLTokenTurnstile{} \HOLSymConst{\HOLTokenForall{}}\HOLBoundVar{P}.
     (\HOLSymConst{\HOLTokenForall{}}\HOLBoundVar{sender} \HOLBoundVar{role} \HOLBoundVar{recipient} \HOLBoundVar{orders}.
        \HOLBoundVar{P} (\HOLConst{CMD} (\HOLConst{From} \HOLBoundVar{sender}) (\HOLConst{As} \HOLBoundVar{role}) (\HOLConst{To} \HOLBoundVar{recipient}) \HOLBoundVar{orders})) \HOLSymConst{\HOLTokenImp{}}
     \HOLSymConst{\HOLTokenForall{}}\HOLBoundVar{v}. \HOLBoundVar{P} \HOLBoundVar{v}
\end{SaveVerbatim}
\newcommand{\HOLmessageCertificateTheoremsordersParametersXXind}{\UseVerbatim{HOLmessageCertificateTheoremsordersParametersXXind}}
\begin{SaveVerbatim}{HOLmessageCertificateTheoremsrcertStaffSatXXthm}
\HOLTokenTurnstile{} \HOLSymConst{\HOLTokenForall{}}\HOLBoundVar{order} \HOLBoundVar{delegateRole} \HOLBoundVar{delegate} \HOLBoundVar{commander} \HOLBoundVar{cmdRole} \HOLBoundVar{Os} \HOLBoundVar{Oi} \HOLBoundVar{M}.
     (\HOLBoundVar{M}\HOLSymConst{,}\HOLBoundVar{Oi}\HOLSymConst{,}\HOLBoundVar{Os}) \HOLConst{sat}
     \HOLConst{Name} (\HOLConst{MKey} (\HOLConst{pubK} (\HOLConst{KStaff} \HOLBoundVar{commander}))) \HOLConst{says}
     \HOLConst{reps} (\HOLConst{Name} (\HOLConst{MStaff} \HOLBoundVar{delegate})) (\HOLConst{Name} (\HOLConst{MRole} \HOLBoundVar{delegateRole}))
       (\HOLConst{prop} \HOLBoundVar{order}) \HOLSymConst{\HOLTokenEquiv{}}
     (\HOLBoundVar{M}\HOLSymConst{,}\HOLBoundVar{Oi}\HOLSymConst{,}\HOLBoundVar{Os}) \HOLConst{rsat}
     \HOLConst{RCert} (\HOLConst{Auth} \HOLBoundVar{commander}) (\HOLConst{As} \HOLBoundVar{cmdRole}) (\HOLConst{For} \HOLBoundVar{delegate})
       (\HOLConst{As} \HOLBoundVar{delegateRole}) \HOLBoundVar{order}
       (\HOLConst{RCrtSig}
          (\HOLConst{sign} (\HOLConst{KStaff} \HOLBoundVar{commander})
             (\HOLConst{hash}
                (\HOLConst{plain}
                   (\HOLBoundVar{commander}\HOLSymConst{,}\HOLBoundVar{cmdRole}\HOLSymConst{,}\HOLBoundVar{delegate}\HOLSymConst{,}\HOLBoundVar{delegateRole}\HOLSymConst{,}
                    \HOLBoundVar{order})))))
\end{SaveVerbatim}
\newcommand{\HOLmessageCertificateTheoremsrcertStaffSatXXthm}{\UseVerbatim{HOLmessageCertificateTheoremsrcertStaffSatXXthm}}
\begin{SaveVerbatim}{HOLmessageCertificateTheoremsrcrtStaffInterpXXthm}
\HOLTokenTurnstile{} \HOLSymConst{\HOLTokenForall{}}\HOLBoundVar{rcertSig} \HOLBoundVar{order} \HOLBoundVar{delegateRole} \HOLBoundVar{delegate} \HOLBoundVar{commander} \HOLBoundVar{cmdRole} \HOLBoundVar{Os} \HOLBoundVar{Oi}
      \HOLBoundVar{M}.
     \HOLConst{checkRCert}
       (\HOLConst{RCert} (\HOLConst{Auth} \HOLBoundVar{commander}) (\HOLConst{As} \HOLBoundVar{cmdRole}) (\HOLConst{For} \HOLBoundVar{delegate})
          (\HOLConst{As} \HOLBoundVar{delegateRole}) \HOLBoundVar{order} (\HOLConst{RCrtSig} \HOLBoundVar{rcertSig})) \HOLSymConst{\HOLTokenImp{}}
     ((\HOLBoundVar{M}\HOLSymConst{,}\HOLBoundVar{Oi}\HOLSymConst{,}\HOLBoundVar{Os}) \HOLConst{rsat}
      \HOLConst{RCert} (\HOLConst{Auth} \HOLBoundVar{commander}) (\HOLConst{As} \HOLBoundVar{cmdRole}) (\HOLConst{For} \HOLBoundVar{delegate})
        (\HOLConst{As} \HOLBoundVar{delegateRole}) \HOLBoundVar{order} (\HOLConst{RCrtSig} \HOLBoundVar{rcertSig}) \HOLSymConst{\HOLTokenEquiv{}}
      (\HOLBoundVar{M}\HOLSymConst{,}\HOLBoundVar{Oi}\HOLSymConst{,}\HOLBoundVar{Os}) \HOLConst{sat}
      \HOLConst{Name} (\HOLConst{MKey} (\HOLConst{pubK} (\HOLConst{KStaff} \HOLBoundVar{commander}))) \HOLConst{says}
      \HOLConst{reps} (\HOLConst{Name} (\HOLConst{MStaff} \HOLBoundVar{delegate})) (\HOLConst{Name} (\HOLConst{MRole} \HOLBoundVar{delegateRole}))
        (\HOLConst{prop} \HOLBoundVar{order}))
\end{SaveVerbatim}
\newcommand{\HOLmessageCertificateTheoremsrcrtStaffInterpXXthm}{\UseVerbatim{HOLmessageCertificateTheoremsrcrtStaffInterpXXthm}}
\begin{SaveVerbatim}{HOLmessageCertificateTheoremsrootkcertCAInterpXXthm}
\HOLTokenTurnstile{} \HOLSymConst{\HOLTokenForall{}}\HOLBoundVar{pubKey} \HOLBoundVar{ca} \HOLBoundVar{Os} \HOLBoundVar{Oi} \HOLBoundVar{M}.
     (\HOLBoundVar{M}\HOLSymConst{,}\HOLBoundVar{Oi}\HOLSymConst{,}\HOLBoundVar{Os}) \HOLConst{rootksat}
     \HOLConst{RootKCert} (\HOLConst{Entity} (\HOLConst{KCA} \HOLBoundVar{ca})) (\HOLConst{SPubKey} \HOLBoundVar{pubKey}) \HOLSymConst{\HOLTokenEquiv{}}
     (\HOLBoundVar{M}\HOLSymConst{,}\HOLBoundVar{Oi}\HOLSymConst{,}\HOLBoundVar{Os}) \HOLConst{sat} \HOLConst{Name} (\HOLConst{MKey} \HOLBoundVar{pubKey}) \HOLConst{speaks_for} \HOLConst{Name} (\HOLConst{MCA} \HOLBoundVar{ca})
\end{SaveVerbatim}
\newcommand{\HOLmessageCertificateTheoremsrootkcertCAInterpXXthm}{\UseVerbatim{HOLmessageCertificateTheoremsrootkcertCAInterpXXthm}}
\begin{SaveVerbatim}{HOLmessageCertificateTheoremsrootkcertStaffInterpXXthm}
\HOLTokenTurnstile{} \HOLSymConst{\HOLTokenForall{}}\HOLBoundVar{staff} \HOLBoundVar{pubKey} \HOLBoundVar{Os} \HOLBoundVar{Oi} \HOLBoundVar{M}.
     (\HOLBoundVar{M}\HOLSymConst{,}\HOLBoundVar{Oi}\HOLSymConst{,}\HOLBoundVar{Os}) \HOLConst{rootksat}
     \HOLConst{RootKCert} (\HOLConst{Entity} (\HOLConst{KStaff} \HOLBoundVar{staff})) (\HOLConst{SPubKey} \HOLBoundVar{pubKey}) \HOLSymConst{\HOLTokenEquiv{}}
     (\HOLBoundVar{M}\HOLSymConst{,}\HOLBoundVar{Oi}\HOLSymConst{,}\HOLBoundVar{Os}) \HOLConst{sat}
     \HOLConst{Name} (\HOLConst{MKey} \HOLBoundVar{pubKey}) \HOLConst{speaks_for} \HOLConst{Name} (\HOLConst{MStaff} \HOLBoundVar{staff})
\end{SaveVerbatim}
\newcommand{\HOLmessageCertificateTheoremsrootkcertStaffInterpXXthm}{\UseVerbatim{HOLmessageCertificateTheoremsrootkcertStaffInterpXXthm}}
\begin{SaveVerbatim}{HOLmessageCertificateTheoremsrootrcertStaffInterpXXthm}
\HOLTokenTurnstile{} \HOLSymConst{\HOLTokenForall{}}\HOLBoundVar{order} \HOLBoundVar{delegateRole} \HOLBoundVar{delegate} \HOLBoundVar{Os} \HOLBoundVar{Oi} \HOLBoundVar{M}.
     (\HOLBoundVar{M}\HOLSymConst{,}\HOLBoundVar{Oi}\HOLSymConst{,}\HOLBoundVar{Os}) \HOLConst{rootrsat}
     \HOLConst{RootRCert} (\HOLConst{For} \HOLBoundVar{delegate}) (\HOLConst{As} \HOLBoundVar{delegateRole}) \HOLBoundVar{order} \HOLSymConst{\HOLTokenEquiv{}}
     (\HOLBoundVar{M}\HOLSymConst{,}\HOLBoundVar{Oi}\HOLSymConst{,}\HOLBoundVar{Os}) \HOLConst{sat}
     \HOLConst{reps} (\HOLConst{Name} (\HOLConst{MStaff} \HOLBoundVar{delegate})) (\HOLConst{Name} (\HOLConst{MRole} \HOLBoundVar{delegateRole}))
       (\HOLConst{prop} \HOLBoundVar{order})
\end{SaveVerbatim}
\newcommand{\HOLmessageCertificateTheoremsrootrcertStaffInterpXXthm}{\UseVerbatim{HOLmessageCertificateTheoremsrootrcertStaffInterpXXthm}}
\newcommand{\HOLmessageCertificateTheorems}{
\HOLThmTag{messageCertificate}{checkKCert_def}\HOLmessageCertificateTheoremscheckKCertXXdef
\HOLThmTag{messageCertificate}{checkKCert_ind}\HOLmessageCertificateTheoremscheckKCertXXind
\HOLThmTag{messageCertificate}{checkKCertOK}\HOLmessageCertificateTheoremscheckKCertOK
\HOLThmTag{messageCertificate}{checkMSG_def}\HOLmessageCertificateTheoremscheckMSGXXdef
\HOLThmTag{messageCertificate}{checkMSG_ind}\HOLmessageCertificateTheoremscheckMSGXXind
\HOLThmTag{messageCertificate}{checkMSGOK}\HOLmessageCertificateTheoremscheckMSGOK
\HOLThmTag{messageCertificate}{checkRCert_def}\HOLmessageCertificateTheoremscheckRCertXXdef
\HOLThmTag{messageCertificate}{checkRCert_ind}\HOLmessageCertificateTheoremscheckRCertXXind
\HOLThmTag{messageCertificate}{checkRCertOK}\HOLmessageCertificateTheoremscheckRCertOK
\HOLThmTag{messageCertificate}{Ekcrt_def}\HOLmessageCertificateTheoremsEkcrtXXdef
\HOLThmTag{messageCertificate}{Ekcrt_ind}\HOLmessageCertificateTheoremsEkcrtXXind
\HOLThmTag{messageCertificate}{Emsg_def}\HOLmessageCertificateTheoremsEmsgXXdef
\HOLThmTag{messageCertificate}{Emsg_ind}\HOLmessageCertificateTheoremsEmsgXXind
\HOLThmTag{messageCertificate}{Ercrt_def}\HOLmessageCertificateTheoremsErcrtXXdef
\HOLThmTag{messageCertificate}{Ercrt_ind}\HOLmessageCertificateTheoremsErcrtXXind
\HOLThmTag{messageCertificate}{Erootkcrt_def}\HOLmessageCertificateTheoremsErootkcrtXXdef
\HOLThmTag{messageCertificate}{Erootkcrt_ind}\HOLmessageCertificateTheoremsErootkcrtXXind
\HOLThmTag{messageCertificate}{Erootrcrt_def}\HOLmessageCertificateTheoremsErootrcrtXXdef
\HOLThmTag{messageCertificate}{Erootrcrt_ind}\HOLmessageCertificateTheoremsErootrcrtXXind
\HOLThmTag{messageCertificate}{getCommand_def}\HOLmessageCertificateTheoremsgetCommandXXdef
\HOLThmTag{messageCertificate}{getCommand_ind}\HOLmessageCertificateTheoremsgetCommandXXind
\HOLThmTag{messageCertificate}{getCommand_thm}\HOLmessageCertificateTheoremsgetCommandXXthm
\HOLThmTag{messageCertificate}{kcertCASat_thm}\HOLmessageCertificateTheoremskcertCASatXXthm
\HOLThmTag{messageCertificate}{kcertStaffSat_thm}\HOLmessageCertificateTheoremskcertStaffSatXXthm
\HOLThmTag{messageCertificate}{kcrtCAInterp_thm}\HOLmessageCertificateTheoremskcrtCAInterpXXthm
\HOLThmTag{messageCertificate}{kcrtStaffInterp_thm}\HOLmessageCertificateTheoremskcrtStaffInterpXXthm
\HOLThmTag{messageCertificate}{msgInterp_thm}\HOLmessageCertificateTheoremsmsgInterpXXthm
\HOLThmTag{messageCertificate}{msgParameters_def}\HOLmessageCertificateTheoremsmsgParametersXXdef
\HOLThmTag{messageCertificate}{msgParameters_ind}\HOLmessageCertificateTheoremsmsgParametersXXind
\HOLThmTag{messageCertificate}{msgSat_thm}\HOLmessageCertificateTheoremsmsgSatXXthm
\HOLThmTag{messageCertificate}{ordersCommand_def}\HOLmessageCertificateTheoremsordersCommandXXdef
\HOLThmTag{messageCertificate}{ordersCommand_ind}\HOLmessageCertificateTheoremsordersCommandXXind
\HOLThmTag{messageCertificate}{ordersParameters_def}\HOLmessageCertificateTheoremsordersParametersXXdef
\HOLThmTag{messageCertificate}{ordersParameters_ind}\HOLmessageCertificateTheoremsordersParametersXXind
\HOLThmTag{messageCertificate}{rcertStaffSat_thm}\HOLmessageCertificateTheoremsrcertStaffSatXXthm
\HOLThmTag{messageCertificate}{rcrtStaffInterp_thm}\HOLmessageCertificateTheoremsrcrtStaffInterpXXthm
\HOLThmTag{messageCertificate}{rootkcertCAInterp_thm}\HOLmessageCertificateTheoremsrootkcertCAInterpXXthm
\HOLThmTag{messageCertificate}{rootkcertStaffInterp_thm}\HOLmessageCertificateTheoremsrootkcertStaffInterpXXthm
\HOLThmTag{messageCertificate}{rootrcertStaffInterp_thm}\HOLmessageCertificateTheoremsrootrcertStaffInterpXXthm
}

\newcommand{\HOLBFOConopsDate}{20 August 2016}
\newcommand{\HOLBFOConopsTime}{12:37}
\begin{SaveVerbatim}{HOLBFOConopsTheoremsBFOXXthm}
\HOLTokenTurnstile{} (\HOLFreeVar{M}\HOLSymConst{,}\HOLFreeVar{Oi}\HOLSymConst{,}\HOLFreeVar{Os}) \HOLConst{sat}
   \HOLConst{Name} (\HOLConst{MKey} (\HOLConst{pubK} (\HOLConst{KStaff} \HOLConst{Alice}))) \HOLConst{quoting}
   \HOLConst{Name} (\HOLConst{MRole} \HOLConst{BFC}) \HOLConst{says} \HOLConst{prop} (\HOLConst{MC} \HOLConst{go}) \HOLSymConst{\HOLTokenImp{}}
   (\HOLFreeVar{M}\HOLSymConst{,}\HOLFreeVar{Oi}\HOLSymConst{,}\HOLFreeVar{Os}) \HOLConst{sat}
   \HOLConst{Name} (\HOLConst{MKey} (\HOLConst{pubK} (\HOLConst{KCA} \HOLConst{BFCA}))) \HOLConst{says}
   \HOLConst{Name} (\HOLConst{MKey} (\HOLConst{pubK} (\HOLConst{KStaff} \HOLConst{Alice}))) \HOLConst{speaks_for}
   \HOLConst{Name} (\HOLConst{MStaff} \HOLConst{Alice}) \HOLSymConst{\HOLTokenImp{}}
   (\HOLFreeVar{M}\HOLSymConst{,}\HOLFreeVar{Oi}\HOLSymConst{,}\HOLFreeVar{Os}) \HOLConst{sat}
   \HOLConst{Name} (\HOLConst{MKey} (\HOLConst{pubK} (\HOLConst{KCA} \HOLConst{JFCA}))) \HOLConst{says}
   \HOLConst{Name} (\HOLConst{MKey} (\HOLConst{pubK} (\HOLConst{KCA} \HOLConst{BFCA}))) \HOLConst{speaks_for} \HOLConst{Name} (\HOLConst{MCA} \HOLConst{BFCA}) \HOLSymConst{\HOLTokenImp{}}
   (\HOLFreeVar{M}\HOLSymConst{,}\HOLFreeVar{Oi}\HOLSymConst{,}\HOLFreeVar{Os}) \HOLConst{sat}
   \HOLConst{reps} (\HOLConst{Name} (\HOLConst{MStaff} \HOLConst{Alice})) (\HOLConst{Name} (\HOLConst{MRole} \HOLConst{BFC}))
     (\HOLConst{prop} (\HOLConst{MC} \HOLConst{go})) \HOLSymConst{\HOLTokenImp{}}
   (\HOLFreeVar{M}\HOLSymConst{,}\HOLFreeVar{Oi}\HOLSymConst{,}\HOLFreeVar{Os}) \HOLConst{sat}
   \HOLConst{Name} (\HOLConst{MKey} (\HOLConst{pubK} (\HOLConst{KCA} \HOLConst{JFCA}))) \HOLConst{speaks_for} \HOLConst{Name} (\HOLConst{MCA} \HOLConst{JFCA}) \HOLSymConst{\HOLTokenImp{}}
   (\HOLFreeVar{M}\HOLSymConst{,}\HOLFreeVar{Oi}\HOLSymConst{,}\HOLFreeVar{Os}) \HOLConst{sat} \HOLConst{Name} (\HOLConst{MRole} \HOLConst{BFC}) \HOLConst{controls} \HOLConst{prop} (\HOLConst{MC} \HOLConst{go}) \HOLSymConst{\HOLTokenImp{}}
   (\HOLFreeVar{M}\HOLSymConst{,}\HOLFreeVar{Oi}\HOLSymConst{,}\HOLFreeVar{Os}) \HOLConst{sat}
   \HOLConst{Name} (\HOLConst{MCA} \HOLConst{JFCA}) \HOLConst{controls}
   \HOLConst{Name} (\HOLConst{MKey} (\HOLConst{pubK} (\HOLConst{KCA} \HOLConst{BFCA}))) \HOLConst{speaks_for} \HOLConst{Name} (\HOLConst{MCA} \HOLConst{BFCA}) \HOLSymConst{\HOLTokenImp{}}
   (\HOLFreeVar{M}\HOLSymConst{,}\HOLFreeVar{Oi}\HOLSymConst{,}\HOLFreeVar{Os}) \HOLConst{sat}
   \HOLConst{Name} (\HOLConst{MCA} \HOLConst{BFCA}) \HOLConst{controls}
   \HOLConst{Name} (\HOLConst{MKey} (\HOLConst{pubK} (\HOLConst{KStaff} \HOLConst{Alice}))) \HOLConst{speaks_for}
   \HOLConst{Name} (\HOLConst{MStaff} \HOLConst{Alice}) \HOLSymConst{\HOLTokenImp{}}
   (\HOLFreeVar{M}\HOLSymConst{,}\HOLFreeVar{Oi}\HOLSymConst{,}\HOLFreeVar{Os}) \HOLConst{sat} \HOLConst{prop} (\HOLConst{MC} \HOLConst{go}) \HOLConst{impf} \HOLConst{prop} (\HOLConst{WC} \HOLConst{launch}) \HOLSymConst{\HOLTokenImp{}}
   (\HOLFreeVar{M}\HOLSymConst{,}\HOLFreeVar{Oi}\HOLSymConst{,}\HOLFreeVar{Os}) \HOLConst{sat}
   \HOLConst{Name} (\HOLConst{MKey} (\HOLConst{pubK} (\HOLConst{KStaff} \HOLConst{Carol}))) \HOLConst{quoting}
   \HOLConst{Name} (\HOLConst{MRole} \HOLConst{BFO}) \HOLConst{says} \HOLConst{prop} (\HOLConst{WC} \HOLConst{launch})
\end{SaveVerbatim}
\newcommand{\HOLBFOConopsTheoremsBFOXXthm}{\UseVerbatim{HOLBFOConopsTheoremsBFOXXthm}}
\begin{SaveVerbatim}{HOLBFOConopsTheoremsblackBoxBFOXXthm}
\HOLTokenTurnstile{} (\HOLFreeVar{M}\HOLSymConst{,}\HOLFreeVar{Oi}\HOLSymConst{,}\HOLFreeVar{Os}) \HOLConst{msat}
   \HOLConst{MSG} (\HOLConst{From} \HOLConst{Alice}) (\HOLConst{As} \HOLConst{BFC}) (\HOLConst{To} \HOLConst{Carol})
     (\HOLConst{Ea} (\HOLConst{pubK} \HOLConst{Carol}) (\HOLConst{plain} (\HOLConst{sym} \HOLFreeVar{dek\sb{\mathrm{1}}})))
     (\HOLConst{Es} (\HOLConst{sym} \HOLFreeVar{dek\sb{\mathrm{1}}})
        (\HOLConst{plain} (\HOLConst{CMD} (\HOLConst{From} \HOLConst{Alice}) (\HOLConst{As} \HOLConst{BFC}) (\HOLConst{To} \HOLConst{Carol}) (\HOLConst{MC} \HOLConst{go}))))
     (\HOLConst{sign} \HOLConst{Alice}
        (\HOLConst{hash}
           (\HOLConst{plain}
              (\HOLConst{CMD} (\HOLConst{From} \HOLConst{Alice}) (\HOLConst{As} \HOLConst{BFC}) (\HOLConst{To} \HOLConst{Carol})
                 (\HOLConst{MC} \HOLConst{go}))))) \HOLSymConst{\HOLTokenImp{}}
   (\HOLFreeVar{M}\HOLSymConst{,}\HOLFreeVar{Oi}\HOLSymConst{,}\HOLFreeVar{Os}) \HOLConst{ksat}
   \HOLConst{KCert} (\HOLConst{CA} \HOLConst{BFCA}) (\HOLConst{Entity} (\HOLConst{KStaff} \HOLConst{Alice}))
     (\HOLConst{SPubKey} (\HOLConst{pubK} (\HOLConst{KStaff} \HOLConst{Alice})))
     (\HOLConst{KCrtSig}
        (\HOLConst{sign} (\HOLConst{KCA} \HOLConst{BFCA})
           (\HOLConst{hash}
              (\HOLConst{plain}
                 (\HOLConst{BFCA}\HOLSymConst{,}\HOLConst{KStaff} \HOLConst{Alice}\HOLSymConst{,}\HOLConst{pubK} (\HOLConst{KStaff} \HOLConst{Alice})))))) \HOLSymConst{\HOLTokenImp{}}
   (\HOLFreeVar{M}\HOLSymConst{,}\HOLFreeVar{Oi}\HOLSymConst{,}\HOLFreeVar{Os}) \HOLConst{ksat}
   \HOLConst{KCert} (\HOLConst{CA} \HOLConst{JFCA}) (\HOLConst{Entity} (\HOLConst{KCA} \HOLConst{BFCA}))
     (\HOLConst{SPubKey} (\HOLConst{pubK} (\HOLConst{KCA} \HOLConst{BFCA})))
     (\HOLConst{KCrtSig}
        (\HOLConst{sign} (\HOLConst{KCA} \HOLConst{JFCA})
           (\HOLConst{hash} (\HOLConst{plain} (\HOLConst{JFCA}\HOLSymConst{,}\HOLConst{KCA} \HOLConst{BFCA}\HOLSymConst{,}\HOLConst{pubK} (\HOLConst{KCA} \HOLConst{BFCA})))))) \HOLSymConst{\HOLTokenImp{}}
   (\HOLFreeVar{M}\HOLSymConst{,}\HOLFreeVar{Oi}\HOLSymConst{,}\HOLFreeVar{Os}) \HOLConst{rootrsat} \HOLConst{RootRCert} (\HOLConst{For} \HOLConst{Alice}) (\HOLConst{As} \HOLConst{BFC}) (\HOLConst{MC} \HOLConst{go}) \HOLSymConst{\HOLTokenImp{}}
   (\HOLFreeVar{M}\HOLSymConst{,}\HOLFreeVar{Oi}\HOLSymConst{,}\HOLFreeVar{Os}) \HOLConst{rootksat}
   \HOLConst{RootKCert} (\HOLConst{Entity} (\HOLConst{KCA} \HOLConst{JFCA})) (\HOLConst{SPubKey} (\HOLConst{pubK} (\HOLConst{KCA} \HOLConst{JFCA}))) \HOLSymConst{\HOLTokenImp{}}
   (\HOLFreeVar{M}\HOLSymConst{,}\HOLFreeVar{Oi}\HOLSymConst{,}\HOLFreeVar{Os}) \HOLConst{sat} \HOLConst{Name} (\HOLConst{MRole} \HOLConst{BFC}) \HOLConst{controls} \HOLConst{prop} (\HOLConst{MC} \HOLConst{go}) \HOLSymConst{\HOLTokenImp{}}
   (\HOLFreeVar{M}\HOLSymConst{,}\HOLFreeVar{Oi}\HOLSymConst{,}\HOLFreeVar{Os}) \HOLConst{sat}
   \HOLConst{Name} (\HOLConst{MCA} \HOLConst{JFCA}) \HOLConst{controls}
   \HOLConst{Name} (\HOLConst{MKey} (\HOLConst{pubK} (\HOLConst{KCA} \HOLConst{BFCA}))) \HOLConst{speaks_for} \HOLConst{Name} (\HOLConst{MCA} \HOLConst{BFCA}) \HOLSymConst{\HOLTokenImp{}}
   (\HOLFreeVar{M}\HOLSymConst{,}\HOLFreeVar{Oi}\HOLSymConst{,}\HOLFreeVar{Os}) \HOLConst{sat}
   \HOLConst{Name} (\HOLConst{MCA} \HOLConst{BFCA}) \HOLConst{controls}
   \HOLConst{Name} (\HOLConst{MKey} (\HOLConst{pubK} (\HOLConst{KStaff} \HOLConst{Alice}))) \HOLConst{speaks_for}
   \HOLConst{Name} (\HOLConst{MStaff} \HOLConst{Alice}) \HOLSymConst{\HOLTokenImp{}}
   (\HOLFreeVar{M}\HOLSymConst{,}\HOLFreeVar{Oi}\HOLSymConst{,}\HOLFreeVar{Os}) \HOLConst{sat} \HOLConst{prop} (\HOLConst{MC} \HOLConst{go}) \HOLConst{impf} \HOLConst{prop} (\HOLConst{WC} \HOLConst{launch}) \HOLSymConst{\HOLTokenImp{}}
   (\HOLFreeVar{M}\HOLSymConst{,}\HOLFreeVar{Oi}\HOLSymConst{,}\HOLFreeVar{Os}) \HOLConst{msat}
   \HOLConst{MSG} (\HOLConst{From} \HOLConst{Carol}) (\HOLConst{As} \HOLConst{BFO}) (\HOLConst{To} \HOLConst{Weapon})
     (\HOLConst{Ea} (\HOLConst{pubK} \HOLConst{Weapon}) (\HOLConst{plain} (\HOLConst{sym} \HOLFreeVar{dek\sb{\mathrm{2}}})))
     (\HOLConst{Es} (\HOLConst{sym} \HOLFreeVar{dek\sb{\mathrm{2}}})
        (\HOLConst{plain}
           (\HOLConst{CMD} (\HOLConst{From} \HOLConst{Carol}) (\HOLConst{As} \HOLConst{BFO}) (\HOLConst{To} \HOLConst{Weapon}) (\HOLConst{WC} \HOLConst{launch}))))
     (\HOLConst{sign} \HOLConst{Carol}
        (\HOLConst{hash}
           (\HOLConst{plain}
              (\HOLConst{CMD} (\HOLConst{From} \HOLConst{Carol}) (\HOLConst{As} \HOLConst{BFO}) (\HOLConst{To} \HOLConst{Weapon})
                 (\HOLConst{WC} \HOLConst{launch})))))
\end{SaveVerbatim}
\newcommand{\HOLBFOConopsTheoremsblackBoxBFOXXthm}{\UseVerbatim{HOLBFOConopsTheoremsblackBoxBFOXXthm}}
\newcommand{\HOLBFOConopsTheorems}{
\HOLThmTag{BFOConops}{BFO_thm}\HOLBFOConopsTheoremsBFOXXthm
\HOLThmTag{BFOConops}{blackBoxBFO_thm}\HOLBFOConopsTheoremsblackBoxBFOXXthm
}


\maketitle
\thispagestyle{empty}
\author{}
\maketitle

\begin{abstract}
  We provide guidance for developing the HOL theories. We show a
  complete theory listing once cipherTheory is correctly
  completed. All the starting source files are listed.
\end{abstract}

\section{Assignment}
\label{sec:assignment}

To simplify things a bit, \redtext{I have eliminated the task of
  proving deciphP\_clauses}. The reasons primarily to keep focused on
\emph{integrity and integrity checking}.  On the other hand, I have
added two theorems you need to prove to make it easier for you to
related signed key certificates to their interpretation:
\redtext{kcertCASat\_thm} and \redtext{kcertStaffSat\_thm}. The list
of datatypes, definitions, and theorems you must prove is below.

\paragraph{cipher Theory}

\begin{enumerate}
\item Definitions
  \begin{enumerate}[{a.}]
  \item sign\_def
  \item signVerify\_def
  \end{enumerate}
\item Theorems: signVerifyOK
\end{enumerate}

\paragraph{messageCertificate Theory}

\begin{enumerate}
\item Datatypes
  \begin{enumerate}[{a.}]
  \item KCertSignature
  \item KeyCertificate
  \item RootKeyCertificate
  \end{enumerate}

\item Definitions
  \begin{enumerate}[{a.}]
  \item checkKCert\_def
  \item Ekcrt\_def
  \item Erootkcrt\_def
  \item ksat\_def
  \item rootksat\_def
  \end{enumerate}

\item Theorems
  \begin{enumerate}[{a.}]
  \item checkKCertOK
  \item kcrtCAInterp\_thm
  \item kcrtStaffInterp\_thm
  \item rootkcertCAInterp\_thm
  \item rootkcertStaffInterp\_thm
  \item kcertCASat\_thm
  \item kcertStaffSat\_thm
  \end{enumerate}

\end{enumerate}

\paragraph{BFOConops Theory}

\begin{enumerate}
\item Theorems: blackBoxBFO\_thm

\end{enumerate}

\section{Development Guidance}
\label{sec:guidance}

\subsection{Using Holmake}
\label{sec:holmake}

I \textbf{strongly suggest} getting cipher Theory fully defined and
verified as quickly as possible. Once this theory is working
correctly, you can use Holmake with the rest of the script files to
make sure everything works together.  Of course, prior to finishing
cipher Theory, if you run Holmake on cipher Theory with the rest of
the script files in the same subdirectory, you will get compilation
errors because many of the other script files depend on signing and
signature checking. \textbf{The listing of theories in the Appendices
  shows all the theories working together (without key certificate
  syntax and semantics) once cipher Theory is working correctly.}

All of the remaining development is in messageCertificate Theory.  Add
the datatypes, then the definitions, followed by the proofs.  It is a
good idea with each addition or change to recompile using Holmake.

The appendices also have a listing of the script files I have made
available to you.
% \vspace*{1.5in}
% \begin{center}
%   \begin{Large}
%     \bluetext{\textbf{Good Luck!}}
%   \end{Large}
% \end{center}

\subsection{Relating Messages and Certificates to Their Interpretations}
\label{sec:interpretations}

\begin{table}[t]
  \centering
  \begin{tabular}{|r>{$}l<{$}|}
    \hline
    \textbf{Classification} & \textbf{Access-Control Statement}\\
    \hline
    \textbf{Signed Message:} & K_{Alice} \quoting BFC \says \action{go}\\
    \hline
    \textbf{Signed Key Certificate:} & 
    K_{bfca} \says K_{Alice} \speaksfor Alice\\
    \textbf{Signed Key Certificate:} & K_{jfca} \says K_{bfca} \speaksfor BFCA\\
    \hline
    \textbf{Trusted Role Certificate:} & \reps{Alice}{BFC}{\action{go}}\\
    \textbf{Trusted Key Certificate:} & K_{jfca} \speaksfor JFCA\\
    \hline
  \end{tabular}
  \caption{Classification of Access-Control Statements According to Function}
  \label{tab:classification-table}
\end{table}

One thing to remember is that the ACL\_ASSUM rules are defined in
terms of the relation \texttt{sat}, and not the message certificate
relations \texttt{msat, ksat, rootksat}, and \texttt{rootrsat}.  To
create sequents for certificates and messages you will have to use
HOL's ASSUME rule.  

The next series of sessions show the creation of various messages and
certificates corresponding to those listed in
Table~\ref{tab:classification-table}.

\begin{session}
  \begin{scriptsize}
\begin{verbatim}

- (* Signed Message from Alice as BFC *)
val BFCMessage =
ASSUME
``((M :(commands, 'world, missionPrincipals, 'Int, 'Sec) Kripke),Oi:'Int po,Os:'Sec po) msat
  MSG (From Alice) (As BFC) (To Carol)
      (Ea (pubK Carol) (plain (sym (dek1 :num))))
      (Es (sym dek1) (plain (MC go)))
      (sign Alice (hash (plain (MC go))))``;
> val BFCMessage =
     [.]
    |- (M,Oi,Os) msat
       MSG (From Alice) (As BFC) (To Carol)
         (Ea (pubK Carol) (plain (sym dek1)))
         (Es (sym dek1) (plain (MC go))) (sign Alice (hash (plain (MC go)))) :
  thm
\end{verbatim}
  \end{scriptsize}
\end{session}

\begin{session}
  \begin{scriptsize}
\begin{verbatim}

- (* Signed Key Certificate for Alice *)
val AliceCert =
ASSUME
``(((M :(commands, 'world, missionPrincipals, 'Int, 'Sec) Kripke),Oi:'Int po,Os:'Sec po) ksat
   KCert (CA BFCA) (Entity (KStaff Alice)) (SPubKey (pubK (KStaff Alice)))
            (KCrtSig (sign (KCA BFCA) (hash (plain (BFCA,KStaff Alice,(pubK (KStaff Alice))))))))``;
> val AliceCert =
     [.]
    |- (M,Oi,Os) ksat
       KCert (CA BFCA) (Entity (KStaff Alice))
         (SPubKey (pubK (KStaff Alice)))
         (KCrtSig
            (sign (KCA BFCA)
               (hash (plain (BFCA,KStaff Alice,pubK (KStaff Alice)))))) : thm
\end{verbatim}
  \end{scriptsize}
\end{session}

\begin{session}
  \begin{scriptsize}
\begin{verbatim}

- (* Signed Key Certificate for BFCA *)
val BFCACert =
ASSUME
``(((M :(commands, 'world, missionPrincipals, 'Int, 'Sec) Kripke),Oi:'Int po,Os:'Sec po) ksat
   KCert (CA JFCA) (Entity (KCA BFCA)) (SPubKey (pubK (KCA BFCA)))
            (KCrtSig (sign (KCA JFCA) (hash (plain (JFCA,KCA BFCA,(pubK (KCA BFCA))))))))``;
> val BFCACert =
     [.]
    |- (M,Oi,Os) ksat
       KCert (CA JFCA) (Entity (KCA BFCA)) (SPubKey (pubK (KCA BFCA)))
         (KCrtSig
            (sign (KCA JFCA)
               (hash (plain (JFCA,KCA BFCA,pubK (KCA BFCA)))))) : thm
\end{verbatim}
  \end{scriptsize}
\end{session}

\begin{session}
  \begin{scriptsize}
\begin{verbatim}

- (* Trusted Root Role Certificate for Alice as BFC *)
val BFCCert =
ASSUME
``((M :(commands, 'world, missionPrincipals, 'Int, 'Sec) Kripke),Oi:'Int po,Os:'Sec po) rootrsat
  RootRCert (For Alice) (As BFC) (MC go)``;
> val BFCCert =
     [.] |- (M,Oi,Os) rootrsat RootRCert (For Alice) (As BFC) (MC go) : thm
\end{verbatim}
  \end{scriptsize}
\end{session}

\begin{session}
  \begin{scriptsize}
\begin{verbatim}

- (* Trusted Root Key Certificate for JFCA *)
val JFCARootCert = 
ASSUME
 ``((M :(commands, 'world, missionPrincipals, 'Int, 'Sec) Kripke),Oi:'Int po,Os:'Sec po) rootksat 
    RootKCert (Entity (KCA JFCA)) (SPubKey (pubK (KCA JFCA)))``;
> val JFCARootCert =
     [.]
    |- (M,Oi,Os) rootksat
       RootKCert (Entity (KCA JFCA)) (SPubKey (pubK (KCA JFCA))) : thm
\end{verbatim}
  \end{scriptsize}
\end{session}

Once we have the input message from Alice and the signed key
certificates and the root role and key certificates, we can convert
them to their interpretations and use HOL's Modus Ponens inference
rule (MATCH\_MP) with BFO\_thm, followed by discharging assumptions in
the appropriate order, to get the blackBoxBFO\_thm. What follows next
illustrates how to do this.

Recall BFO\_thm, as shown below. Note that the first five antecedents
in the implication correspond to the messages and certificates (in the
same order listed) in Table~\ref{tab:classification-table}. The very
first antecedent is Alice's message to Carol.
\begin{session}
  \begin{scriptsize}
\begin{verbatim}

- BFO_thm;
> val it =
    |- (M,Oi,Os) sat
       Name (MKey (pubK (KStaff Alice))) quoting Name (MRole BFC) says
       prop (MC go) ==>
       (M,Oi,Os) sat
       Name (MKey (pubK (KCA BFCA))) says
       Name (MKey (pubK (KStaff Alice))) speaks_for Name (MStaff Alice) ==>
       (M,Oi,Os) sat
       Name (MKey (pubK (KCA JFCA))) says
       Name (MKey (pubK (KCA BFCA))) speaks_for Name (MCA BFCA) ==>
       (M,Oi,Os) sat
       reps (Name (MStaff Alice)) (Name (MRole BFC)) (prop (MC go)) ==>
       (M,Oi,Os) sat
       Name (MKey (pubK (KCA JFCA))) speaks_for Name (MCA JFCA) ==>
       (M,Oi,Os) sat Name (MRole BFC) controls prop (MC go) ==>
       (M,Oi,Os) sat
       Name (MCA JFCA) controls
       Name (MKey (pubK (KCA BFCA))) speaks_for Name (MCA BFCA) ==>
       (M,Oi,Os) sat
       Name (MCA BFCA) controls
       Name (MKey (pubK (KStaff Alice))) speaks_for Name (MStaff Alice) ==>
       (M,Oi,Os) sat prop (MC go) impf prop (WC launch) ==>
       (M,Oi,Os) sat
       Name (MKey (pubK (KStaff Carol))) quoting Name (MRole BFO) says
       prop (WC launch) : thm
\end{verbatim}
  \end{scriptsize}
\end{session}
We convert Alice's message to Carol to its interpretation in the
access-control logic by rewriting it with the msgSat\_thm (with the
order of terms reversed by HOL's GSYM inference rule).

\begin{session}
  \begin{scriptsize}
\begin{verbatim}

- BFCMessage;
> val it =
     [.]
    |- (M,Oi,Os) msat
       MSG (From Alice) (As BFC) (To Carol)
         (Ea (pubK Carol) (plain (sym dek1)))
         (Es (sym dek1) (plain (MC go))) (sign Alice (hash (plain (MC go)))) :
  thm
- msgSat_thm;
> val it =
    |- !recipient dek sender role order M Oi Os.
         (M,Oi,Os) sat
         Name (MKey (pubK (KStaff sender))) quoting Name (MRole role) says
         prop order <=>
         (M,Oi,Os) msat
         MSG (From sender) (As role) (To recipient)
           (Ea (pubK recipient) (plain (sym dek)))
           (Es (sym dek) (plain order)) (sign sender (hash (plain order))) :
  thm
- GSYM msgSat_thm;
> val it =
    |- !recipient dek sender role order M Oi Os.
         (M,Oi,Os) msat
         MSG (From sender) (As role) (To recipient)
           (Ea (pubK recipient) (plain (sym dek)))
           (Es (sym dek) (plain order))
           (sign sender (hash (plain order))) <=>
         (M,Oi,Os) sat
         Name (MKey (pubK (KStaff sender))) quoting Name (MRole role) says
         prop order : thm
- val th6 = REWRITE_RULE [GSYM msgSat_thm] BFCMessage;
> val th6 =
     [.]
    |- (M,Oi,Os) sat
       Name (MKey (pubK (KStaff Alice))) quoting Name (MRole BFC) says
       prop (MC go) : thm
- hyp th6;
> val it =
    [``(M,Oi,Os) msat
       MSG (From Alice) (As BFC) (To Carol)
         (Ea (pubK Carol) (plain (sym dek1))) (Es (sym dek1) (plain (MC go)))
         (sign Alice (hash (plain (MC go))))``] : term list
\end{verbatim}
  \end{scriptsize}
\end{session}
Note that th6 corresponds to the first antecedent in BFO\_thm and that
the hypothesis of th6 is the message.

We use the message and its interpretation to eliminate one of the
antecedents in BFO\_thm as shown below.
\begin{session}
  \begin{scriptsize}
\begin{verbatim}

 - val th7 = MATCH_MP BFO_thm th6;
> val th7 =
     [.]
    |- (M,Oi,Os) sat
       Name (MKey (pubK (KCA BFCA))) says
       Name (MKey (pubK (KStaff Alice))) speaks_for Name (MStaff Alice) ==>
       (M,Oi,Os) sat
       Name (MKey (pubK (KCA JFCA))) says
       Name (MKey (pubK (KCA BFCA))) speaks_for Name (MCA BFCA) ==>
       (M,Oi,Os) sat
       reps (Name (MStaff Alice)) (Name (MRole BFC)) (prop (MC go)) ==>
       (M,Oi,Os) sat
       Name (MKey (pubK (KCA JFCA))) speaks_for Name (MCA JFCA) ==>
       (M,Oi,Os) sat Name (MRole BFC) controls prop (MC go) ==>
       (M,Oi,Os) sat
       Name (MCA JFCA) controls
       Name (MKey (pubK (KCA BFCA))) speaks_for Name (MCA BFCA) ==>
       (M,Oi,Os) sat
       Name (MCA BFCA) controls
       Name (MKey (pubK (KStaff Alice))) speaks_for Name (MStaff Alice) ==>
       (M,Oi,Os) sat prop (MC go) impf prop (WC launch) ==>
       (M,Oi,Os) sat
       Name (MKey (pubK (KStaff Carol))) quoting Name (MRole BFO) says
       prop (WC launch) : thm
\end{verbatim}
  \end{scriptsize}
\end{session}

The corresponding idea is used on key certificates. First we rewrite
the signed key certificate for Alice using kcertStaffSat\_thm and do
MATCH\_MP on th7.  

\begin{session}
  \begin{scriptsize}
\begin{verbatim}

- (* Rewrite AliceCert with its interpretation *)
val th8 = REWRITE_RULE [GSYM kcertStaffSat_thm] AliceCert;
> val th8 =
     [.]
    |- (M,Oi,Os) sat
       Name (MKey (pubK (KCA BFCA))) says
       Name (MKey (pubK (KStaff Alice))) speaks_for Name (MStaff Alice) : thm
- val th9 = MATCH_MP th7 th8;
> val th9 =
     [..]
    |- (M,Oi,Os) sat
       Name (MKey (pubK (KCA JFCA))) says
       Name (MKey (pubK (KCA BFCA))) speaks_for Name (MCA BFCA) ==>
       (M,Oi,Os) sat
       reps (Name (MStaff Alice)) (Name (MRole BFC)) (prop (MC go)) ==>
       (M,Oi,Os) sat
       Name (MKey (pubK (KCA JFCA))) speaks_for Name (MCA JFCA) ==>
       (M,Oi,Os) sat Name (MRole BFC) controls prop (MC go) ==>
       (M,Oi,Os) sat
       Name (MCA JFCA) controls
       Name (MKey (pubK (KCA BFCA))) speaks_for Name (MCA BFCA) ==>
       (M,Oi,Os) sat
       Name (MCA BFCA) controls
       Name (MKey (pubK (KStaff Alice))) speaks_for Name (MStaff Alice) ==>
       (M,Oi,Os) sat prop (MC go) impf prop (WC launch) ==>
       (M,Oi,Os) sat
       Name (MKey (pubK (KStaff Carol))) quoting Name (MRole BFO) says
       prop (WC launch) : thm
\end{verbatim}
  \end{scriptsize}
\end{session}
We repeat the process above for all the certificates. We can move the
messages and certificates back into the conclusion by repeated
application of HOL's DISCH inference rule in the reverse order in
which we want the certificates and messages to appear. 

Suppose we wish the message to be first followed by the key
certificate. We discharge the key certificate first followed by the
message. This is shown below. \standout{\textbf{Note that the first
    two antecedents in th11 now correspond to Alice's message and her
    public key certificate signed by BFCA.}}
\begin{session}
  \begin{scriptsize}
\begin{verbatim}

- val th10 = DISCH ``((M :(commands, 'world, missionPrincipals, 'Int, 'Sec) Kripke),
      (Oi :'Int po),(Os :'Sec po)) ksat
     KCert (CA BFCA) (Entity (KStaff Alice)) (SPubKey (pubK (KStaff Alice)))
       (KCrtSig
          (sign (KCA BFCA)
             (hash (plain (BFCA,KStaff Alice,pubK (KStaff Alice))))))`` th9;
> val th10 =
     [.]
    |- (M,Oi,Os) ksat
       KCert (CA BFCA) (Entity (KStaff Alice))
         (SPubKey (pubK (KStaff Alice)))
         (KCrtSig
            (sign (KCA BFCA)
               (hash (plain (BFCA,KStaff Alice,pubK (KStaff Alice)))))) ==>
       (M,Oi,Os) sat
       Name (MKey (pubK (KCA JFCA))) says
       Name (MKey (pubK (KCA BFCA))) speaks_for Name (MCA BFCA) ==>
       (M,Oi,Os) sat
       reps (Name (MStaff Alice)) (Name (MRole BFC)) (prop (MC go)) ==>
       (M,Oi,Os) sat
       Name (MKey (pubK (KCA JFCA))) speaks_for Name (MCA JFCA) ==>
       (M,Oi,Os) sat Name (MRole BFC) controls prop (MC go) ==>
       (M,Oi,Os) sat
       Name (MCA JFCA) controls
       Name (MKey (pubK (KCA BFCA))) speaks_for Name (MCA BFCA) ==>
       (M,Oi,Os) sat
       Name (MCA BFCA) controls
       Name (MKey (pubK (KStaff Alice))) speaks_for Name (MStaff Alice) ==>
       (M,Oi,Os) sat prop (MC go) impf prop (WC launch) ==>
       (M,Oi,Os) sat
       Name (MKey (pubK (KStaff Carol))) quoting Name (MRole BFO) says
       prop (WC launch) : thm
- val th11 = DISCH ``((M :(commands, 'world, missionPrincipals, 'Int, 'Sec) Kripke),
      (Oi :'Int po),(Os :'Sec po)) msat
     MSG (From Alice) (As BFC) (To Carol)
       (Ea (pubK Carol) (plain (sym (dek1 :num))))
       (Es (sym dek1) (plain (MC go))) (sign Alice (hash (plain (MC go))))`` th10;
> val th11 =
    |- (M,Oi,Os) msat
       MSG (From Alice) (As BFC) (To Carol)
         (Ea (pubK Carol) (plain (sym dek1)))
         (Es (sym dek1) (plain (MC go)))
         (sign Alice (hash (plain (MC go)))) ==>
       (M,Oi,Os) ksat
       KCert (CA BFCA) (Entity (KStaff Alice))
         (SPubKey (pubK (KStaff Alice)))
         (KCrtSig
            (sign (KCA BFCA)
               (hash (plain (BFCA,KStaff Alice,pubK (KStaff Alice)))))) ==>
       (M,Oi,Os) sat
       Name (MKey (pubK (KCA JFCA))) says
       Name (MKey (pubK (KCA BFCA))) speaks_for Name (MCA BFCA) ==>
       (M,Oi,Os) sat
       reps (Name (MStaff Alice)) (Name (MRole BFC)) (prop (MC go)) ==>
       (M,Oi,Os) sat
       Name (MKey (pubK (KCA JFCA))) speaks_for Name (MCA JFCA) ==>
       (M,Oi,Os) sat Name (MRole BFC) controls prop (MC go) ==>
       (M,Oi,Os) sat
       Name (MCA JFCA) controls
       Name (MKey (pubK (KCA BFCA))) speaks_for Name (MCA BFCA) ==>
       (M,Oi,Os) sat
       Name (MCA BFCA) controls
       Name (MKey (pubK (KStaff Alice))) speaks_for Name (MStaff Alice) ==>
       (M,Oi,Os) sat prop (MC go) impf prop (WC launch) ==>
       (M,Oi,Os) sat
       Name (MKey (pubK (KStaff Carol))) quoting Name (MRole BFO) says
       prop (WC launch) : thm
\end{verbatim}
  \end{scriptsize}
\end{session}

There is one last item: converting the conclusion of th11 into the
message put out by Carol to the weapon. This is done by rewriting the
conclusion using msgSat\_thm with it variables suitably specialized.
This is shown below. \textbf{\standout{Note that now the conclusion is
    the message sent by Carol to the Weapon. Once you prove all the
    signed key certificate theorems, you will use them and the role
    certificate theorems, to introduce the remaining certificates into
    the theorem to prove blackBoxBFO\_thm.}}
\begin{session}
  \begin{scriptsize}
\begin{verbatim}

- SPECL [``Weapon``,``dek2:num``,``Carol``,``BFO``,``WC launch``] msgSat_thm;
> val it =
    |- !M Oi Os.
         (M,Oi,Os) sat
         Name (MKey (pubK (KStaff Carol))) quoting Name (MRole BFO) says
         prop (WC launch) <=>
         (M,Oi,Os) msat
         MSG (From Carol) (As BFO) (To Weapon)
           (Ea (pubK Weapon) (plain (sym dek2)))
           (Es (sym dek2) (plain (WC launch)))
           (sign Carol (hash (plain (WC launch)))) : thm
- REWRITE_RULE [SPECL [``Weapon``,``dek2:num``,``Carol``,``BFO``,``WC launch``] msgSat_thm] th11;
> val it =
    |- (M,Oi,Os) msat
       MSG (From Alice) (As BFC) (To Carol)
         (Ea (pubK Carol) (plain (sym dek1)))
         (Es (sym dek1) (plain (MC go)))
         (sign Alice (hash (plain (MC go)))) ==>
       (M,Oi,Os) ksat
       KCert (CA BFCA) (Entity (KStaff Alice))
         (SPubKey (pubK (KStaff Alice)))
         (KCrtSig
            (sign (KCA BFCA)
               (hash (plain (BFCA,KStaff Alice,pubK (KStaff Alice)))))) ==>
       (M,Oi,Os) sat
       Name (MKey (pubK (KCA JFCA))) says
       Name (MKey (pubK (KCA BFCA))) speaks_for Name (MCA BFCA) ==>
       (M,Oi,Os) sat
       reps (Name (MStaff Alice)) (Name (MRole BFC)) (prop (MC go)) ==>
       (M,Oi,Os) sat
       Name (MKey (pubK (KCA JFCA))) speaks_for Name (MCA JFCA) ==>
       (M,Oi,Os) sat Name (MRole BFC) controls prop (MC go) ==>
       (M,Oi,Os) sat
       Name (MCA JFCA) controls
       Name (MKey (pubK (KCA BFCA))) speaks_for Name (MCA BFCA) ==>
       (M,Oi,Os) sat
       Name (MCA BFCA) controls
       Name (MKey (pubK (KStaff Alice))) speaks_for Name (MStaff Alice) ==>
       (M,Oi,Os) sat prop (MC go) impf prop (WC launch) ==>
       (M,Oi,Os) msat
       MSG (From Carol) (As BFO) (To Weapon)
         (Ea (pubK Weapon) (plain (sym dek2)))
         (Es (sym dek2) (plain (WC launch)))
         (sign Carol (hash (plain (WC launch)))) : thm
\end{verbatim}
  \end{scriptsize}
\end{session}



\newpage{}

\part*{Appendices}
\label{part:appendicies}

\HOLpagestyle

% ::::::::::::::::::::::::::::::::::::::::::::::::::::::::::::::::::::::::::
\section{cipher Theory}
\index{cipher Theory@\textbf  {cipher Theory}}
\begin{flushleft}
\textbf{Built:} \HOLcipherDate \\[2pt]
\textbf{Parent Theories:} list
\end{flushleft}
% ::::::::::::::::::::::::::::::::::::::::::::::::::::::::::::::::::::::::::

\subsection{Datatypes}
\index{cipher Theory@\textbf  {cipher Theory}!Datatypes}
% .....................................

\HOLcipherDatatypes

\subsection{Definitions}
\index{cipher Theory@\textbf  {cipher Theory}!Definitions}
% .....................................

\HOLcipherDefinitions

\subsection{Theorems}
\index{cipher Theory@\textbf  {cipher Theory}!Theorems}
% .....................................

\HOLcipherTheorems

% ::::::::::::::::::::::::::::::::::::::::::::::::::::::::::::::::::::::::::
\section{command Theory}
\index{command Theory@\textbf  {command Theory}}
\begin{flushleft}
\textbf{Built:} \HOLcommandDate \\[2pt]
\textbf{Parent Theories:} aclDrules
\end{flushleft}
% ::::::::::::::::::::::::::::::::::::::::::::::::::::::::::::::::::::::::::

\subsection{Datatypes}
\index{command Theory@\textbf  {command Theory}!Datatypes}
% .....................................

\HOLcommandDatatypes

% No definitions

% No theorems

% ::::::::::::::::::::::::::::::::::::::::::::::::::::::::::::::::::::::::::
\section{missionRoles Theory}
\index{missionRoles Theory@\textbf  {missionRoles Theory}}
\begin{flushleft}
\textbf{Built:} \HOLmissionRolesDate \\[2pt]
\textbf{Parent Theories:} command
\end{flushleft}
% ::::::::::::::::::::::::::::::::::::::::::::::::::::::::::::::::::::::::::

\subsection{Datatypes}
\index{missionRoles Theory@\textbf  {missionRoles Theory}!Datatypes}
% .....................................

\HOLmissionRolesDatatypes

% No definitions

\subsection{Theorems}
\index{missionRoles Theory@\textbf  {missionRoles Theory}!Theorems}
% .....................................

\HOLmissionRolesTheorems

% ::::::::::::::::::::::::::::::::::::::::::::::::::::::::::::::::::::::::::
\section{missionStaff Theory}
\index{missionStaff Theory@\textbf  {missionStaff Theory}}
\begin{flushleft}
\textbf{Built:} \HOLmissionStaffDate \\[2pt]
\textbf{Parent Theories:} missionRoles
\end{flushleft}
% ::::::::::::::::::::::::::::::::::::::::::::::::::::::::::::::::::::::::::

\subsection{Datatypes}
\index{missionStaff Theory@\textbf  {missionStaff Theory}!Datatypes}
% .....................................

\HOLmissionStaffDatatypes

% No definitions

\subsection{Theorems}
\index{missionStaff Theory@\textbf  {missionStaff Theory}!Theorems}
% .....................................

\HOLmissionStaffTheorems

% ::::::::::::::::::::::::::::::::::::::::::::::::::::::::::::::::::::::::::
\section{revisedMissionKeys Theory}
\index{revisedMissionKeys Theory@\textbf  {revisedMissionKeys Theory}}
\begin{flushleft}
\textbf{Built:} \HOLrevisedMissionKeysDate \\[2pt]
\textbf{Parent Theories:} missionStaff, cipher
\end{flushleft}
% ::::::::::::::::::::::::::::::::::::::::::::::::::::::::::::::::::::::::::

\subsection{Datatypes}
\index{revisedMissionKeys Theory@\textbf  {revisedMissionKeys Theory}!Datatypes}
% .....................................

\HOLrevisedMissionKeysDatatypes

% No definitions

% No theorems

% ::::::::::::::::::::::::::::::::::::::::::::::::::::::::::::::::::::::::::
\section{messageCertificate Theory}
\index{messageCertificate Theory@\textbf  {messageCertificate Theory}}
\begin{flushleft}
\textbf{Built:} \HOLmessageCertificateDate \\[2pt]
\textbf{Parent Theories:} revisedMissionKeys
\end{flushleft}
% ::::::::::::::::::::::::::::::::::::::::::::::::::::::::::::::::::::::::::

\subsection{Datatypes}
\index{messageCertificate Theory@\textbf  {messageCertificate Theory}!Datatypes}
% .....................................

\HOLmessageCertificateDatatypes

\subsection{Definitions}
\index{messageCertificate Theory@\textbf  {messageCertificate Theory}!Definitions}
% .....................................

\HOLmessageCertificateDefinitions

\subsection{Theorems}
\index{messageCertificate Theory@\textbf  {messageCertificate Theory}!Theorems}
% .....................................

\HOLmessageCertificateTheorems

% ::::::::::::::::::::::::::::::::::::::::::::::::::::::::::::::::::::::::::
\section{BFOConops Theory}
\index{BFOConops Theory@\textbf  {BFOConops Theory}}
\begin{flushleft}
\textbf{Built:} \HOLBFOConopsDate \\[2pt]
\textbf{Parent Theories:} messageCertificate
\end{flushleft}
% ::::::::::::::::::::::::::::::::::::::::::::::::::::::::::::::::::::::::::

% No datatypes

% No definitions

\subsection{Theorems}
\index{BFOConops Theory@\textbf  {BFOConops Theory}!Theorems}
% .....................................

\HOLBFOConopsTheorems

\HOLindex

\section{cipherScript.sml}
\label{sec:cipherScript}

\begin{scriptsize}
  \begin{alltt}
\input{AssignedHOL/cipherScript.sml}
  \end{alltt}
\end{scriptsize}

\section{commandScript.sml}
\begin{scriptsize}
  \begin{alltt}
\input{AssignedHOL/commandScript.sml}
  \end{alltt}
\end{scriptsize}

\section{missionRolesScript.sml}
\label{sec:missionRolesScript}

\begin{scriptsize}
  \begin{alltt}
\input{AssignedHOL/missionRolesScript.sml}
  \end{alltt}
\end{scriptsize}

\section{missionStaffScript.sml}
\label{sec:missionStaffScript}

\begin{scriptsize}
  \begin{alltt}
\input{AssignedHOL/missionStaffScript.sml}
  \end{alltt}
\end{scriptsize}

\section{revisedMissionKeysScript.sml}
\label{sec:revisedMissionKeysScript}

\begin{scriptsize}
  \begin{alltt}
\input{AssignedHOL/revisedMissionKeysScript.sml}
  \end{alltt}
\end{scriptsize}

\section{messageCertificateScript.sml}
\label{sec:messageCertificateScript}

\begin{scriptsize}
  \begin{alltt}
\input{AssignedHOL/messageCertificateScript.sml}
  \end{alltt}
\end{scriptsize}

\section{BFOConopsScript.sml}
\label{sec:BFOConopsScript}

\begin{scriptsize}
  \begin{alltt}
\input{AssignedHOL/BFOConopsScript.sml}
  \end{alltt}
\end{scriptsize}



\end{document}
%  LocalWords:  BoxSpec tuples BoxImplementation BV haSpec haImp COND Holmake
%  LocalWords:  tuple arithmeticTheory andGate xorGate subgoals MULT HOLcommand

% LocalWords:  thm DISCH REFL ANTISYM GENL ty ISPEC ISPECL th indices HD SUC rl
% LocalWords:  CONV listTheory reduceLib maketitle thispagestyle emph textbf fn

% LocalWords:  hw4.sml texttt texttt sequents seqs scriptsize tacticals qquad
% LocalWords:  irule conj-forward-proof conj-tac conj linewidth linewidth hline
% LocalWords:  seq seq fbox forall footnotesize ldots asm-rewrite-tac-proof-1
% LocalWords:  asm-rewrite-tac-proof-2 asm-rewrite-tac-proof-3 Ponens ASM TAC
% LocalWords:  ASSUM Modus EQ subgoal DISJ Contrapositives SYM ELIM bool online
% LocalWords:  contrapositive sequent's DeMorgan's HOL hw sml acl nogo CGs CAs
%  LocalWords:  infRules Kripke pName Auth FactorAuth HOLmissionRoles conops
% LocalWords:  HOLmissionStaff HOLmissionKeys HOLmissionCONOPSOne varphi equiv
% LocalWords:  HOLmissionCONOPSTwo includegraphics deriv-infer-rules missioninf
% LocalWords:  bfo-gfo-launch bfo-gfo-abort subsubsection holboxed hfill andf
% LocalWords:  HOLcommandDatatypesmissionCommands HOLcommandDatatypescommands
% LocalWords:  HOLcommandDatatypesweaponCommands HOLmissionRolesTheorems alltt
% LocalWords:  HOLmissionRolesDatatypesmissionRoles ImpliedCOntrolsSays newpage
% LocalWords:  HOLmissionRolesTheoremsImpliedControlsSaysXXthm BFO
% LocalWords:  HOLmissionRolesTheoremsDualControlXXthm HOLmissionStaffDatatypes
% LocalWords:  HOLmissionRolesTheoremsAlternateControlsOneXXthm BFC
% LocalWords:  HOLmissionRolesTheoremsAlternateControlsTwoXXthm BFS
% LocalWords:  HOLmissionCONOPSOneTheorems HOLmissionStaffTheorems
% LocalWords:  HOLmissionCONOPSTwoTheorems HOLmissionKeysDatatypes
%  LocalWords:  bfca jfca messageCertificate signVerify deciphP Ekcrt
%  LocalWords:  signVerifyOK KCertSignature KerCertificate checkKCert
%  LocalWords:  RootKeyCertificate Erootkcrt ksat rootksat BFOConops
%  LocalWords:  checkKCertOK kcrtCAInterp kcrtStaffInterp blackBoxBFO
%  LocalWords:  rootkcertCAInterp rootkcertStaffInterp missionStaff
%  LocalWords:  missionRoles revisedMissionKeys cipherTheory msat
%  LocalWords:  KeyCertificate subdirectory aclDrules cipherScript
%  LocalWords:  commandScript missionRolesScript missionStaffScript
%  LocalWords:  revisedMissionKeysScript messageCertificateScript
%  LocalWords:  BFOConopsScript kcertCASat kcertStaffSat rootrsat
%  LocalWords:  msgSat GSYM

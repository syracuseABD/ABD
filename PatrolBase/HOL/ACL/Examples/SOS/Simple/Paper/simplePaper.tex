\documentclass[10pt,twoside]{article}
\usepackage{./LaTeX/engineeringAssurance}
% \usepackage{./LaTeX/unclassFOUO}
%Times for rm and math | Helvetica for ss | Courier for tt
\usepackage{mathptmx} % rm & math
\usepackage[scaled=0.90]{helvet} % ss
\usepackage{courier} % tt
\usepackage{amsmath}
\usepackage{enumerate}

%  %listings is used for printing ML source code files
\usepackage{listings}
% %listings paramenters: ML is the source code, we want to print in scriptsize
\lstset{language=ML,
           basicstyle=\scriptsize,
           breaklines=true}

\normalfont
\usepackage[T1]{fontenc}
\newcommand{\action}[1]{\ensuremath{\langle #1 \rangle}}

\newcommand{\name}[1]{\ensuremath{\textit{#1}}}
\newcommand{\filename}[1]{\ensuremath{\mathtt{#1}}}
\newcommand{\rulespace}{\vspace*{2em}}


\newcommand{\pair}[1]{\ensuremath{\langle #1\rangle}}
%\newcommand{\annd}{\ensuremath{\ \&\ }}
\newcommand{\annd}{\ensuremath{\textrm{ and }}}
% \newcommand{\orr}{\ensuremath{\textrm{ or }}}
\newcommand{\pow}[1]{\ensuremath{\mathcal{P}(#1)}}
\newcommand{\set}[1]{\ensuremath{\{#1\}}}
\newcommand{\midset}[2]{\ensuremath{\set{#1 \mid #2}}}
\newcommand{\arrow}{\ensuremath{\rightarrow}}
\newcommand{\id}[1]{\ensuremath{\textsf{id}_{#1}}}
\newcommand{\subst}[3]{\ensuremath{{#1}\boldsymbol{[}#2\boldsymbol{/}#3\boldsymbol{]}}} 

%% For mathematical proofs with ``reasons'' on each step
\newenvironment{mathprf}
  {\begin{displaymath}\begin{array}{rcll}}{\end{array}\end{displaymath}}
\newcommand{\why}[1]{\ensuremath{\quad \text{#1}}}

%% for conventions (spelled out explicitly in text)
\newenvironment{convention}{\begin{description} \item[\textit{Convention:}
    ]}{\end{description}} 

\newcommand{\readernote}[2]{\noindent\shadowbox{\parbox{.95\textwidth}{%
      \begin{description} \item[\textit{#1:}] #2 \end{description}}}}



\newenvironment{indentedExample}{\begin{example}\ \begin{list}{}{}\item }{\end{list}\end{example}}

%\newenvironment{indentedExample}{\begin{example}}{\end{example}}
% for grammars

\newcommand{\isa}{\ensuremath{\; {:}{:}{=} \;}}
\newcommand{\goesto}{\ensuremath{\leadsto}}
%\newcommand{\ora}{\ensuremath{\;\mid\;}}
\newcommand{\ora}{\ensuremath{\;/\;}}
%\newcommand{\syncat}[1]{\hbox{\textcolor{red}{\sc$\langle$#1$\rangle$}}}
%\newcommand{\syncat}[1]{\hbox{{ \sc$\langle$#1$\rangle$}}}
%\newcommand{\syncat}[1]{\hbox{{ \bf #1 }}}
\newcommand{\syncat}[1]{\ensuremath{\textbf{#1}}\xspace}

% Miscellaneous

\newcommand{\defined}{\ensuremath{\quad \triangleq \quad}}
\newcommand{\defn}{\ensuremath{\stackrel{\mathrm{def}}{=}}}


\newcommand{\assign}{\ensuremath{:=}}

%%% for hiding self comments
%\renewcommand{\suebox}[1]{}
%\renewcommand{\chinbox}[1]{}

\newcommand{\key}[1]{\textbf{#1}}

% encryption

\newcommand{\encrypt}[2]{\ensuremath{\mathit{encrypt}(#2,#1)}}
\newcommand{\cat}[1]{\ensuremath{\langle\!\langle #1 \rangle \! \rangle}}

% Syntactic sets
\newcommand{\PName}{\ensuremath{\textbf{PName}}\xspace}
\newcommand{\PExp}{\ensuremath{\textbf{Princ}}\xspace}
\newcommand{\PropVar}{\ensuremath{\textbf{PropVar}}\xspace}
\newcommand{\LExp}{\ensuremath{\textbf{Form}}\xspace}
\newcommand{\LabelConst}{\ensuremath{\textbf{SecLabel}}\xspace}
\newcommand{\Level}{\ensuremath{\textbf{SecLevel}}\xspace}
\newcommand{\IntLabelConst}{\ensuremath{\textbf{IntLabel}}\xspace}
\newcommand{\IntLevel}{\ensuremath{\textbf{IntLevel}}\xspace}
% \newcommand{\PName}{\ensuremath{\textbf{\textsc{PName}}}\xspace}
% \newcommand{\PExp}{\ensuremath{\textbf{\textsc{Princ}}}\xspace}
% \newcommand{\PropVar}{\ensuremath{\textbf{\textsc{PropVar}}}\xspace}
% \newcommand{\LExp}{\ensuremath{\textbf{\textsc{Form}}}\xspace}
%\newcommand{\LExp}{\ensuremath{\textbf{\underline{Form} }}}
%\newcommand{\privs}{\ensuremath{\mathit{privs}}}
%\newcommand{\Targ}{\ensuremath{\mathcal{T}}}


% Principals
\newcommand{\with}{\ensuremath{\;\&\;}}
\newcommand{\quoting}{\ensuremath{\;|\;}}
\newcommand{\for}[1]{\ensuremath{\;\textsf{for}_{#1}\;}}

% Logical expressions
\newcommand{\has}{\ensuremath{\textsf{ has }}}
\newcommand{\says}{\ensuremath{\text{\footnotesize \textsf{ says }}}}
\newcommand{\controls}{\ensuremath{\text{\footnotesize \textsf{ controls }}}}
\newcommand{\serves}{\ensuremath{\textsf{ serves }}}
\newcommand{\speaksfor}{\ensuremath{\Rightarrow}}
\newcommand{\then}{\;\supset\;}
\newcommand{\phiplus}{\ensuremath{\varphi^{+}}}
% SKC - added syntactic sugar for ``represents''
\newcommand{\rreps}{\ensuremath{\text{\footnotesize \textsf{reps} }}}
\newcommand{\reps}[3]{\ensuremath{{#1} \text{\footnotesize \textsf{
          reps }}{#2}{\text{\footnotesize \textsf{ on }}}{#3}}}
\newcommand{\controlsandsays}{\ensuremath{\textsf{ controls+says }}}
\newcommand{\rp}[2]{\ensuremath{{#1} {\text{\footnotesize \textsf{
          reps }}}{#2}{\text{\footnotesize \textsf{ on }}}}}
\newcommand{\slv}[1]{\ensuremath{\text{\textsf{ slev}}(#1)}}
\newcommand{\ilv}[1]{\ensuremath{\text{\textsf{ ilev}}(#1)}}


% Semantics
\newcommand{\struct}[1]{\ensuremath{\langle {#1} \rangle}}
\newcommand{\krip}[1]{\ensuremath{\langle {#1} \rangle}}

\newcommand{\E}[1]{\ensuremath{\mathcal{E}_{\mathcal{M}}[\![#1]\!]}}
\newcommand{\Ee}{\ensuremath{\mathcal{E}_{\mathcal{M}}}}
\newcommand{\Em}[2]{\ensuremath{\mathcal{E}_{#2}[\![#1]\!]}}

%% for actions: 
%%   \actionit puts contents into textit mode
\newcommand{\action}[1]{\ensuremath{\langle #1 \rangle}}
\newcommand{\actionit}[1]{\ensuremath{\langle \textit{#1} \rangle}}

\newcommand{\sig}[1]{\ensuremath{\mathit{Signature}_{#1}}}

\newcommand{\mssmeet}{\ensuremath{\curlywedge}}
\newcommand{\mssbar}{\ensuremath{\|}}
\newcommand{\below}{\ensuremath{\leq}}

\newcommand{\eqmod}[1]{\ensuremath{\approx_{#1}}}

% Derivability

\newcommand{\infrule}[2]
   {\ensuremath{{\textstyle #1}\over{\textstyle #2}}}

\newcommand{\infname}[1]{\textit{#1}}

\newcommand{\irule}[3]
    {\ensuremath{\infname{#3}\quad {\displaystyle \frac{#1}{#2}}}}

\newcommand{\displaynewrule}[3]
{\begin{displaymath}
  \colorbox{LightGray}{\irule{#1}{#2}{#3}}
%  \colorbox{SpringGreen}{\irule{#1}{#2}{#3}}
\end{displaymath}}

\newcommand{\displaycorerule}[4]
{\begin{displaymath}
  \colorbox{lightgray}{\irule{#1}{#2}{#3}{#4}}
%  \colorbox{SkyBlue}{\irule{#1}{#2}{#3}{#4}}
\end{displaymath}}

\newcommand{\displaycoredef}[1]
{\begin{displaymath}
    \colorbox{lightgray}{\ensuremath{#1}}
%  \colorbox{SkyBlue}{\ensuremath{#1}}
\end{displaymath}}

%% Assorted notation for RBAC
\newcommand{\inherits}{\ensuremath{\,\succeq\,}}   % role inheritance
\newcommand{\Users}{\ensuremath{\mathit{Users}}}
\newcommand{\Perms}{\ensuremath{\mathit{Perms}}}
\newcommand{\Roles}{\ensuremath{\mathit{Roles}}}
\newcommand{\Sessions}{\ensuremath{\mathit{Sessions}}}
\newcommand{\SSD}{\ensuremath{\mathit{SSD}}}
\newcommand{\DSD}{\ensuremath{\mathit{DSD}}}
\newcommand{\ausers}{\ensuremath{\mathit{auth\_users}}}
\newcommand{\aperms}{\ensuremath{\mathit{auth\_perms}}}
\newcommand{\suser}{\ensuremath{\mathit{user}}}
\newcommand{\sroles}{\ensuremath{\mathit{roles}}}


% Generalized addresses
% \newcommand{\genAddr}[2]{\ensuremath{\langle\negmedspace\langle #1
%     \rangle\negmedspace\rangle}{\mid #2}}
\newcommand{\segName}[1]{\ensuremath{\langle\negmedspace\langle #1
    \rangle\negmedspace\rangle}}

%% address-descriptor location for named segment
\newcommand{\adLoc}[1]{\ensuremath{|\!| #1 |\!|}}
\newcommand{\segLoc}[1]
    {\ensuremath{\langle\!\!\langle #1 \rangle\!\!\rangle}}
\newcommand{\genAddr}[2]{\ensuremath{[ #1:#2 ]}}
\newcommand{\genAddrPair}[2]{\ensuremath{(#1,#2)}}

%% Formal proof environment

\newenvironment{formalProof}
{\begin{center}\small\begin{tabular}{ll}}{\end{tabular}\end{center}} 


%% Alternate Formal proof environment



\newenvironment{tabProof}{\begin{center}\footnotesize\begin{tabular}{r
        >{$}p{3.2in}<{$}p{1.0in}}}{\end{tabular}\end{center}}  
\newenvironment{tabProof2}{\begin{center}\footnotesize\begin{tabular}{r >{$}p{2.7in}<{$}p{1.5in}}}{\end{tabular}\end{center}} 

% if then else
\newcommand{\ite}[3]
   {\ensuremath{#1 \rightarrow #2 \mid #3}}

%%  Macro(s) for section summaries
\newcommand*{\summproblem}[1]{\def\fromsummproblem{#1}}
\newcommand*{\summanalysis}[1]{\def\fromsummanalysis{#1}}
\newcommand*{\summresults}[1]{\def\fromsummresults{#1}}

\newcommand{\missing}{\textsc{Missing Component}}
\summproblem{\missing}
\summanalysis{\missing}
\summresults{\missing}

\newcommand{\makesummary}
  {%\section{Summary} 
    \begin{enumerate}
    \item \textsc{What is the problem?}
%      \begin{quote}
%        \it What is the purpose of my thinking?  What precise
%        question am I trying to answer?  Within what point of view am I
%        thinking?
%      \end{quote}
%      \ \\ 

      \fromsummproblem
    \item  \textsc{How am I analyzing the problem?}
%      \begin{quote}
%        \it What am I taking for granted, and what assumptions am
%        I making? What information am I using?  How am I interpreting
%        that information?  What concepts or ideas are central to my
%        thinking? 
%      \end{quote}
%      \ \\ 

      \fromsummanalysis
    \item \textsc{What are the results?}
%      \begin{quote}
%        \it What conclusions am I coming to?  If I accept the
%        conclusions, what are the implications?  What would the
%        consequences be, if I put my thought into action?
%      \end{quote}
%      \ \\ 

      \fromsummresults
    \end{enumerate}

  \summproblem{\missing}
  \summanalysis{\missing}
  \summresults{\missing}
    
}


%----redefine implication as horseshoe instead of arrow--------

\renewcommand{\implies}{\supset}
\newcommand{\believes}{\ensuremath\textsf{ believes }}

%%% Local Variables: 
%%% mode: latex
%%% TeX-master: "book"
%%% End: 

\renewcommand{\infrule}[2]
   {\ensuremath{{\textstyle #1}\over{\textstyle #2}}}

\renewcommand{\infname}[1]{\textit{#1}}
% \renewcommand{\irule}[3]
%     {\ensuremath{\infname{#3}\quad {\displaystyle \frac{#1}{#2}}}}

\ifx\pdfoutput\undefined
\usepackage[dvips]{graphicx}
\else
\usepackage[pdftex]{graphicx}
\usepackage{epstopdf}
\pdfcompresslevel=9
\fi

%Notes in text
\newcommand{\chinbox}[1]{%
  \fbox{\parbox[t]{6.0in}{\textsc{Note to Self:} 
      \begin{center}
        #1
      \end{center}}}}

\newcommand{\problembox}[1]
{
  \fbox{\begin{minipage}{0.9\linewidth}
      \begin{center}
        \redtext{\underline{\textbf{\textsc{Assignment}}}}
      \end{center}
      #1
  \end{minipage}}
}


\usepackage{array}

% ---------------------------------------------------------------------
% Input defined macros and commands
% ---------------------------------------------------------------------
\input{./LaTeX/commands}
\input{./LaTeX/pic-macros}
\input{./LaTeX/ref-macros}

\usepackage{url}
\usepackage[line,arrow,frame,matrix]{xy}
\usepackage{./LaTeX/proof}
\usepackage{holtex}
\usepackage{holtexbasic}

% \usepackage[usenames,dvipsnames]{color}
\definecolor{orange}{rgb}{1,0.5,0}
\newcommand{\redtext}[1]{\textcolor{red}{#1}}
\newcommand{\bluetext}[1]{\textcolor{blue}{#1}}
\newcommand{\magtext}[1]{\textcolor{magenta}{#1}}
\newcommand{\greentext}[1]{\textcolor{green}{#1}}
\newcommand{\orangetext}[1]{\textcolor{orange}{#1}}
\newcommand{\standout}[1]{\textcolor{orange}{#1}}

\newcommand{\seq}[2]{\ensuremath{\set{#1} \vdash {#2}}}
\newcommand{\seqs}[2]{\ensuremath{#1 \vdash {#2}}}
\newcommand{\sq}[1]{\ensuremath{\vdash {#1}}}

\newcommand{\goal}[2]{\ensuremath{\set{#1}\;\text{ ?-- }\;{#2}}}
\newcommand{\goals}[2]{\ensuremath{{#1}\;\text{ ?-- }\;{#2}}}
\newcommand{\gls}[1]{\ensuremath{\text{ ?-- }\;{#1}}}

\renewcommand{\irule}[3]
    {\ensuremath{{\displaystyle \frac{#1}{#2}}\quad \infname{#3}}}
\newcommand{\tac}[3]{
  \ensuremath{\begin{tabular}{c}
    {#1}\\\hline\hline{#2}
  \end{tabular}\quad}{#3}}
% HOL theories

\title{\textsc{Assuring Integrity and Security of Transition Systems
    Using Structural Operational Semantics and Access-Control Logic}}

\author{Shiu-Kai Chin}

\date{\today{}}

\makeindex

\begin{document}

% ---------------------------------------------------------------------
% Inputs for HOL reports
% ---------------------------------------------------------------------
\newcommand{\HOLfoundationDate}{20 August 2016}
\newcommand{\HOLfoundationTime}{12:50}
\begin{SaveVerbatim}{HOLfoundationDatatypesblock}
\HOLFreeVar{block} = \HOLConst{BLK} (('Int, 'Sec) \HOLTyOp{messages}) (('Int, 'Sec) \HOLTyOp{configuration})
\end{SaveVerbatim}
\newcommand{\HOLfoundationDatatypesblock}{\UseVerbatim{HOLfoundationDatatypesblock}}
\begin{SaveVerbatim}{HOLfoundationDatatypescertificates}
\HOLFreeVar{certificates} = \HOLConst{Certs} ((\HOLTyOp{Statements}, \HOLTyOp{Roles}, 'Int, 'Sec) \HOLTyOp{Form} \HOLTyOp{list})
\end{SaveVerbatim}
\newcommand{\HOLfoundationDatatypescertificates}{\UseVerbatim{HOLfoundationDatatypescertificates}}
\begin{SaveVerbatim}{HOLfoundationDatatypesCommands}
\HOLFreeVar{Commands} = \HOLConst{enable} \HOLTokenBar{} \HOLConst{disable}
\end{SaveVerbatim}
\newcommand{\HOLfoundationDatatypesCommands}{\UseVerbatim{HOLfoundationDatatypesCommands}}
\begin{SaveVerbatim}{HOLfoundationDatatypesconfiguration}
\HOLFreeVar{configuration} =
    \HOLConst{CFG} \HOLTyOp{modes} (('Int, 'Sec) \HOLTyOp{certificates})
        (('Int, 'Sec) \HOLTyOp{policies})
\end{SaveVerbatim}
\newcommand{\HOLfoundationDatatypesconfiguration}{\UseVerbatim{HOLfoundationDatatypesconfiguration}}
\begin{SaveVerbatim}{HOLfoundationDatatypesmessages}
\HOLFreeVar{messages} = \HOLConst{Messages} ((\HOLTyOp{Statements}, \HOLTyOp{Roles}, 'Int, 'Sec) \HOLTyOp{Form} \HOLTyOp{list})
\end{SaveVerbatim}
\newcommand{\HOLfoundationDatatypesmessages}{\UseVerbatim{HOLfoundationDatatypesmessages}}
\begin{SaveVerbatim}{HOLfoundationDatatypesmodes}
\HOLFreeVar{modes} = \HOLConst{STBY} \HOLTokenBar{} \HOLConst{RDY}
\end{SaveVerbatim}
\newcommand{\HOLfoundationDatatypesmodes}{\UseVerbatim{HOLfoundationDatatypesmodes}}
\begin{SaveVerbatim}{HOLfoundationDatatypespolicies}
\HOLFreeVar{policies} = \HOLConst{Policies} ((\HOLTyOp{Statements}, \HOLTyOp{Roles}, 'Int, 'Sec) \HOLTyOp{Form} \HOLTyOp{list})
\end{SaveVerbatim}
\newcommand{\HOLfoundationDatatypespolicies}{\UseVerbatim{HOLfoundationDatatypespolicies}}
\begin{SaveVerbatim}{HOLfoundationDatatypesRoles}
\HOLFreeVar{Roles} = \HOLConst{Owner}
\end{SaveVerbatim}
\newcommand{\HOLfoundationDatatypesRoles}{\UseVerbatim{HOLfoundationDatatypesRoles}}
\begin{SaveVerbatim}{HOLfoundationDatatypesStatements}
\HOLFreeVar{Statements} = \HOLConst{MODE} \HOLTyOp{modes} \HOLTokenBar{} \HOLConst{CMD} \HOLTyOp{Commands}
\end{SaveVerbatim}
\newcommand{\HOLfoundationDatatypesStatements}{\UseVerbatim{HOLfoundationDatatypesStatements}}
\newcommand{\HOLfoundationDatatypes}{
\HOLfoundationDatatypesblock\HOLfoundationDatatypescertificates\HOLfoundationDatatypesCommands\HOLfoundationDatatypesconfiguration\HOLfoundationDatatypesmessages\HOLfoundationDatatypesmodes\HOLfoundationDatatypespolicies\HOLfoundationDatatypesRoles\HOLfoundationDatatypesStatements}
\begin{SaveVerbatim}{HOLfoundationDefinitionssatListXXdef}
\HOLTokenTurnstile{} \HOLSymConst{\HOLTokenForall{}}\HOLBoundVar{M} \HOLBoundVar{Oi} \HOLBoundVar{Os} \HOLBoundVar{formList}.
     (\HOLBoundVar{M}\HOLSymConst{,}\HOLBoundVar{Oi}\HOLSymConst{,}\HOLBoundVar{Os}) \HOLConst{satList} \HOLBoundVar{formList} \HOLSymConst{\HOLTokenEquiv{}}
     \HOLConst{FOLDR} (\HOLTokenLambda{}\HOLBoundVar{x} \HOLBoundVar{y}. \HOLBoundVar{x} \HOLSymConst{\HOLTokenConj{}} \HOLBoundVar{y}) \HOLConst{T} (\HOLConst{MAP} (\HOLTokenLambda{}\HOLBoundVar{f}. (\HOLBoundVar{M}\HOLSymConst{,}\HOLBoundVar{Oi}\HOLSymConst{,}\HOLBoundVar{Os}) \HOLConst{sat} \HOLBoundVar{f}) \HOLBoundVar{formList})
\end{SaveVerbatim}
\newcommand{\HOLfoundationDefinitionssatListXXdef}{\UseVerbatim{HOLfoundationDefinitionssatListXXdef}}
\newcommand{\HOLfoundationDefinitions}{
\HOLDfnTag{foundation}{satList_def}\HOLfoundationDefinitionssatListXXdef
}
\begin{SaveVerbatim}{HOLfoundationTheoremsblocksatXXdef}
\HOLTokenTurnstile{} (\HOLFreeVar{M}\HOLSymConst{,}\HOLFreeVar{Oi}\HOLSymConst{,}\HOLFreeVar{Os}) \HOLConst{blocksat}
   \HOLConst{BLK} (\HOLConst{Messages} \HOLFreeVar{mlist})
     (\HOLConst{CFG} \HOLFreeVar{mode} (\HOLConst{Certs} \HOLFreeVar{clist}) (\HOLConst{Policies} \HOLFreeVar{plist})) \HOLSymConst{\HOLTokenEquiv{}}
   (\HOLFreeVar{M}\HOLSymConst{,}\HOLFreeVar{Oi}\HOLSymConst{,}\HOLFreeVar{Os}) \HOLConst{satList} (\HOLFreeVar{mlist} \HOLSymConst{++} \HOLFreeVar{clist} \HOLSymConst{++} \HOLFreeVar{plist})
\end{SaveVerbatim}
\newcommand{\HOLfoundationTheoremsblocksatXXdef}{\UseVerbatim{HOLfoundationTheoremsblocksatXXdef}}
\begin{SaveVerbatim}{HOLfoundationTheoremsblocksatXXind}
\HOLTokenTurnstile{} \HOLSymConst{\HOLTokenForall{}}\HOLBoundVar{P}.
     (\HOLSymConst{\HOLTokenForall{}}\HOLBoundVar{M} \HOLBoundVar{Oi} \HOLBoundVar{Os} \HOLBoundVar{mlist} \HOLBoundVar{mode} \HOLBoundVar{clist} \HOLBoundVar{plist}.
        \HOLBoundVar{P} (\HOLBoundVar{M}\HOLSymConst{,}\HOLBoundVar{Oi}\HOLSymConst{,}\HOLBoundVar{Os})
          (\HOLConst{BLK} (\HOLConst{Messages} \HOLBoundVar{mlist})
             (\HOLConst{CFG} \HOLBoundVar{mode} (\HOLConst{Certs} \HOLBoundVar{clist}) (\HOLConst{Policies} \HOLBoundVar{plist})))) \HOLSymConst{\HOLTokenImp{}}
     \HOLSymConst{\HOLTokenForall{}}\HOLBoundVar{v} \HOLBoundVar{v\sb{\mathrm{1}}} \HOLBoundVar{v\sb{\mathrm{2}}} \HOLBoundVar{v\sb{\mathrm{3}}}. \HOLBoundVar{P} (\HOLBoundVar{v}\HOLSymConst{,}\HOLBoundVar{v\sb{\mathrm{1}}}\HOLSymConst{,}\HOLBoundVar{v\sb{\mathrm{2}}}) \HOLBoundVar{v\sb{\mathrm{3}}}
\end{SaveVerbatim}
\newcommand{\HOLfoundationTheoremsblocksatXXind}{\UseVerbatim{HOLfoundationTheoremsblocksatXXind}}
\newcommand{\HOLfoundationTheorems}{
\HOLThmTag{foundation}{blocksat_def}\HOLfoundationTheoremsblocksatXXdef
\HOLThmTag{foundation}{blocksat_ind}\HOLfoundationTheoremsblocksatXXind
}

\newcommand{\HOLsosZeroDate}{20 August 2016}
\newcommand{\HOLsosZeroTime}{12:51}
\begin{SaveVerbatim}{HOLsosZeroDefinitionsTRZeroXXdef}
\HOLTokenTurnstile{} \HOLConst{TR0} \HOLSymConst{=}
   (\HOLTokenLambda{}\HOLBoundVar{a\sb{\mathrm{0}}} \HOLBoundVar{a\sb{\mathrm{1}}} \HOLBoundVar{a\sb{\mathrm{2}}}.
      \HOLSymConst{\HOLTokenForall{}}\HOLBoundVar{TR\sb{\mathrm{0}}\sp{\prime}}.
        (\HOLSymConst{\HOLTokenForall{}}\HOLBoundVar{a\sb{\mathrm{0}}} \HOLBoundVar{a\sb{\mathrm{1}}} \HOLBoundVar{a\sb{\mathrm{2}}}.
           (\HOLSymConst{\HOLTokenExists{}}\HOLBoundVar{M} \HOLBoundVar{Oi} \HOLBoundVar{Os}.
              (\HOLBoundVar{a\sb{\mathrm{0}}} \HOLSymConst{=} (\HOLBoundVar{M}\HOLSymConst{,}\HOLBoundVar{Oi}\HOLSymConst{,}\HOLBoundVar{Os})) \HOLSymConst{\HOLTokenConj{}} (\HOLBoundVar{a\sb{\mathrm{1}}} \HOLSymConst{=} \HOLConst{STBY}) \HOLSymConst{\HOLTokenConj{}} (\HOLBoundVar{a\sb{\mathrm{2}}} \HOLSymConst{=} \HOLConst{RDY}) \HOLSymConst{\HOLTokenConj{}}
              (\HOLBoundVar{M}\HOLSymConst{,}\HOLBoundVar{Oi}\HOLSymConst{,}\HOLBoundVar{Os}) \HOLConst{sat} \HOLConst{prop} (\HOLConst{MODE} \HOLConst{RDY})) \HOLSymConst{\HOLTokenDisj{}}
           (\HOLSymConst{\HOLTokenExists{}}\HOLBoundVar{M} \HOLBoundVar{Oi} \HOLBoundVar{Os}.
              (\HOLBoundVar{a\sb{\mathrm{0}}} \HOLSymConst{=} (\HOLBoundVar{M}\HOLSymConst{,}\HOLBoundVar{Oi}\HOLSymConst{,}\HOLBoundVar{Os})) \HOLSymConst{\HOLTokenConj{}} (\HOLBoundVar{a\sb{\mathrm{1}}} \HOLSymConst{=} \HOLConst{RDY}) \HOLSymConst{\HOLTokenConj{}} (\HOLBoundVar{a\sb{\mathrm{2}}} \HOLSymConst{=} \HOLConst{STBY}) \HOLSymConst{\HOLTokenConj{}}
              (\HOLBoundVar{M}\HOLSymConst{,}\HOLBoundVar{Oi}\HOLSymConst{,}\HOLBoundVar{Os}) \HOLConst{sat} \HOLConst{prop} (\HOLConst{MODE} \HOLConst{STBY})) \HOLSymConst{\HOLTokenImp{}}
           \HOLBoundVar{TR\sb{\mathrm{0}}\sp{\prime}} \HOLBoundVar{a\sb{\mathrm{0}}} \HOLBoundVar{a\sb{\mathrm{1}}} \HOLBoundVar{a\sb{\mathrm{2}}}) \HOLSymConst{\HOLTokenImp{}}
        \HOLBoundVar{TR\sb{\mathrm{0}}\sp{\prime}} \HOLBoundVar{a\sb{\mathrm{0}}} \HOLBoundVar{a\sb{\mathrm{1}}} \HOLBoundVar{a\sb{\mathrm{2}}})
\end{SaveVerbatim}
\newcommand{\HOLsosZeroDefinitionsTRZeroXXdef}{\UseVerbatim{HOLsosZeroDefinitionsTRZeroXXdef}}
\newcommand{\HOLsosZeroDefinitions}{
\HOLDfnTag{sos0}{TR0_def}\HOLsosZeroDefinitionsTRZeroXXdef
}
\begin{SaveVerbatim}{HOLsosZeroTheoremsSOSZeroXXSecurityXXthm}
\HOLTokenTurnstile{} \HOLSymConst{\HOLTokenForall{}}\HOLBoundVar{M} \HOLBoundVar{Oi} \HOLBoundVar{Os} \HOLBoundVar{M\sb{\mathrm{1}}} \HOLBoundVar{M\sb{\mathrm{2}}}.
     \HOLConst{TR0} (\HOLBoundVar{M}\HOLSymConst{,}\HOLBoundVar{Oi}\HOLSymConst{,}\HOLBoundVar{Os}) \HOLBoundVar{M\sb{\mathrm{1}}} \HOLBoundVar{M\sb{\mathrm{2}}} \HOLSymConst{\HOLTokenImp{}} (\HOLBoundVar{M}\HOLSymConst{,}\HOLBoundVar{Oi}\HOLSymConst{,}\HOLBoundVar{Os}) \HOLConst{sat} \HOLConst{prop} (\HOLConst{MODE} \HOLBoundVar{M\sb{\mathrm{2}}})
\end{SaveVerbatim}
\newcommand{\HOLsosZeroTheoremsSOSZeroXXSecurityXXthm}{\UseVerbatim{HOLsosZeroTheoremsSOSZeroXXSecurityXXthm}}
\begin{SaveVerbatim}{HOLsosZeroTheoremsTRZeroXXcases}
\HOLTokenTurnstile{} \HOLSymConst{\HOLTokenForall{}}\HOLBoundVar{a\sb{\mathrm{0}}} \HOLBoundVar{a\sb{\mathrm{1}}} \HOLBoundVar{a\sb{\mathrm{2}}}.
     \HOLConst{TR0} \HOLBoundVar{a\sb{\mathrm{0}}} \HOLBoundVar{a\sb{\mathrm{1}}} \HOLBoundVar{a\sb{\mathrm{2}}} \HOLSymConst{\HOLTokenEquiv{}}
     (\HOLSymConst{\HOLTokenExists{}}\HOLBoundVar{M} \HOLBoundVar{Oi} \HOLBoundVar{Os}.
        (\HOLBoundVar{a\sb{\mathrm{0}}} \HOLSymConst{=} (\HOLBoundVar{M}\HOLSymConst{,}\HOLBoundVar{Oi}\HOLSymConst{,}\HOLBoundVar{Os})) \HOLSymConst{\HOLTokenConj{}} (\HOLBoundVar{a\sb{\mathrm{1}}} \HOLSymConst{=} \HOLConst{STBY}) \HOLSymConst{\HOLTokenConj{}} (\HOLBoundVar{a\sb{\mathrm{2}}} \HOLSymConst{=} \HOLConst{RDY}) \HOLSymConst{\HOLTokenConj{}}
        (\HOLBoundVar{M}\HOLSymConst{,}\HOLBoundVar{Oi}\HOLSymConst{,}\HOLBoundVar{Os}) \HOLConst{sat} \HOLConst{prop} (\HOLConst{MODE} \HOLConst{RDY})) \HOLSymConst{\HOLTokenDisj{}}
     \HOLSymConst{\HOLTokenExists{}}\HOLBoundVar{M} \HOLBoundVar{Oi} \HOLBoundVar{Os}.
       (\HOLBoundVar{a\sb{\mathrm{0}}} \HOLSymConst{=} (\HOLBoundVar{M}\HOLSymConst{,}\HOLBoundVar{Oi}\HOLSymConst{,}\HOLBoundVar{Os})) \HOLSymConst{\HOLTokenConj{}} (\HOLBoundVar{a\sb{\mathrm{1}}} \HOLSymConst{=} \HOLConst{RDY}) \HOLSymConst{\HOLTokenConj{}} (\HOLBoundVar{a\sb{\mathrm{2}}} \HOLSymConst{=} \HOLConst{STBY}) \HOLSymConst{\HOLTokenConj{}}
       (\HOLBoundVar{M}\HOLSymConst{,}\HOLBoundVar{Oi}\HOLSymConst{,}\HOLBoundVar{Os}) \HOLConst{sat} \HOLConst{prop} (\HOLConst{MODE} \HOLConst{STBY})
\end{SaveVerbatim}
\newcommand{\HOLsosZeroTheoremsTRZeroXXcases}{\UseVerbatim{HOLsosZeroTheoremsTRZeroXXcases}}
\begin{SaveVerbatim}{HOLsosZeroTheoremsTRZeroXXind}
\HOLTokenTurnstile{} \HOLSymConst{\HOLTokenForall{}}\HOLBoundVar{TR\sb{\mathrm{0}}\sp{\prime}}.
     (\HOLSymConst{\HOLTokenForall{}}\HOLBoundVar{M} \HOLBoundVar{Oi} \HOLBoundVar{Os}.
        (\HOLBoundVar{M}\HOLSymConst{,}\HOLBoundVar{Oi}\HOLSymConst{,}\HOLBoundVar{Os}) \HOLConst{sat} \HOLConst{prop} (\HOLConst{MODE} \HOLConst{RDY}) \HOLSymConst{\HOLTokenImp{}}
        \HOLBoundVar{TR\sb{\mathrm{0}}\sp{\prime}} (\HOLBoundVar{M}\HOLSymConst{,}\HOLBoundVar{Oi}\HOLSymConst{,}\HOLBoundVar{Os}) \HOLConst{STBY} \HOLConst{RDY}) \HOLSymConst{\HOLTokenConj{}}
     (\HOLSymConst{\HOLTokenForall{}}\HOLBoundVar{M} \HOLBoundVar{Oi} \HOLBoundVar{Os}.
        (\HOLBoundVar{M}\HOLSymConst{,}\HOLBoundVar{Oi}\HOLSymConst{,}\HOLBoundVar{Os}) \HOLConst{sat} \HOLConst{prop} (\HOLConst{MODE} \HOLConst{STBY}) \HOLSymConst{\HOLTokenImp{}}
        \HOLBoundVar{TR\sb{\mathrm{0}}\sp{\prime}} (\HOLBoundVar{M}\HOLSymConst{,}\HOLBoundVar{Oi}\HOLSymConst{,}\HOLBoundVar{Os}) \HOLConst{RDY} \HOLConst{STBY}) \HOLSymConst{\HOLTokenImp{}}
     \HOLSymConst{\HOLTokenForall{}}\HOLBoundVar{a\sb{\mathrm{0}}} \HOLBoundVar{a\sb{\mathrm{1}}} \HOLBoundVar{a\sb{\mathrm{2}}}. \HOLConst{TR0} \HOLBoundVar{a\sb{\mathrm{0}}} \HOLBoundVar{a\sb{\mathrm{1}}} \HOLBoundVar{a\sb{\mathrm{2}}} \HOLSymConst{\HOLTokenImp{}} \HOLBoundVar{TR\sb{\mathrm{0}}\sp{\prime}} \HOLBoundVar{a\sb{\mathrm{0}}} \HOLBoundVar{a\sb{\mathrm{1}}} \HOLBoundVar{a\sb{\mathrm{2}}}
\end{SaveVerbatim}
\newcommand{\HOLsosZeroTheoremsTRZeroXXind}{\UseVerbatim{HOLsosZeroTheoremsTRZeroXXind}}
\begin{SaveVerbatim}{HOLsosZeroTheoremsTRZeroXXrules}
\HOLTokenTurnstile{} (\HOLSymConst{\HOLTokenForall{}}\HOLBoundVar{M} \HOLBoundVar{Oi} \HOLBoundVar{Os}.
      (\HOLBoundVar{M}\HOLSymConst{,}\HOLBoundVar{Oi}\HOLSymConst{,}\HOLBoundVar{Os}) \HOLConst{sat} \HOLConst{prop} (\HOLConst{MODE} \HOLConst{RDY}) \HOLSymConst{\HOLTokenImp{}} \HOLConst{TR0} (\HOLBoundVar{M}\HOLSymConst{,}\HOLBoundVar{Oi}\HOLSymConst{,}\HOLBoundVar{Os}) \HOLConst{STBY} \HOLConst{RDY}) \HOLSymConst{\HOLTokenConj{}}
   \HOLSymConst{\HOLTokenForall{}}\HOLBoundVar{M} \HOLBoundVar{Oi} \HOLBoundVar{Os}.
     (\HOLBoundVar{M}\HOLSymConst{,}\HOLBoundVar{Oi}\HOLSymConst{,}\HOLBoundVar{Os}) \HOLConst{sat} \HOLConst{prop} (\HOLConst{MODE} \HOLConst{STBY}) \HOLSymConst{\HOLTokenImp{}} \HOLConst{TR0} (\HOLBoundVar{M}\HOLSymConst{,}\HOLBoundVar{Oi}\HOLSymConst{,}\HOLBoundVar{Os}) \HOLConst{RDY} \HOLConst{STBY}
\end{SaveVerbatim}
\newcommand{\HOLsosZeroTheoremsTRZeroXXrules}{\UseVerbatim{HOLsosZeroTheoremsTRZeroXXrules}}
\begin{SaveVerbatim}{HOLsosZeroTheoremsTRZeroXXstrongind}
\HOLTokenTurnstile{} \HOLSymConst{\HOLTokenForall{}}\HOLBoundVar{TR\sb{\mathrm{0}}\sp{\prime}}.
     (\HOLSymConst{\HOLTokenForall{}}\HOLBoundVar{M} \HOLBoundVar{Oi} \HOLBoundVar{Os}.
        (\HOLBoundVar{M}\HOLSymConst{,}\HOLBoundVar{Oi}\HOLSymConst{,}\HOLBoundVar{Os}) \HOLConst{sat} \HOLConst{prop} (\HOLConst{MODE} \HOLConst{RDY}) \HOLSymConst{\HOLTokenImp{}}
        \HOLBoundVar{TR\sb{\mathrm{0}}\sp{\prime}} (\HOLBoundVar{M}\HOLSymConst{,}\HOLBoundVar{Oi}\HOLSymConst{,}\HOLBoundVar{Os}) \HOLConst{STBY} \HOLConst{RDY}) \HOLSymConst{\HOLTokenConj{}}
     (\HOLSymConst{\HOLTokenForall{}}\HOLBoundVar{M} \HOLBoundVar{Oi} \HOLBoundVar{Os}.
        (\HOLBoundVar{M}\HOLSymConst{,}\HOLBoundVar{Oi}\HOLSymConst{,}\HOLBoundVar{Os}) \HOLConst{sat} \HOLConst{prop} (\HOLConst{MODE} \HOLConst{STBY}) \HOLSymConst{\HOLTokenImp{}}
        \HOLBoundVar{TR\sb{\mathrm{0}}\sp{\prime}} (\HOLBoundVar{M}\HOLSymConst{,}\HOLBoundVar{Oi}\HOLSymConst{,}\HOLBoundVar{Os}) \HOLConst{RDY} \HOLConst{STBY}) \HOLSymConst{\HOLTokenImp{}}
     \HOLSymConst{\HOLTokenForall{}}\HOLBoundVar{a\sb{\mathrm{0}}} \HOLBoundVar{a\sb{\mathrm{1}}} \HOLBoundVar{a\sb{\mathrm{2}}}. \HOLConst{TR0} \HOLBoundVar{a\sb{\mathrm{0}}} \HOLBoundVar{a\sb{\mathrm{1}}} \HOLBoundVar{a\sb{\mathrm{2}}} \HOLSymConst{\HOLTokenImp{}} \HOLBoundVar{TR\sb{\mathrm{0}}\sp{\prime}} \HOLBoundVar{a\sb{\mathrm{0}}} \HOLBoundVar{a\sb{\mathrm{1}}} \HOLBoundVar{a\sb{\mathrm{2}}}
\end{SaveVerbatim}
\newcommand{\HOLsosZeroTheoremsTRZeroXXstrongind}{\UseVerbatim{HOLsosZeroTheoremsTRZeroXXstrongind}}
\newcommand{\HOLsosZeroTheorems}{
\HOLThmTag{sos0}{SOS0_Security_thm}\HOLsosZeroTheoremsSOSZeroXXSecurityXXthm
\HOLThmTag{sos0}{TR0_cases}\HOLsosZeroTheoremsTRZeroXXcases
\HOLThmTag{sos0}{TR0_ind}\HOLsosZeroTheoremsTRZeroXXind
\HOLThmTag{sos0}{TR0_rules}\HOLsosZeroTheoremsTRZeroXXrules
\HOLThmTag{sos0}{TR0_strongind}\HOLsosZeroTheoremsTRZeroXXstrongind
}

\newcommand{\HOLsosOneDate}{20 August 2016}
\newcommand{\HOLsosOneTime}{12:51}
\begin{SaveVerbatim}{HOLsosOneDefinitionsTROneXXdef}
\HOLTokenTurnstile{} \HOLConst{TR1} \HOLSymConst{=}
   (\HOLTokenLambda{}\HOLBoundVar{a\sb{\mathrm{0}}} \HOLBoundVar{a\sb{\mathrm{1}}} \HOLBoundVar{a\sb{\mathrm{2}}} \HOLBoundVar{a\sb{\mathrm{3}}}.
      \HOLSymConst{\HOLTokenForall{}}\HOLBoundVar{TR\sb{\mathrm{1}}\sp{\prime}}.
        (\HOLSymConst{\HOLTokenForall{}}\HOLBoundVar{a\sb{\mathrm{0}}} \HOLBoundVar{a\sb{\mathrm{1}}} \HOLBoundVar{a\sb{\mathrm{2}}} \HOLBoundVar{a\sb{\mathrm{3}}}.
           (\HOLSymConst{\HOLTokenExists{}}\HOLBoundVar{M} \HOLBoundVar{Oi} \HOLBoundVar{Os} \HOLBoundVar{RDYclist} \HOLBoundVar{RDYplist} \HOLBoundVar{STBYclist} \HOLBoundVar{STBYmlist}
               \HOLBoundVar{STBYplist}.
              (\HOLBoundVar{a\sb{\mathrm{0}}} \HOLSymConst{=} (\HOLBoundVar{M}\HOLSymConst{,}\HOLBoundVar{Oi}\HOLSymConst{,}\HOLBoundVar{Os})) \HOLSymConst{\HOLTokenConj{}} (\HOLBoundVar{a\sb{\mathrm{1}}} \HOLSymConst{=} \HOLConst{Messages} \HOLBoundVar{STBYmlist}) \HOLSymConst{\HOLTokenConj{}}
              (\HOLBoundVar{a\sb{\mathrm{2}}} \HOLSymConst{=}
               \HOLConst{CFG} \HOLConst{STBY} (\HOLConst{Certs} \HOLBoundVar{STBYclist})
                 (\HOLConst{Policies} \HOLBoundVar{STBYplist})) \HOLSymConst{\HOLTokenConj{}}
              (\HOLBoundVar{a\sb{\mathrm{3}}} \HOLSymConst{=}
               \HOLConst{CFG} \HOLConst{RDY} (\HOLConst{Certs} \HOLBoundVar{RDYclist}) (\HOLConst{Policies} \HOLBoundVar{RDYplist})) \HOLSymConst{\HOLTokenConj{}}
              (\HOLBoundVar{M}\HOLSymConst{,}\HOLBoundVar{Oi}\HOLSymConst{,}\HOLBoundVar{Os}) \HOLConst{blocksat}
              \HOLConst{BLK} (\HOLConst{Messages} \HOLBoundVar{STBYmlist})
                (\HOLConst{CFG} \HOLConst{STBY} (\HOLConst{Certs} \HOLBoundVar{STBYclist})
                   (\HOLConst{Policies} \HOLBoundVar{STBYplist})) \HOLSymConst{\HOLTokenConj{}}
              (\HOLBoundVar{M}\HOLSymConst{,}\HOLBoundVar{Oi}\HOLSymConst{,}\HOLBoundVar{Os}) \HOLConst{sat} \HOLConst{prop} (\HOLConst{MODE} \HOLConst{RDY})) \HOLSymConst{\HOLTokenDisj{}}
           (\HOLSymConst{\HOLTokenExists{}}\HOLBoundVar{M} \HOLBoundVar{Oi} \HOLBoundVar{Os} \HOLBoundVar{RDYclist} \HOLBoundVar{RDYmlist} \HOLBoundVar{RDYplist} \HOLBoundVar{STBYclist}
               \HOLBoundVar{STBYplist}.
              (\HOLBoundVar{a\sb{\mathrm{0}}} \HOLSymConst{=} (\HOLBoundVar{M}\HOLSymConst{,}\HOLBoundVar{Oi}\HOLSymConst{,}\HOLBoundVar{Os})) \HOLSymConst{\HOLTokenConj{}} (\HOLBoundVar{a\sb{\mathrm{1}}} \HOLSymConst{=} \HOLConst{Messages} \HOLBoundVar{RDYmlist}) \HOLSymConst{\HOLTokenConj{}}
              (\HOLBoundVar{a\sb{\mathrm{2}}} \HOLSymConst{=}
               \HOLConst{CFG} \HOLConst{RDY} (\HOLConst{Certs} \HOLBoundVar{RDYclist}) (\HOLConst{Policies} \HOLBoundVar{RDYplist})) \HOLSymConst{\HOLTokenConj{}}
              (\HOLBoundVar{a\sb{\mathrm{3}}} \HOLSymConst{=}
               \HOLConst{CFG} \HOLConst{STBY} (\HOLConst{Certs} \HOLBoundVar{STBYclist})
                 (\HOLConst{Policies} \HOLBoundVar{STBYplist})) \HOLSymConst{\HOLTokenConj{}}
              (\HOLBoundVar{M}\HOLSymConst{,}\HOLBoundVar{Oi}\HOLSymConst{,}\HOLBoundVar{Os}) \HOLConst{blocksat}
              \HOLConst{BLK} (\HOLConst{Messages} \HOLBoundVar{RDYmlist})
                (\HOLConst{CFG} \HOLConst{RDY} (\HOLConst{Certs} \HOLBoundVar{RDYclist}) (\HOLConst{Policies} \HOLBoundVar{RDYplist})) \HOLSymConst{\HOLTokenConj{}}
              (\HOLBoundVar{M}\HOLSymConst{,}\HOLBoundVar{Oi}\HOLSymConst{,}\HOLBoundVar{Os}) \HOLConst{sat} \HOLConst{prop} (\HOLConst{MODE} \HOLConst{STBY})) \HOLSymConst{\HOLTokenImp{}}
           \HOLBoundVar{TR\sb{\mathrm{1}}\sp{\prime}} \HOLBoundVar{a\sb{\mathrm{0}}} \HOLBoundVar{a\sb{\mathrm{1}}} \HOLBoundVar{a\sb{\mathrm{2}}} \HOLBoundVar{a\sb{\mathrm{3}}}) \HOLSymConst{\HOLTokenImp{}}
        \HOLBoundVar{TR\sb{\mathrm{1}}\sp{\prime}} \HOLBoundVar{a\sb{\mathrm{0}}} \HOLBoundVar{a\sb{\mathrm{1}}} \HOLBoundVar{a\sb{\mathrm{2}}} \HOLBoundVar{a\sb{\mathrm{3}}})
\end{SaveVerbatim}
\newcommand{\HOLsosOneDefinitionsTROneXXdef}{\UseVerbatim{HOLsosOneDefinitionsTROneXXdef}}
\newcommand{\HOLsosOneDefinitions}{
\HOLDfnTag{sos1}{TR1_def}\HOLsosOneDefinitionsTROneXXdef
}
\begin{SaveVerbatim}{HOLsosOneTheoremsabsOneXXdef}
\HOLTokenTurnstile{} \HOLConst{abs1} (\HOLConst{CFG} \HOLFreeVar{mode} (\HOLConst{Certs} \HOLFreeVar{clist}) (\HOLConst{Policies} \HOLFreeVar{plist})) \HOLSymConst{=} \HOLFreeVar{mode}
\end{SaveVerbatim}
\newcommand{\HOLsosOneTheoremsabsOneXXdef}{\UseVerbatim{HOLsosOneTheoremsabsOneXXdef}}
\begin{SaveVerbatim}{HOLsosOneTheoremsabsOneXXind}
\HOLTokenTurnstile{} \HOLSymConst{\HOLTokenForall{}}\HOLBoundVar{P}.
     (\HOLSymConst{\HOLTokenForall{}}\HOLBoundVar{mode} \HOLBoundVar{clist} \HOLBoundVar{plist}.
        \HOLBoundVar{P} (\HOLConst{CFG} \HOLBoundVar{mode} (\HOLConst{Certs} \HOLBoundVar{clist}) (\HOLConst{Policies} \HOLBoundVar{plist}))) \HOLSymConst{\HOLTokenImp{}}
     \HOLSymConst{\HOLTokenForall{}}\HOLBoundVar{v}. \HOLBoundVar{P} \HOLBoundVar{v}
\end{SaveVerbatim}
\newcommand{\HOLsosOneTheoremsabsOneXXind}{\UseVerbatim{HOLsosOneTheoremsabsOneXXind}}
\begin{SaveVerbatim}{HOLsosOneTheoremsRDYXXSTBYXXrule}
\HOLTokenTurnstile{} (\HOLFreeVar{M}\HOLSymConst{,}\HOLFreeVar{Oi}\HOLSymConst{,}\HOLFreeVar{Os}) \HOLConst{sat} \HOLConst{Name} \HOLConst{Owner} \HOLConst{says} \HOLConst{prop} (\HOLConst{CMD} \HOLConst{disable}) \HOLSymConst{\HOLTokenImp{}}
   (\HOLFreeVar{M}\HOLSymConst{,}\HOLFreeVar{Oi}\HOLSymConst{,}\HOLFreeVar{Os}) \HOLConst{sat} \HOLConst{Name} \HOLConst{Owner} \HOLConst{controls} \HOLConst{prop} (\HOLConst{CMD} \HOLConst{disable}) \HOLSymConst{\HOLTokenImp{}}
   (\HOLFreeVar{M}\HOLSymConst{,}\HOLFreeVar{Oi}\HOLSymConst{,}\HOLFreeVar{Os}) \HOLConst{sat} \HOLConst{prop} (\HOLConst{CMD} \HOLConst{disable}) \HOLConst{impf} \HOLConst{prop} (\HOLConst{MODE} \HOLConst{STBY}) \HOLSymConst{\HOLTokenImp{}}
   (\HOLFreeVar{M}\HOLSymConst{,}\HOLFreeVar{Oi}\HOLSymConst{,}\HOLFreeVar{Os}) \HOLConst{sat} \HOLConst{prop} (\HOLConst{MODE} \HOLConst{STBY})
\end{SaveVerbatim}
\newcommand{\HOLsosOneTheoremsRDYXXSTBYXXrule}{\UseVerbatim{HOLsosOneTheoremsRDYXXSTBYXXrule}}
\begin{SaveVerbatim}{HOLsosOneTheoremsRDYXXSTBYXXTRANSXXrule}
\HOLTokenTurnstile{} (\HOLFreeVar{M}\HOLSymConst{,}\HOLFreeVar{Oi}\HOLSymConst{,}\HOLFreeVar{Os}) \HOLConst{blocksat}
   \HOLConst{BLK} (\HOLConst{Messages} [\HOLConst{Name} \HOLConst{Owner} \HOLConst{says} \HOLConst{prop} (\HOLConst{CMD} \HOLConst{disable})])
     (\HOLConst{CFG} \HOLConst{RDY} (\HOLConst{Certs} [])
        (\HOLConst{Policies}
           [\HOLConst{Name} \HOLConst{Owner} \HOLConst{controls} \HOLConst{prop} (\HOLConst{CMD} \HOLConst{disable});
            \HOLConst{prop} (\HOLConst{CMD} \HOLConst{disable}) \HOLConst{impf} \HOLConst{prop} (\HOLConst{MODE} \HOLConst{STBY})])) \HOLSymConst{\HOLTokenImp{}}
   \HOLConst{TR1} (\HOLFreeVar{M}\HOLSymConst{,}\HOLFreeVar{Oi}\HOLSymConst{,}\HOLFreeVar{Os}) (\HOLConst{Messages} [\HOLConst{Name} \HOLConst{Owner} \HOLConst{says} \HOLConst{prop} (\HOLConst{CMD} \HOLConst{disable})])
     (\HOLConst{CFG} \HOLConst{RDY} (\HOLConst{Certs} [])
        (\HOLConst{Policies}
           [\HOLConst{Name} \HOLConst{Owner} \HOLConst{controls} \HOLConst{prop} (\HOLConst{CMD} \HOLConst{disable});
            \HOLConst{prop} (\HOLConst{CMD} \HOLConst{disable}) \HOLConst{impf} \HOLConst{prop} (\HOLConst{MODE} \HOLConst{STBY})]))
     (\HOLConst{CFG} \HOLConst{STBY} (\HOLConst{Certs} \HOLFreeVar{STBYclist}) (\HOLConst{Policies} \HOLFreeVar{STBYplist}))
\end{SaveVerbatim}
\newcommand{\HOLsosOneTheoremsRDYXXSTBYXXTRANSXXrule}{\UseVerbatim{HOLsosOneTheoremsRDYXXSTBYXXTRANSXXrule}}
\begin{SaveVerbatim}{HOLsosOneTheoremsSOneImplementsSZeroXXthm}
\HOLTokenTurnstile{} \HOLSymConst{\HOLTokenForall{}}\HOLBoundVar{M} \HOLBoundVar{Oi} \HOLBoundVar{Os} \HOLBoundVar{M\sb{\mathrm{1}}} \HOLBoundVar{M\sb{\mathrm{2}}} \HOLBoundVar{mlist} \HOLBoundVar{clist\sb{\mathrm{1}}} \HOLBoundVar{clist\sb{\mathrm{2}}} \HOLBoundVar{plist\sb{\mathrm{1}}} \HOLBoundVar{plist\sb{\mathrm{2}}}.
     \HOLConst{TR1} (\HOLBoundVar{M}\HOLSymConst{,}\HOLBoundVar{Oi}\HOLSymConst{,}\HOLBoundVar{Os}) (\HOLConst{Messages} \HOLBoundVar{mlist})
       (\HOLConst{CFG} \HOLBoundVar{M\sb{\mathrm{1}}} (\HOLConst{Certs} \HOLBoundVar{clist\sb{\mathrm{1}}}) (\HOLConst{Policies} \HOLBoundVar{plist\sb{\mathrm{1}}}))
       (\HOLConst{CFG} \HOLBoundVar{M\sb{\mathrm{2}}} (\HOLConst{Certs} \HOLBoundVar{clist\sb{\mathrm{2}}}) (\HOLConst{Policies} \HOLBoundVar{plist\sb{\mathrm{2}}})) \HOLSymConst{\HOLTokenImp{}}
     \HOLConst{TR0} (\HOLBoundVar{M}\HOLSymConst{,}\HOLBoundVar{Oi}\HOLSymConst{,}\HOLBoundVar{Os})
       (\HOLConst{abs1} (\HOLConst{CFG} \HOLBoundVar{M\sb{\mathrm{1}}} (\HOLConst{Certs} \HOLBoundVar{clist\sb{\mathrm{1}}}) (\HOLConst{Policies} \HOLBoundVar{plist\sb{\mathrm{1}}})))
       (\HOLConst{abs1} (\HOLConst{CFG} \HOLBoundVar{M\sb{\mathrm{2}}} (\HOLConst{Certs} \HOLBoundVar{clist\sb{\mathrm{2}}}) (\HOLConst{Policies} \HOLBoundVar{plist\sb{\mathrm{2}}}))) \HOLSymConst{\HOLTokenConj{}}
     (\HOLBoundVar{M}\HOLSymConst{,}\HOLBoundVar{Oi}\HOLSymConst{,}\HOLBoundVar{Os}) \HOLConst{sat}
     \HOLConst{prop}
       (\HOLConst{MODE} (\HOLConst{abs1} (\HOLConst{CFG} \HOLBoundVar{M\sb{\mathrm{2}}} (\HOLConst{Certs} \HOLBoundVar{clist\sb{\mathrm{2}}}) (\HOLConst{Policies} \HOLBoundVar{plist\sb{\mathrm{2}}}))))
\end{SaveVerbatim}
\newcommand{\HOLsosOneTheoremsSOneImplementsSZeroXXthm}{\UseVerbatim{HOLsosOneTheoremsSOneImplementsSZeroXXthm}}
\begin{SaveVerbatim}{HOLsosOneTheoremsSecuritySimulationXXthm}
\HOLTokenTurnstile{} \HOLSymConst{\HOLTokenForall{}}\HOLBoundVar{M} \HOLBoundVar{Oi} \HOLBoundVar{Os} \HOLBoundVar{M\sb{\mathrm{1}}} \HOLBoundVar{M\sb{\mathrm{2}}} \HOLBoundVar{mlist} \HOLBoundVar{clist\sb{\mathrm{1}}} \HOLBoundVar{clist\sb{\mathrm{2}}} \HOLBoundVar{plist\sb{\mathrm{1}}} \HOLBoundVar{plist\sb{\mathrm{2}}}.
     \HOLConst{TR1} (\HOLBoundVar{M}\HOLSymConst{,}\HOLBoundVar{Oi}\HOLSymConst{,}\HOLBoundVar{Os}) (\HOLConst{Messages} \HOLBoundVar{mlist})
       (\HOLConst{CFG} \HOLBoundVar{M\sb{\mathrm{1}}} (\HOLConst{Certs} \HOLBoundVar{clist\sb{\mathrm{1}}}) (\HOLConst{Policies} \HOLBoundVar{plist\sb{\mathrm{1}}}))
       (\HOLConst{CFG} \HOLBoundVar{M\sb{\mathrm{2}}} (\HOLConst{Certs} \HOLBoundVar{clist\sb{\mathrm{2}}}) (\HOLConst{Policies} \HOLBoundVar{plist\sb{\mathrm{2}}})) \HOLSymConst{\HOLTokenImp{}}
     \HOLConst{TR0} (\HOLBoundVar{M}\HOLSymConst{,}\HOLBoundVar{Oi}\HOLSymConst{,}\HOLBoundVar{Os}) \HOLBoundVar{M\sb{\mathrm{1}}} \HOLBoundVar{M\sb{\mathrm{2}}} \HOLSymConst{\HOLTokenConj{}} (\HOLBoundVar{M}\HOLSymConst{,}\HOLBoundVar{Oi}\HOLSymConst{,}\HOLBoundVar{Os}) \HOLConst{sat} \HOLConst{prop} (\HOLConst{MODE} \HOLBoundVar{M\sb{\mathrm{2}}})
\end{SaveVerbatim}
\newcommand{\HOLsosOneTheoremsSecuritySimulationXXthm}{\UseVerbatim{HOLsosOneTheoremsSecuritySimulationXXthm}}
\begin{SaveVerbatim}{HOLsosOneTheoremsSOSOneXXSecurityXXthm}
\HOLTokenTurnstile{} \HOLSymConst{\HOLTokenForall{}}\HOLBoundVar{M} \HOLBoundVar{Oi} \HOLBoundVar{Os} \HOLBoundVar{M\sb{\mathrm{1}}} \HOLBoundVar{M\sb{\mathrm{2}}} \HOLBoundVar{mlist} \HOLBoundVar{clist\sb{\mathrm{1}}} \HOLBoundVar{clist\sb{\mathrm{2}}} \HOLBoundVar{plist\sb{\mathrm{1}}} \HOLBoundVar{plist\sb{\mathrm{2}}}.
     \HOLConst{TR1} (\HOLBoundVar{M}\HOLSymConst{,}\HOLBoundVar{Oi}\HOLSymConst{,}\HOLBoundVar{Os}) (\HOLConst{Messages} \HOLBoundVar{mlist})
       (\HOLConst{CFG} \HOLBoundVar{M\sb{\mathrm{1}}} (\HOLConst{Certs} \HOLBoundVar{clist\sb{\mathrm{1}}}) (\HOLConst{Policies} \HOLBoundVar{plist\sb{\mathrm{1}}}))
       (\HOLConst{CFG} \HOLBoundVar{M\sb{\mathrm{2}}} (\HOLConst{Certs} \HOLBoundVar{clist\sb{\mathrm{2}}}) (\HOLConst{Policies} \HOLBoundVar{plist\sb{\mathrm{2}}})) \HOLSymConst{\HOLTokenImp{}}
     (\HOLBoundVar{M}\HOLSymConst{,}\HOLBoundVar{Oi}\HOLSymConst{,}\HOLBoundVar{Os}) \HOLConst{sat} \HOLConst{prop} (\HOLConst{MODE} \HOLBoundVar{M\sb{\mathrm{2}}})
\end{SaveVerbatim}
\newcommand{\HOLsosOneTheoremsSOSOneXXSecurityXXthm}{\UseVerbatim{HOLsosOneTheoremsSOSOneXXSecurityXXthm}}
\begin{SaveVerbatim}{HOLsosOneTheoremsSTBYXXRDYXXrule}
\HOLTokenTurnstile{} (\HOLFreeVar{M}\HOLSymConst{,}\HOLFreeVar{Oi}\HOLSymConst{,}\HOLFreeVar{Os}) \HOLConst{sat} \HOLConst{Name} \HOLConst{Owner} \HOLConst{says} \HOLConst{prop} (\HOLConst{CMD} \HOLConst{enable}) \HOLSymConst{\HOLTokenImp{}}
   (\HOLFreeVar{M}\HOLSymConst{,}\HOLFreeVar{Oi}\HOLSymConst{,}\HOLFreeVar{Os}) \HOLConst{sat} \HOLConst{Name} \HOLConst{Owner} \HOLConst{controls} \HOLConst{prop} (\HOLConst{CMD} \HOLConst{enable}) \HOLSymConst{\HOLTokenImp{}}
   (\HOLFreeVar{M}\HOLSymConst{,}\HOLFreeVar{Oi}\HOLSymConst{,}\HOLFreeVar{Os}) \HOLConst{sat} \HOLConst{prop} (\HOLConst{CMD} \HOLConst{enable}) \HOLConst{impf} \HOLConst{prop} (\HOLConst{MODE} \HOLConst{RDY}) \HOLSymConst{\HOLTokenImp{}}
   (\HOLFreeVar{M}\HOLSymConst{,}\HOLFreeVar{Oi}\HOLSymConst{,}\HOLFreeVar{Os}) \HOLConst{sat} \HOLConst{prop} (\HOLConst{MODE} \HOLConst{RDY})
\end{SaveVerbatim}
\newcommand{\HOLsosOneTheoremsSTBYXXRDYXXrule}{\UseVerbatim{HOLsosOneTheoremsSTBYXXRDYXXrule}}
\begin{SaveVerbatim}{HOLsosOneTheoremsSTBYXXRDYXXTRANSXXrule}
\HOLTokenTurnstile{} (\HOLFreeVar{M}\HOLSymConst{,}\HOLFreeVar{Oi}\HOLSymConst{,}\HOLFreeVar{Os}) \HOLConst{blocksat}
   \HOLConst{BLK} (\HOLConst{Messages} [\HOLConst{Name} \HOLConst{Owner} \HOLConst{says} \HOLConst{prop} (\HOLConst{CMD} \HOLConst{enable})])
     (\HOLConst{CFG} \HOLConst{STBY} (\HOLConst{Certs} [])
        (\HOLConst{Policies}
           [\HOLConst{Name} \HOLConst{Owner} \HOLConst{controls} \HOLConst{prop} (\HOLConst{CMD} \HOLConst{enable});
            \HOLConst{prop} (\HOLConst{CMD} \HOLConst{enable}) \HOLConst{impf} \HOLConst{prop} (\HOLConst{MODE} \HOLConst{RDY})])) \HOLSymConst{\HOLTokenImp{}}
   \HOLConst{TR1} (\HOLFreeVar{M}\HOLSymConst{,}\HOLFreeVar{Oi}\HOLSymConst{,}\HOLFreeVar{Os}) (\HOLConst{Messages} [\HOLConst{Name} \HOLConst{Owner} \HOLConst{says} \HOLConst{prop} (\HOLConst{CMD} \HOLConst{enable})])
     (\HOLConst{CFG} \HOLConst{STBY} (\HOLConst{Certs} [])
        (\HOLConst{Policies}
           [\HOLConst{Name} \HOLConst{Owner} \HOLConst{controls} \HOLConst{prop} (\HOLConst{CMD} \HOLConst{enable});
            \HOLConst{prop} (\HOLConst{CMD} \HOLConst{enable}) \HOLConst{impf} \HOLConst{prop} (\HOLConst{MODE} \HOLConst{RDY})]))
     (\HOLConst{CFG} \HOLConst{RDY} (\HOLConst{Certs} \HOLFreeVar{RDYclist}) (\HOLConst{Policies} \HOLFreeVar{RDYplist}))
\end{SaveVerbatim}
\newcommand{\HOLsosOneTheoremsSTBYXXRDYXXTRANSXXrule}{\UseVerbatim{HOLsosOneTheoremsSTBYXXRDYXXTRANSXXrule}}
\begin{SaveVerbatim}{HOLsosOneTheoremsTROneXXcases}
\HOLTokenTurnstile{} \HOLSymConst{\HOLTokenForall{}}\HOLBoundVar{a\sb{\mathrm{0}}} \HOLBoundVar{a\sb{\mathrm{1}}} \HOLBoundVar{a\sb{\mathrm{2}}} \HOLBoundVar{a\sb{\mathrm{3}}}.
     \HOLConst{TR1} \HOLBoundVar{a\sb{\mathrm{0}}} \HOLBoundVar{a\sb{\mathrm{1}}} \HOLBoundVar{a\sb{\mathrm{2}}} \HOLBoundVar{a\sb{\mathrm{3}}} \HOLSymConst{\HOLTokenEquiv{}}
     (\HOLSymConst{\HOLTokenExists{}}\HOLBoundVar{M} \HOLBoundVar{Oi} \HOLBoundVar{Os} \HOLBoundVar{RDYclist} \HOLBoundVar{RDYplist} \HOLBoundVar{STBYclist} \HOLBoundVar{STBYmlist} \HOLBoundVar{STBYplist}.
        (\HOLBoundVar{a\sb{\mathrm{0}}} \HOLSymConst{=} (\HOLBoundVar{M}\HOLSymConst{,}\HOLBoundVar{Oi}\HOLSymConst{,}\HOLBoundVar{Os})) \HOLSymConst{\HOLTokenConj{}} (\HOLBoundVar{a\sb{\mathrm{1}}} \HOLSymConst{=} \HOLConst{Messages} \HOLBoundVar{STBYmlist}) \HOLSymConst{\HOLTokenConj{}}
        (\HOLBoundVar{a\sb{\mathrm{2}}} \HOLSymConst{=} \HOLConst{CFG} \HOLConst{STBY} (\HOLConst{Certs} \HOLBoundVar{STBYclist}) (\HOLConst{Policies} \HOLBoundVar{STBYplist})) \HOLSymConst{\HOLTokenConj{}}
        (\HOLBoundVar{a\sb{\mathrm{3}}} \HOLSymConst{=} \HOLConst{CFG} \HOLConst{RDY} (\HOLConst{Certs} \HOLBoundVar{RDYclist}) (\HOLConst{Policies} \HOLBoundVar{RDYplist})) \HOLSymConst{\HOLTokenConj{}}
        (\HOLBoundVar{M}\HOLSymConst{,}\HOLBoundVar{Oi}\HOLSymConst{,}\HOLBoundVar{Os}) \HOLConst{blocksat}
        \HOLConst{BLK} (\HOLConst{Messages} \HOLBoundVar{STBYmlist})
          (\HOLConst{CFG} \HOLConst{STBY} (\HOLConst{Certs} \HOLBoundVar{STBYclist}) (\HOLConst{Policies} \HOLBoundVar{STBYplist})) \HOLSymConst{\HOLTokenConj{}}
        (\HOLBoundVar{M}\HOLSymConst{,}\HOLBoundVar{Oi}\HOLSymConst{,}\HOLBoundVar{Os}) \HOLConst{sat} \HOLConst{prop} (\HOLConst{MODE} \HOLConst{RDY})) \HOLSymConst{\HOLTokenDisj{}}
     \HOLSymConst{\HOLTokenExists{}}\HOLBoundVar{M} \HOLBoundVar{Oi} \HOLBoundVar{Os} \HOLBoundVar{RDYclist} \HOLBoundVar{RDYmlist} \HOLBoundVar{RDYplist} \HOLBoundVar{STBYclist} \HOLBoundVar{STBYplist}.
       (\HOLBoundVar{a\sb{\mathrm{0}}} \HOLSymConst{=} (\HOLBoundVar{M}\HOLSymConst{,}\HOLBoundVar{Oi}\HOLSymConst{,}\HOLBoundVar{Os})) \HOLSymConst{\HOLTokenConj{}} (\HOLBoundVar{a\sb{\mathrm{1}}} \HOLSymConst{=} \HOLConst{Messages} \HOLBoundVar{RDYmlist}) \HOLSymConst{\HOLTokenConj{}}
       (\HOLBoundVar{a\sb{\mathrm{2}}} \HOLSymConst{=} \HOLConst{CFG} \HOLConst{RDY} (\HOLConst{Certs} \HOLBoundVar{RDYclist}) (\HOLConst{Policies} \HOLBoundVar{RDYplist})) \HOLSymConst{\HOLTokenConj{}}
       (\HOLBoundVar{a\sb{\mathrm{3}}} \HOLSymConst{=} \HOLConst{CFG} \HOLConst{STBY} (\HOLConst{Certs} \HOLBoundVar{STBYclist}) (\HOLConst{Policies} \HOLBoundVar{STBYplist})) \HOLSymConst{\HOLTokenConj{}}
       (\HOLBoundVar{M}\HOLSymConst{,}\HOLBoundVar{Oi}\HOLSymConst{,}\HOLBoundVar{Os}) \HOLConst{blocksat}
       \HOLConst{BLK} (\HOLConst{Messages} \HOLBoundVar{RDYmlist})
         (\HOLConst{CFG} \HOLConst{RDY} (\HOLConst{Certs} \HOLBoundVar{RDYclist}) (\HOLConst{Policies} \HOLBoundVar{RDYplist})) \HOLSymConst{\HOLTokenConj{}}
       (\HOLBoundVar{M}\HOLSymConst{,}\HOLBoundVar{Oi}\HOLSymConst{,}\HOLBoundVar{Os}) \HOLConst{sat} \HOLConst{prop} (\HOLConst{MODE} \HOLConst{STBY})
\end{SaveVerbatim}
\newcommand{\HOLsosOneTheoremsTROneXXcases}{\UseVerbatim{HOLsosOneTheoremsTROneXXcases}}
\begin{SaveVerbatim}{HOLsosOneTheoremsTROneXXind}
\HOLTokenTurnstile{} \HOLSymConst{\HOLTokenForall{}}\HOLBoundVar{TR\sb{\mathrm{1}}\sp{\prime}}.
     (\HOLSymConst{\HOLTokenForall{}}\HOLBoundVar{M} \HOLBoundVar{Oi} \HOLBoundVar{Os} \HOLBoundVar{RDYclist} \HOLBoundVar{RDYplist} \HOLBoundVar{STBYclist} \HOLBoundVar{STBYmlist} \HOLBoundVar{STBYplist}.
        (\HOLBoundVar{M}\HOLSymConst{,}\HOLBoundVar{Oi}\HOLSymConst{,}\HOLBoundVar{Os}) \HOLConst{blocksat}
        \HOLConst{BLK} (\HOLConst{Messages} \HOLBoundVar{STBYmlist})
          (\HOLConst{CFG} \HOLConst{STBY} (\HOLConst{Certs} \HOLBoundVar{STBYclist}) (\HOLConst{Policies} \HOLBoundVar{STBYplist})) \HOLSymConst{\HOLTokenConj{}}
        (\HOLBoundVar{M}\HOLSymConst{,}\HOLBoundVar{Oi}\HOLSymConst{,}\HOLBoundVar{Os}) \HOLConst{sat} \HOLConst{prop} (\HOLConst{MODE} \HOLConst{RDY}) \HOLSymConst{\HOLTokenImp{}}
        \HOLBoundVar{TR\sb{\mathrm{1}}\sp{\prime}} (\HOLBoundVar{M}\HOLSymConst{,}\HOLBoundVar{Oi}\HOLSymConst{,}\HOLBoundVar{Os}) (\HOLConst{Messages} \HOLBoundVar{STBYmlist})
          (\HOLConst{CFG} \HOLConst{STBY} (\HOLConst{Certs} \HOLBoundVar{STBYclist}) (\HOLConst{Policies} \HOLBoundVar{STBYplist}))
          (\HOLConst{CFG} \HOLConst{RDY} (\HOLConst{Certs} \HOLBoundVar{RDYclist}) (\HOLConst{Policies} \HOLBoundVar{RDYplist}))) \HOLSymConst{\HOLTokenConj{}}
     (\HOLSymConst{\HOLTokenForall{}}\HOLBoundVar{M} \HOLBoundVar{Oi} \HOLBoundVar{Os} \HOLBoundVar{RDYclist} \HOLBoundVar{RDYmlist} \HOLBoundVar{RDYplist} \HOLBoundVar{STBYclist} \HOLBoundVar{STBYplist}.
        (\HOLBoundVar{M}\HOLSymConst{,}\HOLBoundVar{Oi}\HOLSymConst{,}\HOLBoundVar{Os}) \HOLConst{blocksat}
        \HOLConst{BLK} (\HOLConst{Messages} \HOLBoundVar{RDYmlist})
          (\HOLConst{CFG} \HOLConst{RDY} (\HOLConst{Certs} \HOLBoundVar{RDYclist}) (\HOLConst{Policies} \HOLBoundVar{RDYplist})) \HOLSymConst{\HOLTokenConj{}}
        (\HOLBoundVar{M}\HOLSymConst{,}\HOLBoundVar{Oi}\HOLSymConst{,}\HOLBoundVar{Os}) \HOLConst{sat} \HOLConst{prop} (\HOLConst{MODE} \HOLConst{STBY}) \HOLSymConst{\HOLTokenImp{}}
        \HOLBoundVar{TR\sb{\mathrm{1}}\sp{\prime}} (\HOLBoundVar{M}\HOLSymConst{,}\HOLBoundVar{Oi}\HOLSymConst{,}\HOLBoundVar{Os}) (\HOLConst{Messages} \HOLBoundVar{RDYmlist})
          (\HOLConst{CFG} \HOLConst{RDY} (\HOLConst{Certs} \HOLBoundVar{RDYclist}) (\HOLConst{Policies} \HOLBoundVar{RDYplist}))
          (\HOLConst{CFG} \HOLConst{STBY} (\HOLConst{Certs} \HOLBoundVar{STBYclist}) (\HOLConst{Policies} \HOLBoundVar{STBYplist}))) \HOLSymConst{\HOLTokenImp{}}
     \HOLSymConst{\HOLTokenForall{}}\HOLBoundVar{a\sb{\mathrm{0}}} \HOLBoundVar{a\sb{\mathrm{1}}} \HOLBoundVar{a\sb{\mathrm{2}}} \HOLBoundVar{a\sb{\mathrm{3}}}. \HOLConst{TR1} \HOLBoundVar{a\sb{\mathrm{0}}} \HOLBoundVar{a\sb{\mathrm{1}}} \HOLBoundVar{a\sb{\mathrm{2}}} \HOLBoundVar{a\sb{\mathrm{3}}} \HOLSymConst{\HOLTokenImp{}} \HOLBoundVar{TR\sb{\mathrm{1}}\sp{\prime}} \HOLBoundVar{a\sb{\mathrm{0}}} \HOLBoundVar{a\sb{\mathrm{1}}} \HOLBoundVar{a\sb{\mathrm{2}}} \HOLBoundVar{a\sb{\mathrm{3}}}
\end{SaveVerbatim}
\newcommand{\HOLsosOneTheoremsTROneXXind}{\UseVerbatim{HOLsosOneTheoremsTROneXXind}}
\begin{SaveVerbatim}{HOLsosOneTheoremsTROneXXrules}
\HOLTokenTurnstile{} (\HOLSymConst{\HOLTokenForall{}}\HOLBoundVar{M} \HOLBoundVar{Oi} \HOLBoundVar{Os} \HOLBoundVar{RDYclist} \HOLBoundVar{RDYplist} \HOLBoundVar{STBYclist} \HOLBoundVar{STBYmlist} \HOLBoundVar{STBYplist}.
      (\HOLBoundVar{M}\HOLSymConst{,}\HOLBoundVar{Oi}\HOLSymConst{,}\HOLBoundVar{Os}) \HOLConst{blocksat}
      \HOLConst{BLK} (\HOLConst{Messages} \HOLBoundVar{STBYmlist})
        (\HOLConst{CFG} \HOLConst{STBY} (\HOLConst{Certs} \HOLBoundVar{STBYclist}) (\HOLConst{Policies} \HOLBoundVar{STBYplist})) \HOLSymConst{\HOLTokenConj{}}
      (\HOLBoundVar{M}\HOLSymConst{,}\HOLBoundVar{Oi}\HOLSymConst{,}\HOLBoundVar{Os}) \HOLConst{sat} \HOLConst{prop} (\HOLConst{MODE} \HOLConst{RDY}) \HOLSymConst{\HOLTokenImp{}}
      \HOLConst{TR1} (\HOLBoundVar{M}\HOLSymConst{,}\HOLBoundVar{Oi}\HOLSymConst{,}\HOLBoundVar{Os}) (\HOLConst{Messages} \HOLBoundVar{STBYmlist})
        (\HOLConst{CFG} \HOLConst{STBY} (\HOLConst{Certs} \HOLBoundVar{STBYclist}) (\HOLConst{Policies} \HOLBoundVar{STBYplist}))
        (\HOLConst{CFG} \HOLConst{RDY} (\HOLConst{Certs} \HOLBoundVar{RDYclist}) (\HOLConst{Policies} \HOLBoundVar{RDYplist}))) \HOLSymConst{\HOLTokenConj{}}
   \HOLSymConst{\HOLTokenForall{}}\HOLBoundVar{M} \HOLBoundVar{Oi} \HOLBoundVar{Os} \HOLBoundVar{RDYclist} \HOLBoundVar{RDYmlist} \HOLBoundVar{RDYplist} \HOLBoundVar{STBYclist} \HOLBoundVar{STBYplist}.
     (\HOLBoundVar{M}\HOLSymConst{,}\HOLBoundVar{Oi}\HOLSymConst{,}\HOLBoundVar{Os}) \HOLConst{blocksat}
     \HOLConst{BLK} (\HOLConst{Messages} \HOLBoundVar{RDYmlist})
       (\HOLConst{CFG} \HOLConst{RDY} (\HOLConst{Certs} \HOLBoundVar{RDYclist}) (\HOLConst{Policies} \HOLBoundVar{RDYplist})) \HOLSymConst{\HOLTokenConj{}}
     (\HOLBoundVar{M}\HOLSymConst{,}\HOLBoundVar{Oi}\HOLSymConst{,}\HOLBoundVar{Os}) \HOLConst{sat} \HOLConst{prop} (\HOLConst{MODE} \HOLConst{STBY}) \HOLSymConst{\HOLTokenImp{}}
     \HOLConst{TR1} (\HOLBoundVar{M}\HOLSymConst{,}\HOLBoundVar{Oi}\HOLSymConst{,}\HOLBoundVar{Os}) (\HOLConst{Messages} \HOLBoundVar{RDYmlist})
       (\HOLConst{CFG} \HOLConst{RDY} (\HOLConst{Certs} \HOLBoundVar{RDYclist}) (\HOLConst{Policies} \HOLBoundVar{RDYplist}))
       (\HOLConst{CFG} \HOLConst{STBY} (\HOLConst{Certs} \HOLBoundVar{STBYclist}) (\HOLConst{Policies} \HOLBoundVar{STBYplist}))
\end{SaveVerbatim}
\newcommand{\HOLsosOneTheoremsTROneXXrules}{\UseVerbatim{HOLsosOneTheoremsTROneXXrules}}
\begin{SaveVerbatim}{HOLsosOneTheoremsTROneXXstrongind}
\HOLTokenTurnstile{} \HOLSymConst{\HOLTokenForall{}}\HOLBoundVar{TR\sb{\mathrm{1}}\sp{\prime}}.
     (\HOLSymConst{\HOLTokenForall{}}\HOLBoundVar{M} \HOLBoundVar{Oi} \HOLBoundVar{Os} \HOLBoundVar{RDYclist} \HOLBoundVar{RDYplist} \HOLBoundVar{STBYclist} \HOLBoundVar{STBYmlist} \HOLBoundVar{STBYplist}.
        (\HOLBoundVar{M}\HOLSymConst{,}\HOLBoundVar{Oi}\HOLSymConst{,}\HOLBoundVar{Os}) \HOLConst{blocksat}
        \HOLConst{BLK} (\HOLConst{Messages} \HOLBoundVar{STBYmlist})
          (\HOLConst{CFG} \HOLConst{STBY} (\HOLConst{Certs} \HOLBoundVar{STBYclist}) (\HOLConst{Policies} \HOLBoundVar{STBYplist})) \HOLSymConst{\HOLTokenConj{}}
        (\HOLBoundVar{M}\HOLSymConst{,}\HOLBoundVar{Oi}\HOLSymConst{,}\HOLBoundVar{Os}) \HOLConst{sat} \HOLConst{prop} (\HOLConst{MODE} \HOLConst{RDY}) \HOLSymConst{\HOLTokenImp{}}
        \HOLBoundVar{TR\sb{\mathrm{1}}\sp{\prime}} (\HOLBoundVar{M}\HOLSymConst{,}\HOLBoundVar{Oi}\HOLSymConst{,}\HOLBoundVar{Os}) (\HOLConst{Messages} \HOLBoundVar{STBYmlist})
          (\HOLConst{CFG} \HOLConst{STBY} (\HOLConst{Certs} \HOLBoundVar{STBYclist}) (\HOLConst{Policies} \HOLBoundVar{STBYplist}))
          (\HOLConst{CFG} \HOLConst{RDY} (\HOLConst{Certs} \HOLBoundVar{RDYclist}) (\HOLConst{Policies} \HOLBoundVar{RDYplist}))) \HOLSymConst{\HOLTokenConj{}}
     (\HOLSymConst{\HOLTokenForall{}}\HOLBoundVar{M} \HOLBoundVar{Oi} \HOLBoundVar{Os} \HOLBoundVar{RDYclist} \HOLBoundVar{RDYmlist} \HOLBoundVar{RDYplist} \HOLBoundVar{STBYclist} \HOLBoundVar{STBYplist}.
        (\HOLBoundVar{M}\HOLSymConst{,}\HOLBoundVar{Oi}\HOLSymConst{,}\HOLBoundVar{Os}) \HOLConst{blocksat}
        \HOLConst{BLK} (\HOLConst{Messages} \HOLBoundVar{RDYmlist})
          (\HOLConst{CFG} \HOLConst{RDY} (\HOLConst{Certs} \HOLBoundVar{RDYclist}) (\HOLConst{Policies} \HOLBoundVar{RDYplist})) \HOLSymConst{\HOLTokenConj{}}
        (\HOLBoundVar{M}\HOLSymConst{,}\HOLBoundVar{Oi}\HOLSymConst{,}\HOLBoundVar{Os}) \HOLConst{sat} \HOLConst{prop} (\HOLConst{MODE} \HOLConst{STBY}) \HOLSymConst{\HOLTokenImp{}}
        \HOLBoundVar{TR\sb{\mathrm{1}}\sp{\prime}} (\HOLBoundVar{M}\HOLSymConst{,}\HOLBoundVar{Oi}\HOLSymConst{,}\HOLBoundVar{Os}) (\HOLConst{Messages} \HOLBoundVar{RDYmlist})
          (\HOLConst{CFG} \HOLConst{RDY} (\HOLConst{Certs} \HOLBoundVar{RDYclist}) (\HOLConst{Policies} \HOLBoundVar{RDYplist}))
          (\HOLConst{CFG} \HOLConst{STBY} (\HOLConst{Certs} \HOLBoundVar{STBYclist}) (\HOLConst{Policies} \HOLBoundVar{STBYplist}))) \HOLSymConst{\HOLTokenImp{}}
     \HOLSymConst{\HOLTokenForall{}}\HOLBoundVar{a\sb{\mathrm{0}}} \HOLBoundVar{a\sb{\mathrm{1}}} \HOLBoundVar{a\sb{\mathrm{2}}} \HOLBoundVar{a\sb{\mathrm{3}}}. \HOLConst{TR1} \HOLBoundVar{a\sb{\mathrm{0}}} \HOLBoundVar{a\sb{\mathrm{1}}} \HOLBoundVar{a\sb{\mathrm{2}}} \HOLBoundVar{a\sb{\mathrm{3}}} \HOLSymConst{\HOLTokenImp{}} \HOLBoundVar{TR\sb{\mathrm{1}}\sp{\prime}} \HOLBoundVar{a\sb{\mathrm{0}}} \HOLBoundVar{a\sb{\mathrm{1}}} \HOLBoundVar{a\sb{\mathrm{2}}} \HOLBoundVar{a\sb{\mathrm{3}}}
\end{SaveVerbatim}
\newcommand{\HOLsosOneTheoremsTROneXXstrongind}{\UseVerbatim{HOLsosOneTheoremsTROneXXstrongind}}
\newcommand{\HOLsosOneTheorems}{
\HOLThmTag{sos1}{abs1_def}\HOLsosOneTheoremsabsOneXXdef
\HOLThmTag{sos1}{abs1_ind}\HOLsosOneTheoremsabsOneXXind
\HOLThmTag{sos1}{RDY_STBY_rule}\HOLsosOneTheoremsRDYXXSTBYXXrule
\HOLThmTag{sos1}{RDY_STBY_TRANS_rule}\HOLsosOneTheoremsRDYXXSTBYXXTRANSXXrule
\HOLThmTag{sos1}{S1ImplementsS0_thm}\HOLsosOneTheoremsSOneImplementsSZeroXXthm
\HOLThmTag{sos1}{SecuritySimulation_thm}\HOLsosOneTheoremsSecuritySimulationXXthm
\HOLThmTag{sos1}{SOS1_Security_thm}\HOLsosOneTheoremsSOSOneXXSecurityXXthm
\HOLThmTag{sos1}{STBY_RDY_rule}\HOLsosOneTheoremsSTBYXXRDYXXrule
\HOLThmTag{sos1}{STBY_RDY_TRANS_rule}\HOLsosOneTheoremsSTBYXXRDYXXTRANSXXrule
\HOLThmTag{sos1}{TR1_cases}\HOLsosOneTheoremsTROneXXcases
\HOLThmTag{sos1}{TR1_ind}\HOLsosOneTheoremsTROneXXind
\HOLThmTag{sos1}{TR1_rules}\HOLsosOneTheoremsTROneXXrules
\HOLThmTag{sos1}{TR1_strongind}\HOLsosOneTheoremsTROneXXstrongind
}


\maketitle
\thispagestyle{empty}
\author{}
\maketitle

\begin{abstract}
  Describing how systems change their behavior is required at all
  levels of abstraction, from simple finite state machines in hardware
  to abstract concepts of operations for accomplishing missions. These
  descriptions entail the definition of transition relations, the
  conditions under which transitions occur, and proofs of crucial
  system properties that must be assured for all system states and
  operating modes. For systems where integrity and security properties
  are important, we can use a combination of structural operational
  semantics and access-control logic to reason about a system's
  behavior and security properties.
\end{abstract}

\section{Introduction}
\label{sec:introduction}

\section{Background}
\label{sec:background}

\section{Rigorous Representations of Transition Systems and Concepts of Operations}
\label{sec:conops}

\section{Formal Description and Verification in HOL}
\label{sec:hol}

\section{Conclusions}
\label{sec:conclusions}



\bibliography{references}
\bibliographystyle{alpha}

\newpage{}
\part*{Appendices}
\label{part:appendices}

% ::::::::::::::::::::::::::::::::::::::::::::::::::::::::::::::::::::::::::
\section{foundation Theory}
\index{foundation Theory@\textbf  {foundation Theory}}
\begin{flushleft}
\textbf{Built:} \HOLfoundationDate \\[2pt]
\textbf{Parent Theories:} aclDrules
\end{flushleft}
% ::::::::::::::::::::::::::::::::::::::::::::::::::::::::::::::::::::::::::

\subsection{Datatypes}
\index{foundation Theory@\textbf  {foundation Theory}!Datatypes}
% .....................................

\HOLfoundationDatatypes

\index{foundation Theory@\textbf  {foundation Theory}!Definitions}
% .....................................

\subsection{Definitions}
\index{foundation Theory@\textbf  {foundation Theory}!Definitions}
% .....................................

\HOLfoundationDefinitions

% No theorems

% ::::::::::::::::::::::::::::::::::::::::::::::::::::::::::::::::::::::::::
\section{sos0 Theory}
\index{sos0 Theory@\textbf  {sos0 Theory}}
\begin{flushleft}
\textbf{Built:} \HOLsosZeroDate \\[2pt]
\textbf{Parent Theories:} foundation
\end{flushleft}
% ::::::::::::::::::::::::::::::::::::::::::::::::::::::::::::::::::::::::::

% No datatypes

\subsection{Definitions}
\index{sos0 Theory@\textbf  {sos0 Theory}!Definitions}
% .....................................

\HOLsosZeroDefinitions

\subsection{Theorems}
\index{sos0 Theory@\textbf  {sos0 Theory}!Theorems}
% .....................................

\HOLsosZeroTheorems

% ::::::::::::::::::::::::::::::::::::::::::::::::::::::::::::::::::::::::::
\section{sos1 Theory}
\index{sos1 Theory@\textbf  {sos1 Theory}}
\begin{flushleft}
\textbf{Built:} \HOLsosOneDate \\[2pt]
\textbf{Parent Theories:} rich_list, sos0
\end{flushleft}
% ::::::::::::::::::::::::::::::::::::::::::::::::::::::::::::::::::::::::::

% No datatypes

\subsection{Definitions}
\index{sos1 Theory@\textbf  {sos1 Theory}!Definitions}
% .....................................

\HOLsosOneDefinitions

\subsection{Theorems}
\index{sos1 Theory@\textbf  {sos1 Theory}!Theorems}
% .....................................

\HOLsosOneTheorems

\section{foundationScript.sml}
\label{sec:foundationScript}

\lstinputlisting{../foundationScript.sml}

\section{sos0Script.sml}
\label{sec:sos0Script}

\lstinputlisting{../sos0Script.sml}

\section{sos1Script.sml}
\label{sec:sos1Script}

\lstinputlisting{../sos1Script.sml}
\end{document}

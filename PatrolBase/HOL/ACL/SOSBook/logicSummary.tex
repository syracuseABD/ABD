\chapter{Summary of the Access-Control Logic}
\label{cha:logic-summary}

This appendix provides a summary of the syntax and inference
rules of the access-control logic.  It is intended as a canonical
reference for the syntax and rules (both core and derived) introduced
throughout the book.

\section{Syntax}
We define \PName to be the collection of all simple principal names.
The set \PExp of all principal expressions is given by the following BNF
specification:
\begin{eqnarray*}
  \PExp &\isa& \PName \ora  \PExp \with \PExp \ora \PExp \quoting \PExp
\end{eqnarray*}
The convention for compound principals is that \with binds more tightly
than \quoting.

We let \PropVar be the collection of all propositional
variables.  The set \LExp of all well-formed expressions is given by
the following BNF specification:
\begin{eqnarray*}
  \LExp &::=&  \PropVar \ora
               \neg\ \LExp \ora
               (\LExp \vee \LExp) \ora \\
               & & 
               (\LExp \wedge \LExp) \ora 
               (\LExp \implies \LExp) \ora 
               (\LExp \equiv \LExp) \ora \\
               & & 
               (\PExp \speaksfor \PExp) \ora
               (\PExp \says \LExp) \ora
               (\PExp \controls \LExp) \\
               & & 
               (\reps{\PExp}{\PExp}{\varphi}) 
\end{eqnarray*}
Parentheses can be omitted according to the following conventions for
operator precedence, in decreasing tightness of bindings:
\begin{eqnarray*}
  & & \neg \\
  & & \says \quad \controls \quad \rreps \\
  & & \wedge \\
  & & \vee \\
  & & \implies \\
  & & \equiv \\
  & & 
\end{eqnarray*}

The definition of \LExp defines the core syntax of the logic. Three
extensions are made to describe confidentiality, integrity, and
role-based access control policies.

\paragraph*{Confidentiality}

We define \LabelConst to be the collection of \emph{simple security
  labels}, which are used as names for the various levels associated
with confidentiality. In addition to these specific security labels,
we will often want to refer abstractly to the security level assigned
to a particular principal $P$.  For this reason, we define the larger
set \Level of \emph{all} possible security-level expressions:
\begin{eqnarray*}
  \Level &\isa& \LabelConst \ora  \slv{\PName}
\end{eqnarray*}
That is, a security-level expression is either a simple security label
or an expression of the form $\slv{A}$, where $A$ is a simple principal
name.\footnote{This syntax precludes security-level expressions such as
  $\slv{P\with Q}$ or $\slv{P\quoting Q}$, because there is no standard
  technique for associating security classification labels with compound
  principals.} Informally, $\slv{A}$ refers to the security level of
principal $A$.

Finally, we extend our definition of well-formed formulas to support
comparisons of security levels:
\begin{eqnarray*}
    \LExp &::=&  \Level \leq_s \Level \ora \Level =_s \Level 
\end{eqnarray*}

\paragraph*{Integrity}

We define \IntLabelConst to be the collection of \emph{simple
  integrity labels}, and we define \IntLevel to be the set of
\emph{all} possible integrity-level expressions:
\begin{eqnarray*}
  \IntLevel &\isa& \IntLabelConst \ora  \ilv{\PName}
\end{eqnarray*}
Informally, $\ilv{A}$ refers to the integrity---i.e., quality or
trustworthiness---level of principal $A$.

We then extend our definition of well-formed formulas to support
comparisons of security levels:
\begin{eqnarray*}
    \LExp &::=&  \IntLevel \leq_i \IntLevel \ora \IntLevel =_i \IntLevel 
\end{eqnarray*}
The symbol $\leq_i$ denotes a partial ordering on integrity levels, in
the same way that $\leq_s$ denotes a partial ordering on security
levels.  In particular, $\leq_i$ is reflexive, transitive, and
antisymmetric.

\paragraph*{Role-based access control}

We extend the syntax of the logic to accommodate statements that
express equality among principals:
\[ \LExp ::=  (\PExp = \PExp) \]

%\section{Semantics}
\section{Core Rules, Derived Rules, and Extensions}

The following figures summarize the core rules, derived rules, and
extensions for the access-control logic:
\begin{itemize}
\item Figure~\ref{fig:summary-core-rules} is a summary of the core
  inference rules.
\item Figure~\ref{fig:summary-derived-rules} is a summary of
  frequently used derived inference rules.
\item Figure~\ref{fig:summary-delegation-rules} is a summary of
  inference rules for delegation.
\item Figure~\ref{fig:summary-security-level-rules} is a summary of
  inference rules for relating security levels.
\item Figure~\ref{fig:summary-integrity-level-rules} is a summary of
  inference rules for relating integrity levels.
\item Figure~\ref{fig:summary-principal-equality-rules} is a summary
  of inference rules regarding principal equality.
\end{itemize}


\begin{figure}[tbp]
  \begin{center}
    \ \rulespace

    \irule{}{\qquad\varphi\qquad}{Taut} $\qquad$ \mbox{if $\varphi$ is an 
      instance of a prop-logic tautology}    \rulespace

    \irule{\varphi \qquad  \varphi \implies \varphi'}{\varphi'}{Modus\
      Ponens}
%    \rulespace
    \qquad
    \irule{\varphi}{P \says \varphi}{Says}
%  \quad  \mbox{(for    all $P$)} 
    \rulespace

    \irule{}{(P \says (\varphi \implies \varphi')) \implies
      (P \says \varphi \implies P \says \varphi')}{MP\ Says} \rulespace

    \irule{}{P \speaksfor Q   \implies
      (P \says \varphi \implies Q \says \varphi)}{Speaks For} \rulespace

    \irule{}{(P \with Q \says \varphi) \equiv   ((P \says \varphi) \wedge
      (Q\says \varphi))}{\with Says}
    \rulespace

    \irule{}{(P \quoting Q \says \varphi) \equiv   (P \says Q\says
      \varphi)}{Quoting} 
    \rulespace

    \irule{}{P \speaksfor P}{Idempotency of $\speaksfor$} 
    \rulespace

    \irule{P\speaksfor Q\qquad Q \speaksfor R}{P\speaksfor
      R}{\parbox[c]{12ex}{\centering Transitivity\\ of $\speaksfor$}} 
    \qquad
%    \rulespace 
%
    \irule{P\speaksfor P' \quad Q \speaksfor Q'}{P\quoting Q \speaksfor
      P'\quoting Q'}{\parbox[c]{13ex}{\centering Monotonicity \\ of $\speaksfor$}}
    \rulespace

    \irule{\varphi_1\equiv \varphi_2 \qquad \subst{\psi}{\varphi_1}{q}}
    {\subst{\psi}{\varphi_2}{q}}{\infname{Equivalence}}
    \rulespace
  
    $P \controls \varphi \quad \defn \quad (P \says \varphi) \implies
    \varphi$
    \rulespace
    \caption{Summary of core rules for the access-control logic}
    \label{fig:summary-core-rules}
  \end{center}
\end{figure}

% \section{Derived Rules and Extensions}

\begin{figure}[tbp]
  \centering

    \irule{\varphi_1 \qquad \varphi_2}{\varphi_1 \wedge
      \varphi_2}{Conjunction} \rulespace 

    \irule{\varphi_1 \wedge \varphi_2}{\varphi_1}{Simplification (1)} 
    \hspace*{2em}
    \irule{\varphi_1 \wedge \varphi_2}{\varphi_2}{Simplification (2)} 
    \rulespace

    \irule{\varphi_1}{\varphi_1 \vee \varphi_2}{Disjunction (1)} 
    \hspace*{2em}
    \irule{\varphi_2}{\varphi_1 \vee \varphi_2}{Disjunction (2)} 
    \rulespace
   
    \irule{\varphi_1 \implies \varphi_2 \qquad \neg \varphi_2}{\neg
      \varphi_1}{Modus\ Tollens}
%    \hspace*{2em}
%     \irule{\varphi_1 \equiv \varphi_2 \qquad \varphi_1}
%     {\varphi_2}{Equivalence} 
     \rulespace
%
    \irule{\neg \neg \varphi}{\varphi}{Double\ negation}
    \rulespace

    \irule{\varphi_1 \vee \varphi_2 \qquad \neg
      \varphi_1}{\varphi_2}{\parbox[c]{11ex}{\centering Disjunctive\\ Syllogism}}
%    \rulespace
    \qquad
    \irule{\varphi_1 \implies \varphi_2 \qquad \varphi_2 \implies
      \varphi_3}{\varphi_1 \implies
      \varphi_3}{\parbox[c]{12ex}{\centering Hypothetical\\ Syllogism}} 
    \rulespace
  
    \irule{P \controls \varphi \quad P \says \varphi}{\varphi}{Controls}
    \rulespace

    \irule{P \speaksfor Q \quad P\says \varphi}{Q \says
      \varphi}{\parbox[c]{11ex}{\centering Derived \\ Speaks For}}
    \qquad
    \irule{P \speaksfor Q \quad Q \controls \varphi}{P \controls
      \varphi}{\parbox[c]{9ex}{\centering Derived Controls}}
    \rulespace

    \irule{P \says (\varphi_1 \wedge \varphi_2)}{P \says
      \varphi_1}{\parbox[c]{18ex}{\centering Says \\ Simplification (1)}}
    \quad 
    \irule{P \says (\varphi_1 \wedge \varphi_2)}{P \says \varphi_2}{\parbox[c]{18ex}{\centering Says \\ Simplification (2)}}
  \caption{Summary of useful derived rules}
  \label{fig:summary-derived-rules}
\end{figure}

% \begin{figure}
%   \centering

%   \infrule{\begin{array}[b]{c}
%       \textit{subject} \says \varphi \quad
%       \textit{authority} \controls (\textit{subject} \controls \varphi) \\
%       \textit{ticket} \speaksfor \textit{authority} \quad
%       \textit{ticket} \says (\textit{subject} \controls \varphi)
%     \end{array}}{\varphi}{Ticket} \rulespace

%   \infrule
%   {
%     \begin{matrix}
%       (\textrm{factor}_1 \with \textrm{factor}_2) \says \varphi \\
%       \textrm{authority}_1 \controls (\textrm{subject} \controls
%       \varphi) \\
%       \textrm{authority}_1 \says (\textrm{subject} \controls \varphi) \\
%       \textrm{authority}_2 \controls ((\textrm{factor}_1 \with
%       \textrm{factor}_2) \speaksfor \textrm{subject}) \\
%       \textrm{authority}_2 \says ((\textrm{factor}_1 \with \textrm{factor}_2)
%     \speaksfor \textrm{subject}) \rulespace
%     \end{matrix}
%     }
%     {\varphi}
%     {Two-Factor ACL}

%     \infrule{\begin{matrix}
%          \textrm{factor} \says \varphi_1\\
%          A_1 \controls ((\textrm{subject} \controls \varphi_2) \implies
%          \textrm{subject} \controls \varphi_1) \\
%          A_1 \says ((\textrm{subject} \controls \varphi_2) \implies
%          \textrm{subject} \controls \varphi_1) \\
%          A_2 \controls (\textrm{factor} \speaksfor \textrm{subject}) \\
%          \textrm{certificate} \says (\textrm{factor} \speaksfor
%          \textrm{subject}) \\
%          \textrm{certificate} \speaksfor A_2 \\
%          A_3 \controls (\textrm{subject} \controls \varphi_2) \\
%          \textrm{ticket} \says (\textrm{subject} \controls \varphi_2) \\
%          \textrm{ticket} \speaksfor A_3 \\
%        \end{matrix}} 
%      {\varphi_1} {ID \& Ticket} \rulespace

%   \caption{Summary of derived rules for basic access-control concepts}
  
% \end{figure}

% \begin{figure}
%   \centering
%   \infrule{\begin{array}[b]{c}
%       K_{ca} \speaksfor \name{Certificate Authority} \qquad K_{ca}
%       \says  (K_P \speaksfor P)  \\
%       \name{Certificate Authority} \controls (K_P \speaksfor P) 
%     \end{array}}{K_P \speaksfor P}{\parbox[c]{.8in}{Certificate Verification}}

  
%   \caption{Summary of derived rules for digital authentication}
%   \label{fig:dig-auth-rules}
% \end{figure}

\begin{figure}
  \centering

  $\reps{P}{Q}{\varphi} \defn P \quoting Q \says \varphi \implies Q
  \says \varphi$

  \rulespace

  $\irule{Q \controls \varphi \quad \reps{P}{Q}{\varphi} \quad P
    \quoting Q \says \varphi}{\varphi}{Reps}$

  \rulespace

  $\irule{}{\reps{A}{B}{\varphi} \equiv (A \controls (B \says
    \varphi))}{Rep Controls}$

  \rulespace

  $\irule{\begin{array}[b]{c} \reps{A}{B}{\varphi} \quad A | B \says
      \varphi
    \end{array}} {B \says \varphi} {Rep Says}$

  \rulespace

%   \irule
%   {
%     \begin{array}[b]{c}
%       Signature _Q \quoting \sig{P} \says (\langle \textit{pay
%         \emph{amt}}, Q \rangle \wedge \langle \textit{debit
%         \emph{amt}}, acct_P \rangle)\\
%       \sig{Q} \speaksfor Q \quad \sig{P} \speaksfor P \quad
%       \langle \textit{\emph{amt} covered}, acct_P \rangle \\
%       \langle \textit{\emph{amt} covered}, acct_P \rangle \implies
%       P\controls (\langle \textit{pay \emph{amt}}, Q \rangle \wedge
%       \langle \textit{debit \emph{amt}}, acct_P \rangle)\\
%       \reps{Q}{P}{(\langle \textit{pay \emph{amt}}, Q \rangle \wedge
%       \langle \textit{debit \emph{amt}}, acct_P \rangle)}\\
%     \end{array}
%   }
%   {\langle \textit{pay \emph{amt}}, Q \rangle
%     \wedge \langle \textit{debit \emph{amt}}, acct_P \rangle}
%   {\parbox[m]{0.5in}{Simple Checking}}.

  \caption{Summary of rules for delegation}
  \label{fig:summary-delegation-rules}
\end{figure}

\begin{figure}
  \centering

  $\ell_1 =_s \ell_2 \defn (\ell_1 \leq_s \ell_2) \wedge (\ell_2 \leq_s
  \ell_1)$ \vspace*{2em} 
  
  \irule{}{\ell \leq_s \ell}{Reflexivity of $\leq_s$} \vspace*{2em} 

  \irule{\ell_1 \leq_s \ell_2 \qquad \ell_2 \leq_s \ell_3}{\ell_1 \leq_s
    \ell_3}{Transitivity of $\leq_s$} 

  \caption{Inference rules for relating security levels}
  \label{fig:summary-security-level-rules}
\end{figure}



\begin{figure}
  \centering

  $\ell_1 =_i \ell_2 \defn (\ell_1 \leq_i \ell_2) \wedge (\ell_2 \leq_i
  \ell_1)$ \vspace*{2em} 
  
  \irule{}{\ell \leq_i \ell}{Reflexivity of $\leq_i$} \vspace*{2em} 

  \irule{\ell_1 \leq_i \ell_2 \qquad \ell_2 \leq_i \ell_3}{\ell_1 \leq_i
    \ell_3}{Transitivity of $\leq_i$} 

  \caption{Inference rules for relating integrity levels}
  \label{fig:summary-integrity-level-rules}
\end{figure}

\begin{figure}[t]
  \centering
   $P = Q \defn  P \speaksfor Q \;\wedge\;Q \speaksfor P$
   \rulespace
  
    \irule{P = Q \qquad \subst{\varphi}{P}{A}}
    {\subst{\varphi}{Q}{A}}{\infname{Principal Equality}}
    \rulespace


   \irule{}{P\quoting (R_1 \with \cdots \with R_k) = (P \quoting R_1)
    \with \cdots \with (P \quoting R_k)}{Distributivity of \quoting}
  ($k\geq 1$)
  \rulespace

    \irule{P \quoting (Q\with R) \says \varphi}{P\quoting Q \says \varphi}
      {Quoting Simplification} 
  \caption{Logical rules regarding principal equality}
  \label{fig:summary-principal-equality-rules}
\end{figure}


% \begin{figure}
%   \centering
%     \irule
%   {
%     \begin{array}[b]{c}
%       Signature _Q \quoting \sig{P} \says (\langle \textit{pay
%         \emph{amt}}, Q \rangle \wedge \langle \textit{debit
%         \emph{amt}}, acct_P \rangle)\\
%       \sig{Q} \speaksfor Q \quad \sig{P} \speaksfor P \quad
%       \langle \textit{\emph{amt} covered}, acct_P \rangle \\
%       \langle \textit{\emph{amt} covered}, acct_P \rangle \implies
%       P\controls (\langle \textit{pay \emph{amt}}, Q \rangle \wedge
%       \langle \textit{debit \emph{amt}}, acct_P \rangle)\\
%       \reps{Q}{P}{(\langle \textit{pay \emph{amt}}, Q \rangle \wedge
%       \langle \textit{debit \emph{amt}}, acct_P \rangle)}\\
%     \end{array}
%   }
%   {\langle \textit{pay \emph{amt}}, Q \rangle
%     \wedge \langle \textit{debit \emph{amt}}, acct_P \rangle}
%   {\parbox[m]{0.5in}{Simple Checking}}
%   \caption{Summary of rules for simple checking}
%   \label{fig:simple-checking-rules}
% \end{figure}

% \begin{figure}
%   \centering
  
%   \caption{Summary of derived rules for SSL/TLS}
%   \label{fig:ssl-rules}
% \end{figure}

% \begin{figure}
%   \centering
%     \irule
%     {
%       \begin{array}[t]{c}
%         Authority \controls \action{\textit{use session key}, K_\mathit{session}} \\
%         K \says \action{\textit{use session key}, K_\mathit{session}} \\
%         K \speaksfor Authority 
%       \end{array}
%     }
%     {\action{\textit{use session key}, K_\mathit{session}}}
%   {\parbox[c]{.6in}{Session Key \\
%       Receipt}} 
%   \rulespace

%     \irule
%     {
%       \begin{array}[t]{c}
%         K \says (K_\mathit{session} \controls \action{\name{get ticket,
%             service, P@\textit{add}}}) \\
%         K \speaksfor \name{AS} \\
%         \name{AS} \controls (K_\mathit{session} \controls
%         \action{\name{get ticket, service, P@\textit{add}}}) \\
%         K_\mathit{session} \says \action{\name{get ticket,
%             service, P@\textit{add} }} \\
%       \end{array}
%     }
%     {\action{\name{get ticket, service, P@\textit{add}}}}
%   {\parbox[c]{.7in}{Grant Service \\ Ticket}}
%   \rulespace

%     \irule
%     {
%       \begin{array}[t]{c}
%         \name{TGS} \controls (K_\mathit{session} \speaksfor P) \\
%         K \says (K_\mathit{session} \speaksfor P) \\
%         K \speaksfor \name{TGS} \\
%         K_\mathit{session} \says \mathit{service request}
%       \end{array}
%     }
%     {P \says \mathit{service request}}
%   {\parbox[c]{.7in}{Service Request}}
%   \rulespace

%   \caption{Summary of derived rules for Kerberos}
%   \label{fig:kerberos-rules}
% \end{figure}

% \begin{figure}
%   \centering
%       \irule
%     {
%       \begin{array}[t]{c}
%        Signature _Q \quoting Signature_P \says (\langle \textit{Pay
%          \emph{amt}}, Q \rangle \wedge \langle \textit{debit
%          \emph{amt}}, acct_P \rangle)\\
%        Signature_Q \says \langle \textit{credit \emph{amt}},acct_Q
%        \rangle \\
%        Signature_Q \speaksfor Q\\
%        \hspace*{-.8in}
%        (Q \quoting  Signature_P  \says  (\langle \textit{Pay
%          \emph{amt}}, Q \rangle \wedge \langle \textit{debit
%          \emph{amt}}, acct_P \rangle)) \implies \\
%        \hspace*{1in}
%        \langle \textit{Pay \emph{amt}}, Q \rangle \wedge (Q
%        \controls \langle \textit{credit \emph{amt}},acct_Q \rangle) \\ 
%       \end{array}
%     }
%     {\langle \textit{Pay \emph{amt}}, Q \rangle \wedge \langle
%       \textit{credit \emph{amt}}, acct_Q \rangle} 
%     {\parbox[b]{0.5in}{ACH Check Deposit}}
%     \rulespace

%     \irule
%     {
%       \begin{array}[t]{c}
%         Signature _Q \quoting Signature_P \says (\langle \textit{Pay
%           \emph{amt}}, Q \rangle \wedge \langle \textit{debit
%           \emph{amt}}, acct_P \rangle)\\
%         Signature_Q \speaksfor Q
%       \end{array}
%     }
%     {Bank_Q  \says  (ECC_{Bank_Q}  \says
%       (Q \quoting Signature_P) \says (\langle \textit{Pay
%         \emph{amt}}, Q \rangle \wedge \langle \textit{debit
%         \emph{amt}}, acct_P \rangle))}
%     {\parbox[c]{0.5in}{ACH Check Presentation}}
%     \rulespace

%     \irule
%     {
%       \begin{array}[b]{l}
%         Bank_Q  \says  (ECC_{Bank_Q} \says % \\
%          (Q \quoting Signature_P) \says (\langle \textit{Pay
%           \emph{amt}}, Q \rangle \wedge \langle \textit{debit
%           \emph{amt}}, acct_P \rangle)) \\
%         Bank_Q  \controls  (ECC_{Bank_Q} \says %\\
%          (Q \quoting Signature_P) \says (\langle \textit{Pay
%           \emph{amt}}, Q \rangle \wedge \langle \textit{debit
%           \emph{amt}}, acct_P \rangle)) %\\
% %         \rp{Bank_Q}{ECC_{Bank_Q}} \\
% %         \quad (Q \quoting Signature_P) \says (\langle \textit{Pay
% %           \emph{amt}}, Q \rangle \wedge \langle \textit{debit
% %           \emph{amt}}, acct_P \rangle))
%       \end{array}
%     }
%     {
%       \begin{array}[b]{l}
%         (ACH \with Bank_Q)  \says  (ECC_{Bank_Q}  \says
%         ((Q \quoting   Signature_P) \says %\\
%          (\langle \textit{Pay \emph{amt}},  Q \rangle \wedge \langle
%         \textit{debit \emph{amt}}, acct_P \rangle)))
%       \end{array}
%     }
%     {\parbox[c]{0.5in}{ACH \\ Countersign \\ Rule}}
%     \rulespace

%     \irule
%     {
%       \begin{array}[b]{c}
%         (ACH \with Bank_Q)  \says  (ECC_{Bank_Q}  \says
%         ((Q \quoting   Signature_P) \says 
%         (\langle \textit{Pay \emph{amt}},  Q \rangle \wedge \langle
%         \textit{debit \emph{amt}}, acct_P \rangle)))\\
%         %
%         \rp{((ACH \with Bank_Q)  \quoting  ECC_{Bank_Q})}{Q}
%         (Signature_P \says
%         (\langle \textit{Pay \emph{amt}},  Q \rangle \wedge \langle
%         \textit{debit \emph{amt}}, acct_P \rangle)) \\
%         Signature_P \speaksfor P\\
%         P \controls (\langle \textit{Pay \emph{amt}}, Q \rangle \wedge
%         \langle \textit{debit \emph{amt}}, acct_P \rangle)\\
%         \reps{Q}{P}{(\langle \textit{Pay \emph{amt}}, Q \rangle \wedge
%       \langle \textit{debit \emph{amt}}, acct_P \rangle)}
%       \end{array}
% }
%     {(Bank_P \says \langle \textit{Pay \emph{amt}}, Q \rangle)
%       \wedge \langle \textit{debit \emph{amt}}, acct_P \rangle}
%     {\parbox[b]{0.5in}{ACH Check Funding}}
%     \rulespace

%         \irule
%     {
%       \begin{array}[t]{c}
%         Bank_P \says \langle \textit{Pay \emph{amt}}, Q \rangle\\
%         Bank_P \controls \langle \textit{Pay \emph{amt}}, Q \rangle
%       \end{array}
%     }
%     {ACH \says \langle \textit{Pay \emph{amt}}, Q \rangle}
%     {ACH Check Settlement Rule}

%   \caption{Summary of derived rules for interbank checking}
%   \label{fig:ACH-checking-rules}
% \end{figure}

%%% Local Variables: 
%%% mode: latex
%%% TeX-master: "book"
%%% End: 


\chapter{Simple Virtual Machines}
\label{chap:simpleVM}

In this chapter we develop formal descriptions of processors and
virtual machines at the \emph{instruction set architecture} (ISA)
level. A machine's ISA description is its \emph{machine language}: the
instructions directly implemented by the hardware.

\fbox{
  \begin{minipage}{1.0\linewidth}
    Elements of our simple machine:
    \begin{itemize}
    \item An accumulator-based architecture
    \item We omit the details of finite word length and treat all data as numbers
    \item Machine state is given by: accumulator and program counter
    \item User memory holds programs
    \item To simplify our design and analysis, we use a tagged
      architecture approach whereby the types of what is stored in
      memory is explicit
    \end{itemize}

  \end{minipage}
}

\fbox{
  \begin{minipage}{1.0\linewidth}
    Things we need to prove:
    \begin{itemize}
    \item User's view of the program implies desired configuration
      changes: this is the converse of the ISA transition relation
      defined inductively
    \item User's view of a program is the same as what happens in a
      VM, subject to certain conditions
      \begin{itemize}
      \item A simplifying condition: assume physical memory is unlimited
      \item Think about how to describe non-overlapping segments,
        first abstraction: model segments as $S_0, S_1, \cdots$
      \end{itemize}
    \item An instruction set is virtualizable
    \end{itemize}

  \end{minipage}
}
% ---- this points LaTeX to book.tex ---- 
% Local Variables: 
% TeX-master: "book"
% End:

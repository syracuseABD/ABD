
\chapter*{Preface}
\label{preface}
\addcontentsline{toc}{chapter}{\numberline{}Preface}

Our purpose in writing this textbook is to serve the needs of
engineers and computer scientists who are responsible for designing,
implementing, and verifying secure computer and information
systems. This purpose is unchanged from that of our previous textbook,
\emph{Access Control, Security, and Trust: A Logical Approach},
\cite{ACST}. As before, our methods are based on the application of
logic as a means for describing, reasoning about, and verifying the
properties of systems. We use logic from the conceptualization stage,
through the design phase, and up to and including verification and
certification. Our intent is to make this material accessible to upper
division undergraduate students in computer science and engineering.
What we present here has been tried out on undergraduate students from
many different undergraduate colleges and universities.

This text book builds upon our previous work on access control and
adds two more dimensions: (a) transition systems, and (b) formal
verification computer-assisted reasoning tools. Why these additions?

First, all but the simplest combinational logic functions are
transition systems, i.e., they have some notion of state. The logic of
transition systems is useful to describe both the behavior and
properties of systems. When combined with the multi-agent modal logic
that is the access-control logic used for describing and reasoning
about the elements of access-control decisions, we are able to
describe the behavior of systems over time while accounting for the
security policies affecting system operations.

Second, assurance of correctness matters. No matter how simple a
calculation or proof might be, the possibility of human error is ever
present.  Just as the use of computer-aided design (CAD) tools is
essential for producing microprocessors with billions of transistors
on a single chip, so is the use of computer-assisted reasoning tools,
such as theorem provers, necessary to formally verify assurance
arguments. The use of reasoning tools such as theorem provers provide a
degree of confidence unmatched by less rigorous means depending on
either pencil and paper calculations or simulations that often are
incomplete case analyses of a system. \emph{These tools are an
  effective and essential antidote to self delusion}.

CAD tools and programming languages sometimes change and evolve
quickly. The particular theorem prover we use is the the Cambridge
University Higher Order Logic (HOL) theorem prover. There are other
theorem provers and our use of HOL is not intended as an implicit
judgment of its superiority over the rest. Rather, it reflects HOL's
longevity, openness, and extensive libraries of theories developed
since its inception in the early 1980s. To the extent possible, we
have tried to minimize the dating this book by focusing on the
long-lasting parts of HOL.

As with our previous book, we developed much of the content of this
book in our own courses at Syracuse University and for the Air Force
Research Laboratory Information Directorate's Information Assurance
Research internships. Our students benefited from having illustrations
and exercises to learn from and gain deeper insights.  Thus, we have
included numerous examples to illustrate principles, as well as many
exercises to serve as assessments of knowledge.  We have annotated
each exercise to indicate the level of knowledge it assesses,
according to the following legend:
\begin{itemize}
\item The symbol \appn\ denotes exercises at the \emph{application}
  level of knowledge.  These exercises typically ask the reader to apply
  particular knowledge or use particular techniques to solve a problem
  in a new context.  Straightforward calculations fall into this
  category of exercises.
\item The symbol $\analysis$ denotes exercises at the \emph{analysis}
  level of knowledge.  These exercises generally require the reader to
  decompose a problem into its constituent parts in order to perform the
  necessary calculations or experiments necessary to solve the problem.
\item The symbol $\synthesis$ denotes exercises at the \emph{synthesis}
  level of knowledge.  These exercises typically require the reader to
  integrate various techniques or components to design, construct, or
  formulate an entirely new structure, pattern, or proof.
\item The symbol $\eval$ denotes exercises at the \emph{evaluation}
  level of knowledge.  These exercises require the reader to identify
  and use relevant criteria to assess and judge the suitability of a
  solution to a given problem.
\end{itemize}

Exercises at higher levels of knowledge are not necessarily harder than
exercises at lower levels of knowledge. We recommend that readers try at
least one exercise at each level of knowledge to get the most benefit.


\paragraph*{Acknowledgments}

We are grateful for the support of numerous colleagues and
students. In particular, we are grateful for the partnership we have
enjoyed with the Air Force Research Laboratory since 2003. The
partnership has allowed us access to young and fresh minds who are
tomorrow's leaders.

% ---- this points LaTeX to book.tex ---- %
% Local Variables: 
% TeX-master: "book" 
% End:



\chapter{Using Type Definitions in HOL}
\label{chap:types}

\section{An Algebraic Model of Cryptographic Operations}
\label{sec:crypto-ops}

% %% HOl/LaTeX files for SecureMessages from HOL/ACL/Examples/SecureMessages
\newcommand{\HOLcipherDate}{20 August 2016}
\newcommand{\HOLcipherTime}{12:36}
\begin{SaveVerbatim}{HOLcipherDatatypesasymMsg}
\HOLFreeVar{asymMsg} = \HOLConst{Ea} ('princ \HOLTyOp{pKey}) ('message \HOLTyOp{contents})
\end{SaveVerbatim}
\newcommand{\HOLcipherDatatypesasymMsg}{\UseVerbatim{HOLcipherDatatypesasymMsg}}
\begin{SaveVerbatim}{HOLcipherDatatypescontents}
\HOLFreeVar{contents} = \HOLConst{plain} 'message \HOLTokenBar{} \HOLConst{unknown}
\end{SaveVerbatim}
\newcommand{\HOLcipherDatatypescontents}{\UseVerbatim{HOLcipherDatatypescontents}}
\begin{SaveVerbatim}{HOLcipherDatatypesdigest}
\HOLFreeVar{digest} = \HOLConst{hash} ('message \HOLTyOp{contents})
\end{SaveVerbatim}
\newcommand{\HOLcipherDatatypesdigest}{\UseVerbatim{HOLcipherDatatypesdigest}}
\begin{SaveVerbatim}{HOLcipherDatatypespKey}
\HOLFreeVar{pKey} = \HOLConst{pubK} 'princ \HOLTokenBar{} \HOLConst{privK} 'princ
\end{SaveVerbatim}
\newcommand{\HOLcipherDatatypespKey}{\UseVerbatim{HOLcipherDatatypespKey}}
\begin{SaveVerbatim}{HOLcipherDatatypesSymKey}
\HOLFreeVar{SymKey} = \HOLConst{sym} \HOLTyOp{num}
\end{SaveVerbatim}
\newcommand{\HOLcipherDatatypesSymKey}{\UseVerbatim{HOLcipherDatatypesSymKey}}
\begin{SaveVerbatim}{HOLcipherDatatypessymMsg}
\HOLFreeVar{symMsg} = \HOLConst{Es} \HOLTyOp{SymKey} ('message \HOLTyOp{contents})
\end{SaveVerbatim}
\newcommand{\HOLcipherDatatypessymMsg}{\UseVerbatim{HOLcipherDatatypessymMsg}}
\newcommand{\HOLcipherDatatypes}{
\HOLcipherDatatypesasymMsg\HOLcipherDatatypescontents\HOLcipherDatatypesdigest\HOLcipherDatatypespKey\HOLcipherDatatypesSymKey\HOLcipherDatatypessymMsg}
\begin{SaveVerbatim}{HOLcipherDefinitionsgetMessageXXdef}
\HOLTokenTurnstile{} \HOLSymConst{\HOLTokenForall{}}\HOLBoundVar{msg}. \HOLConst{getMessage} (\HOLConst{plain} \HOLBoundVar{msg}) \HOLSymConst{=} \HOLBoundVar{msg}
\end{SaveVerbatim}
\newcommand{\HOLcipherDefinitionsgetMessageXXdef}{\UseVerbatim{HOLcipherDefinitionsgetMessageXXdef}}
\begin{SaveVerbatim}{HOLcipherDefinitionssignXXdef}
\HOLTokenTurnstile{} \HOLSymConst{\HOLTokenForall{}}\HOLBoundVar{P} \HOLBoundVar{dgst}. \HOLConst{sign} \HOLBoundVar{P} \HOLBoundVar{dgst} \HOLSymConst{=} \HOLConst{Ea} (\HOLConst{privK} \HOLBoundVar{P}) (\HOLConst{plain} \HOLBoundVar{dgst})
\end{SaveVerbatim}
\newcommand{\HOLcipherDefinitionssignXXdef}{\UseVerbatim{HOLcipherDefinitionssignXXdef}}
\begin{SaveVerbatim}{HOLcipherDefinitionssignVerifyXXdef}
\HOLTokenTurnstile{} \HOLSymConst{\HOLTokenForall{}}\HOLBoundVar{P} \HOLBoundVar{signature} \HOLBoundVar{msgContents}.
     \HOLConst{signVerify} \HOLBoundVar{P} \HOLBoundVar{signature} \HOLBoundVar{msgContents} \HOLSymConst{\HOLTokenEquiv{}}
     (\HOLConst{plain} (\HOLConst{hash} \HOLBoundVar{msgContents}) \HOLSymConst{=} \HOLConst{deciphP} (\HOLConst{pubK} \HOLBoundVar{P}) \HOLBoundVar{signature})
\end{SaveVerbatim}
\newcommand{\HOLcipherDefinitionssignVerifyXXdef}{\UseVerbatim{HOLcipherDefinitionssignVerifyXXdef}}
\newcommand{\HOLcipherDefinitions}{
\HOLDfnTag{cipher}{getMessage_def}\HOLcipherDefinitionsgetMessageXXdef
\HOLDfnTag{cipher}{sign_def}\HOLcipherDefinitionssignXXdef
\HOLDfnTag{cipher}{signVerify_def}\HOLcipherDefinitionssignVerifyXXdef
}
\begin{SaveVerbatim}{HOLcipherTheoremsdeciphPXXclauses}
\HOLTokenTurnstile{} (\HOLSymConst{\HOLTokenForall{}}\HOLBoundVar{P} \HOLBoundVar{text}.
      (\HOLConst{deciphP} (\HOLConst{pubK} \HOLBoundVar{P}) (\HOLConst{Ea} (\HOLConst{privK} \HOLBoundVar{P}) (\HOLConst{plain} \HOLBoundVar{text})) \HOLSymConst{=}
       \HOLConst{plain} \HOLBoundVar{text}) \HOLSymConst{\HOLTokenConj{}}
      (\HOLConst{deciphP} (\HOLConst{privK} \HOLBoundVar{P}) (\HOLConst{Ea} (\HOLConst{pubK} \HOLBoundVar{P}) (\HOLConst{plain} \HOLBoundVar{text})) \HOLSymConst{=}
       \HOLConst{plain} \HOLBoundVar{text})) \HOLSymConst{\HOLTokenConj{}}
   (\HOLSymConst{\HOLTokenForall{}}\HOLBoundVar{k} \HOLBoundVar{P} \HOLBoundVar{text}.
      (\HOLConst{deciphP} \HOLBoundVar{k} (\HOLConst{Ea} (\HOLConst{privK} \HOLBoundVar{P}) (\HOLConst{plain} \HOLBoundVar{text})) \HOLSymConst{=} \HOLConst{plain} \HOLBoundVar{text}) \HOLSymConst{\HOLTokenImp{}}
      (\HOLBoundVar{k} \HOLSymConst{=} \HOLConst{pubK} \HOLBoundVar{P})) \HOLSymConst{\HOLTokenConj{}}
   \HOLSymConst{\HOLTokenForall{}}\HOLBoundVar{k} \HOLBoundVar{P} \HOLBoundVar{text}.
     (\HOLConst{deciphP} \HOLBoundVar{k} (\HOLConst{Ea} (\HOLConst{pubK} \HOLBoundVar{P}) (\HOLConst{plain} \HOLBoundVar{text})) \HOLSymConst{=} \HOLConst{plain} \HOLBoundVar{text}) \HOLSymConst{\HOLTokenImp{}}
     (\HOLBoundVar{k} \HOLSymConst{=} \HOLConst{privK} \HOLBoundVar{P})
\end{SaveVerbatim}
\newcommand{\HOLcipherTheoremsdeciphPXXclauses}{\UseVerbatim{HOLcipherTheoremsdeciphPXXclauses}}
\begin{SaveVerbatim}{HOLcipherTheoremsdeciphPXXdef}
\HOLTokenTurnstile{} (\HOLConst{deciphP} \HOLFreeVar{key} (\HOLConst{Ea} (\HOLConst{privK} \HOLFreeVar{P}) (\HOLConst{plain} \HOLFreeVar{text})) \HOLSymConst{=}
    \HOLKeyword{if} \HOLFreeVar{key} \HOLSymConst{=} \HOLConst{pubK} \HOLFreeVar{P} \HOLKeyword{then} \HOLConst{plain} \HOLFreeVar{text} \HOLKeyword{else} \HOLConst{unknown}) \HOLSymConst{\HOLTokenConj{}}
   (\HOLConst{deciphP} \HOLFreeVar{key} (\HOLConst{Ea} (\HOLConst{pubK} \HOLFreeVar{P}) (\HOLConst{plain} \HOLFreeVar{text})) \HOLSymConst{=}
    \HOLKeyword{if} \HOLFreeVar{key} \HOLSymConst{=} \HOLConst{privK} \HOLFreeVar{P} \HOLKeyword{then} \HOLConst{plain} \HOLFreeVar{text} \HOLKeyword{else} \HOLConst{unknown})
\end{SaveVerbatim}
\newcommand{\HOLcipherTheoremsdeciphPXXdef}{\UseVerbatim{HOLcipherTheoremsdeciphPXXdef}}
\begin{SaveVerbatim}{HOLcipherTheoremsdeciphPXXind}
\HOLTokenTurnstile{} \HOLSymConst{\HOLTokenForall{}}\HOLBoundVar{P\sp{\prime}}.
     (\HOLSymConst{\HOLTokenForall{}}\HOLBoundVar{key} \HOLBoundVar{P} \HOLBoundVar{text}. \HOLBoundVar{P\sp{\prime}} \HOLBoundVar{key} (\HOLConst{Ea} (\HOLConst{privK} \HOLBoundVar{P}) (\HOLConst{plain} \HOLBoundVar{text}))) \HOLSymConst{\HOLTokenConj{}}
     (\HOLSymConst{\HOLTokenForall{}}\HOLBoundVar{key} \HOLBoundVar{P} \HOLBoundVar{text}. \HOLBoundVar{P\sp{\prime}} \HOLBoundVar{key} (\HOLConst{Ea} (\HOLConst{pubK} \HOLBoundVar{P}) (\HOLConst{plain} \HOLBoundVar{text}))) \HOLSymConst{\HOLTokenConj{}}
     (\HOLSymConst{\HOLTokenForall{}}\HOLBoundVar{key} \HOLBoundVar{v\sb{\mathrm{7}}}. \HOLBoundVar{P\sp{\prime}} \HOLBoundVar{key} (\HOLConst{Ea} \HOLBoundVar{v\sb{\mathrm{7}}} \HOLConst{unknown})) \HOLSymConst{\HOLTokenImp{}}
     \HOLSymConst{\HOLTokenForall{}}\HOLBoundVar{v} \HOLBoundVar{v\sb{\mathrm{1}}}. \HOLBoundVar{P\sp{\prime}} \HOLBoundVar{v} \HOLBoundVar{v\sb{\mathrm{1}}}
\end{SaveVerbatim}
\newcommand{\HOLcipherTheoremsdeciphPXXind}{\UseVerbatim{HOLcipherTheoremsdeciphPXXind}}
\begin{SaveVerbatim}{HOLcipherTheoremsdeciphSXXclauses}
\HOLTokenTurnstile{} (\HOLSymConst{\HOLTokenForall{}}\HOLBoundVar{k} \HOLBoundVar{text}. \HOLConst{deciphS} \HOLBoundVar{k} (\HOLConst{Es} \HOLBoundVar{k} (\HOLConst{plain} \HOLBoundVar{text})) \HOLSymConst{=} \HOLConst{plain} \HOLBoundVar{text}) \HOLSymConst{\HOLTokenConj{}}
   \HOLSymConst{\HOLTokenForall{}}\HOLBoundVar{k\sb{\mathrm{1}}} \HOLBoundVar{k\sb{\mathrm{2}}} \HOLBoundVar{text}.
     (\HOLConst{deciphS} \HOLBoundVar{k\sb{\mathrm{1}}} (\HOLConst{Es} \HOLBoundVar{k\sb{\mathrm{2}}} (\HOLConst{plain} \HOLBoundVar{text})) \HOLSymConst{=} \HOLConst{plain} \HOLBoundVar{text}) \HOLSymConst{\HOLTokenImp{}} (\HOLBoundVar{k\sb{\mathrm{1}}} \HOLSymConst{=} \HOLBoundVar{k\sb{\mathrm{2}}})
\end{SaveVerbatim}
\newcommand{\HOLcipherTheoremsdeciphSXXclauses}{\UseVerbatim{HOLcipherTheoremsdeciphSXXclauses}}
\begin{SaveVerbatim}{HOLcipherTheoremsdeciphSXXdef}
\HOLTokenTurnstile{} \HOLConst{deciphS} \HOLFreeVar{k\sb{\mathrm{1}}} (\HOLConst{Es} \HOLFreeVar{k\sb{\mathrm{2}}} (\HOLConst{plain} \HOLFreeVar{text})) \HOLSymConst{=}
   \HOLKeyword{if} \HOLFreeVar{k\sb{\mathrm{1}}} \HOLSymConst{=} \HOLFreeVar{k\sb{\mathrm{2}}} \HOLKeyword{then} \HOLConst{plain} \HOLFreeVar{text} \HOLKeyword{else} \HOLConst{unknown}
\end{SaveVerbatim}
\newcommand{\HOLcipherTheoremsdeciphSXXdef}{\UseVerbatim{HOLcipherTheoremsdeciphSXXdef}}
\begin{SaveVerbatim}{HOLcipherTheoremsdeciphSXXind}
\HOLTokenTurnstile{} \HOLSymConst{\HOLTokenForall{}}\HOLBoundVar{P}.
     (\HOLSymConst{\HOLTokenForall{}}\HOLBoundVar{k\sb{\mathrm{1}}} \HOLBoundVar{k\sb{\mathrm{2}}} \HOLBoundVar{text}. \HOLBoundVar{P} \HOLBoundVar{k\sb{\mathrm{1}}} (\HOLConst{Es} \HOLBoundVar{k\sb{\mathrm{2}}} (\HOLConst{plain} \HOLBoundVar{text}))) \HOLSymConst{\HOLTokenConj{}}
     (\HOLSymConst{\HOLTokenForall{}}\HOLBoundVar{v\sb{\mathrm{6}}} \HOLBoundVar{v\sb{\mathrm{5}}}. \HOLBoundVar{P} \HOLBoundVar{v\sb{\mathrm{6}}} (\HOLConst{Es} \HOLBoundVar{v\sb{\mathrm{5}}} \HOLConst{unknown})) \HOLSymConst{\HOLTokenImp{}}
     \HOLSymConst{\HOLTokenForall{}}\HOLBoundVar{v} \HOLBoundVar{v\sb{\mathrm{1}}}. \HOLBoundVar{P} \HOLBoundVar{v} \HOLBoundVar{v\sb{\mathrm{1}}}
\end{SaveVerbatim}
\newcommand{\HOLcipherTheoremsdeciphSXXind}{\UseVerbatim{HOLcipherTheoremsdeciphSXXind}}
\begin{SaveVerbatim}{HOLcipherTheoremssignVerifyXXOneOne}
\HOLTokenTurnstile{} \HOLSymConst{\HOLTokenForall{}}\HOLBoundVar{P} \HOLBoundVar{m\sb{\mathrm{1}}} \HOLBoundVar{m\sb{\mathrm{2}}}.
     \HOLConst{signVerify} \HOLBoundVar{P} (\HOLConst{Ea} (\HOLConst{privK} \HOLBoundVar{P}) (\HOLConst{plain} (\HOLConst{hash} (\HOLConst{plain} \HOLBoundVar{m\sb{\mathrm{1}}}))))
       (\HOLConst{plain} \HOLBoundVar{m\sb{\mathrm{2}}}) \HOLSymConst{\HOLTokenEquiv{}} (\HOLBoundVar{m\sb{\mathrm{1}}} \HOLSymConst{=} \HOLBoundVar{m\sb{\mathrm{2}}})
\end{SaveVerbatim}
\newcommand{\HOLcipherTheoremssignVerifyXXOneOne}{\UseVerbatim{HOLcipherTheoremssignVerifyXXOneOne}}
\begin{SaveVerbatim}{HOLcipherTheoremssignVerifyOK}
\HOLTokenTurnstile{} \HOLSymConst{\HOLTokenForall{}}\HOLBoundVar{P} \HOLBoundVar{msg}. \HOLConst{signVerify} \HOLBoundVar{P} (\HOLConst{sign} \HOLBoundVar{P} (\HOLConst{hash} (\HOLConst{plain} \HOLBoundVar{msg}))) (\HOLConst{plain} \HOLBoundVar{msg})
\end{SaveVerbatim}
\newcommand{\HOLcipherTheoremssignVerifyOK}{\UseVerbatim{HOLcipherTheoremssignVerifyOK}}
\newcommand{\HOLcipherTheorems}{
\HOLThmTag{cipher}{deciphP_clauses}\HOLcipherTheoremsdeciphPXXclauses
\HOLThmTag{cipher}{deciphP_def}\HOLcipherTheoremsdeciphPXXdef
\HOLThmTag{cipher}{deciphP_ind}\HOLcipherTheoremsdeciphPXXind
\HOLThmTag{cipher}{deciphS_clauses}\HOLcipherTheoremsdeciphSXXclauses
\HOLThmTag{cipher}{deciphS_def}\HOLcipherTheoremsdeciphSXXdef
\HOLThmTag{cipher}{deciphS_ind}\HOLcipherTheoremsdeciphSXXind
\HOLThmTag{cipher}{signVerify_11}\HOLcipherTheoremssignVerifyXXOneOne
\HOLThmTag{cipher}{signVerifyOK}\HOLcipherTheoremssignVerifyOK
}


\begin{figure}[t]
  \centering
  \begin{minipage}{1.0\linewidth}
    \HOLcipherDatatypescontents
    \bluetext{[getMessage\_def]}\vspace*{-0.1in}
    \HOLcipherDefinitionsgetMessageXXdef
  \end{minipage}
  \caption{Contents and Content Accessor Functions}
  \label{fig:contents}
\end{figure}

\begin{figure}[tb]
  \centering
  \begin{minipage}{1.0\linewidth}
    \HOLcipherDatatypesSymKey
    \HOLcipherDatatypessymMsg
    \bluetext{[deciphS\_def]}\vspace*{-0.1in}
    \HOLcipherDefinitionsdeciphSXXdef
  \end{minipage}

  \caption{cipher Theory: Symmetric Key Operations}
  \label{fig:cipher-theory-symmetric}
\end{figure}

When devising and assuring systems, reasoning about operations and
their sequencing is crucial. For cryptographic operations, where
asymmetric, symmetric, and cryptographic hash functions are viewed as
components, we focus on their \emph{properties} as opposed to their
algorithmic details.

Figures~\ref{fig:public key}, \ref{fig:private key}, \ref{fig:digital
  signature}, and \ref{fig:signature verification} in
Chapter~\ref{cha:pki} show public-key cryptographic operations for
encryption, decryption, signature generation, and signature
verification. At the level of the access-control logic, cryptographic
operations only allow us to determine which principal made a statement
or sent a message. Figure~\ref{fig:dig-sig-proof} is an example.
Specifically, if a received message passes its integrity check, we can
conclude who sent the message, e.g., $Ellen \says m$, if we know that
$K_{Ellen} \speaksfor Ellen$. The access-control logic alone does not
have sufficient expressive power to describe the cryptographic
operations and their sequencing in Figures~\ref{fig:public key},
\ref{fig:private key}, \ref{fig:digital signature}, and
\ref{fig:signature verification}.

However, it is important to know that a proposed signature integrity
checking scheme will work at the level of sequencing various
cryptographic operations on various message fields. To address this
need, we use algebraic types in HOL to describe structures and
properties of cryptographic operations. In this section, we give
detailed examples of how this is done.

For symmetrically encrypted messages we do the following:
\begin{enumerate}
\item Define the datatype \emph{contents}, which has two forms: (a)
  \emph{plain message}, used in cases where we can determine the
  unencrypted message, and (b) \emph{unknown}, used in cases where we
  cannot determine the unencrypted contents. In the first case,
  \emph{plain} is a \emph{polymorphic} type operator that takes any
  message type \emph{'message}. In the second case, \emph{unknown} is
  a type operator that takes no arguments and is also of type
  \emph{'message}. This is shown in Figure~\ref{fig:contents}.
\item Define the accessor function \emph{getMessage} as
  \emph{getMessage(plain msg) = msg}. This is shown in
  Figure~\ref{fig:contents}.
\item Define \emph{symmetric} encryption keys as a type \emph{SymKey},
  which has the form \emph{sym number}, i.e., a type operator
  \emph{sym} operating on arguments of type \emph{num}. This is shown
  in Figure~\ref{fig:cipher-theory-symmetric}.
\item Define symmetrically encrypted messages to be of type
  \emph{symMsg}, which are created by using the type operator
  \emph{Es} applied to a symmetric key \emph{SymKey} followed by
  \emph{message\_contents}.  This is shown in
  Figure~\ref{fig:cipher-theory-symmetric}.
\item Finally, we define the accessor function \emph{deciphS} whose
  definition mimics exactly the properties of symmetric-key
  decryption, i.e., if the same key is used to both encrypt and
  decrypt messages, then the original plain text is
  accessible. Otherwise, the decryption result is \emph{unknown}. The
  definition is shown in Figure~\ref{fig:cipher-theory-symmetric}.
\end{enumerate}



In the HOL session below, we see the types created by \emph{plain} and
the type of the argument returned by \emph{getMessage}.  We also see
from the theorem that \emph{getMessage} returns the plain text of a
message. Notice that the value of $getMessage \;unknown$ is
\emph{undefined}. Thus, it is usually a sign that something is wrong
if we ever need to determine the value of $getMessage \;unknown$.

\setcounter{sessioncount}{0}
\begin{session}
  \begin{scriptsize}
\begin{verbatim}

- type_of ``plain (msg:'message)``;
> val it = ``:'message contents``
     : hol_type
- type_of ``getMessage(plain (msg:'message))``;
> val it = ``:'message`` : hol_type
- REWRITE_RULE[getMessage_def](ASSUME``x = getMessage(plain(msg:'message))``);
> val it =
     [.] |- (x :'message) = (msg :'message)
     : thm
\end{verbatim}
  \end{scriptsize}
\end{session}


The HOL session below shows the types \emph{SymKey} and \emph{symMsg}
created by \emph{sym} and \emph{Es}, respectively.  The theorem shows
that the original message contents is retrieved when decrypting with
the same key used for encryption.
\begin{session}
  \begin{scriptsize}
\begin{verbatim}

- type_of ``sym (dek:num)``;
> val it = ``:SymKey`` : hol_type
- type_of ``Es (sym (dek:num)) (plain (msg:'message))``;
> val it = ``:'message symMsg`` :
- REWRITE_RULE
  [deciphS_def] 
  (ASSUME
  ``x = deciphS (sym (dek :num))(Es (sym (dek:num)) (plain (msg:'message)))``);
> val it =
     [.] |- (x :'message contents) = plain (msg :'message)
     : thm
\end{verbatim}
  \end{scriptsize}
\end{session}

For asymmetric or public-key cryptographic operations we take a
similar approach to that used for symmetric-key cryptography.
\begin{enumerate}
\item As before, \emph{'message contents} are operated on with public
  keys to created asymmetrically encrypted messages.
\item Public keys \emph{pKey} have a bit more abstract representation
  than symmetric keys. As public (and their corresponding private)
  keys are associated with principals, public and private keys are
  associated with principals: \emph{pubK principal} or \emph{privK
    principal}.  This is shown in Figure~\ref{fig:public-key}.
\end{enumerate}


\begin{figure}[tb]
  \centering
  \begin{minipage}{1.0\linewidth}
    \HOLcipherDatatypespKey
    \HOLcipherDatatypesasymMsg
    \bluetext{[deciphP\_def]}\vspace*{-0.1in}
    \HOLcipherDefinitionsdeciphPXXdef
  \end{minipage}
  \caption{Public Keys, Datatypes, and Decryption}
  \label{fig:public-key}
\end{figure}

We develop a structure and algebra for public-key
operations. Figure~\ref{fig:public-key} shows the datatypes for public
keys (\emph{pKey}) and asymmetrically encrypted messages
(\emph{asymMsg}), and the definition \emph{deciphP\_def} for
public-key decryption. In our model, public keys are associated with
\emph{principals}, in this case represented with the type variable
\texttt{'princ}---deliberately chosen to be the same as the type
variable name used for principals in our HOL implementation of the
access-control logic. Our model of asymmetric cryptographic keys is to
have public and private keys indexed by principals, e.g., \emph{pubK
  Alice} and \emph{privK Alice}. Asymmetrically encrypted contents are
encrypted using either public or private keys, e.g., \emph{Ea (pubK
  Alice) (plain msg)}. Asymmetrically encrypted contents can be
retrieved only if the corresponding private or public key is
used. Otherwise, the message contents are \emph{unknown}.

The HOL session below shows the types \emph{pKey} and \emph{asymMsg}
created by \emph{pubK}, \emph{privK}, and \emph{Ea}.  The theorem
shows that the message contents is retrieved when using the
corresponding private or public key.
\begin{session}
  \begin{scriptsize}
\begin{verbatim}

- type_of ``pubK (Alice:'princ)``;
> val it = ``:'princ pKey`` : hol_type
- type_of ``privK (Alice:'princ)``;
> val it = ``:'princ pKey`` : hol_type
- type_of ``Ea (pubK (Alice:'princ))(plain (msg:'message))``;
> val it = ``:('message, 'princ) asymMsg`` : hol_type
- REWRITE_RULE[deciphP_def]
  (ASSUME
   ``x = deciphP (privK (Alice:'princ))
          (Ea (pubK (Alice:'princ))(plain (msg:'message)))``);
> val it =
     [.] |- (x :'message contents) = plain (msg :'message) : thm
- REWRITE_RULE[deciphP_def]
  (ASSUME
   ``x = deciphP (pubK (Alice:'princ))
          (Ea (privK (Alice:'princ))(plain (msg:'message)))``);
> val it =
     [.] |- (x :'message contents) = plain (msg :'message) : thm
\end{verbatim}
  \end{scriptsize}
\end{session}

\begin{figure}[tb]
  \centering
  \begin{minipage}{1.0\linewidth}
    \HOLcipherDatatypesdigest
    \bluetext{[sign\_def]}\vspace*{-0.1in}
    \HOLcipherDefinitionssignXXdef
    \bluetext{[signVerify\_def]}\vspace*{-0.1in}
    \HOLcipherDefinitionssignVerifyXXdef
  \end{minipage}
  \caption{Hash and Signature Definitions}
  \label{fig:hash-signature}
\end{figure}

The last set of datatypes and definitions associated with
cryptographic operations deals with digital
signatures. Figure~\ref{fig:hash-signature} shows the definition of
the datatype \emph{digest}, and digital signature functions
\emph{sign\_def} and \emph{signVerify\_def}. The \emph{digest}
datatype models cryptographic hashes by the use of the \emph{hash}
constructor function applied to arguments of type \emph{'message
  contents}. Digital signatures are created using \emph{sign}, which
uses public-key encryption to encrypt message digests, i.e.,
\emph{hash (msg:'message)}, with a principal \emph{P}'s private key,
\emph{privK P}. Digital signatures are checked using
\emph{signVerify}, which deciphers the encrypted message digest using
principal \emph{P}'s public key \emph{pubK P} and compares the result
with the digest of the message contents received.

The HOL session below shows the type of message digests and the result
of applying the signature verification function to signed
messages. The theorem shows that when \emph{signVerify} is applied to
\emph{sign} with the appropriate keys and message digests,
\emph{signVerify} evaluates to \emph{true}.
\begin{session}
  \begin{scriptsize}
\begin{verbatim}

- type_of ``hash (plain (msg:'message))``;
> val it = ``:'message digest`` : hol_type
- type_of ``sign (Alice:'princ)(hash(plain (msg:'message)))``;
> val it =
    ``:('message digest, 'princ) asymMsg`` : hol_type
- REWRITE_RULE [signVerify_def,sign_def,deciphP_def]
  (ASSUME
   ``signVerify
     (Alice:'princ)
     (sign (Alice:'princ)(hash(plain (msg:'message))))
     (plain (msg:'message))``);
> val it =  [.] |- T : thm
\end{verbatim}
  \end{scriptsize}
\end{session}

\begin{figure}[tb]
  \centering
  \begin{minipage}{1.0\linewidth}
    \tealtext{[deciphS\_clauses]}\vspace*{-0.1in}
    \HOLcipherTheoremsdeciphSXXclauses
    \tealtext{[deciphP\_clauses]}\vspace*{-0.1in}
    \HOLcipherTheoremsdeciphPXXclauses
    \tealtext{[signVerifyOK]}\vspace*{-0.1in}
    \HOLcipherTheoremssignVerifyOK
    \tealtext{[signVerify\_11]}\vspace*{-0.1in}
    \HOLcipherTheoremssignVerifyXXOneOne
  \end{minipage}
  \caption{Theorems for Symmetric, Asymmetric, and Signature Operations}
  \label{fig:cipher-theorems}
\end{figure}

Finally, Figure~\ref{fig:cipher-theorems} lists four theorems for
symmetric, asymmetric, and signature operations. A summary of the four
theorems is as follows.
\begin{enumerate}[{-}]
\item \tealtext{[deciphS\_clauses]} This theorem states that (1)
  \emph{deciphS k} inverts \emph{Es k}, and (2) if \emph{deciphS k1}
  inverts \emph{deciphS k2} that it must be the case that \emph{k1} and
  \emph{k2} are identical.
\item \tealtext{[deciphP\_clauses]} This theorem states that (1)
  \emph{deciphP (pubK P)} and \emph{deciphP (privK P} inverts \emph{Es
    (privK P} and \emph{Es (pubK P)}, respectively, (2) if
  \emph{deciphP k} inverts \emph{Es (privK P)} then \emph{k} must be
  \emph{pubK P}, and (3) if \emph{deciphP k} inverts \emph{Es (pubK
    p)} then \emph{k} must be \emph{privK P}.
\item \tealtext{[signVerifyOK]} This theorem states that a message
  signature created by \emph{sign} will pass the message integrity
  checking function \emph{signVerify} when the corresponding public
  keys and message contents are supplied.
\item \tealtext{[signVerify\_11]} This theorem states that whenever
  the message integrity check is satisfied, then the received message
  is the message that was hashed to generate the digital signature.
\end{enumerate}

\section{Linking Message and Certificate Structures to Cryptographic Operations and Interpretations}
\label{sec:structure-interpretation}

We now look at how to make formal connections among concepts of
operations, specific message and certificate structures, the semantics
of messages and certificates, and cryptographic operations. Connecting
CONOPS to messages, certificates, cryptographic operations, and
semantics amounts to a \emph{refinement} of conceptual descriptions to
concrete implementations.

% As a concrete example, we refine the example command and control
% CONOPS introduced in Chapter~\ref{cha:c2conops} to include specific
% message and certificate structures, and their interpretations in the
% access-control logic.  The foundational theories for mission commands,
% roles, and staff remain the same as shown in
% Figures~\ref{fig:command-theory}, \ref{fig:mission-roles}, and
% \ref{fig:mission-staff}. To these theories we add Cipher Theory as
% described in Section~\ref{sec:crypto-ops}.

A \emph{Message} is intended to communicate \emph{Orders} securely.  

\newcommand{\HOLrevisedMissionKeysDate}{20 August 2016}
\newcommand{\HOLrevisedMissionKeysTime}{12:36}
\begin{SaveVerbatim}{HOLrevisedMissionKeysDatatypesmissionCA}
\HOLFreeVar{missionCA} = \HOLConst{JFCA} \HOLTokenBar{} \HOLConst{BFCA} \HOLTokenBar{} \HOLConst{GFCA}
\end{SaveVerbatim}
\newcommand{\HOLrevisedMissionKeysDatatypesmissionCA}{\UseVerbatim{HOLrevisedMissionKeysDatatypesmissionCA}}
\begin{SaveVerbatim}{HOLrevisedMissionKeysDatatypesmissionPrincipals}
\HOLFreeVar{missionPrincipals} =
    \HOLConst{MRole} \HOLTyOp{missionRoles}
  \HOLTokenBar{} \HOLConst{MStaff} \HOLTyOp{missionStaff}
  \HOLTokenBar{} \HOLConst{MCA} \HOLTyOp{missionCA}
  \HOLTokenBar{} \HOLConst{MKey} (\HOLTyOp{missionStaffCA} \HOLTyOp{pKey})
\end{SaveVerbatim}
\newcommand{\HOLrevisedMissionKeysDatatypesmissionPrincipals}{\UseVerbatim{HOLrevisedMissionKeysDatatypesmissionPrincipals}}
\begin{SaveVerbatim}{HOLrevisedMissionKeysDatatypesmissionStaffCA}
\HOLFreeVar{missionStaffCA} = \HOLConst{KStaff} \HOLTyOp{missionStaff} \HOLTokenBar{} \HOLConst{KCA} \HOLTyOp{missionCA}
\end{SaveVerbatim}
\newcommand{\HOLrevisedMissionKeysDatatypesmissionStaffCA}{\UseVerbatim{HOLrevisedMissionKeysDatatypesmissionStaffCA}}
\newcommand{\HOLrevisedMissionKeysDatatypes}{
\HOLrevisedMissionKeysDatatypesmissionCA\HOLrevisedMissionKeysDatatypesmissionPrincipals\HOLrevisedMissionKeysDatatypesmissionStaffCA}


\begin{figure}[tb]
  \centering
  \HOLrevisedMissionKeysDatatypesmissionCA
  \HOLrevisedMissionKeysDatatypesmissionPrincipals
  \HOLrevisedMissionKeysDatatypesmissionStaffCA
  \caption{Revised Mission Keys Theory}
  \label{fig:revised-mission-keys}
\end{figure}

\newcommand{\HOLmessageCertificateDate}{20 August 2016}
\newcommand{\HOLmessageCertificateTime}{12:36}
\begin{SaveVerbatim}{HOLmessageCertificateDatatypesAuthority}
\HOLFreeVar{Authority} = \HOLConst{Auth} \HOLTyOp{missionStaff}
\end{SaveVerbatim}
\newcommand{\HOLmessageCertificateDatatypesAuthority}{\UseVerbatim{HOLmessageCertificateDatatypesAuthority}}
\begin{SaveVerbatim}{HOLmessageCertificateDatatypesDelegate}
\HOLFreeVar{Delegate} = \HOLConst{For} \HOLTyOp{missionStaff}
\end{SaveVerbatim}
\newcommand{\HOLmessageCertificateDatatypesDelegate}{\UseVerbatim{HOLmessageCertificateDatatypesDelegate}}
\begin{SaveVerbatim}{HOLmessageCertificateDatatypesDestination}
\HOLFreeVar{Destination} = \HOLConst{To} \HOLTyOp{missionStaff}
\end{SaveVerbatim}
\newcommand{\HOLmessageCertificateDatatypesDestination}{\UseVerbatim{HOLmessageCertificateDatatypesDestination}}
\begin{SaveVerbatim}{HOLmessageCertificateDatatypesIssuer}
\HOLFreeVar{Issuer} = \HOLConst{CA} \HOLTyOp{missionCA}
\end{SaveVerbatim}
\newcommand{\HOLmessageCertificateDatatypesIssuer}{\UseVerbatim{HOLmessageCertificateDatatypesIssuer}}
\begin{SaveVerbatim}{HOLmessageCertificateDatatypesKCertSignature}
\HOLFreeVar{KCertSignature} =
    \HOLConst{KCrtSig} (((\HOLTyOp{missionCA} \HOLTokenProd{} \HOLTyOp{missionStaffCA} \HOLTokenProd{} \HOLTyOp{missionStaffCA} \HOLTyOp{pKey})
              \HOLTyOp{digest}, \HOLTyOp{missionStaffCA}) \HOLTyOp{asymMsg})
\end{SaveVerbatim}
\newcommand{\HOLmessageCertificateDatatypesKCertSignature}{\UseVerbatim{HOLmessageCertificateDatatypesKCertSignature}}
\begin{SaveVerbatim}{HOLmessageCertificateDatatypesKeyCertificate}
\HOLFreeVar{KeyCertificate} = \HOLConst{KCert} \HOLTyOp{Issuer} \HOLTyOp{Subject} \HOLTyOp{SubPubKey} \HOLTyOp{KCertSignature}
\end{SaveVerbatim}
\newcommand{\HOLmessageCertificateDatatypesKeyCertificate}{\UseVerbatim{HOLmessageCertificateDatatypesKeyCertificate}}
\begin{SaveVerbatim}{HOLmessageCertificateDatatypesMessage}
\HOLFreeVar{Message} =
    \HOLConst{MSG} \HOLTyOp{Origin} \HOLTyOp{Role} \HOLTyOp{Destination} ((\HOLTyOp{SymKey}, \HOLTyOp{missionStaff}) \HOLTyOp{asymMsg})
        (\HOLTyOp{Orders} \HOLTyOp{symMsg}) ((\HOLTyOp{Orders} \HOLTyOp{digest}, \HOLTyOp{missionStaff}) \HOLTyOp{asymMsg})
\end{SaveVerbatim}
\newcommand{\HOLmessageCertificateDatatypesMessage}{\UseVerbatim{HOLmessageCertificateDatatypesMessage}}
\begin{SaveVerbatim}{HOLmessageCertificateDatatypesOrders}
\HOLFreeVar{Orders} = \HOLConst{CMD} \HOLTyOp{Origin} \HOLTyOp{Role} \HOLTyOp{Destination} \HOLTyOp{commands}
\end{SaveVerbatim}
\newcommand{\HOLmessageCertificateDatatypesOrders}{\UseVerbatim{HOLmessageCertificateDatatypesOrders}}
\begin{SaveVerbatim}{HOLmessageCertificateDatatypesOrigin}
\HOLFreeVar{Origin} = \HOLConst{From} \HOLTyOp{missionStaff}
\end{SaveVerbatim}
\newcommand{\HOLmessageCertificateDatatypesOrigin}{\UseVerbatim{HOLmessageCertificateDatatypesOrigin}}
\begin{SaveVerbatim}{HOLmessageCertificateDatatypesRCertSignature}
\HOLFreeVar{RCertSignature} =
    \HOLConst{RCrtSig} (((\HOLTyOp{missionStaff} \HOLTokenProd{}
               \HOLTyOp{missionRoles} \HOLTokenProd{}
               \HOLTyOp{missionStaff} \HOLTokenProd{} \HOLTyOp{missionRoles} \HOLTokenProd{} \HOLTyOp{commands}) \HOLTyOp{digest},
              \HOLTyOp{missionStaffCA}) \HOLTyOp{asymMsg})
\end{SaveVerbatim}
\newcommand{\HOLmessageCertificateDatatypesRCertSignature}{\UseVerbatim{HOLmessageCertificateDatatypesRCertSignature}}
\begin{SaveVerbatim}{HOLmessageCertificateDatatypesRole}
\HOLFreeVar{Role} = \HOLConst{As} \HOLTyOp{missionRoles}
\end{SaveVerbatim}
\newcommand{\HOLmessageCertificateDatatypesRole}{\UseVerbatim{HOLmessageCertificateDatatypesRole}}
\begin{SaveVerbatim}{HOLmessageCertificateDatatypesRoleCertificate}
\HOLFreeVar{RoleCertificate} =
    \HOLConst{RCert} \HOLTyOp{Authority} \HOLTyOp{Role} \HOLTyOp{Delegate} \HOLTyOp{Role} \HOLTyOp{commands} \HOLTyOp{RCertSignature}
\end{SaveVerbatim}
\newcommand{\HOLmessageCertificateDatatypesRoleCertificate}{\UseVerbatim{HOLmessageCertificateDatatypesRoleCertificate}}
\begin{SaveVerbatim}{HOLmessageCertificateDatatypesRootKeyCertificate}
\HOLFreeVar{RootKeyCertificate} = \HOLConst{RootKCert} \HOLTyOp{Subject} \HOLTyOp{SubPubKey}
\end{SaveVerbatim}
\newcommand{\HOLmessageCertificateDatatypesRootKeyCertificate}{\UseVerbatim{HOLmessageCertificateDatatypesRootKeyCertificate}}
\begin{SaveVerbatim}{HOLmessageCertificateDatatypesRootRoleCertificate}
\HOLFreeVar{RootRoleCertificate} = \HOLConst{RootRCert} \HOLTyOp{Delegate} \HOLTyOp{Role} \HOLTyOp{commands}
\end{SaveVerbatim}
\newcommand{\HOLmessageCertificateDatatypesRootRoleCertificate}{\UseVerbatim{HOLmessageCertificateDatatypesRootRoleCertificate}}
\begin{SaveVerbatim}{HOLmessageCertificateDatatypesSubject}
\HOLFreeVar{Subject} = \HOLConst{Entity} \HOLTyOp{missionStaffCA}
\end{SaveVerbatim}
\newcommand{\HOLmessageCertificateDatatypesSubject}{\UseVerbatim{HOLmessageCertificateDatatypesSubject}}
\begin{SaveVerbatim}{HOLmessageCertificateDatatypesSubPubKey}
\HOLFreeVar{SubPubKey} = \HOLConst{SPubKey} (\HOLTyOp{missionStaffCA} \HOLTyOp{pKey})
\end{SaveVerbatim}
\newcommand{\HOLmessageCertificateDatatypesSubPubKey}{\UseVerbatim{HOLmessageCertificateDatatypesSubPubKey}}
\newcommand{\HOLmessageCertificateDatatypes}{
\HOLmessageCertificateDatatypesAuthority\HOLmessageCertificateDatatypesDelegate\HOLmessageCertificateDatatypesDestination\HOLmessageCertificateDatatypesIssuer\HOLmessageCertificateDatatypesKCertSignature\HOLmessageCertificateDatatypesKeyCertificate\HOLmessageCertificateDatatypesMessage\HOLmessageCertificateDatatypesOrders\HOLmessageCertificateDatatypesOrigin\HOLmessageCertificateDatatypesRCertSignature\HOLmessageCertificateDatatypesRole\HOLmessageCertificateDatatypesRoleCertificate\HOLmessageCertificateDatatypesRootKeyCertificate\HOLmessageCertificateDatatypesRootRoleCertificate\HOLmessageCertificateDatatypesSubject\HOLmessageCertificateDatatypesSubPubKey}
\begin{SaveVerbatim}{HOLmessageCertificateDefinitionsksatXXdef}
\HOLTokenTurnstile{} \HOLSymConst{\HOLTokenForall{}}\HOLBoundVar{M} \HOLBoundVar{Oi} \HOLBoundVar{Os} \HOLBoundVar{kcert}.
     (\HOLBoundVar{M}\HOLSymConst{,}\HOLBoundVar{Oi}\HOLSymConst{,}\HOLBoundVar{Os}) \HOLConst{ksat} \HOLBoundVar{kcert} \HOLSymConst{\HOLTokenEquiv{}} (\HOLConst{Ekcrt} \HOLBoundVar{Oi} \HOLBoundVar{Os} \HOLBoundVar{M} \HOLBoundVar{kcert} \HOLSymConst{=} \ensuremath{\cal{U}}(:'world))
\end{SaveVerbatim}
\newcommand{\HOLmessageCertificateDefinitionsksatXXdef}{\UseVerbatim{HOLmessageCertificateDefinitionsksatXXdef}}
\begin{SaveVerbatim}{HOLmessageCertificateDefinitionsmsatXXdef}
\HOLTokenTurnstile{} \HOLSymConst{\HOLTokenForall{}}\HOLBoundVar{M} \HOLBoundVar{Oi} \HOLBoundVar{Os} \HOLBoundVar{msg}.
     (\HOLBoundVar{M}\HOLSymConst{,}\HOLBoundVar{Oi}\HOLSymConst{,}\HOLBoundVar{Os}) \HOLConst{msat} \HOLBoundVar{msg} \HOLSymConst{\HOLTokenEquiv{}} (\HOLConst{Emsg} \HOLBoundVar{Oi} \HOLBoundVar{Os} \HOLBoundVar{M} \HOLBoundVar{msg} \HOLSymConst{=} \ensuremath{\cal{U}}(:'world))
\end{SaveVerbatim}
\newcommand{\HOLmessageCertificateDefinitionsmsatXXdef}{\UseVerbatim{HOLmessageCertificateDefinitionsmsatXXdef}}
\begin{SaveVerbatim}{HOLmessageCertificateDefinitionsroleOriginXXdef}
\HOLTokenTurnstile{} \HOLSymConst{\HOLTokenForall{}}\HOLBoundVar{role}. \HOLConst{roleOrigin} (\HOLConst{As} \HOLBoundVar{role}) \HOLSymConst{=} \HOLBoundVar{role}
\end{SaveVerbatim}
\newcommand{\HOLmessageCertificateDefinitionsroleOriginXXdef}{\UseVerbatim{HOLmessageCertificateDefinitionsroleOriginXXdef}}
\begin{SaveVerbatim}{HOLmessageCertificateDefinitionsrootksatXXdef}
\HOLTokenTurnstile{} \HOLSymConst{\HOLTokenForall{}}\HOLBoundVar{M} \HOLBoundVar{Oi} \HOLBoundVar{Os} \HOLBoundVar{rootkcert}.
     (\HOLBoundVar{M}\HOLSymConst{,}\HOLBoundVar{Oi}\HOLSymConst{,}\HOLBoundVar{Os}) \HOLConst{rootksat} \HOLBoundVar{rootkcert} \HOLSymConst{\HOLTokenEquiv{}}
     (\HOLConst{Erootkcrt} \HOLBoundVar{Oi} \HOLBoundVar{Os} \HOLBoundVar{M} \HOLBoundVar{rootkcert} \HOLSymConst{=} \ensuremath{\cal{U}}(:'world))
\end{SaveVerbatim}
\newcommand{\HOLmessageCertificateDefinitionsrootksatXXdef}{\UseVerbatim{HOLmessageCertificateDefinitionsrootksatXXdef}}
\begin{SaveVerbatim}{HOLmessageCertificateDefinitionsrootrsatXXdef}
\HOLTokenTurnstile{} \HOLSymConst{\HOLTokenForall{}}\HOLBoundVar{M} \HOLBoundVar{Oi} \HOLBoundVar{Os} \HOLBoundVar{rootrcert}.
     (\HOLBoundVar{M}\HOLSymConst{,}\HOLBoundVar{Oi}\HOLSymConst{,}\HOLBoundVar{Os}) \HOLConst{rootrsat} \HOLBoundVar{rootrcert} \HOLSymConst{\HOLTokenEquiv{}}
     (\HOLConst{Erootrcrt} \HOLBoundVar{Oi} \HOLBoundVar{Os} \HOLBoundVar{M} \HOLBoundVar{rootrcert} \HOLSymConst{=} \ensuremath{\cal{U}}(:'world))
\end{SaveVerbatim}
\newcommand{\HOLmessageCertificateDefinitionsrootrsatXXdef}{\UseVerbatim{HOLmessageCertificateDefinitionsrootrsatXXdef}}
\begin{SaveVerbatim}{HOLmessageCertificateDefinitionsrsatXXdef}
\HOLTokenTurnstile{} \HOLSymConst{\HOLTokenForall{}}\HOLBoundVar{M} \HOLBoundVar{Oi} \HOLBoundVar{Os} \HOLBoundVar{rcert}.
     (\HOLBoundVar{M}\HOLSymConst{,}\HOLBoundVar{Oi}\HOLSymConst{,}\HOLBoundVar{Os}) \HOLConst{rsat} \HOLBoundVar{rcert} \HOLSymConst{\HOLTokenEquiv{}} (\HOLConst{Ercrt} \HOLBoundVar{Oi} \HOLBoundVar{Os} \HOLBoundVar{M} \HOLBoundVar{rcert} \HOLSymConst{=} \ensuremath{\cal{U}}(:'world))
\end{SaveVerbatim}
\newcommand{\HOLmessageCertificateDefinitionsrsatXXdef}{\UseVerbatim{HOLmessageCertificateDefinitionsrsatXXdef}}
\begin{SaveVerbatim}{HOLmessageCertificateDefinitionsstaffDestinationXXdef}
\HOLTokenTurnstile{} \HOLSymConst{\HOLTokenForall{}}\HOLBoundVar{staff}. \HOLConst{staffDestination} (\HOLConst{To} \HOLBoundVar{staff}) \HOLSymConst{=} \HOLBoundVar{staff}
\end{SaveVerbatim}
\newcommand{\HOLmessageCertificateDefinitionsstaffDestinationXXdef}{\UseVerbatim{HOLmessageCertificateDefinitionsstaffDestinationXXdef}}
\begin{SaveVerbatim}{HOLmessageCertificateDefinitionsstaffOriginXXdef}
\HOLTokenTurnstile{} \HOLSymConst{\HOLTokenForall{}}\HOLBoundVar{staff}. \HOLConst{staffOrigin} (\HOLConst{From} \HOLBoundVar{staff}) \HOLSymConst{=} \HOLBoundVar{staff}
\end{SaveVerbatim}
\newcommand{\HOLmessageCertificateDefinitionsstaffOriginXXdef}{\UseVerbatim{HOLmessageCertificateDefinitionsstaffOriginXXdef}}
\newcommand{\HOLmessageCertificateDefinitions}{
\HOLDfnTag{messageCertificate}{ksat_def}\HOLmessageCertificateDefinitionsksatXXdef
\HOLDfnTag{messageCertificate}{msat_def}\HOLmessageCertificateDefinitionsmsatXXdef
\HOLDfnTag{messageCertificate}{roleOrigin_def}\HOLmessageCertificateDefinitionsroleOriginXXdef
\HOLDfnTag{messageCertificate}{rootksat_def}\HOLmessageCertificateDefinitionsrootksatXXdef
\HOLDfnTag{messageCertificate}{rootrsat_def}\HOLmessageCertificateDefinitionsrootrsatXXdef
\HOLDfnTag{messageCertificate}{rsat_def}\HOLmessageCertificateDefinitionsrsatXXdef
\HOLDfnTag{messageCertificate}{staffDestination_def}\HOLmessageCertificateDefinitionsstaffDestinationXXdef
\HOLDfnTag{messageCertificate}{staffOrigin_def}\HOLmessageCertificateDefinitionsstaffOriginXXdef
}
\begin{SaveVerbatim}{HOLmessageCertificateTheoremscheckKCertXXdef}
\HOLTokenTurnstile{} \HOLConst{checkKCert}
     (\HOLConst{KCert} (\HOLConst{CA} \HOLFreeVar{mca}) (\HOLConst{Entity} \HOLFreeVar{mStaffCA}) (\HOLConst{SPubKey} \HOLFreeVar{pubKey})
        (\HOLConst{KCrtSig} \HOLFreeVar{kcertSig})) \HOLSymConst{\HOLTokenEquiv{}}
   \HOLConst{signVerify} (\HOLConst{KCA} \HOLFreeVar{mca}) \HOLFreeVar{kcertSig} (\HOLConst{plain} (\HOLFreeVar{mca}\HOLSymConst{,}\HOLFreeVar{mStaffCA}\HOLSymConst{,}\HOLFreeVar{pubKey}))
\end{SaveVerbatim}
\newcommand{\HOLmessageCertificateTheoremscheckKCertXXdef}{\UseVerbatim{HOLmessageCertificateTheoremscheckKCertXXdef}}
\begin{SaveVerbatim}{HOLmessageCertificateTheoremscheckKCertXXind}
\HOLTokenTurnstile{} \HOLSymConst{\HOLTokenForall{}}\HOLBoundVar{P}.
     (\HOLSymConst{\HOLTokenForall{}}\HOLBoundVar{mca} \HOLBoundVar{mStaffCA} \HOLBoundVar{pubKey} \HOLBoundVar{kcertSig}.
        \HOLBoundVar{P}
          (\HOLConst{KCert} (\HOLConst{CA} \HOLBoundVar{mca}) (\HOLConst{Entity} \HOLBoundVar{mStaffCA}) (\HOLConst{SPubKey} \HOLBoundVar{pubKey})
             (\HOLConst{KCrtSig} \HOLBoundVar{kcertSig}))) \HOLSymConst{\HOLTokenImp{}}
     \HOLSymConst{\HOLTokenForall{}}\HOLBoundVar{v}. \HOLBoundVar{P} \HOLBoundVar{v}
\end{SaveVerbatim}
\newcommand{\HOLmessageCertificateTheoremscheckKCertXXind}{\UseVerbatim{HOLmessageCertificateTheoremscheckKCertXXind}}
\begin{SaveVerbatim}{HOLmessageCertificateTheoremscheckKCertOK}
\HOLTokenTurnstile{} \HOLConst{checkKCert}
     (\HOLConst{KCert} (\HOLConst{CA} \HOLFreeVar{mca}) (\HOLConst{Entity} \HOLFreeVar{mStaffCA}) (\HOLConst{SPubKey} \HOLFreeVar{pubKey})
        (\HOLConst{KCrtSig}
           (\HOLConst{sign} (\HOLConst{KCA} \HOLFreeVar{mca})
              (\HOLConst{hash} (\HOLConst{plain} (\HOLFreeVar{mca}\HOLSymConst{,}\HOLFreeVar{mStaffCA}\HOLSymConst{,}\HOLFreeVar{pubKey}))))))
\end{SaveVerbatim}
\newcommand{\HOLmessageCertificateTheoremscheckKCertOK}{\UseVerbatim{HOLmessageCertificateTheoremscheckKCertOK}}
\begin{SaveVerbatim}{HOLmessageCertificateTheoremscheckMSGXXdef}
\HOLTokenTurnstile{} \HOLConst{checkMSG}
     (\HOLConst{MSG} (\HOLConst{From} \HOLFreeVar{sender}) (\HOLConst{As} \HOLFreeVar{role}) (\HOLConst{To} \HOLFreeVar{recipient}) \HOLFreeVar{enDEK} \HOLFreeVar{enCMDMsg}
        \HOLFreeVar{CMDMsgSig}) \HOLSymConst{\HOLTokenEquiv{}}
   (\HOLKeyword{let} \HOLBoundVar{dek} = \HOLConst{getMessage} (\HOLConst{deciphP} (\HOLConst{privK} \HOLFreeVar{recipient}) \HOLFreeVar{enDEK}) \HOLKeyword{in}
    \HOLKeyword{let} \HOLBoundVar{msgContents} = \HOLConst{deciphS} \HOLBoundVar{dek} \HOLFreeVar{enCMDMsg}
    \HOLKeyword{in}
      \HOLConst{signVerify} \HOLFreeVar{sender} \HOLFreeVar{CMDMsgSig} \HOLBoundVar{msgContents})
\end{SaveVerbatim}
\newcommand{\HOLmessageCertificateTheoremscheckMSGXXdef}{\UseVerbatim{HOLmessageCertificateTheoremscheckMSGXXdef}}
\begin{SaveVerbatim}{HOLmessageCertificateTheoremscheckMSGXXind}
\HOLTokenTurnstile{} \HOLSymConst{\HOLTokenForall{}}\HOLBoundVar{P}.
     (\HOLSymConst{\HOLTokenForall{}}\HOLBoundVar{sender} \HOLBoundVar{role} \HOLBoundVar{recipient} \HOLBoundVar{enDEK} \HOLBoundVar{enCMDMsg} \HOLBoundVar{CMDMsgSig}.
        \HOLBoundVar{P}
          (\HOLConst{MSG} (\HOLConst{From} \HOLBoundVar{sender}) (\HOLConst{As} \HOLBoundVar{role}) (\HOLConst{To} \HOLBoundVar{recipient}) \HOLBoundVar{enDEK}
             \HOLBoundVar{enCMDMsg} \HOLBoundVar{CMDMsgSig})) \HOLSymConst{\HOLTokenImp{}}
     \HOLSymConst{\HOLTokenForall{}}\HOLBoundVar{v}. \HOLBoundVar{P} \HOLBoundVar{v}
\end{SaveVerbatim}
\newcommand{\HOLmessageCertificateTheoremscheckMSGXXind}{\UseVerbatim{HOLmessageCertificateTheoremscheckMSGXXind}}
\begin{SaveVerbatim}{HOLmessageCertificateTheoremscheckMSGOK}
\HOLTokenTurnstile{} \HOLConst{checkMSG}
     (\HOLConst{MSG} (\HOLConst{From} \HOLFreeVar{sender}) (\HOLConst{As} \HOLFreeVar{role}) (\HOLConst{To} \HOLFreeVar{recipient})
        (\HOLConst{Ea} (\HOLConst{pubK} \HOLFreeVar{recipient}) (\HOLConst{plain} (\HOLConst{sym} \HOLFreeVar{dek})))
        (\HOLConst{Es} (\HOLConst{sym} \HOLFreeVar{dek}) (\HOLConst{plain} \HOLFreeVar{order}))
        (\HOLConst{sign} \HOLFreeVar{sender} (\HOLConst{hash} (\HOLConst{plain} \HOLFreeVar{order}))))
\end{SaveVerbatim}
\newcommand{\HOLmessageCertificateTheoremscheckMSGOK}{\UseVerbatim{HOLmessageCertificateTheoremscheckMSGOK}}
\begin{SaveVerbatim}{HOLmessageCertificateTheoremscheckRCertXXdef}
\HOLTokenTurnstile{} \HOLConst{checkRCert}
     (\HOLConst{RCert} (\HOLConst{Auth} \HOLFreeVar{commander}) (\HOLConst{As} \HOLFreeVar{cmdRole}) (\HOLConst{For} \HOLFreeVar{delegate})
        (\HOLConst{As} \HOLFreeVar{delegateRole}) \HOLFreeVar{order} (\HOLConst{RCrtSig} \HOLFreeVar{rcertSig})) \HOLSymConst{\HOLTokenEquiv{}}
   \HOLConst{signVerify} (\HOLConst{KStaff} \HOLFreeVar{commander}) \HOLFreeVar{rcertSig}
     (\HOLConst{plain} (\HOLFreeVar{commander}\HOLSymConst{,}\HOLFreeVar{cmdRole}\HOLSymConst{,}\HOLFreeVar{delegate}\HOLSymConst{,}\HOLFreeVar{delegateRole}\HOLSymConst{,}\HOLFreeVar{order}))
\end{SaveVerbatim}
\newcommand{\HOLmessageCertificateTheoremscheckRCertXXdef}{\UseVerbatim{HOLmessageCertificateTheoremscheckRCertXXdef}}
\begin{SaveVerbatim}{HOLmessageCertificateTheoremscheckRCertXXind}
\HOLTokenTurnstile{} \HOLSymConst{\HOLTokenForall{}}\HOLBoundVar{P}.
     (\HOLSymConst{\HOLTokenForall{}}\HOLBoundVar{commander} \HOLBoundVar{cmdRole} \HOLBoundVar{delegate} \HOLBoundVar{delegateRole} \HOLBoundVar{order} \HOLBoundVar{rcertSig}.
        \HOLBoundVar{P}
          (\HOLConst{RCert} (\HOLConst{Auth} \HOLBoundVar{commander}) (\HOLConst{As} \HOLBoundVar{cmdRole}) (\HOLConst{For} \HOLBoundVar{delegate})
             (\HOLConst{As} \HOLBoundVar{delegateRole}) \HOLBoundVar{order} (\HOLConst{RCrtSig} \HOLBoundVar{rcertSig}))) \HOLSymConst{\HOLTokenImp{}}
     \HOLSymConst{\HOLTokenForall{}}\HOLBoundVar{v}. \HOLBoundVar{P} \HOLBoundVar{v}
\end{SaveVerbatim}
\newcommand{\HOLmessageCertificateTheoremscheckRCertXXind}{\UseVerbatim{HOLmessageCertificateTheoremscheckRCertXXind}}
\begin{SaveVerbatim}{HOLmessageCertificateTheoremscheckRCertOK}
\HOLTokenTurnstile{} \HOLConst{checkRCert}
     (\HOLConst{RCert} (\HOLConst{Auth} \HOLFreeVar{commander}) (\HOLConst{As} \HOLFreeVar{cmdRole}) (\HOLConst{For} \HOLFreeVar{delegate})
        (\HOLConst{As} \HOLFreeVar{delegateRole}) \HOLFreeVar{order}
        (\HOLConst{RCrtSig}
           (\HOLConst{sign} (\HOLConst{KStaff} \HOLFreeVar{commander})
              (\HOLConst{hash}
                 (\HOLConst{plain}
                    (\HOLFreeVar{commander}\HOLSymConst{,}\HOLFreeVar{cmdRole}\HOLSymConst{,}\HOLFreeVar{delegate}\HOLSymConst{,}\HOLFreeVar{delegateRole}\HOLSymConst{,}
                     \HOLFreeVar{order}))))))
\end{SaveVerbatim}
\newcommand{\HOLmessageCertificateTheoremscheckRCertOK}{\UseVerbatim{HOLmessageCertificateTheoremscheckRCertOK}}
\begin{SaveVerbatim}{HOLmessageCertificateTheoremsEkcrtXXdef}
\HOLTokenTurnstile{} (\HOLConst{Ekcrt} \HOLFreeVar{Oi} \HOLFreeVar{Os} \HOLFreeVar{M}
      (\HOLConst{KCert} (\HOLConst{CA} \HOLFreeVar{mca}) (\HOLConst{Entity} (\HOLConst{KStaff} \HOLFreeVar{staff})) (\HOLConst{SPubKey} \HOLFreeVar{pubKey})
         (\HOLConst{KCrtSig} \HOLFreeVar{kcertSig})) \HOLSymConst{=}
    \HOLKeyword{if}
      \HOLConst{checkKCert}
        (\HOLConst{KCert} (\HOLConst{CA} \HOLFreeVar{mca}) (\HOLConst{Entity} (\HOLConst{KStaff} \HOLFreeVar{staff})) (\HOLConst{SPubKey} \HOLFreeVar{pubKey})
           (\HOLConst{KCrtSig} \HOLFreeVar{kcertSig}))
    \HOLKeyword{then}
      \HOLConst{Efn} \HOLFreeVar{Oi} \HOLFreeVar{Os} \HOLFreeVar{M}
        (\HOLConst{Name} (\HOLConst{MKey} (\HOLConst{pubK} (\HOLConst{KCA} \HOLFreeVar{mca}))) \HOLConst{says}
         \HOLConst{Name} (\HOLConst{MKey} \HOLFreeVar{pubKey}) \HOLConst{speaks_for} \HOLConst{Name} (\HOLConst{MStaff} \HOLFreeVar{staff}))
    \HOLKeyword{else} \HOLTokenLeftbrace{}\HOLTokenRightbrace{}) \HOLSymConst{\HOLTokenConj{}}
   (\HOLConst{Ekcrt} \HOLFreeVar{Oi} \HOLFreeVar{Os} \HOLFreeVar{M}
      (\HOLConst{KCert} (\HOLConst{CA} \HOLFreeVar{mca}) (\HOLConst{Entity} (\HOLConst{KCA} \HOLFreeVar{ca\sb{\mathrm{2}}})) (\HOLConst{SPubKey} \HOLFreeVar{pubKey})
         (\HOLConst{KCrtSig} \HOLFreeVar{kcertSig})) \HOLSymConst{=}
    \HOLKeyword{if}
      \HOLConst{checkKCert}
        (\HOLConst{KCert} (\HOLConst{CA} \HOLFreeVar{mca}) (\HOLConst{Entity} (\HOLConst{KCA} \HOLFreeVar{ca\sb{\mathrm{2}}})) (\HOLConst{SPubKey} \HOLFreeVar{pubKey})
           (\HOLConst{KCrtSig} \HOLFreeVar{kcertSig}))
    \HOLKeyword{then}
      \HOLConst{Efn} \HOLFreeVar{Oi} \HOLFreeVar{Os} \HOLFreeVar{M}
        (\HOLConst{Name} (\HOLConst{MKey} (\HOLConst{pubK} (\HOLConst{KCA} \HOLFreeVar{mca}))) \HOLConst{says}
         \HOLConst{Name} (\HOLConst{MKey} \HOLFreeVar{pubKey}) \HOLConst{speaks_for} \HOLConst{Name} (\HOLConst{MCA} \HOLFreeVar{ca\sb{\mathrm{2}}}))
    \HOLKeyword{else} \HOLTokenLeftbrace{}\HOLTokenRightbrace{})
\end{SaveVerbatim}
\newcommand{\HOLmessageCertificateTheoremsEkcrtXXdef}{\UseVerbatim{HOLmessageCertificateTheoremsEkcrtXXdef}}
\begin{SaveVerbatim}{HOLmessageCertificateTheoremsEkcrtXXind}
\HOLTokenTurnstile{} \HOLSymConst{\HOLTokenForall{}}\HOLBoundVar{P}.
     (\HOLSymConst{\HOLTokenForall{}}\HOLBoundVar{Oi} \HOLBoundVar{Os} \HOLBoundVar{M} \HOLBoundVar{mca} \HOLBoundVar{staff} \HOLBoundVar{pubKey} \HOLBoundVar{kcertSig}.
        \HOLBoundVar{P} \HOLBoundVar{Oi} \HOLBoundVar{Os} \HOLBoundVar{M}
          (\HOLConst{KCert} (\HOLConst{CA} \HOLBoundVar{mca}) (\HOLConst{Entity} (\HOLConst{KStaff} \HOLBoundVar{staff}))
             (\HOLConst{SPubKey} \HOLBoundVar{pubKey}) (\HOLConst{KCrtSig} \HOLBoundVar{kcertSig}))) \HOLSymConst{\HOLTokenConj{}}
     (\HOLSymConst{\HOLTokenForall{}}\HOLBoundVar{Oi} \HOLBoundVar{Os} \HOLBoundVar{M} \HOLBoundVar{mca} \HOLBoundVar{ca\sb{\mathrm{2}}} \HOLBoundVar{pubKey} \HOLBoundVar{kcertSig}.
        \HOLBoundVar{P} \HOLBoundVar{Oi} \HOLBoundVar{Os} \HOLBoundVar{M}
          (\HOLConst{KCert} (\HOLConst{CA} \HOLBoundVar{mca}) (\HOLConst{Entity} (\HOLConst{KCA} \HOLBoundVar{ca\sb{\mathrm{2}}})) (\HOLConst{SPubKey} \HOLBoundVar{pubKey})
             (\HOLConst{KCrtSig} \HOLBoundVar{kcertSig}))) \HOLSymConst{\HOLTokenImp{}}
     \HOLSymConst{\HOLTokenForall{}}\HOLBoundVar{v} \HOLBoundVar{v\sb{\mathrm{1}}} \HOLBoundVar{v\sb{\mathrm{2}}} \HOLBoundVar{v\sb{\mathrm{3}}}. \HOLBoundVar{P} \HOLBoundVar{v} \HOLBoundVar{v\sb{\mathrm{1}}} \HOLBoundVar{v\sb{\mathrm{2}}} \HOLBoundVar{v\sb{\mathrm{3}}}
\end{SaveVerbatim}
\newcommand{\HOLmessageCertificateTheoremsEkcrtXXind}{\UseVerbatim{HOLmessageCertificateTheoremsEkcrtXXind}}
\begin{SaveVerbatim}{HOLmessageCertificateTheoremsEmsgXXdef}
\HOLTokenTurnstile{} \HOLConst{Emsg} \HOLFreeVar{Oi} \HOLFreeVar{Os} \HOLFreeVar{M}
     (\HOLConst{MSG} (\HOLConst{From} \HOLFreeVar{sender}) (\HOLConst{As} \HOLFreeVar{role}) (\HOLConst{To} \HOLFreeVar{recipient}) \HOLFreeVar{enDEK} \HOLFreeVar{enCMD}
        \HOLFreeVar{CMDsig}) \HOLSymConst{=}
   \HOLKeyword{if}
     \HOLConst{checkMSG}
       (\HOLConst{MSG} (\HOLConst{From} \HOLFreeVar{sender}) (\HOLConst{As} \HOLFreeVar{role}) (\HOLConst{To} \HOLFreeVar{recipient}) \HOLFreeVar{enDEK} \HOLFreeVar{enCMD}
          \HOLFreeVar{CMDsig})
   \HOLKeyword{then}
     (\HOLKeyword{let} \HOLBoundVar{order} =
            \HOLConst{getCommand}
              (\HOLConst{MSG} (\HOLConst{From} \HOLFreeVar{sender}) (\HOLConst{As} \HOLFreeVar{role}) (\HOLConst{To} \HOLFreeVar{recipient}) \HOLFreeVar{enDEK}
                 \HOLFreeVar{enCMD} \HOLFreeVar{CMDsig})
      \HOLKeyword{in}
      \HOLKeyword{let} \HOLBoundVar{x} = \HOLConst{ordersParameters} \HOLBoundVar{order}
      \HOLKeyword{in}
        \HOLKeyword{if} \HOLBoundVar{x} \HOLSymConst{=} (\HOLFreeVar{sender}\HOLSymConst{,}\HOLFreeVar{role}\HOLSymConst{,}\HOLFreeVar{recipient}) \HOLKeyword{then}
          \HOLConst{Efn} \HOLFreeVar{Oi} \HOLFreeVar{Os} \HOLFreeVar{M}
            (\HOLConst{Name} (\HOLConst{MKey} (\HOLConst{pubK} (\HOLConst{KStaff} \HOLFreeVar{sender}))) \HOLConst{quoting}
             \HOLConst{Name} (\HOLConst{MRole} \HOLFreeVar{role}) \HOLConst{says} \HOLConst{prop} (\HOLConst{ordersCommand} \HOLBoundVar{order}))
        \HOLKeyword{else} \HOLTokenLeftbrace{}\HOLTokenRightbrace{})
   \HOLKeyword{else} \HOLTokenLeftbrace{}\HOLTokenRightbrace{}
\end{SaveVerbatim}
\newcommand{\HOLmessageCertificateTheoremsEmsgXXdef}{\UseVerbatim{HOLmessageCertificateTheoremsEmsgXXdef}}
\begin{SaveVerbatim}{HOLmessageCertificateTheoremsEmsgXXind}
\HOLTokenTurnstile{} \HOLSymConst{\HOLTokenForall{}}\HOLBoundVar{P}.
     (\HOLSymConst{\HOLTokenForall{}}\HOLBoundVar{Oi} \HOLBoundVar{Os} \HOLBoundVar{M} \HOLBoundVar{sender} \HOLBoundVar{role} \HOLBoundVar{recipient} \HOLBoundVar{enDEK} \HOLBoundVar{enCMD} \HOLBoundVar{CMDsig}.
        \HOLBoundVar{P} \HOLBoundVar{Oi} \HOLBoundVar{Os} \HOLBoundVar{M}
          (\HOLConst{MSG} (\HOLConst{From} \HOLBoundVar{sender}) (\HOLConst{As} \HOLBoundVar{role}) (\HOLConst{To} \HOLBoundVar{recipient}) \HOLBoundVar{enDEK}
             \HOLBoundVar{enCMD} \HOLBoundVar{CMDsig})) \HOLSymConst{\HOLTokenImp{}}
     \HOLSymConst{\HOLTokenForall{}}\HOLBoundVar{v} \HOLBoundVar{v\sb{\mathrm{1}}} \HOLBoundVar{v\sb{\mathrm{2}}} \HOLBoundVar{v\sb{\mathrm{3}}}. \HOLBoundVar{P} \HOLBoundVar{v} \HOLBoundVar{v\sb{\mathrm{1}}} \HOLBoundVar{v\sb{\mathrm{2}}} \HOLBoundVar{v\sb{\mathrm{3}}}
\end{SaveVerbatim}
\newcommand{\HOLmessageCertificateTheoremsEmsgXXind}{\UseVerbatim{HOLmessageCertificateTheoremsEmsgXXind}}
\begin{SaveVerbatim}{HOLmessageCertificateTheoremsErcrtXXdef}
\HOLTokenTurnstile{} \HOLConst{Ercrt} \HOLFreeVar{Oi} \HOLFreeVar{Os} \HOLFreeVar{M}
     (\HOLConst{RCert} (\HOLConst{Auth} \HOLFreeVar{commander}) (\HOLConst{As} \HOLFreeVar{cmdRole}) (\HOLConst{For} \HOLFreeVar{delegate})
        (\HOLConst{As} \HOLFreeVar{delegateRole}) \HOLFreeVar{order} (\HOLConst{RCrtSig} \HOLFreeVar{rcertSig})) \HOLSymConst{=}
   \HOLKeyword{if}
     \HOLConst{checkRCert}
       (\HOLConst{RCert} (\HOLConst{Auth} \HOLFreeVar{commander}) (\HOLConst{As} \HOLFreeVar{cmdRole}) (\HOLConst{For} \HOLFreeVar{delegate})
          (\HOLConst{As} \HOLFreeVar{delegateRole}) \HOLFreeVar{order} (\HOLConst{RCrtSig} \HOLFreeVar{rcertSig}))
   \HOLKeyword{then}
     \HOLConst{Efn} \HOLFreeVar{Oi} \HOLFreeVar{Os} \HOLFreeVar{M}
       (\HOLConst{Name} (\HOLConst{MKey} (\HOLConst{pubK} (\HOLConst{KStaff} \HOLFreeVar{commander}))) \HOLConst{says}
        \HOLConst{reps} (\HOLConst{Name} (\HOLConst{MStaff} \HOLFreeVar{delegate}))
          (\HOLConst{Name} (\HOLConst{MRole} \HOLFreeVar{delegateRole})) (\HOLConst{prop} \HOLFreeVar{order}))
   \HOLKeyword{else} \HOLTokenLeftbrace{}\HOLTokenRightbrace{}
\end{SaveVerbatim}
\newcommand{\HOLmessageCertificateTheoremsErcrtXXdef}{\UseVerbatim{HOLmessageCertificateTheoremsErcrtXXdef}}
\begin{SaveVerbatim}{HOLmessageCertificateTheoremsErcrtXXind}
\HOLTokenTurnstile{} \HOLSymConst{\HOLTokenForall{}}\HOLBoundVar{P}.
     (\HOLSymConst{\HOLTokenForall{}}\HOLBoundVar{Oi} \HOLBoundVar{Os} \HOLBoundVar{M} \HOLBoundVar{commander} \HOLBoundVar{cmdRole} \HOLBoundVar{delegate} \HOLBoundVar{delegateRole} \HOLBoundVar{order}
         \HOLBoundVar{rcertSig}.
        \HOLBoundVar{P} \HOLBoundVar{Oi} \HOLBoundVar{Os} \HOLBoundVar{M}
          (\HOLConst{RCert} (\HOLConst{Auth} \HOLBoundVar{commander}) (\HOLConst{As} \HOLBoundVar{cmdRole}) (\HOLConst{For} \HOLBoundVar{delegate})
             (\HOLConst{As} \HOLBoundVar{delegateRole}) \HOLBoundVar{order} (\HOLConst{RCrtSig} \HOLBoundVar{rcertSig}))) \HOLSymConst{\HOLTokenImp{}}
     \HOLSymConst{\HOLTokenForall{}}\HOLBoundVar{v} \HOLBoundVar{v\sb{\mathrm{1}}} \HOLBoundVar{v\sb{\mathrm{2}}} \HOLBoundVar{v\sb{\mathrm{3}}}. \HOLBoundVar{P} \HOLBoundVar{v} \HOLBoundVar{v\sb{\mathrm{1}}} \HOLBoundVar{v\sb{\mathrm{2}}} \HOLBoundVar{v\sb{\mathrm{3}}}
\end{SaveVerbatim}
\newcommand{\HOLmessageCertificateTheoremsErcrtXXind}{\UseVerbatim{HOLmessageCertificateTheoremsErcrtXXind}}
\begin{SaveVerbatim}{HOLmessageCertificateTheoremsErootkcrtXXdef}
\HOLTokenTurnstile{} (\HOLConst{Erootkcrt} \HOLFreeVar{Oi} \HOLFreeVar{Os} \HOLFreeVar{M}
      (\HOLConst{RootKCert} (\HOLConst{Entity} (\HOLConst{KStaff} \HOLFreeVar{staff})) (\HOLConst{SPubKey} \HOLFreeVar{pubKey})) \HOLSymConst{=}
    \HOLConst{Efn} \HOLFreeVar{Oi} \HOLFreeVar{Os} \HOLFreeVar{M}
      (\HOLConst{Name} (\HOLConst{MKey} \HOLFreeVar{pubKey}) \HOLConst{speaks_for} \HOLConst{Name} (\HOLConst{MStaff} \HOLFreeVar{staff}))) \HOLSymConst{\HOLTokenConj{}}
   (\HOLConst{Erootkcrt} \HOLFreeVar{Oi} \HOLFreeVar{Os} \HOLFreeVar{M}
      (\HOLConst{RootKCert} (\HOLConst{Entity} (\HOLConst{KCA} \HOLFreeVar{ca\sb{\mathrm{2}}})) (\HOLConst{SPubKey} \HOLFreeVar{pubKey})) \HOLSymConst{=}
    \HOLConst{Efn} \HOLFreeVar{Oi} \HOLFreeVar{Os} \HOLFreeVar{M} (\HOLConst{Name} (\HOLConst{MKey} \HOLFreeVar{pubKey}) \HOLConst{speaks_for} \HOLConst{Name} (\HOLConst{MCA} \HOLFreeVar{ca\sb{\mathrm{2}}})))
\end{SaveVerbatim}
\newcommand{\HOLmessageCertificateTheoremsErootkcrtXXdef}{\UseVerbatim{HOLmessageCertificateTheoremsErootkcrtXXdef}}
\begin{SaveVerbatim}{HOLmessageCertificateTheoremsErootkcrtXXind}
\HOLTokenTurnstile{} \HOLSymConst{\HOLTokenForall{}}\HOLBoundVar{P}.
     (\HOLSymConst{\HOLTokenForall{}}\HOLBoundVar{Oi} \HOLBoundVar{Os} \HOLBoundVar{M} \HOLBoundVar{staff} \HOLBoundVar{pubKey}.
        \HOLBoundVar{P} \HOLBoundVar{Oi} \HOLBoundVar{Os} \HOLBoundVar{M}
          (\HOLConst{RootKCert} (\HOLConst{Entity} (\HOLConst{KStaff} \HOLBoundVar{staff}))
             (\HOLConst{SPubKey} \HOLBoundVar{pubKey}))) \HOLSymConst{\HOLTokenConj{}}
     (\HOLSymConst{\HOLTokenForall{}}\HOLBoundVar{Oi} \HOLBoundVar{Os} \HOLBoundVar{M} \HOLBoundVar{ca\sb{\mathrm{2}}} \HOLBoundVar{pubKey}.
        \HOLBoundVar{P} \HOLBoundVar{Oi} \HOLBoundVar{Os} \HOLBoundVar{M}
          (\HOLConst{RootKCert} (\HOLConst{Entity} (\HOLConst{KCA} \HOLBoundVar{ca\sb{\mathrm{2}}})) (\HOLConst{SPubKey} \HOLBoundVar{pubKey}))) \HOLSymConst{\HOLTokenImp{}}
     \HOLSymConst{\HOLTokenForall{}}\HOLBoundVar{v} \HOLBoundVar{v\sb{\mathrm{1}}} \HOLBoundVar{v\sb{\mathrm{2}}} \HOLBoundVar{v\sb{\mathrm{3}}}. \HOLBoundVar{P} \HOLBoundVar{v} \HOLBoundVar{v\sb{\mathrm{1}}} \HOLBoundVar{v\sb{\mathrm{2}}} \HOLBoundVar{v\sb{\mathrm{3}}}
\end{SaveVerbatim}
\newcommand{\HOLmessageCertificateTheoremsErootkcrtXXind}{\UseVerbatim{HOLmessageCertificateTheoremsErootkcrtXXind}}
\begin{SaveVerbatim}{HOLmessageCertificateTheoremsErootrcrtXXdef}
\HOLTokenTurnstile{} \HOLConst{Erootrcrt} \HOLFreeVar{Oi} \HOLFreeVar{Os} \HOLFreeVar{M}
     (\HOLConst{RootRCert} (\HOLConst{For} \HOLFreeVar{delegate}) (\HOLConst{As} \HOLFreeVar{delegateRole}) \HOLFreeVar{order}) \HOLSymConst{=}
   \HOLConst{Efn} \HOLFreeVar{Oi} \HOLFreeVar{Os} \HOLFreeVar{M}
     (\HOLConst{reps} (\HOLConst{Name} (\HOLConst{MStaff} \HOLFreeVar{delegate})) (\HOLConst{Name} (\HOLConst{MRole} \HOLFreeVar{delegateRole}))
        (\HOLConst{prop} \HOLFreeVar{order}))
\end{SaveVerbatim}
\newcommand{\HOLmessageCertificateTheoremsErootrcrtXXdef}{\UseVerbatim{HOLmessageCertificateTheoremsErootrcrtXXdef}}
\begin{SaveVerbatim}{HOLmessageCertificateTheoremsErootrcrtXXind}
\HOLTokenTurnstile{} \HOLSymConst{\HOLTokenForall{}}\HOLBoundVar{P}.
     (\HOLSymConst{\HOLTokenForall{}}\HOLBoundVar{Oi} \HOLBoundVar{Os} \HOLBoundVar{M} \HOLBoundVar{delegate} \HOLBoundVar{delegateRole} \HOLBoundVar{order}.
        \HOLBoundVar{P} \HOLBoundVar{Oi} \HOLBoundVar{Os} \HOLBoundVar{M}
          (\HOLConst{RootRCert} (\HOLConst{For} \HOLBoundVar{delegate}) (\HOLConst{As} \HOLBoundVar{delegateRole}) \HOLBoundVar{order})) \HOLSymConst{\HOLTokenImp{}}
     \HOLSymConst{\HOLTokenForall{}}\HOLBoundVar{v} \HOLBoundVar{v\sb{\mathrm{1}}} \HOLBoundVar{v\sb{\mathrm{2}}} \HOLBoundVar{v\sb{\mathrm{3}}}. \HOLBoundVar{P} \HOLBoundVar{v} \HOLBoundVar{v\sb{\mathrm{1}}} \HOLBoundVar{v\sb{\mathrm{2}}} \HOLBoundVar{v\sb{\mathrm{3}}}
\end{SaveVerbatim}
\newcommand{\HOLmessageCertificateTheoremsErootrcrtXXind}{\UseVerbatim{HOLmessageCertificateTheoremsErootrcrtXXind}}
\begin{SaveVerbatim}{HOLmessageCertificateTheoremsgetCommandXXdef}
\HOLTokenTurnstile{} \HOLConst{getCommand}
     (\HOLConst{MSG} (\HOLConst{From} \HOLFreeVar{sender}) (\HOLConst{As} \HOLFreeVar{role}) (\HOLConst{To} \HOLFreeVar{recipient}) \HOLFreeVar{enDEK} \HOLFreeVar{enCMDMsg}
        \HOLFreeVar{CMDMsgSig}) \HOLSymConst{=}
   (\HOLKeyword{let} \HOLBoundVar{dek} = \HOLConst{getMessage} (\HOLConst{deciphP} (\HOLConst{privK} \HOLFreeVar{recipient}) \HOLFreeVar{enDEK})
    \HOLKeyword{in}
      \HOLConst{getMessage} (\HOLConst{deciphS} \HOLBoundVar{dek} \HOLFreeVar{enCMDMsg}))
\end{SaveVerbatim}
\newcommand{\HOLmessageCertificateTheoremsgetCommandXXdef}{\UseVerbatim{HOLmessageCertificateTheoremsgetCommandXXdef}}
\begin{SaveVerbatim}{HOLmessageCertificateTheoremsgetCommandXXind}
\HOLTokenTurnstile{} \HOLSymConst{\HOLTokenForall{}}\HOLBoundVar{P}.
     (\HOLSymConst{\HOLTokenForall{}}\HOLBoundVar{sender} \HOLBoundVar{role} \HOLBoundVar{recipient} \HOLBoundVar{enDEK} \HOLBoundVar{enCMDMsg} \HOLBoundVar{CMDMsgSig}.
        \HOLBoundVar{P}
          (\HOLConst{MSG} (\HOLConst{From} \HOLBoundVar{sender}) (\HOLConst{As} \HOLBoundVar{role}) (\HOLConst{To} \HOLBoundVar{recipient}) \HOLBoundVar{enDEK}
             \HOLBoundVar{enCMDMsg} \HOLBoundVar{CMDMsgSig})) \HOLSymConst{\HOLTokenImp{}}
     \HOLSymConst{\HOLTokenForall{}}\HOLBoundVar{v}. \HOLBoundVar{P} \HOLBoundVar{v}
\end{SaveVerbatim}
\newcommand{\HOLmessageCertificateTheoremsgetCommandXXind}{\UseVerbatim{HOLmessageCertificateTheoremsgetCommandXXind}}
\begin{SaveVerbatim}{HOLmessageCertificateTheoremsgetCommandXXthm}
\HOLTokenTurnstile{} \HOLConst{getCommand}
     (\HOLConst{MSG} (\HOLConst{From} \HOLFreeVar{sender}) (\HOLConst{As} \HOLFreeVar{role}) (\HOLConst{To} \HOLFreeVar{recipient})
        (\HOLConst{Ea} (\HOLConst{pubK} \HOLFreeVar{recipient}) (\HOLConst{plain} (\HOLConst{sym} \HOLFreeVar{dek})))
        (\HOLConst{Es} (\HOLConst{sym} \HOLFreeVar{dek}) (\HOLConst{plain} \HOLFreeVar{order}))
        (\HOLConst{sign} \HOLFreeVar{sender} (\HOLConst{hash} (\HOLConst{plain} \HOLFreeVar{order})))) \HOLSymConst{=}
   \HOLFreeVar{order}
\end{SaveVerbatim}
\newcommand{\HOLmessageCertificateTheoremsgetCommandXXthm}{\UseVerbatim{HOLmessageCertificateTheoremsgetCommandXXthm}}
\begin{SaveVerbatim}{HOLmessageCertificateTheoremskcertCASatXXthm}
\HOLTokenTurnstile{} \HOLSymConst{\HOLTokenForall{}}\HOLBoundVar{mca} \HOLBoundVar{ca\sb{\mathrm{2}}} \HOLBoundVar{Os} \HOLBoundVar{Oi} \HOLBoundVar{M}.
     (\HOLBoundVar{M}\HOLSymConst{,}\HOLBoundVar{Oi}\HOLSymConst{,}\HOLBoundVar{Os}) \HOLConst{sat}
     \HOLConst{Name} (\HOLConst{MKey} (\HOLConst{pubK} (\HOLConst{KCA} \HOLBoundVar{mca}))) \HOLConst{says}
     \HOLConst{Name} (\HOLConst{MKey} (\HOLConst{pubK} (\HOLConst{KCA} \HOLBoundVar{ca\sb{\mathrm{2}}}))) \HOLConst{speaks_for} \HOLConst{Name} (\HOLConst{MCA} \HOLBoundVar{ca\sb{\mathrm{2}}}) \HOLSymConst{\HOLTokenEquiv{}}
     (\HOLBoundVar{M}\HOLSymConst{,}\HOLBoundVar{Oi}\HOLSymConst{,}\HOLBoundVar{Os}) \HOLConst{ksat}
     \HOLConst{KCert} (\HOLConst{CA} \HOLBoundVar{mca}) (\HOLConst{Entity} (\HOLConst{KCA} \HOLBoundVar{ca\sb{\mathrm{2}}}))
       (\HOLConst{SPubKey} (\HOLConst{pubK} (\HOLConst{KCA} \HOLBoundVar{ca\sb{\mathrm{2}}})))
       (\HOLConst{KCrtSig}
          (\HOLConst{sign} (\HOLConst{KCA} \HOLBoundVar{mca})
             (\HOLConst{hash} (\HOLConst{plain} (\HOLBoundVar{mca}\HOLSymConst{,}\HOLConst{KCA} \HOLBoundVar{ca\sb{\mathrm{2}}}\HOLSymConst{,}\HOLConst{pubK} (\HOLConst{KCA} \HOLBoundVar{ca\sb{\mathrm{2}}}))))))
\end{SaveVerbatim}
\newcommand{\HOLmessageCertificateTheoremskcertCASatXXthm}{\UseVerbatim{HOLmessageCertificateTheoremskcertCASatXXthm}}
\begin{SaveVerbatim}{HOLmessageCertificateTheoremskcertStaffSatXXthm}
\HOLTokenTurnstile{} \HOLSymConst{\HOLTokenForall{}}\HOLBoundVar{staff} \HOLBoundVar{mca} \HOLBoundVar{Os} \HOLBoundVar{Oi} \HOLBoundVar{M}.
     (\HOLBoundVar{M}\HOLSymConst{,}\HOLBoundVar{Oi}\HOLSymConst{,}\HOLBoundVar{Os}) \HOLConst{sat}
     \HOLConst{Name} (\HOLConst{MKey} (\HOLConst{pubK} (\HOLConst{KCA} \HOLBoundVar{mca}))) \HOLConst{says}
     \HOLConst{Name} (\HOLConst{MKey} (\HOLConst{pubK} (\HOLConst{KStaff} \HOLBoundVar{staff}))) \HOLConst{speaks_for}
     \HOLConst{Name} (\HOLConst{MStaff} \HOLBoundVar{staff}) \HOLSymConst{\HOLTokenEquiv{}}
     (\HOLBoundVar{M}\HOLSymConst{,}\HOLBoundVar{Oi}\HOLSymConst{,}\HOLBoundVar{Os}) \HOLConst{ksat}
     \HOLConst{KCert} (\HOLConst{CA} \HOLBoundVar{mca}) (\HOLConst{Entity} (\HOLConst{KStaff} \HOLBoundVar{staff}))
       (\HOLConst{SPubKey} (\HOLConst{pubK} (\HOLConst{KStaff} \HOLBoundVar{staff})))
       (\HOLConst{KCrtSig}
          (\HOLConst{sign} (\HOLConst{KCA} \HOLBoundVar{mca})
             (\HOLConst{hash}
                (\HOLConst{plain}
                   (\HOLBoundVar{mca}\HOLSymConst{,}\HOLConst{KStaff} \HOLBoundVar{staff}\HOLSymConst{,}\HOLConst{pubK} (\HOLConst{KStaff} \HOLBoundVar{staff}))))))
\end{SaveVerbatim}
\newcommand{\HOLmessageCertificateTheoremskcertStaffSatXXthm}{\UseVerbatim{HOLmessageCertificateTheoremskcertStaffSatXXthm}}
\begin{SaveVerbatim}{HOLmessageCertificateTheoremskcrtCAInterpXXthm}
\HOLTokenTurnstile{} \HOLSymConst{\HOLTokenForall{}}\HOLBoundVar{pubKey} \HOLBoundVar{mca} \HOLBoundVar{kcertSig} \HOLBoundVar{ca\sb{\mathrm{2}}} \HOLBoundVar{Os} \HOLBoundVar{Oi} \HOLBoundVar{M}.
     \HOLConst{checkKCert}
       (\HOLConst{KCert} (\HOLConst{CA} \HOLBoundVar{mca}) (\HOLConst{Entity} (\HOLConst{KCA} \HOLBoundVar{ca\sb{\mathrm{2}}})) (\HOLConst{SPubKey} \HOLBoundVar{pubKey})
          (\HOLConst{KCrtSig} \HOLBoundVar{kcertSig})) \HOLSymConst{\HOLTokenImp{}}
     ((\HOLBoundVar{M}\HOLSymConst{,}\HOLBoundVar{Oi}\HOLSymConst{,}\HOLBoundVar{Os}) \HOLConst{ksat}
      \HOLConst{KCert} (\HOLConst{CA} \HOLBoundVar{mca}) (\HOLConst{Entity} (\HOLConst{KCA} \HOLBoundVar{ca\sb{\mathrm{2}}})) (\HOLConst{SPubKey} \HOLBoundVar{pubKey})
        (\HOLConst{KCrtSig} \HOLBoundVar{kcertSig}) \HOLSymConst{\HOLTokenEquiv{}}
      (\HOLBoundVar{M}\HOLSymConst{,}\HOLBoundVar{Oi}\HOLSymConst{,}\HOLBoundVar{Os}) \HOLConst{sat}
      \HOLConst{Name} (\HOLConst{MKey} (\HOLConst{pubK} (\HOLConst{KCA} \HOLBoundVar{mca}))) \HOLConst{says}
      \HOLConst{Name} (\HOLConst{MKey} \HOLBoundVar{pubKey}) \HOLConst{speaks_for} \HOLConst{Name} (\HOLConst{MCA} \HOLBoundVar{ca\sb{\mathrm{2}}}))
\end{SaveVerbatim}
\newcommand{\HOLmessageCertificateTheoremskcrtCAInterpXXthm}{\UseVerbatim{HOLmessageCertificateTheoremskcrtCAInterpXXthm}}
\begin{SaveVerbatim}{HOLmessageCertificateTheoremskcrtStaffInterpXXthm}
\HOLTokenTurnstile{} \HOLSymConst{\HOLTokenForall{}}\HOLBoundVar{staff} \HOLBoundVar{pubKey} \HOLBoundVar{mca} \HOLBoundVar{kcertSig} \HOLBoundVar{Os} \HOLBoundVar{Oi} \HOLBoundVar{M}.
     \HOLConst{checkKCert}
       (\HOLConst{KCert} (\HOLConst{CA} \HOLBoundVar{mca}) (\HOLConst{Entity} (\HOLConst{KStaff} \HOLBoundVar{staff})) (\HOLConst{SPubKey} \HOLBoundVar{pubKey})
          (\HOLConst{KCrtSig} \HOLBoundVar{kcertSig})) \HOLSymConst{\HOLTokenImp{}}
     ((\HOLBoundVar{M}\HOLSymConst{,}\HOLBoundVar{Oi}\HOLSymConst{,}\HOLBoundVar{Os}) \HOLConst{ksat}
      \HOLConst{KCert} (\HOLConst{CA} \HOLBoundVar{mca}) (\HOLConst{Entity} (\HOLConst{KStaff} \HOLBoundVar{staff})) (\HOLConst{SPubKey} \HOLBoundVar{pubKey})
        (\HOLConst{KCrtSig} \HOLBoundVar{kcertSig}) \HOLSymConst{\HOLTokenEquiv{}}
      (\HOLBoundVar{M}\HOLSymConst{,}\HOLBoundVar{Oi}\HOLSymConst{,}\HOLBoundVar{Os}) \HOLConst{sat}
      \HOLConst{Name} (\HOLConst{MKey} (\HOLConst{pubK} (\HOLConst{KCA} \HOLBoundVar{mca}))) \HOLConst{says}
      \HOLConst{Name} (\HOLConst{MKey} \HOLBoundVar{pubKey}) \HOLConst{speaks_for} \HOLConst{Name} (\HOLConst{MStaff} \HOLBoundVar{staff}))
\end{SaveVerbatim}
\newcommand{\HOLmessageCertificateTheoremskcrtStaffInterpXXthm}{\UseVerbatim{HOLmessageCertificateTheoremskcrtStaffInterpXXthm}}
\begin{SaveVerbatim}{HOLmessageCertificateTheoremsmsgInterpXXthm}
\HOLTokenTurnstile{} \HOLSymConst{\HOLTokenForall{}}\HOLBoundVar{sender} \HOLBoundVar{role} \HOLBoundVar{recipient} \HOLBoundVar{dek} \HOLBoundVar{action} \HOLBoundVar{Os} \HOLBoundVar{Oi} \HOLBoundVar{M}.
     \HOLConst{checkMSG}
       (\HOLConst{MSG} (\HOLConst{From} \HOLBoundVar{sender}) (\HOLConst{As} \HOLBoundVar{role}) (\HOLConst{To} \HOLBoundVar{recipient})
          (\HOLConst{Ea} (\HOLConst{pubK} \HOLBoundVar{recipient}) (\HOLConst{plain} (\HOLConst{sym} \HOLBoundVar{dek})))
          (\HOLConst{Es} (\HOLConst{sym} \HOLBoundVar{dek})
             (\HOLConst{plain}
                (\HOLConst{CMD} (\HOLConst{From} \HOLBoundVar{sender}) (\HOLConst{As} \HOLBoundVar{role}) (\HOLConst{To} \HOLBoundVar{recipient})
                   \HOLBoundVar{action})))
          (\HOLConst{sign} \HOLBoundVar{sender}
             (\HOLConst{hash}
                (\HOLConst{plain}
                   (\HOLConst{CMD} (\HOLConst{From} \HOLBoundVar{sender}) (\HOLConst{As} \HOLBoundVar{role}) (\HOLConst{To} \HOLBoundVar{recipient})
                      \HOLBoundVar{action}))))) \HOLSymConst{\HOLTokenImp{}}
     (\HOLBoundVar{M}\HOLSymConst{,}\HOLBoundVar{Oi}\HOLSymConst{,}\HOLBoundVar{Os}) \HOLConst{msat}
     \HOLConst{MSG} (\HOLConst{From} \HOLBoundVar{sender}) (\HOLConst{As} \HOLBoundVar{role}) (\HOLConst{To} \HOLBoundVar{recipient})
       (\HOLConst{Ea} (\HOLConst{pubK} \HOLBoundVar{recipient}) (\HOLConst{plain} (\HOLConst{sym} \HOLBoundVar{dek})))
       (\HOLConst{Es} (\HOLConst{sym} \HOLBoundVar{dek})
          (\HOLConst{plain}
             (\HOLConst{CMD} (\HOLConst{From} \HOLBoundVar{sender}) (\HOLConst{As} \HOLBoundVar{role}) (\HOLConst{To} \HOLBoundVar{recipient})
                \HOLBoundVar{action})))
       (\HOLConst{sign} \HOLBoundVar{sender}
          (\HOLConst{hash}
             (\HOLConst{plain}
                (\HOLConst{CMD} (\HOLConst{From} \HOLBoundVar{sender}) (\HOLConst{As} \HOLBoundVar{role}) (\HOLConst{To} \HOLBoundVar{recipient})
                   \HOLBoundVar{action})))) \HOLSymConst{\HOLTokenEquiv{}}
     (\HOLBoundVar{M}\HOLSymConst{,}\HOLBoundVar{Oi}\HOLSymConst{,}\HOLBoundVar{Os}) \HOLConst{sat}
     \HOLConst{Name} (\HOLConst{MKey} (\HOLConst{pubK} (\HOLConst{KStaff} \HOLBoundVar{sender}))) \HOLConst{quoting}
     \HOLConst{Name} (\HOLConst{MRole} \HOLBoundVar{role}) \HOLConst{says} \HOLConst{prop} \HOLBoundVar{action}
\end{SaveVerbatim}
\newcommand{\HOLmessageCertificateTheoremsmsgInterpXXthm}{\UseVerbatim{HOLmessageCertificateTheoremsmsgInterpXXthm}}
\begin{SaveVerbatim}{HOLmessageCertificateTheoremsmsgParametersXXdef}
\HOLTokenTurnstile{} \HOLConst{msgParameters}
     (\HOLConst{MSG} (\HOLConst{From} \HOLFreeVar{sender}) (\HOLConst{As} \HOLFreeVar{role}) (\HOLConst{To} \HOLFreeVar{recipient}) \HOLFreeVar{enDEK} \HOLFreeVar{enCMDMsg}
        \HOLFreeVar{CMDMsgSig}) \HOLSymConst{=}
   (\HOLFreeVar{sender}\HOLSymConst{,}\HOLFreeVar{role}\HOLSymConst{,}\HOLFreeVar{recipient}\HOLSymConst{,}\HOLFreeVar{enDEK}\HOLSymConst{,}\HOLFreeVar{enCMDMsg}\HOLSymConst{,}\HOLFreeVar{CMDMsgSig})
\end{SaveVerbatim}
\newcommand{\HOLmessageCertificateTheoremsmsgParametersXXdef}{\UseVerbatim{HOLmessageCertificateTheoremsmsgParametersXXdef}}
\begin{SaveVerbatim}{HOLmessageCertificateTheoremsmsgParametersXXind}
\HOLTokenTurnstile{} \HOLSymConst{\HOLTokenForall{}}\HOLBoundVar{P}.
     (\HOLSymConst{\HOLTokenForall{}}\HOLBoundVar{sender} \HOLBoundVar{role} \HOLBoundVar{recipient} \HOLBoundVar{enDEK} \HOLBoundVar{enCMDMsg} \HOLBoundVar{CMDMsgSig}.
        \HOLBoundVar{P}
          (\HOLConst{MSG} (\HOLConst{From} \HOLBoundVar{sender}) (\HOLConst{As} \HOLBoundVar{role}) (\HOLConst{To} \HOLBoundVar{recipient}) \HOLBoundVar{enDEK}
             \HOLBoundVar{enCMDMsg} \HOLBoundVar{CMDMsgSig})) \HOLSymConst{\HOLTokenImp{}}
     \HOLSymConst{\HOLTokenForall{}}\HOLBoundVar{v}. \HOLBoundVar{P} \HOLBoundVar{v}
\end{SaveVerbatim}
\newcommand{\HOLmessageCertificateTheoremsmsgParametersXXind}{\UseVerbatim{HOLmessageCertificateTheoremsmsgParametersXXind}}
\begin{SaveVerbatim}{HOLmessageCertificateTheoremsmsgSatXXthm}
\HOLTokenTurnstile{} \HOLSymConst{\HOLTokenForall{}}\HOLBoundVar{recipient} \HOLBoundVar{dek} \HOLBoundVar{sender} \HOLBoundVar{role} \HOLBoundVar{order} \HOLBoundVar{M} \HOLBoundVar{Oi} \HOLBoundVar{Os}.
     (\HOLBoundVar{M}\HOLSymConst{,}\HOLBoundVar{Oi}\HOLSymConst{,}\HOLBoundVar{Os}) \HOLConst{sat}
     \HOLConst{Name} (\HOLConst{MKey} (\HOLConst{pubK} (\HOLConst{KStaff} \HOLBoundVar{sender}))) \HOLConst{quoting}
     \HOLConst{Name} (\HOLConst{MRole} \HOLBoundVar{role}) \HOLConst{says} \HOLConst{prop} \HOLFreeVar{action} \HOLSymConst{\HOLTokenEquiv{}}
     (\HOLBoundVar{M}\HOLSymConst{,}\HOLBoundVar{Oi}\HOLSymConst{,}\HOLBoundVar{Os}) \HOLConst{msat}
     \HOLConst{MSG} (\HOLConst{From} \HOLBoundVar{sender}) (\HOLConst{As} \HOLBoundVar{role}) (\HOLConst{To} \HOLBoundVar{recipient})
       (\HOLConst{Ea} (\HOLConst{pubK} \HOLBoundVar{recipient}) (\HOLConst{plain} (\HOLConst{sym} \HOLBoundVar{dek})))
       (\HOLConst{Es} (\HOLConst{sym} \HOLBoundVar{dek})
          (\HOLConst{plain}
             (\HOLConst{CMD} (\HOLConst{From} \HOLBoundVar{sender}) (\HOLConst{As} \HOLBoundVar{role}) (\HOLConst{To} \HOLBoundVar{recipient})
                \HOLFreeVar{action})))
       (\HOLConst{sign} \HOLBoundVar{sender}
          (\HOLConst{hash}
             (\HOLConst{plain}
                (\HOLConst{CMD} (\HOLConst{From} \HOLBoundVar{sender}) (\HOLConst{As} \HOLBoundVar{role}) (\HOLConst{To} \HOLBoundVar{recipient})
                   \HOLFreeVar{action}))))
\end{SaveVerbatim}
\newcommand{\HOLmessageCertificateTheoremsmsgSatXXthm}{\UseVerbatim{HOLmessageCertificateTheoremsmsgSatXXthm}}
\begin{SaveVerbatim}{HOLmessageCertificateTheoremsordersCommandXXdef}
\HOLTokenTurnstile{} \HOLConst{ordersCommand}
     (\HOLConst{CMD} (\HOLConst{From} \HOLFreeVar{sender}) (\HOLConst{As} \HOLFreeVar{role}) (\HOLConst{To} \HOLFreeVar{recipient}) \HOLFreeVar{orders}) \HOLSymConst{=}
   \HOLFreeVar{orders}
\end{SaveVerbatim}
\newcommand{\HOLmessageCertificateTheoremsordersCommandXXdef}{\UseVerbatim{HOLmessageCertificateTheoremsordersCommandXXdef}}
\begin{SaveVerbatim}{HOLmessageCertificateTheoremsordersCommandXXind}
\HOLTokenTurnstile{} \HOLSymConst{\HOLTokenForall{}}\HOLBoundVar{P}.
     (\HOLSymConst{\HOLTokenForall{}}\HOLBoundVar{sender} \HOLBoundVar{role} \HOLBoundVar{recipient} \HOLBoundVar{orders}.
        \HOLBoundVar{P} (\HOLConst{CMD} (\HOLConst{From} \HOLBoundVar{sender}) (\HOLConst{As} \HOLBoundVar{role}) (\HOLConst{To} \HOLBoundVar{recipient}) \HOLBoundVar{orders})) \HOLSymConst{\HOLTokenImp{}}
     \HOLSymConst{\HOLTokenForall{}}\HOLBoundVar{v}. \HOLBoundVar{P} \HOLBoundVar{v}
\end{SaveVerbatim}
\newcommand{\HOLmessageCertificateTheoremsordersCommandXXind}{\UseVerbatim{HOLmessageCertificateTheoremsordersCommandXXind}}
\begin{SaveVerbatim}{HOLmessageCertificateTheoremsordersParametersXXdef}
\HOLTokenTurnstile{} \HOLConst{ordersParameters}
     (\HOLConst{CMD} (\HOLConst{From} \HOLFreeVar{sender}) (\HOLConst{As} \HOLFreeVar{role}) (\HOLConst{To} \HOLFreeVar{recipient}) \HOLFreeVar{orders}) \HOLSymConst{=}
   (\HOLFreeVar{sender}\HOLSymConst{,}\HOLFreeVar{role}\HOLSymConst{,}\HOLFreeVar{recipient})
\end{SaveVerbatim}
\newcommand{\HOLmessageCertificateTheoremsordersParametersXXdef}{\UseVerbatim{HOLmessageCertificateTheoremsordersParametersXXdef}}
\begin{SaveVerbatim}{HOLmessageCertificateTheoremsordersParametersXXind}
\HOLTokenTurnstile{} \HOLSymConst{\HOLTokenForall{}}\HOLBoundVar{P}.
     (\HOLSymConst{\HOLTokenForall{}}\HOLBoundVar{sender} \HOLBoundVar{role} \HOLBoundVar{recipient} \HOLBoundVar{orders}.
        \HOLBoundVar{P} (\HOLConst{CMD} (\HOLConst{From} \HOLBoundVar{sender}) (\HOLConst{As} \HOLBoundVar{role}) (\HOLConst{To} \HOLBoundVar{recipient}) \HOLBoundVar{orders})) \HOLSymConst{\HOLTokenImp{}}
     \HOLSymConst{\HOLTokenForall{}}\HOLBoundVar{v}. \HOLBoundVar{P} \HOLBoundVar{v}
\end{SaveVerbatim}
\newcommand{\HOLmessageCertificateTheoremsordersParametersXXind}{\UseVerbatim{HOLmessageCertificateTheoremsordersParametersXXind}}
\begin{SaveVerbatim}{HOLmessageCertificateTheoremsrcertStaffSatXXthm}
\HOLTokenTurnstile{} \HOLSymConst{\HOLTokenForall{}}\HOLBoundVar{order} \HOLBoundVar{delegateRole} \HOLBoundVar{delegate} \HOLBoundVar{commander} \HOLBoundVar{cmdRole} \HOLBoundVar{Os} \HOLBoundVar{Oi} \HOLBoundVar{M}.
     (\HOLBoundVar{M}\HOLSymConst{,}\HOLBoundVar{Oi}\HOLSymConst{,}\HOLBoundVar{Os}) \HOLConst{sat}
     \HOLConst{Name} (\HOLConst{MKey} (\HOLConst{pubK} (\HOLConst{KStaff} \HOLBoundVar{commander}))) \HOLConst{says}
     \HOLConst{reps} (\HOLConst{Name} (\HOLConst{MStaff} \HOLBoundVar{delegate})) (\HOLConst{Name} (\HOLConst{MRole} \HOLBoundVar{delegateRole}))
       (\HOLConst{prop} \HOLBoundVar{order}) \HOLSymConst{\HOLTokenEquiv{}}
     (\HOLBoundVar{M}\HOLSymConst{,}\HOLBoundVar{Oi}\HOLSymConst{,}\HOLBoundVar{Os}) \HOLConst{rsat}
     \HOLConst{RCert} (\HOLConst{Auth} \HOLBoundVar{commander}) (\HOLConst{As} \HOLBoundVar{cmdRole}) (\HOLConst{For} \HOLBoundVar{delegate})
       (\HOLConst{As} \HOLBoundVar{delegateRole}) \HOLBoundVar{order}
       (\HOLConst{RCrtSig}
          (\HOLConst{sign} (\HOLConst{KStaff} \HOLBoundVar{commander})
             (\HOLConst{hash}
                (\HOLConst{plain}
                   (\HOLBoundVar{commander}\HOLSymConst{,}\HOLBoundVar{cmdRole}\HOLSymConst{,}\HOLBoundVar{delegate}\HOLSymConst{,}\HOLBoundVar{delegateRole}\HOLSymConst{,}
                    \HOLBoundVar{order})))))
\end{SaveVerbatim}
\newcommand{\HOLmessageCertificateTheoremsrcertStaffSatXXthm}{\UseVerbatim{HOLmessageCertificateTheoremsrcertStaffSatXXthm}}
\begin{SaveVerbatim}{HOLmessageCertificateTheoremsrcrtStaffInterpXXthm}
\HOLTokenTurnstile{} \HOLSymConst{\HOLTokenForall{}}\HOLBoundVar{rcertSig} \HOLBoundVar{order} \HOLBoundVar{delegateRole} \HOLBoundVar{delegate} \HOLBoundVar{commander} \HOLBoundVar{cmdRole} \HOLBoundVar{Os} \HOLBoundVar{Oi}
      \HOLBoundVar{M}.
     \HOLConst{checkRCert}
       (\HOLConst{RCert} (\HOLConst{Auth} \HOLBoundVar{commander}) (\HOLConst{As} \HOLBoundVar{cmdRole}) (\HOLConst{For} \HOLBoundVar{delegate})
          (\HOLConst{As} \HOLBoundVar{delegateRole}) \HOLBoundVar{order} (\HOLConst{RCrtSig} \HOLBoundVar{rcertSig})) \HOLSymConst{\HOLTokenImp{}}
     ((\HOLBoundVar{M}\HOLSymConst{,}\HOLBoundVar{Oi}\HOLSymConst{,}\HOLBoundVar{Os}) \HOLConst{rsat}
      \HOLConst{RCert} (\HOLConst{Auth} \HOLBoundVar{commander}) (\HOLConst{As} \HOLBoundVar{cmdRole}) (\HOLConst{For} \HOLBoundVar{delegate})
        (\HOLConst{As} \HOLBoundVar{delegateRole}) \HOLBoundVar{order} (\HOLConst{RCrtSig} \HOLBoundVar{rcertSig}) \HOLSymConst{\HOLTokenEquiv{}}
      (\HOLBoundVar{M}\HOLSymConst{,}\HOLBoundVar{Oi}\HOLSymConst{,}\HOLBoundVar{Os}) \HOLConst{sat}
      \HOLConst{Name} (\HOLConst{MKey} (\HOLConst{pubK} (\HOLConst{KStaff} \HOLBoundVar{commander}))) \HOLConst{says}
      \HOLConst{reps} (\HOLConst{Name} (\HOLConst{MStaff} \HOLBoundVar{delegate})) (\HOLConst{Name} (\HOLConst{MRole} \HOLBoundVar{delegateRole}))
        (\HOLConst{prop} \HOLBoundVar{order}))
\end{SaveVerbatim}
\newcommand{\HOLmessageCertificateTheoremsrcrtStaffInterpXXthm}{\UseVerbatim{HOLmessageCertificateTheoremsrcrtStaffInterpXXthm}}
\begin{SaveVerbatim}{HOLmessageCertificateTheoremsrootkcertCAInterpXXthm}
\HOLTokenTurnstile{} \HOLSymConst{\HOLTokenForall{}}\HOLBoundVar{pubKey} \HOLBoundVar{ca} \HOLBoundVar{Os} \HOLBoundVar{Oi} \HOLBoundVar{M}.
     (\HOLBoundVar{M}\HOLSymConst{,}\HOLBoundVar{Oi}\HOLSymConst{,}\HOLBoundVar{Os}) \HOLConst{rootksat}
     \HOLConst{RootKCert} (\HOLConst{Entity} (\HOLConst{KCA} \HOLBoundVar{ca})) (\HOLConst{SPubKey} \HOLBoundVar{pubKey}) \HOLSymConst{\HOLTokenEquiv{}}
     (\HOLBoundVar{M}\HOLSymConst{,}\HOLBoundVar{Oi}\HOLSymConst{,}\HOLBoundVar{Os}) \HOLConst{sat} \HOLConst{Name} (\HOLConst{MKey} \HOLBoundVar{pubKey}) \HOLConst{speaks_for} \HOLConst{Name} (\HOLConst{MCA} \HOLBoundVar{ca})
\end{SaveVerbatim}
\newcommand{\HOLmessageCertificateTheoremsrootkcertCAInterpXXthm}{\UseVerbatim{HOLmessageCertificateTheoremsrootkcertCAInterpXXthm}}
\begin{SaveVerbatim}{HOLmessageCertificateTheoremsrootkcertStaffInterpXXthm}
\HOLTokenTurnstile{} \HOLSymConst{\HOLTokenForall{}}\HOLBoundVar{staff} \HOLBoundVar{pubKey} \HOLBoundVar{Os} \HOLBoundVar{Oi} \HOLBoundVar{M}.
     (\HOLBoundVar{M}\HOLSymConst{,}\HOLBoundVar{Oi}\HOLSymConst{,}\HOLBoundVar{Os}) \HOLConst{rootksat}
     \HOLConst{RootKCert} (\HOLConst{Entity} (\HOLConst{KStaff} \HOLBoundVar{staff})) (\HOLConst{SPubKey} \HOLBoundVar{pubKey}) \HOLSymConst{\HOLTokenEquiv{}}
     (\HOLBoundVar{M}\HOLSymConst{,}\HOLBoundVar{Oi}\HOLSymConst{,}\HOLBoundVar{Os}) \HOLConst{sat}
     \HOLConst{Name} (\HOLConst{MKey} \HOLBoundVar{pubKey}) \HOLConst{speaks_for} \HOLConst{Name} (\HOLConst{MStaff} \HOLBoundVar{staff})
\end{SaveVerbatim}
\newcommand{\HOLmessageCertificateTheoremsrootkcertStaffInterpXXthm}{\UseVerbatim{HOLmessageCertificateTheoremsrootkcertStaffInterpXXthm}}
\begin{SaveVerbatim}{HOLmessageCertificateTheoremsrootrcertStaffInterpXXthm}
\HOLTokenTurnstile{} \HOLSymConst{\HOLTokenForall{}}\HOLBoundVar{order} \HOLBoundVar{delegateRole} \HOLBoundVar{delegate} \HOLBoundVar{Os} \HOLBoundVar{Oi} \HOLBoundVar{M}.
     (\HOLBoundVar{M}\HOLSymConst{,}\HOLBoundVar{Oi}\HOLSymConst{,}\HOLBoundVar{Os}) \HOLConst{rootrsat}
     \HOLConst{RootRCert} (\HOLConst{For} \HOLBoundVar{delegate}) (\HOLConst{As} \HOLBoundVar{delegateRole}) \HOLBoundVar{order} \HOLSymConst{\HOLTokenEquiv{}}
     (\HOLBoundVar{M}\HOLSymConst{,}\HOLBoundVar{Oi}\HOLSymConst{,}\HOLBoundVar{Os}) \HOLConst{sat}
     \HOLConst{reps} (\HOLConst{Name} (\HOLConst{MStaff} \HOLBoundVar{delegate})) (\HOLConst{Name} (\HOLConst{MRole} \HOLBoundVar{delegateRole}))
       (\HOLConst{prop} \HOLBoundVar{order})
\end{SaveVerbatim}
\newcommand{\HOLmessageCertificateTheoremsrootrcertStaffInterpXXthm}{\UseVerbatim{HOLmessageCertificateTheoremsrootrcertStaffInterpXXthm}}
\newcommand{\HOLmessageCertificateTheorems}{
\HOLThmTag{messageCertificate}{checkKCert_def}\HOLmessageCertificateTheoremscheckKCertXXdef
\HOLThmTag{messageCertificate}{checkKCert_ind}\HOLmessageCertificateTheoremscheckKCertXXind
\HOLThmTag{messageCertificate}{checkKCertOK}\HOLmessageCertificateTheoremscheckKCertOK
\HOLThmTag{messageCertificate}{checkMSG_def}\HOLmessageCertificateTheoremscheckMSGXXdef
\HOLThmTag{messageCertificate}{checkMSG_ind}\HOLmessageCertificateTheoremscheckMSGXXind
\HOLThmTag{messageCertificate}{checkMSGOK}\HOLmessageCertificateTheoremscheckMSGOK
\HOLThmTag{messageCertificate}{checkRCert_def}\HOLmessageCertificateTheoremscheckRCertXXdef
\HOLThmTag{messageCertificate}{checkRCert_ind}\HOLmessageCertificateTheoremscheckRCertXXind
\HOLThmTag{messageCertificate}{checkRCertOK}\HOLmessageCertificateTheoremscheckRCertOK
\HOLThmTag{messageCertificate}{Ekcrt_def}\HOLmessageCertificateTheoremsEkcrtXXdef
\HOLThmTag{messageCertificate}{Ekcrt_ind}\HOLmessageCertificateTheoremsEkcrtXXind
\HOLThmTag{messageCertificate}{Emsg_def}\HOLmessageCertificateTheoremsEmsgXXdef
\HOLThmTag{messageCertificate}{Emsg_ind}\HOLmessageCertificateTheoremsEmsgXXind
\HOLThmTag{messageCertificate}{Ercrt_def}\HOLmessageCertificateTheoremsErcrtXXdef
\HOLThmTag{messageCertificate}{Ercrt_ind}\HOLmessageCertificateTheoremsErcrtXXind
\HOLThmTag{messageCertificate}{Erootkcrt_def}\HOLmessageCertificateTheoremsErootkcrtXXdef
\HOLThmTag{messageCertificate}{Erootkcrt_ind}\HOLmessageCertificateTheoremsErootkcrtXXind
\HOLThmTag{messageCertificate}{Erootrcrt_def}\HOLmessageCertificateTheoremsErootrcrtXXdef
\HOLThmTag{messageCertificate}{Erootrcrt_ind}\HOLmessageCertificateTheoremsErootrcrtXXind
\HOLThmTag{messageCertificate}{getCommand_def}\HOLmessageCertificateTheoremsgetCommandXXdef
\HOLThmTag{messageCertificate}{getCommand_ind}\HOLmessageCertificateTheoremsgetCommandXXind
\HOLThmTag{messageCertificate}{getCommand_thm}\HOLmessageCertificateTheoremsgetCommandXXthm
\HOLThmTag{messageCertificate}{kcertCASat_thm}\HOLmessageCertificateTheoremskcertCASatXXthm
\HOLThmTag{messageCertificate}{kcertStaffSat_thm}\HOLmessageCertificateTheoremskcertStaffSatXXthm
\HOLThmTag{messageCertificate}{kcrtCAInterp_thm}\HOLmessageCertificateTheoremskcrtCAInterpXXthm
\HOLThmTag{messageCertificate}{kcrtStaffInterp_thm}\HOLmessageCertificateTheoremskcrtStaffInterpXXthm
\HOLThmTag{messageCertificate}{msgInterp_thm}\HOLmessageCertificateTheoremsmsgInterpXXthm
\HOLThmTag{messageCertificate}{msgParameters_def}\HOLmessageCertificateTheoremsmsgParametersXXdef
\HOLThmTag{messageCertificate}{msgParameters_ind}\HOLmessageCertificateTheoremsmsgParametersXXind
\HOLThmTag{messageCertificate}{msgSat_thm}\HOLmessageCertificateTheoremsmsgSatXXthm
\HOLThmTag{messageCertificate}{ordersCommand_def}\HOLmessageCertificateTheoremsordersCommandXXdef
\HOLThmTag{messageCertificate}{ordersCommand_ind}\HOLmessageCertificateTheoremsordersCommandXXind
\HOLThmTag{messageCertificate}{ordersParameters_def}\HOLmessageCertificateTheoremsordersParametersXXdef
\HOLThmTag{messageCertificate}{ordersParameters_ind}\HOLmessageCertificateTheoremsordersParametersXXind
\HOLThmTag{messageCertificate}{rcertStaffSat_thm}\HOLmessageCertificateTheoremsrcertStaffSatXXthm
\HOLThmTag{messageCertificate}{rcrtStaffInterp_thm}\HOLmessageCertificateTheoremsrcrtStaffInterpXXthm
\HOLThmTag{messageCertificate}{rootkcertCAInterp_thm}\HOLmessageCertificateTheoremsrootkcertCAInterpXXthm
\HOLThmTag{messageCertificate}{rootkcertStaffInterp_thm}\HOLmessageCertificateTheoremsrootkcertStaffInterpXXthm
\HOLThmTag{messageCertificate}{rootrcertStaffInterp_thm}\HOLmessageCertificateTheoremsrootrcertStaffInterpXXthm
}

\begin{figure}[tb]
  \centering
  \begin{minipage}{1.0\linewidth}
    \HOLmessageCertificateDatatypesOrders
    \HOLmessageCertificateDatatypesOrigin
    \HOLmessageCertificateDatatypesRole
    \HOLmessageCertificateDatatypesDestination
    \HOLmessageCertificateDatatypesMessage
    \bluetext{[checkMSG\_def]}\vspace*{-0.1in}
    \HOLmessageCertificateDefinitionscheckMSGXXdef
    \bluetext{[Emsg\_def]}\vspace*{-0.1in}
    \HOLmessageCertificateDefinitionsEmsgXXdef
    \bluetext{[msat\_def]}\vspace*{-0.1in}
    \HOLmessageCertificateDefinitionsmsatXXdef
  \end{minipage}

  \caption{messageCertificate Theory: Datatypes and Definitions for Command Messages}
  \label{fig:messageCertificate-messages}
\end{figure}

\begin{figure}[tb]
  \centering
  \begin{minipage}{1.0\linewidth}
    \tealtext{[checkMSGOK]}\vspace*{-0.1in}
    \HOLmessageCertificateTheoremscheckMSGOK
    % \tealtext{[msgInterp\_thm]}\vspace*{-0.1in}
    % \HOLmessageCertificateTheoremsmsgInterpXXthm
    \tealtext{[msgSat\_thm]}\vspace*{-0.1in}
    \HOLmessageCertificateTheoremsmsgSatXXthm
  \end{minipage}
  \caption{messageCertificate Theory: Message Theorems}
  \label{fig:messageCertificate-message-theorems}
\end{figure}

Figure~\ref{fig:messageCertificate-messages} shows the datatype
definition for \emph{Messages}. The structure of messages is as
follows.
\begin{enumerate}
\item \emph{Origin} identifying who sent the message
\item \emph{Role} in which sender is acting
\item \emph{Destination} identifying who is the intended recipient
\item \emph{Data encryption key} encrypted with the recipients public key
\item \emph{Command} symmetrically encrypted with the data encryption key
\item \emph{Digital signature} computed by encrypting the hash of the
  command with the sender's private key
\end{enumerate}

A sample message given the datatypes in
Figure~\ref{fig:messageCertificate-messages} is shown in the following
HOL session.
\begin{session}
  \begin{scriptsize}
\begin{verbatim}

- val order = 
  ``MSG (From Alice) (As BFC) (To Carol)(Ea (pubK Carol)(plain (sym dek)))
        (Es (sym dek)(plain (MC go)))(sign Alice (hash(plain(MC go))))``;
> val order =
    ``MSG (From Alice) (As BFC) (To Carol)
        (Ea (pubK Carol) (plain (sym (dek :num))))
        (Es (sym dek) (plain (MC go))) (sign Alice (hash (plain (MC go))))``
     : term
- SPECL 
  [``Carol``,``dek:num``,``Alice``,``BFO``,``MC go``,
   ``M:(commands,'world,missionPrincipals,'Int,'Sec) Kripke``,
   ``Oi:'Int po``,``Os:'Sec po``] msgSat_thm;
> val it =
    |- ((M :(commands, 'world, missionPrincipals, 'Int, 'Sec) Kripke),
        (Oi :'Int po),(Os :'Sec po)) sat
       Name (MKey (pubK (KStaff Alice))) quoting Name (MRole BFO) says
       (prop (MC go) :(commands, missionPrincipals, 'Int, 'Sec) Form) <=>
       (M,Oi,Os) msat
       MSG (From Alice) (As BFO) (To Carol)
         (Ea (pubK Carol) (plain (sym (dek :num))))
         (Es (sym dek) (plain (MC go))) (sign Alice (hash (plain (MC go))))
     : thm
\end{verbatim}
  \end{scriptsize}
\end{session}

\clearpage{}


% ---- this points LaTeX to book.tex ---- 
% Local Variables: 
% TeX-master: "book"
% End:


%  LocalWords:  Coq Nuprl ACL bool hol num accessor FST SND dest mk
%  LocalWords:  Deconstructor ty vartype disj eq forall cond xs ys th
%  LocalWords:  deconstructor AndImp ImpImp VLSI certifiers HOL's pre
% LocalWords:  boolean Kripke thm sequents booleans DISCH REFL ELIM
% LocalWords:  ANTISYM GENL ISPEC ISPECL indices HD SUC rl CONV Modus
% LocalWords:  listTheory reduceLib Tollens ConstructiveDilemma TAC
% LocalWords:  ABSORP subgoals tacticals fn destructor tac subgoal
%  LocalWords:  ASM THENL Ponens Disch SML ASSUM Contrapositives SYM
%  LocalWords:  contrapositive online sequent's DeMorgan's

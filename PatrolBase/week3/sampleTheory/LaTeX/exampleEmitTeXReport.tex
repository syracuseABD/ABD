%% ---------------------------------------------------
%% Notice that we use the "report" class instead of "article"
%% ---------------------------------------------------
\documentclass{report}

\title{Sample Report Using EmitTeX Macros}
\author{\textbf{Shiu-Kai Chin}}
\date{\textbf{15 February 2017}}

%% ---------------------------------------------------
%% CSBDformat specifies the format of our reports
%% ---------------------------------------------------
\usepackage{634format}

%% ---------------------------------------------------
%% enumerate 
%% ---------------------------------------------------
\usepackage{enumerate}

%% ---------------------------------------------------
%% listings is used for including our source code in reports
%% ---------------------------------------------------
\usepackage{listings}
\usepackage{textcomp}

%% ---------------------------------------------------
%% Packages for math environments
%% ---------------------------------------------------
\usepackage{amsmath}

%% ---------------------------------------------------
%% Packages for URLs and hotlinks in the table of contents
%% and symbolic cross references using \ref
%% ---------------------------------------------------
\usepackage{hyperref}

%% ---------------------------------------------------
%% Packages for using HOL-generated macros and displays
%% ---------------------------------------------------
\usepackage{holtex}
\usepackage{holtexbasic}
\input{commands}
\newcommand{\HOLexampleOneDate}{29 June 2017}
\newcommand{\HOLexampleOneTime}{09:51}
\begin{SaveVerbatim}{HOLexampleOneTheoremsdemoTheorem}
\HOLTokenTurnstile{} \HOLSymConst{\HOLTokenForall{}}\HOLBoundVar{p} \HOLBoundVar{q}. \HOLBoundVar{p} \HOLSymConst{\HOLTokenImp{}} (\HOLBoundVar{p} \HOLSymConst{\HOLTokenImp{}} \HOLBoundVar{q}) \HOLSymConst{\HOLTokenImp{}} \HOLBoundVar{q}
\end{SaveVerbatim}
\newcommand{\HOLexampleOneTheoremsdemoTheorem}{\UseVerbatim{HOLexampleOneTheoremsdemoTheorem}}
\begin{SaveVerbatim}{HOLexampleOneTheoremsprobOneTheorem}
\HOLTokenTurnstile{} \HOLSymConst{\HOLTokenForall{}}\HOLBoundVar{p} \HOLBoundVar{q}. \HOLBoundVar{p} \HOLSymConst{\HOLTokenImp{}} (\HOLBoundVar{p} \HOLSymConst{\HOLTokenImp{}} \HOLBoundVar{q}) \HOLSymConst{\HOLTokenImp{}} \HOLBoundVar{q}
\end{SaveVerbatim}
\newcommand{\HOLexampleOneTheoremsprobOneTheorem}{\UseVerbatim{HOLexampleOneTheoremsprobOneTheorem}}
\newcommand{\HOLexampleOneTheorems}{
\HOLThmTag{example1}{demoTheorem}\HOLexampleOneTheoremsdemoTheorem
\HOLThmTag{example1}{prob1Theorem}\HOLexampleOneTheoremsprobOneTheorem
}


\begin{document}

%% --------------------------------------------------- the listings
%% parameter "language" is set to "ML"
%% ---------------------------------------------------
\lstset{language=ML}


\maketitle{}

\begin{abstract}
  We demonstrate using the EmitTeX structure functions in HOL to
  typeset HOL terms, types, theorems, and theories. We use
  \emph{example1} theory as our example theory to print.
\end{abstract}

\begin{acknowledgments}
  We gratefully acknowledge the hard work, trust, and dedication of
  our past students in the Syracuse University Cyber Engineering
  Semester and the Air Force Research Laboratory's Advanced Course
  (ACE) in Engineering Cybersecurity Boot Camp.  They bridged dreams
  and reality.
\end{acknowledgments}

\tableofcontents{}

\chapter{Executive Summary}
\label{cha:executive-summary}

\textbf{All requirements for this project are satisfied}.  In
particular, we prove the example theorem, pretty print the HOL theory,
and make use of the \emph{EmitTeX} structure to typeset HOL theorems
in this report.

The following theorems are proved and their corresponding \LaTeX{}
macros used in this report.
\begin{quote}
  \HOLexampleOneTheorems
\end{quote}

\chapter{Proof of prob1Theorem}
\label{cha:proof-prob1theorem}

\section{Problem Statement}
\label{sec:problem-statement}
Our task is to prove the theorem \HOLexampleOneTheoremsprobOneTheorem

Here it is again. \HOLTokenTurnstile{} \HOLSymConst{\HOLTokenForall{}}\HOLBoundVar{p} \HOLBoundVar{q}. \HOLBoundVar{p} \HOLSymConst{\HOLTokenImp{}} (\HOLBoundVar{p} \HOLSymConst{\HOLTokenImp{}} \HOLBoundVar{q}) \HOLSymConst{\HOLTokenImp{}} \HOLBoundVar{q}
\section{HOL Code Proving prob1Theorem}
\label{sec:hol-code-proving}


\begin{lstlisting}[frame=trBL]
val prob1Theorem =
let
  val th1 = ASSUME ``p:bool``
  val th2 = ASSUME ``p ==> q``
  val th3 = MP th2 th1
  val terma = hd (hyp th2)
  val th4 = DISCH terma th3
  val termb = hd (hyp th1)
  val th5 = DISCH termb th4
in
  GENL [``p:bool``,``q:bool``] th5
end  
\end{lstlisting}

\section{Session Transcript}
\label{sec:session-transcript}

\setcounter{sessioncount}{0}
\begin{session}
  \begin{scriptsize}
\begin{verbatim}

> val prob1Theorem =
let
  val th1 = ASSUME ``p:bool``
  val th2 = ASSUME ``p ==> q``
  val th3 = MP th2 th1
  val terma = hd (hyp th2)
  val th4 = DISCH terma th3
  val termb = hd (hyp th1)
  val th5 = DISCH termb th4
in
  GENL [``p:bool``,``q:bool``] th5
end;;;;
# # # # # # # # # # # val prob1Theorem =
   |- !p q. p ==> (p ==> q) ==> q:
   thm
\end{verbatim}
  \end{scriptsize}
\end{session}

\chapter{Proof of demoTheorem}
\label{cha:proof-demotheorem}

\section{Problem Statement}
\label{sec:problem-statement-1}

Our task is to prove the following theorem using \texttt{PROVE}.
\HOLexampleOneTheoremsdemoTheorem

\HOLTokenTurnstile{} \HOLSymConst{\HOLTokenForall{}}\HOLBoundVar{p} \HOLBoundVar{q}. \HOLBoundVar{p} \HOLSymConst{\HOLTokenImp{}} (\HOLBoundVar{p} \HOLSymConst{\HOLTokenImp{}} \HOLBoundVar{q}) \HOLSymConst{\HOLTokenImp{}} \HOLBoundVar{q}


\section{HOL Code Proving demoTheorem}
\label{sec:hol-code-proving-1}

\begin{lstlisting}[frame=trBL]
val demoTheorem = PROVE [] (concl prob1Theorem);  
\end{lstlisting}

\section{Session Transcript}
\label{sec:session-transcript-1}

\begin{session}
  \begin{scriptsize}
\begin{verbatim}

> val demoTheorem = PROVE [] (concl prob1Theorem);
Meson search level: ....
val demoTheorem =
   |- !p q. p ==> (p ==> q) ==> q:
   thm
\end{verbatim}
  \end{scriptsize}
\end{session}

%% ------------------------------------------
%% Change to letters for appendix
%% ------------------------------------------

%% ------------------------------------------
%% this restarts the section numbering
%% ------------------------------------------
\appendix{} 


%% ------------------------------------------
% label using capital letters
%% ------------------------------------------
\renewcommand{\thechapter}{\Alph{chapter}} 

\chapter{Source Code for example1Script}
\label{cha:source-code-sample}

The following code is from \emph{example1Script.sml}, which is located
in a different subdirectory than this file.
\lstinputlisting{../example1Script.sml}



\end{document}

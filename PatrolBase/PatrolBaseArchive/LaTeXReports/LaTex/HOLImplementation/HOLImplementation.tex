\newcommand{\HOLPBTypeDate}{09 July 2017}
\newcommand{\HOLPBTypeTime}{22:25}
\begin{SaveVerbatim}{HOLPBTypeDatatypesslCommand}
\HOLFreeVar{slCommand} = \HOLConst{crossLD} \HOLTokenBar{} \HOLConst{conductORP} \HOLTokenBar{} \HOLConst{moveToPB} \HOLTokenBar{} \HOLConst{conductPB}
          \HOLTokenBar{} \HOLConst{completePB} \HOLTokenBar{} \HOLConst{incomplete}
\end{SaveVerbatim}
\newcommand{\HOLPBTypeDatatypesslCommand}{\UseVerbatim{HOLPBTypeDatatypesslCommand}}
\begin{SaveVerbatim}{HOLPBTypeDatatypesslOutput}
\HOLFreeVar{slOutput} = \HOLConst{PlanPB} \HOLTokenBar{} \HOLConst{MoveToORP} \HOLTokenBar{} \HOLConst{ConductORP} \HOLTokenBar{} \HOLConst{MoveToPB}
         \HOLTokenBar{} \HOLConst{ConductPB} \HOLTokenBar{} \HOLConst{CompletePB} \HOLTokenBar{} \HOLConst{unAuthenticated}
\end{SaveVerbatim}
\newcommand{\HOLPBTypeDatatypesslOutput}{\UseVerbatim{HOLPBTypeDatatypesslOutput}}
\begin{SaveVerbatim}{HOLPBTypeDatatypesslState}
\HOLFreeVar{slState} = \HOLConst{PLAN_PB} \HOLTokenBar{} \HOLConst{MOVE_TO_ORP} \HOLTokenBar{} \HOLConst{CONDUCT_ORP} \HOLTokenBar{} \HOLConst{MOVE_TO_PB}
        \HOLTokenBar{} \HOLConst{CONDUCT_PB} \HOLTokenBar{} \HOLConst{COMPLETE_PB}
\end{SaveVerbatim}
\newcommand{\HOLPBTypeDatatypesslState}{\UseVerbatim{HOLPBTypeDatatypesslState}}
\begin{SaveVerbatim}{HOLPBTypeDatatypesstateRole}
\HOLFreeVar{stateRole} = \HOLConst{PlatoonLeader}
\end{SaveVerbatim}
\newcommand{\HOLPBTypeDatatypesstateRole}{\UseVerbatim{HOLPBTypeDatatypesstateRole}}
\newcommand{\HOLPBTypeDatatypes}{
\HOLPBTypeDatatypesslCommand\HOLPBTypeDatatypesslOutput\HOLPBTypeDatatypesslState\HOLPBTypeDatatypesstateRole}
\begin{SaveVerbatim}{HOLPBTypeTheoremsslCommandXXdistinctXXclauses}
\HOLTokenTurnstile{} \HOLConst{crossLD} \HOLSymConst{\HOLTokenNotEqual{}} \HOLConst{conductORP} \HOLSymConst{\HOLTokenConj{}} \HOLConst{crossLD} \HOLSymConst{\HOLTokenNotEqual{}} \HOLConst{moveToPB} \HOLSymConst{\HOLTokenConj{}}
   \HOLConst{crossLD} \HOLSymConst{\HOLTokenNotEqual{}} \HOLConst{conductPB} \HOLSymConst{\HOLTokenConj{}} \HOLConst{crossLD} \HOLSymConst{\HOLTokenNotEqual{}} \HOLConst{completePB} \HOLSymConst{\HOLTokenConj{}}
   \HOLConst{crossLD} \HOLSymConst{\HOLTokenNotEqual{}} \HOLConst{incomplete} \HOLSymConst{\HOLTokenConj{}} \HOLConst{conductORP} \HOLSymConst{\HOLTokenNotEqual{}} \HOLConst{moveToPB} \HOLSymConst{\HOLTokenConj{}}
   \HOLConst{conductORP} \HOLSymConst{\HOLTokenNotEqual{}} \HOLConst{conductPB} \HOLSymConst{\HOLTokenConj{}} \HOLConst{conductORP} \HOLSymConst{\HOLTokenNotEqual{}} \HOLConst{completePB} \HOLSymConst{\HOLTokenConj{}}
   \HOLConst{conductORP} \HOLSymConst{\HOLTokenNotEqual{}} \HOLConst{incomplete} \HOLSymConst{\HOLTokenConj{}} \HOLConst{moveToPB} \HOLSymConst{\HOLTokenNotEqual{}} \HOLConst{conductPB} \HOLSymConst{\HOLTokenConj{}}
   \HOLConst{moveToPB} \HOLSymConst{\HOLTokenNotEqual{}} \HOLConst{completePB} \HOLSymConst{\HOLTokenConj{}} \HOLConst{moveToPB} \HOLSymConst{\HOLTokenNotEqual{}} \HOLConst{incomplete} \HOLSymConst{\HOLTokenConj{}}
   \HOLConst{conductPB} \HOLSymConst{\HOLTokenNotEqual{}} \HOLConst{completePB} \HOLSymConst{\HOLTokenConj{}} \HOLConst{conductPB} \HOLSymConst{\HOLTokenNotEqual{}} \HOLConst{incomplete} \HOLSymConst{\HOLTokenConj{}}
   \HOLConst{completePB} \HOLSymConst{\HOLTokenNotEqual{}} \HOLConst{incomplete}
\end{SaveVerbatim}
\newcommand{\HOLPBTypeTheoremsslCommandXXdistinctXXclauses}{\UseVerbatim{HOLPBTypeTheoremsslCommandXXdistinctXXclauses}}
\begin{SaveVerbatim}{HOLPBTypeTheoremsslOutputXXdistinctXXclauses}
\HOLTokenTurnstile{} \HOLConst{PlanPB} \HOLSymConst{\HOLTokenNotEqual{}} \HOLConst{MoveToORP} \HOLSymConst{\HOLTokenConj{}} \HOLConst{PlanPB} \HOLSymConst{\HOLTokenNotEqual{}} \HOLConst{ConductORP} \HOLSymConst{\HOLTokenConj{}}
   \HOLConst{PlanPB} \HOLSymConst{\HOLTokenNotEqual{}} \HOLConst{MoveToPB} \HOLSymConst{\HOLTokenConj{}} \HOLConst{PlanPB} \HOLSymConst{\HOLTokenNotEqual{}} \HOLConst{ConductPB} \HOLSymConst{\HOLTokenConj{}}
   \HOLConst{PlanPB} \HOLSymConst{\HOLTokenNotEqual{}} \HOLConst{CompletePB} \HOLSymConst{\HOLTokenConj{}} \HOLConst{PlanPB} \HOLSymConst{\HOLTokenNotEqual{}} \HOLConst{unAuthenticated} \HOLSymConst{\HOLTokenConj{}}
   \HOLConst{MoveToORP} \HOLSymConst{\HOLTokenNotEqual{}} \HOLConst{ConductORP} \HOLSymConst{\HOLTokenConj{}} \HOLConst{MoveToORP} \HOLSymConst{\HOLTokenNotEqual{}} \HOLConst{MoveToPB} \HOLSymConst{\HOLTokenConj{}}
   \HOLConst{MoveToORP} \HOLSymConst{\HOLTokenNotEqual{}} \HOLConst{ConductPB} \HOLSymConst{\HOLTokenConj{}} \HOLConst{MoveToORP} \HOLSymConst{\HOLTokenNotEqual{}} \HOLConst{CompletePB} \HOLSymConst{\HOLTokenConj{}}
   \HOLConst{MoveToORP} \HOLSymConst{\HOLTokenNotEqual{}} \HOLConst{unAuthenticated} \HOLSymConst{\HOLTokenConj{}} \HOLConst{ConductORP} \HOLSymConst{\HOLTokenNotEqual{}} \HOLConst{MoveToPB} \HOLSymConst{\HOLTokenConj{}}
   \HOLConst{ConductORP} \HOLSymConst{\HOLTokenNotEqual{}} \HOLConst{ConductPB} \HOLSymConst{\HOLTokenConj{}} \HOLConst{ConductORP} \HOLSymConst{\HOLTokenNotEqual{}} \HOLConst{CompletePB} \HOLSymConst{\HOLTokenConj{}}
   \HOLConst{ConductORP} \HOLSymConst{\HOLTokenNotEqual{}} \HOLConst{unAuthenticated} \HOLSymConst{\HOLTokenConj{}} \HOLConst{MoveToPB} \HOLSymConst{\HOLTokenNotEqual{}} \HOLConst{ConductPB} \HOLSymConst{\HOLTokenConj{}}
   \HOLConst{MoveToPB} \HOLSymConst{\HOLTokenNotEqual{}} \HOLConst{CompletePB} \HOLSymConst{\HOLTokenConj{}} \HOLConst{MoveToPB} \HOLSymConst{\HOLTokenNotEqual{}} \HOLConst{unAuthenticated} \HOLSymConst{\HOLTokenConj{}}
   \HOLConst{ConductPB} \HOLSymConst{\HOLTokenNotEqual{}} \HOLConst{CompletePB} \HOLSymConst{\HOLTokenConj{}} \HOLConst{ConductPB} \HOLSymConst{\HOLTokenNotEqual{}} \HOLConst{unAuthenticated} \HOLSymConst{\HOLTokenConj{}}
   \HOLConst{CompletePB} \HOLSymConst{\HOLTokenNotEqual{}} \HOLConst{unAuthenticated}
\end{SaveVerbatim}
\newcommand{\HOLPBTypeTheoremsslOutputXXdistinctXXclauses}{\UseVerbatim{HOLPBTypeTheoremsslOutputXXdistinctXXclauses}}
\begin{SaveVerbatim}{HOLPBTypeTheoremsslStateXXdistinctXXclauses}
\HOLTokenTurnstile{} \HOLConst{PLAN_PB} \HOLSymConst{\HOLTokenNotEqual{}} \HOLConst{MOVE_TO_ORP} \HOLSymConst{\HOLTokenConj{}} \HOLConst{PLAN_PB} \HOLSymConst{\HOLTokenNotEqual{}} \HOLConst{CONDUCT_ORP} \HOLSymConst{\HOLTokenConj{}}
   \HOLConst{PLAN_PB} \HOLSymConst{\HOLTokenNotEqual{}} \HOLConst{MOVE_TO_PB} \HOLSymConst{\HOLTokenConj{}} \HOLConst{PLAN_PB} \HOLSymConst{\HOLTokenNotEqual{}} \HOLConst{CONDUCT_PB} \HOLSymConst{\HOLTokenConj{}}
   \HOLConst{PLAN_PB} \HOLSymConst{\HOLTokenNotEqual{}} \HOLConst{COMPLETE_PB} \HOLSymConst{\HOLTokenConj{}} \HOLConst{MOVE_TO_ORP} \HOLSymConst{\HOLTokenNotEqual{}} \HOLConst{CONDUCT_ORP} \HOLSymConst{\HOLTokenConj{}}
   \HOLConst{MOVE_TO_ORP} \HOLSymConst{\HOLTokenNotEqual{}} \HOLConst{MOVE_TO_PB} \HOLSymConst{\HOLTokenConj{}} \HOLConst{MOVE_TO_ORP} \HOLSymConst{\HOLTokenNotEqual{}} \HOLConst{CONDUCT_PB} \HOLSymConst{\HOLTokenConj{}}
   \HOLConst{MOVE_TO_ORP} \HOLSymConst{\HOLTokenNotEqual{}} \HOLConst{COMPLETE_PB} \HOLSymConst{\HOLTokenConj{}} \HOLConst{CONDUCT_ORP} \HOLSymConst{\HOLTokenNotEqual{}} \HOLConst{MOVE_TO_PB} \HOLSymConst{\HOLTokenConj{}}
   \HOLConst{CONDUCT_ORP} \HOLSymConst{\HOLTokenNotEqual{}} \HOLConst{CONDUCT_PB} \HOLSymConst{\HOLTokenConj{}} \HOLConst{CONDUCT_ORP} \HOLSymConst{\HOLTokenNotEqual{}} \HOLConst{COMPLETE_PB} \HOLSymConst{\HOLTokenConj{}}
   \HOLConst{MOVE_TO_PB} \HOLSymConst{\HOLTokenNotEqual{}} \HOLConst{CONDUCT_PB} \HOLSymConst{\HOLTokenConj{}} \HOLConst{MOVE_TO_PB} \HOLSymConst{\HOLTokenNotEqual{}} \HOLConst{COMPLETE_PB} \HOLSymConst{\HOLTokenConj{}}
   \HOLConst{CONDUCT_PB} \HOLSymConst{\HOLTokenNotEqual{}} \HOLConst{COMPLETE_PB}
\end{SaveVerbatim}
\newcommand{\HOLPBTypeTheoremsslStateXXdistinctXXclauses}{\UseVerbatim{HOLPBTypeTheoremsslStateXXdistinctXXclauses}}
\newcommand{\HOLPBTypeTheorems}{
\HOLThmTag{PBType}{slCommand_distinct_clauses}\HOLPBTypeTheoremsslCommandXXdistinctXXclauses
\HOLThmTag{PBType}{slOutput_distinct_clauses}\HOLPBTypeTheoremsslOutputXXdistinctXXclauses
\HOLThmTag{PBType}{slState_distinct_clauses}\HOLPBTypeTheoremsslStateXXdistinctXXclauses
}

\newcommand{\HOLssmPBDate}{09 July 2017}
\newcommand{\HOLssmPBTime}{22:25}
\begin{SaveVerbatim}{HOLssmPBDefinitionssecContextXXdef}
\HOLTokenTurnstile{} \HOLSymConst{\HOLTokenForall{}}\HOLBoundVar{cmd}.
     \HOLConst{secContext} \HOLBoundVar{cmd} \HOLSymConst{=}
     [\HOLConst{Name} \HOLConst{PlatoonLeader} \HOLConst{controls} \HOLConst{prop} (\HOLConst{SOME} (\HOLConst{SLc} \HOLBoundVar{cmd}))]
\end{SaveVerbatim}
\newcommand{\HOLssmPBDefinitionssecContextXXdef}{\UseVerbatim{HOLssmPBDefinitionssecContextXXdef}}
\begin{SaveVerbatim}{HOLssmPBDefinitionsssmPBStateInterpXXdef}
\HOLTokenTurnstile{} \HOLSymConst{\HOLTokenForall{}}\HOLBoundVar{state}. \HOLConst{ssmPBStateInterp} \HOLBoundVar{state} \HOLSymConst{=} \HOLConst{TT}
\end{SaveVerbatim}
\newcommand{\HOLssmPBDefinitionsssmPBStateInterpXXdef}{\UseVerbatim{HOLssmPBDefinitionsssmPBStateInterpXXdef}}
\newcommand{\HOLssmPBDefinitions}{
\HOLDfnTag{ssmPB}{secContext_def}\HOLssmPBDefinitionssecContextXXdef
\HOLDfnTag{ssmPB}{ssmPBStateInterp_def}\HOLssmPBDefinitionsssmPBStateInterpXXdef
}
\begin{SaveVerbatim}{HOLssmPBTheoremsauthenticationTestXXcmdXXrejectXXlemma}
\HOLTokenTurnstile{} \HOLSymConst{\HOLTokenForall{}}\HOLBoundVar{cmd}. \HOLSymConst{\HOLTokenNeg{}}\HOLConst{authenticationTest} (\HOLConst{prop} (\HOLConst{SOME} \HOLBoundVar{cmd}))
\end{SaveVerbatim}
\newcommand{\HOLssmPBTheoremsauthenticationTestXXcmdXXrejectXXlemma}{\UseVerbatim{HOLssmPBTheoremsauthenticationTestXXcmdXXrejectXXlemma}}
\begin{SaveVerbatim}{HOLssmPBTheoremsauthenticationTestXXdef}
\HOLTokenTurnstile{} (\HOLConst{authenticationTest} (\HOLConst{Name} \HOLConst{PlatoonLeader} \HOLConst{says} \HOLConst{prop} \HOLFreeVar{cmd}) \HOLSymConst{\HOLTokenEquiv{}}
    \HOLConst{T}) \HOLSymConst{\HOLTokenConj{}} (\HOLConst{authenticationTest} \HOLConst{TT} \HOLSymConst{\HOLTokenEquiv{}} \HOLConst{F}) \HOLSymConst{\HOLTokenConj{}}
   (\HOLConst{authenticationTest} \HOLConst{FF} \HOLSymConst{\HOLTokenEquiv{}} \HOLConst{F}) \HOLSymConst{\HOLTokenConj{}}
   (\HOLConst{authenticationTest} (\HOLConst{prop} \HOLFreeVar{v}) \HOLSymConst{\HOLTokenEquiv{}} \HOLConst{F}) \HOLSymConst{\HOLTokenConj{}}
   (\HOLConst{authenticationTest} (\HOLConst{notf} \HOLFreeVar{v\sb{\mathrm{1}}}) \HOLSymConst{\HOLTokenEquiv{}} \HOLConst{F}) \HOLSymConst{\HOLTokenConj{}}
   (\HOLConst{authenticationTest} (\HOLFreeVar{v\sb{\mathrm{2}}} \HOLConst{andf} \HOLFreeVar{v\sb{\mathrm{3}}}) \HOLSymConst{\HOLTokenEquiv{}} \HOLConst{F}) \HOLSymConst{\HOLTokenConj{}}
   (\HOLConst{authenticationTest} (\HOLFreeVar{v\sb{\mathrm{4}}} \HOLConst{orf} \HOLFreeVar{v\sb{\mathrm{5}}}) \HOLSymConst{\HOLTokenEquiv{}} \HOLConst{F}) \HOLSymConst{\HOLTokenConj{}}
   (\HOLConst{authenticationTest} (\HOLFreeVar{v\sb{\mathrm{6}}} \HOLConst{impf} \HOLFreeVar{v\sb{\mathrm{7}}}) \HOLSymConst{\HOLTokenEquiv{}} \HOLConst{F}) \HOLSymConst{\HOLTokenConj{}}
   (\HOLConst{authenticationTest} (\HOLFreeVar{v\sb{\mathrm{8}}} \HOLConst{eqf} \HOLFreeVar{v\sb{\mathrm{9}}}) \HOLSymConst{\HOLTokenEquiv{}} \HOLConst{F}) \HOLSymConst{\HOLTokenConj{}}
   (\HOLConst{authenticationTest} (\HOLFreeVar{v\sb{\mathrm{10}}} \HOLConst{says} \HOLConst{TT}) \HOLSymConst{\HOLTokenEquiv{}} \HOLConst{F}) \HOLSymConst{\HOLTokenConj{}}
   (\HOLConst{authenticationTest} (\HOLFreeVar{v\sb{\mathrm{10}}} \HOLConst{says} \HOLConst{FF}) \HOLSymConst{\HOLTokenEquiv{}} \HOLConst{F}) \HOLSymConst{\HOLTokenConj{}}
   (\HOLConst{authenticationTest} (\HOLFreeVar{v133} \HOLConst{meet} \HOLFreeVar{v134} \HOLConst{says} \HOLConst{prop} \HOLFreeVar{v\sb{\mathrm{66}}}) \HOLSymConst{\HOLTokenEquiv{}} \HOLConst{F}) \HOLSymConst{\HOLTokenConj{}}
   (\HOLConst{authenticationTest} (\HOLFreeVar{v135} \HOLConst{quoting} \HOLFreeVar{v136} \HOLConst{says} \HOLConst{prop} \HOLFreeVar{v\sb{\mathrm{66}}}) \HOLSymConst{\HOLTokenEquiv{}} \HOLConst{F}) \HOLSymConst{\HOLTokenConj{}}
   (\HOLConst{authenticationTest} (\HOLFreeVar{v\sb{\mathrm{10}}} \HOLConst{says} \HOLConst{notf} \HOLFreeVar{v\sb{\mathrm{67}}}) \HOLSymConst{\HOLTokenEquiv{}} \HOLConst{F}) \HOLSymConst{\HOLTokenConj{}}
   (\HOLConst{authenticationTest} (\HOLFreeVar{v\sb{\mathrm{10}}} \HOLConst{says} (\HOLFreeVar{v\sb{\mathrm{68}}} \HOLConst{andf} \HOLFreeVar{v\sb{\mathrm{69}}})) \HOLSymConst{\HOLTokenEquiv{}} \HOLConst{F}) \HOLSymConst{\HOLTokenConj{}}
   (\HOLConst{authenticationTest} (\HOLFreeVar{v\sb{\mathrm{10}}} \HOLConst{says} (\HOLFreeVar{v\sb{\mathrm{70}}} \HOLConst{orf} \HOLFreeVar{v\sb{\mathrm{71}}})) \HOLSymConst{\HOLTokenEquiv{}} \HOLConst{F}) \HOLSymConst{\HOLTokenConj{}}
   (\HOLConst{authenticationTest} (\HOLFreeVar{v\sb{\mathrm{10}}} \HOLConst{says} (\HOLFreeVar{v\sb{\mathrm{72}}} \HOLConst{impf} \HOLFreeVar{v\sb{\mathrm{73}}})) \HOLSymConst{\HOLTokenEquiv{}} \HOLConst{F}) \HOLSymConst{\HOLTokenConj{}}
   (\HOLConst{authenticationTest} (\HOLFreeVar{v\sb{\mathrm{10}}} \HOLConst{says} (\HOLFreeVar{v\sb{\mathrm{74}}} \HOLConst{eqf} \HOLFreeVar{v\sb{\mathrm{75}}})) \HOLSymConst{\HOLTokenEquiv{}} \HOLConst{F}) \HOLSymConst{\HOLTokenConj{}}
   (\HOLConst{authenticationTest} (\HOLFreeVar{v\sb{\mathrm{10}}} \HOLConst{says} \HOLFreeVar{v\sb{\mathrm{76}}} \HOLConst{says} \HOLFreeVar{v\sb{\mathrm{77}}}) \HOLSymConst{\HOLTokenEquiv{}} \HOLConst{F}) \HOLSymConst{\HOLTokenConj{}}
   (\HOLConst{authenticationTest} (\HOLFreeVar{v\sb{\mathrm{10}}} \HOLConst{says} \HOLFreeVar{v\sb{\mathrm{78}}} \HOLConst{speaks_for} \HOLFreeVar{v\sb{\mathrm{79}}}) \HOLSymConst{\HOLTokenEquiv{}} \HOLConst{F}) \HOLSymConst{\HOLTokenConj{}}
   (\HOLConst{authenticationTest} (\HOLFreeVar{v\sb{\mathrm{10}}} \HOLConst{says} \HOLFreeVar{v\sb{\mathrm{80}}} \HOLConst{controls} \HOLFreeVar{v\sb{\mathrm{81}}}) \HOLSymConst{\HOLTokenEquiv{}} \HOLConst{F}) \HOLSymConst{\HOLTokenConj{}}
   (\HOLConst{authenticationTest} (\HOLFreeVar{v\sb{\mathrm{10}}} \HOLConst{says} \HOLConst{reps} \HOLFreeVar{v\sb{\mathrm{82}}} \HOLFreeVar{v\sb{\mathrm{83}}} \HOLFreeVar{v\sb{\mathrm{84}}}) \HOLSymConst{\HOLTokenEquiv{}} \HOLConst{F}) \HOLSymConst{\HOLTokenConj{}}
   (\HOLConst{authenticationTest} (\HOLFreeVar{v\sb{\mathrm{10}}} \HOLConst{says} \HOLFreeVar{v\sb{\mathrm{85}}} \HOLConst{domi} \HOLFreeVar{v\sb{\mathrm{86}}}) \HOLSymConst{\HOLTokenEquiv{}} \HOLConst{F}) \HOLSymConst{\HOLTokenConj{}}
   (\HOLConst{authenticationTest} (\HOLFreeVar{v\sb{\mathrm{10}}} \HOLConst{says} \HOLFreeVar{v\sb{\mathrm{87}}} \HOLConst{eqi} \HOLFreeVar{v\sb{\mathrm{88}}}) \HOLSymConst{\HOLTokenEquiv{}} \HOLConst{F}) \HOLSymConst{\HOLTokenConj{}}
   (\HOLConst{authenticationTest} (\HOLFreeVar{v\sb{\mathrm{10}}} \HOLConst{says} \HOLFreeVar{v\sb{\mathrm{89}}} \HOLConst{doms} \HOLFreeVar{v\sb{\mathrm{90}}}) \HOLSymConst{\HOLTokenEquiv{}} \HOLConst{F}) \HOLSymConst{\HOLTokenConj{}}
   (\HOLConst{authenticationTest} (\HOLFreeVar{v\sb{\mathrm{10}}} \HOLConst{says} \HOLFreeVar{v\sb{\mathrm{91}}} \HOLConst{eqs} \HOLFreeVar{v\sb{\mathrm{92}}}) \HOLSymConst{\HOLTokenEquiv{}} \HOLConst{F}) \HOLSymConst{\HOLTokenConj{}}
   (\HOLConst{authenticationTest} (\HOLFreeVar{v\sb{\mathrm{10}}} \HOLConst{says} \HOLFreeVar{v\sb{\mathrm{93}}} \HOLConst{eqn} \HOLFreeVar{v\sb{\mathrm{94}}}) \HOLSymConst{\HOLTokenEquiv{}} \HOLConst{F}) \HOLSymConst{\HOLTokenConj{}}
   (\HOLConst{authenticationTest} (\HOLFreeVar{v\sb{\mathrm{10}}} \HOLConst{says} \HOLFreeVar{v\sb{\mathrm{95}}} \HOLConst{lte} \HOLFreeVar{v\sb{\mathrm{96}}}) \HOLSymConst{\HOLTokenEquiv{}} \HOLConst{F}) \HOLSymConst{\HOLTokenConj{}}
   (\HOLConst{authenticationTest} (\HOLFreeVar{v\sb{\mathrm{10}}} \HOLConst{says} \HOLFreeVar{v\sb{\mathrm{97}}} \HOLConst{lt} \HOLFreeVar{v\sb{\mathrm{98}}}) \HOLSymConst{\HOLTokenEquiv{}} \HOLConst{F}) \HOLSymConst{\HOLTokenConj{}}
   (\HOLConst{authenticationTest} (\HOLFreeVar{v\sb{\mathrm{12}}} \HOLConst{speaks_for} \HOLFreeVar{v\sb{\mathrm{13}}}) \HOLSymConst{\HOLTokenEquiv{}} \HOLConst{F}) \HOLSymConst{\HOLTokenConj{}}
   (\HOLConst{authenticationTest} (\HOLFreeVar{v\sb{\mathrm{14}}} \HOLConst{controls} \HOLFreeVar{v\sb{\mathrm{15}}}) \HOLSymConst{\HOLTokenEquiv{}} \HOLConst{F}) \HOLSymConst{\HOLTokenConj{}}
   (\HOLConst{authenticationTest} (\HOLConst{reps} \HOLFreeVar{v\sb{\mathrm{16}}} \HOLFreeVar{v\sb{\mathrm{17}}} \HOLFreeVar{v\sb{\mathrm{18}}}) \HOLSymConst{\HOLTokenEquiv{}} \HOLConst{F}) \HOLSymConst{\HOLTokenConj{}}
   (\HOLConst{authenticationTest} (\HOLFreeVar{v\sb{\mathrm{19}}} \HOLConst{domi} \HOLFreeVar{v\sb{\mathrm{20}}}) \HOLSymConst{\HOLTokenEquiv{}} \HOLConst{F}) \HOLSymConst{\HOLTokenConj{}}
   (\HOLConst{authenticationTest} (\HOLFreeVar{v\sb{\mathrm{21}}} \HOLConst{eqi} \HOLFreeVar{v\sb{\mathrm{22}}}) \HOLSymConst{\HOLTokenEquiv{}} \HOLConst{F}) \HOLSymConst{\HOLTokenConj{}}
   (\HOLConst{authenticationTest} (\HOLFreeVar{v\sb{\mathrm{23}}} \HOLConst{doms} \HOLFreeVar{v\sb{\mathrm{24}}}) \HOLSymConst{\HOLTokenEquiv{}} \HOLConst{F}) \HOLSymConst{\HOLTokenConj{}}
   (\HOLConst{authenticationTest} (\HOLFreeVar{v\sb{\mathrm{25}}} \HOLConst{eqs} \HOLFreeVar{v\sb{\mathrm{26}}}) \HOLSymConst{\HOLTokenEquiv{}} \HOLConst{F}) \HOLSymConst{\HOLTokenConj{}}
   (\HOLConst{authenticationTest} (\HOLFreeVar{v\sb{\mathrm{27}}} \HOLConst{eqn} \HOLFreeVar{v\sb{\mathrm{28}}}) \HOLSymConst{\HOLTokenEquiv{}} \HOLConst{F}) \HOLSymConst{\HOLTokenConj{}}
   (\HOLConst{authenticationTest} (\HOLFreeVar{v\sb{\mathrm{29}}} \HOLConst{lte} \HOLFreeVar{v\sb{\mathrm{30}}}) \HOLSymConst{\HOLTokenEquiv{}} \HOLConst{F}) \HOLSymConst{\HOLTokenConj{}}
   (\HOLConst{authenticationTest} (\HOLFreeVar{v\sb{\mathrm{31}}} \HOLConst{lt} \HOLFreeVar{v\sb{\mathrm{32}}}) \HOLSymConst{\HOLTokenEquiv{}} \HOLConst{F})
\end{SaveVerbatim}
\newcommand{\HOLssmPBTheoremsauthenticationTestXXdef}{\UseVerbatim{HOLssmPBTheoremsauthenticationTestXXdef}}
\begin{SaveVerbatim}{HOLssmPBTheoremsauthenticationTestXXind}
\HOLTokenTurnstile{} \HOLSymConst{\HOLTokenForall{}}\HOLBoundVar{P}.
     (\HOLSymConst{\HOLTokenForall{}}\HOLBoundVar{cmd}. \HOLBoundVar{P} (\HOLConst{Name} \HOLConst{PlatoonLeader} \HOLConst{says} \HOLConst{prop} \HOLBoundVar{cmd})) \HOLSymConst{\HOLTokenConj{}} \HOLBoundVar{P} \HOLConst{TT} \HOLSymConst{\HOLTokenConj{}}
     \HOLBoundVar{P} \HOLConst{FF} \HOLSymConst{\HOLTokenConj{}} (\HOLSymConst{\HOLTokenForall{}}\HOLBoundVar{v}. \HOLBoundVar{P} (\HOLConst{prop} \HOLBoundVar{v})) \HOLSymConst{\HOLTokenConj{}} (\HOLSymConst{\HOLTokenForall{}}\HOLBoundVar{v\sb{\mathrm{1}}}. \HOLBoundVar{P} (\HOLConst{notf} \HOLBoundVar{v\sb{\mathrm{1}}})) \HOLSymConst{\HOLTokenConj{}}
     (\HOLSymConst{\HOLTokenForall{}}\HOLBoundVar{v\sb{\mathrm{2}}} \HOLBoundVar{v\sb{\mathrm{3}}}. \HOLBoundVar{P} (\HOLBoundVar{v\sb{\mathrm{2}}} \HOLConst{andf} \HOLBoundVar{v\sb{\mathrm{3}}})) \HOLSymConst{\HOLTokenConj{}} (\HOLSymConst{\HOLTokenForall{}}\HOLBoundVar{v\sb{\mathrm{4}}} \HOLBoundVar{v\sb{\mathrm{5}}}. \HOLBoundVar{P} (\HOLBoundVar{v\sb{\mathrm{4}}} \HOLConst{orf} \HOLBoundVar{v\sb{\mathrm{5}}})) \HOLSymConst{\HOLTokenConj{}}
     (\HOLSymConst{\HOLTokenForall{}}\HOLBoundVar{v\sb{\mathrm{6}}} \HOLBoundVar{v\sb{\mathrm{7}}}. \HOLBoundVar{P} (\HOLBoundVar{v\sb{\mathrm{6}}} \HOLConst{impf} \HOLBoundVar{v\sb{\mathrm{7}}})) \HOLSymConst{\HOLTokenConj{}} (\HOLSymConst{\HOLTokenForall{}}\HOLBoundVar{v\sb{\mathrm{8}}} \HOLBoundVar{v\sb{\mathrm{9}}}. \HOLBoundVar{P} (\HOLBoundVar{v\sb{\mathrm{8}}} \HOLConst{eqf} \HOLBoundVar{v\sb{\mathrm{9}}})) \HOLSymConst{\HOLTokenConj{}}
     (\HOLSymConst{\HOLTokenForall{}}\HOLBoundVar{v\sb{\mathrm{10}}}. \HOLBoundVar{P} (\HOLBoundVar{v\sb{\mathrm{10}}} \HOLConst{says} \HOLConst{TT})) \HOLSymConst{\HOLTokenConj{}} (\HOLSymConst{\HOLTokenForall{}}\HOLBoundVar{v\sb{\mathrm{10}}}. \HOLBoundVar{P} (\HOLBoundVar{v\sb{\mathrm{10}}} \HOLConst{says} \HOLConst{FF})) \HOLSymConst{\HOLTokenConj{}}
     (\HOLSymConst{\HOLTokenForall{}}\HOLBoundVar{v133} \HOLBoundVar{v134} \HOLBoundVar{v\sb{\mathrm{66}}}. \HOLBoundVar{P} (\HOLBoundVar{v133} \HOLConst{meet} \HOLBoundVar{v134} \HOLConst{says} \HOLConst{prop} \HOLBoundVar{v\sb{\mathrm{66}}})) \HOLSymConst{\HOLTokenConj{}}
     (\HOLSymConst{\HOLTokenForall{}}\HOLBoundVar{v135} \HOLBoundVar{v136} \HOLBoundVar{v\sb{\mathrm{66}}}. \HOLBoundVar{P} (\HOLBoundVar{v135} \HOLConst{quoting} \HOLBoundVar{v136} \HOLConst{says} \HOLConst{prop} \HOLBoundVar{v\sb{\mathrm{66}}})) \HOLSymConst{\HOLTokenConj{}}
     (\HOLSymConst{\HOLTokenForall{}}\HOLBoundVar{v\sb{\mathrm{10}}} \HOLBoundVar{v\sb{\mathrm{67}}}. \HOLBoundVar{P} (\HOLBoundVar{v\sb{\mathrm{10}}} \HOLConst{says} \HOLConst{notf} \HOLBoundVar{v\sb{\mathrm{67}}})) \HOLSymConst{\HOLTokenConj{}}
     (\HOLSymConst{\HOLTokenForall{}}\HOLBoundVar{v\sb{\mathrm{10}}} \HOLBoundVar{v\sb{\mathrm{68}}} \HOLBoundVar{v\sb{\mathrm{69}}}. \HOLBoundVar{P} (\HOLBoundVar{v\sb{\mathrm{10}}} \HOLConst{says} (\HOLBoundVar{v\sb{\mathrm{68}}} \HOLConst{andf} \HOLBoundVar{v\sb{\mathrm{69}}}))) \HOLSymConst{\HOLTokenConj{}}
     (\HOLSymConst{\HOLTokenForall{}}\HOLBoundVar{v\sb{\mathrm{10}}} \HOLBoundVar{v\sb{\mathrm{70}}} \HOLBoundVar{v\sb{\mathrm{71}}}. \HOLBoundVar{P} (\HOLBoundVar{v\sb{\mathrm{10}}} \HOLConst{says} (\HOLBoundVar{v\sb{\mathrm{70}}} \HOLConst{orf} \HOLBoundVar{v\sb{\mathrm{71}}}))) \HOLSymConst{\HOLTokenConj{}}
     (\HOLSymConst{\HOLTokenForall{}}\HOLBoundVar{v\sb{\mathrm{10}}} \HOLBoundVar{v\sb{\mathrm{72}}} \HOLBoundVar{v\sb{\mathrm{73}}}. \HOLBoundVar{P} (\HOLBoundVar{v\sb{\mathrm{10}}} \HOLConst{says} (\HOLBoundVar{v\sb{\mathrm{72}}} \HOLConst{impf} \HOLBoundVar{v\sb{\mathrm{73}}}))) \HOLSymConst{\HOLTokenConj{}}
     (\HOLSymConst{\HOLTokenForall{}}\HOLBoundVar{v\sb{\mathrm{10}}} \HOLBoundVar{v\sb{\mathrm{74}}} \HOLBoundVar{v\sb{\mathrm{75}}}. \HOLBoundVar{P} (\HOLBoundVar{v\sb{\mathrm{10}}} \HOLConst{says} (\HOLBoundVar{v\sb{\mathrm{74}}} \HOLConst{eqf} \HOLBoundVar{v\sb{\mathrm{75}}}))) \HOLSymConst{\HOLTokenConj{}}
     (\HOLSymConst{\HOLTokenForall{}}\HOLBoundVar{v\sb{\mathrm{10}}} \HOLBoundVar{v\sb{\mathrm{76}}} \HOLBoundVar{v\sb{\mathrm{77}}}. \HOLBoundVar{P} (\HOLBoundVar{v\sb{\mathrm{10}}} \HOLConst{says} \HOLBoundVar{v\sb{\mathrm{76}}} \HOLConst{says} \HOLBoundVar{v\sb{\mathrm{77}}})) \HOLSymConst{\HOLTokenConj{}}
     (\HOLSymConst{\HOLTokenForall{}}\HOLBoundVar{v\sb{\mathrm{10}}} \HOLBoundVar{v\sb{\mathrm{78}}} \HOLBoundVar{v\sb{\mathrm{79}}}. \HOLBoundVar{P} (\HOLBoundVar{v\sb{\mathrm{10}}} \HOLConst{says} \HOLBoundVar{v\sb{\mathrm{78}}} \HOLConst{speaks_for} \HOLBoundVar{v\sb{\mathrm{79}}})) \HOLSymConst{\HOLTokenConj{}}
     (\HOLSymConst{\HOLTokenForall{}}\HOLBoundVar{v\sb{\mathrm{10}}} \HOLBoundVar{v\sb{\mathrm{80}}} \HOLBoundVar{v\sb{\mathrm{81}}}. \HOLBoundVar{P} (\HOLBoundVar{v\sb{\mathrm{10}}} \HOLConst{says} \HOLBoundVar{v\sb{\mathrm{80}}} \HOLConst{controls} \HOLBoundVar{v\sb{\mathrm{81}}})) \HOLSymConst{\HOLTokenConj{}}
     (\HOLSymConst{\HOLTokenForall{}}\HOLBoundVar{v\sb{\mathrm{10}}} \HOLBoundVar{v\sb{\mathrm{82}}} \HOLBoundVar{v\sb{\mathrm{83}}} \HOLBoundVar{v\sb{\mathrm{84}}}. \HOLBoundVar{P} (\HOLBoundVar{v\sb{\mathrm{10}}} \HOLConst{says} \HOLConst{reps} \HOLBoundVar{v\sb{\mathrm{82}}} \HOLBoundVar{v\sb{\mathrm{83}}} \HOLBoundVar{v\sb{\mathrm{84}}})) \HOLSymConst{\HOLTokenConj{}}
     (\HOLSymConst{\HOLTokenForall{}}\HOLBoundVar{v\sb{\mathrm{10}}} \HOLBoundVar{v\sb{\mathrm{85}}} \HOLBoundVar{v\sb{\mathrm{86}}}. \HOLBoundVar{P} (\HOLBoundVar{v\sb{\mathrm{10}}} \HOLConst{says} \HOLBoundVar{v\sb{\mathrm{85}}} \HOLConst{domi} \HOLBoundVar{v\sb{\mathrm{86}}})) \HOLSymConst{\HOLTokenConj{}}
     (\HOLSymConst{\HOLTokenForall{}}\HOLBoundVar{v\sb{\mathrm{10}}} \HOLBoundVar{v\sb{\mathrm{87}}} \HOLBoundVar{v\sb{\mathrm{88}}}. \HOLBoundVar{P} (\HOLBoundVar{v\sb{\mathrm{10}}} \HOLConst{says} \HOLBoundVar{v\sb{\mathrm{87}}} \HOLConst{eqi} \HOLBoundVar{v\sb{\mathrm{88}}})) \HOLSymConst{\HOLTokenConj{}}
     (\HOLSymConst{\HOLTokenForall{}}\HOLBoundVar{v\sb{\mathrm{10}}} \HOLBoundVar{v\sb{\mathrm{89}}} \HOLBoundVar{v\sb{\mathrm{90}}}. \HOLBoundVar{P} (\HOLBoundVar{v\sb{\mathrm{10}}} \HOLConst{says} \HOLBoundVar{v\sb{\mathrm{89}}} \HOLConst{doms} \HOLBoundVar{v\sb{\mathrm{90}}})) \HOLSymConst{\HOLTokenConj{}}
     (\HOLSymConst{\HOLTokenForall{}}\HOLBoundVar{v\sb{\mathrm{10}}} \HOLBoundVar{v\sb{\mathrm{91}}} \HOLBoundVar{v\sb{\mathrm{92}}}. \HOLBoundVar{P} (\HOLBoundVar{v\sb{\mathrm{10}}} \HOLConst{says} \HOLBoundVar{v\sb{\mathrm{91}}} \HOLConst{eqs} \HOLBoundVar{v\sb{\mathrm{92}}})) \HOLSymConst{\HOLTokenConj{}}
     (\HOLSymConst{\HOLTokenForall{}}\HOLBoundVar{v\sb{\mathrm{10}}} \HOLBoundVar{v\sb{\mathrm{93}}} \HOLBoundVar{v\sb{\mathrm{94}}}. \HOLBoundVar{P} (\HOLBoundVar{v\sb{\mathrm{10}}} \HOLConst{says} \HOLBoundVar{v\sb{\mathrm{93}}} \HOLConst{eqn} \HOLBoundVar{v\sb{\mathrm{94}}})) \HOLSymConst{\HOLTokenConj{}}
     (\HOLSymConst{\HOLTokenForall{}}\HOLBoundVar{v\sb{\mathrm{10}}} \HOLBoundVar{v\sb{\mathrm{95}}} \HOLBoundVar{v\sb{\mathrm{96}}}. \HOLBoundVar{P} (\HOLBoundVar{v\sb{\mathrm{10}}} \HOLConst{says} \HOLBoundVar{v\sb{\mathrm{95}}} \HOLConst{lte} \HOLBoundVar{v\sb{\mathrm{96}}})) \HOLSymConst{\HOLTokenConj{}}
     (\HOLSymConst{\HOLTokenForall{}}\HOLBoundVar{v\sb{\mathrm{10}}} \HOLBoundVar{v\sb{\mathrm{97}}} \HOLBoundVar{v\sb{\mathrm{98}}}. \HOLBoundVar{P} (\HOLBoundVar{v\sb{\mathrm{10}}} \HOLConst{says} \HOLBoundVar{v\sb{\mathrm{97}}} \HOLConst{lt} \HOLBoundVar{v\sb{\mathrm{98}}})) \HOLSymConst{\HOLTokenConj{}}
     (\HOLSymConst{\HOLTokenForall{}}\HOLBoundVar{v\sb{\mathrm{12}}} \HOLBoundVar{v\sb{\mathrm{13}}}. \HOLBoundVar{P} (\HOLBoundVar{v\sb{\mathrm{12}}} \HOLConst{speaks_for} \HOLBoundVar{v\sb{\mathrm{13}}})) \HOLSymConst{\HOLTokenConj{}}
     (\HOLSymConst{\HOLTokenForall{}}\HOLBoundVar{v\sb{\mathrm{14}}} \HOLBoundVar{v\sb{\mathrm{15}}}. \HOLBoundVar{P} (\HOLBoundVar{v\sb{\mathrm{14}}} \HOLConst{controls} \HOLBoundVar{v\sb{\mathrm{15}}})) \HOLSymConst{\HOLTokenConj{}}
     (\HOLSymConst{\HOLTokenForall{}}\HOLBoundVar{v\sb{\mathrm{16}}} \HOLBoundVar{v\sb{\mathrm{17}}} \HOLBoundVar{v\sb{\mathrm{18}}}. \HOLBoundVar{P} (\HOLConst{reps} \HOLBoundVar{v\sb{\mathrm{16}}} \HOLBoundVar{v\sb{\mathrm{17}}} \HOLBoundVar{v\sb{\mathrm{18}}})) \HOLSymConst{\HOLTokenConj{}}
     (\HOLSymConst{\HOLTokenForall{}}\HOLBoundVar{v\sb{\mathrm{19}}} \HOLBoundVar{v\sb{\mathrm{20}}}. \HOLBoundVar{P} (\HOLBoundVar{v\sb{\mathrm{19}}} \HOLConst{domi} \HOLBoundVar{v\sb{\mathrm{20}}})) \HOLSymConst{\HOLTokenConj{}}
     (\HOLSymConst{\HOLTokenForall{}}\HOLBoundVar{v\sb{\mathrm{21}}} \HOLBoundVar{v\sb{\mathrm{22}}}. \HOLBoundVar{P} (\HOLBoundVar{v\sb{\mathrm{21}}} \HOLConst{eqi} \HOLBoundVar{v\sb{\mathrm{22}}})) \HOLSymConst{\HOLTokenConj{}}
     (\HOLSymConst{\HOLTokenForall{}}\HOLBoundVar{v\sb{\mathrm{23}}} \HOLBoundVar{v\sb{\mathrm{24}}}. \HOLBoundVar{P} (\HOLBoundVar{v\sb{\mathrm{23}}} \HOLConst{doms} \HOLBoundVar{v\sb{\mathrm{24}}})) \HOLSymConst{\HOLTokenConj{}}
     (\HOLSymConst{\HOLTokenForall{}}\HOLBoundVar{v\sb{\mathrm{25}}} \HOLBoundVar{v\sb{\mathrm{26}}}. \HOLBoundVar{P} (\HOLBoundVar{v\sb{\mathrm{25}}} \HOLConst{eqs} \HOLBoundVar{v\sb{\mathrm{26}}})) \HOLSymConst{\HOLTokenConj{}} (\HOLSymConst{\HOLTokenForall{}}\HOLBoundVar{v\sb{\mathrm{27}}} \HOLBoundVar{v\sb{\mathrm{28}}}. \HOLBoundVar{P} (\HOLBoundVar{v\sb{\mathrm{27}}} \HOLConst{eqn} \HOLBoundVar{v\sb{\mathrm{28}}})) \HOLSymConst{\HOLTokenConj{}}
     (\HOLSymConst{\HOLTokenForall{}}\HOLBoundVar{v\sb{\mathrm{29}}} \HOLBoundVar{v\sb{\mathrm{30}}}. \HOLBoundVar{P} (\HOLBoundVar{v\sb{\mathrm{29}}} \HOLConst{lte} \HOLBoundVar{v\sb{\mathrm{30}}})) \HOLSymConst{\HOLTokenConj{}} (\HOLSymConst{\HOLTokenForall{}}\HOLBoundVar{v\sb{\mathrm{31}}} \HOLBoundVar{v\sb{\mathrm{32}}}. \HOLBoundVar{P} (\HOLBoundVar{v\sb{\mathrm{31}}} \HOLConst{lt} \HOLBoundVar{v\sb{\mathrm{32}}})) \HOLSymConst{\HOLTokenImp{}}
     \HOLSymConst{\HOLTokenForall{}}\HOLBoundVar{v}. \HOLBoundVar{P} \HOLBoundVar{v}
\end{SaveVerbatim}
\newcommand{\HOLssmPBTheoremsauthenticationTestXXind}{\UseVerbatim{HOLssmPBTheoremsauthenticationTestXXind}}
\begin{SaveVerbatim}{HOLssmPBTheoremsPBNSXXdef}
\HOLTokenTurnstile{} (\HOLConst{PBNS} \HOLConst{PLAN_PB} (\HOLConst{exec} (\HOLConst{SLc} \HOLConst{crossLD})) \HOLSymConst{=} \HOLConst{MOVE_TO_ORP}) \HOLSymConst{\HOLTokenConj{}}
   (\HOLConst{PBNS} \HOLConst{PLAN_PB} (\HOLConst{exec} (\HOLConst{SLc} \HOLConst{incomplete})) \HOLSymConst{=} \HOLConst{PLAN_PB}) \HOLSymConst{\HOLTokenConj{}}
   (\HOLConst{PBNS} \HOLConst{MOVE_TO_ORP} (\HOLConst{exec} (\HOLConst{SLc} \HOLConst{conductORP})) \HOLSymConst{=} \HOLConst{CONDUCT_ORP}) \HOLSymConst{\HOLTokenConj{}}
   (\HOLConst{PBNS} \HOLConst{MOVE_TO_ORP} (\HOLConst{exec} (\HOLConst{SLc} \HOLConst{incomplete})) \HOLSymConst{=} \HOLConst{MOVE_TO_ORP}) \HOLSymConst{\HOLTokenConj{}}
   (\HOLConst{PBNS} \HOLConst{CONDUCT_ORP} (\HOLConst{exec} (\HOLConst{SLc} \HOLConst{moveToPB})) \HOLSymConst{=} \HOLConst{MOVE_TO_PB}) \HOLSymConst{\HOLTokenConj{}}
   (\HOLConst{PBNS} \HOLConst{CONDUCT_ORP} (\HOLConst{exec} (\HOLConst{SLc} \HOLConst{incomplete})) \HOLSymConst{=} \HOLConst{CONDUCT_ORP}) \HOLSymConst{\HOLTokenConj{}}
   (\HOLConst{PBNS} \HOLConst{MOVE_TO_PB} (\HOLConst{exec} (\HOLConst{SLc} \HOLConst{conductPB})) \HOLSymConst{=} \HOLConst{CONDUCT_PB}) \HOLSymConst{\HOLTokenConj{}}
   (\HOLConst{PBNS} \HOLConst{MOVE_TO_PB} (\HOLConst{exec} (\HOLConst{SLc} \HOLConst{incomplete})) \HOLSymConst{=} \HOLConst{MOVE_TO_PB}) \HOLSymConst{\HOLTokenConj{}}
   (\HOLConst{PBNS} \HOLConst{CONDUCT_PB} (\HOLConst{exec} (\HOLConst{SLc} \HOLConst{completePB})) \HOLSymConst{=} \HOLConst{COMPLETE_PB}) \HOLSymConst{\HOLTokenConj{}}
   (\HOLConst{PBNS} \HOLConst{CONDUCT_PB} (\HOLConst{exec} (\HOLConst{SLc} \HOLConst{incomplete})) \HOLSymConst{=} \HOLConst{CONDUCT_PB}) \HOLSymConst{\HOLTokenConj{}}
   (\HOLConst{PBNS} \HOLConst{PLAN_PB} (\HOLConst{trap} (\HOLConst{SLc} \HOLConst{crossLD})) \HOLSymConst{=} \HOLConst{PLAN_PB}) \HOLSymConst{\HOLTokenConj{}}
   (\HOLConst{PBNS} \HOLConst{PLAN_PB} (\HOLConst{trap} (\HOLConst{SLc} \HOLConst{incomplete})) \HOLSymConst{=} \HOLConst{PLAN_PB}) \HOLSymConst{\HOLTokenConj{}}
   (\HOLConst{PBNS} \HOLConst{MOVE_TO_ORP} (\HOLConst{trap} (\HOLConst{SLc} \HOLConst{conductORP})) \HOLSymConst{=} \HOLConst{MOVE_TO_ORP}) \HOLSymConst{\HOLTokenConj{}}
   (\HOLConst{PBNS} \HOLConst{MOVE_TO_ORP} (\HOLConst{trap} (\HOLConst{SLc} \HOLConst{incomplete})) \HOLSymConst{=} \HOLConst{MOVE_TO_ORP}) \HOLSymConst{\HOLTokenConj{}}
   (\HOLConst{PBNS} \HOLConst{CONDUCT_ORP} (\HOLConst{trap} (\HOLConst{SLc} \HOLConst{moveToPB})) \HOLSymConst{=} \HOLConst{CONDUCT_ORP}) \HOLSymConst{\HOLTokenConj{}}
   (\HOLConst{PBNS} \HOLConst{CONDUCT_ORP} (\HOLConst{trap} (\HOLConst{SLc} \HOLConst{incomplete})) \HOLSymConst{=} \HOLConst{CONDUCT_ORP}) \HOLSymConst{\HOLTokenConj{}}
   (\HOLConst{PBNS} \HOLConst{MOVE_TO_PB} (\HOLConst{trap} (\HOLConst{SLc} \HOLConst{conductPB})) \HOLSymConst{=} \HOLConst{MOVE_TO_PB}) \HOLSymConst{\HOLTokenConj{}}
   (\HOLConst{PBNS} \HOLConst{MOVE_TO_PB} (\HOLConst{trap} (\HOLConst{SLc} \HOLConst{incomplete})) \HOLSymConst{=} \HOLConst{MOVE_TO_PB}) \HOLSymConst{\HOLTokenConj{}}
   (\HOLConst{PBNS} \HOLConst{CONDUCT_PB} (\HOLConst{trap} (\HOLConst{SLc} \HOLConst{completePB})) \HOLSymConst{=} \HOLConst{CONDUCT_PB}) \HOLSymConst{\HOLTokenConj{}}
   (\HOLConst{PBNS} \HOLConst{CONDUCT_PB} (\HOLConst{trap} (\HOLConst{SLc} \HOLConst{incomplete})) \HOLSymConst{=} \HOLConst{CONDUCT_PB}) \HOLSymConst{\HOLTokenConj{}}
   (\HOLConst{PBNS} \HOLConst{PLAN_PB} (\HOLConst{discard} (\HOLConst{SLc} \HOLConst{crossLD})) \HOLSymConst{=} \HOLConst{PLAN_PB}) \HOLSymConst{\HOLTokenConj{}}
   (\HOLConst{PBNS} \HOLConst{MOVE_TO_ORP} (\HOLConst{discard} (\HOLConst{SLc} \HOLConst{conductORP})) \HOLSymConst{=} \HOLConst{MOVE_TO_ORP}) \HOLSymConst{\HOLTokenConj{}}
   (\HOLConst{PBNS} \HOLConst{CONDUCT_ORP} (\HOLConst{discard} (\HOLConst{SLc} \HOLConst{moveToPB})) \HOLSymConst{=} \HOLConst{CONDUCT_ORP}) \HOLSymConst{\HOLTokenConj{}}
   (\HOLConst{PBNS} \HOLConst{MOVE_TO_PB} (\HOLConst{discard} (\HOLConst{SLc} \HOLConst{conductPB})) \HOLSymConst{=} \HOLConst{MOVE_TO_PB}) \HOLSymConst{\HOLTokenConj{}}
   (\HOLConst{PBNS} \HOLConst{CONDUCT_PB} (\HOLConst{discard} (\HOLConst{SLc} \HOLConst{completePB})) \HOLSymConst{=} \HOLConst{CONDUCT_PB})
\end{SaveVerbatim}
\newcommand{\HOLssmPBTheoremsPBNSXXdef}{\UseVerbatim{HOLssmPBTheoremsPBNSXXdef}}
\begin{SaveVerbatim}{HOLssmPBTheoremsPBNSXXind}
\HOLTokenTurnstile{} \HOLSymConst{\HOLTokenForall{}}\HOLBoundVar{P}.
     \HOLBoundVar{P} \HOLConst{PLAN_PB} (\HOLConst{exec} (\HOLConst{SLc} \HOLConst{crossLD})) \HOLSymConst{\HOLTokenConj{}}
     \HOLBoundVar{P} \HOLConst{PLAN_PB} (\HOLConst{exec} (\HOLConst{SLc} \HOLConst{incomplete})) \HOLSymConst{\HOLTokenConj{}}
     \HOLBoundVar{P} \HOLConst{MOVE_TO_ORP} (\HOLConst{exec} (\HOLConst{SLc} \HOLConst{conductORP})) \HOLSymConst{\HOLTokenConj{}}
     \HOLBoundVar{P} \HOLConst{MOVE_TO_ORP} (\HOLConst{exec} (\HOLConst{SLc} \HOLConst{incomplete})) \HOLSymConst{\HOLTokenConj{}}
     \HOLBoundVar{P} \HOLConst{CONDUCT_ORP} (\HOLConst{exec} (\HOLConst{SLc} \HOLConst{moveToPB})) \HOLSymConst{\HOLTokenConj{}}
     \HOLBoundVar{P} \HOLConst{CONDUCT_ORP} (\HOLConst{exec} (\HOLConst{SLc} \HOLConst{incomplete})) \HOLSymConst{\HOLTokenConj{}}
     \HOLBoundVar{P} \HOLConst{MOVE_TO_PB} (\HOLConst{exec} (\HOLConst{SLc} \HOLConst{conductPB})) \HOLSymConst{\HOLTokenConj{}}
     \HOLBoundVar{P} \HOLConst{MOVE_TO_PB} (\HOLConst{exec} (\HOLConst{SLc} \HOLConst{incomplete})) \HOLSymConst{\HOLTokenConj{}}
     \HOLBoundVar{P} \HOLConst{CONDUCT_PB} (\HOLConst{exec} (\HOLConst{SLc} \HOLConst{completePB})) \HOLSymConst{\HOLTokenConj{}}
     \HOLBoundVar{P} \HOLConst{CONDUCT_PB} (\HOLConst{exec} (\HOLConst{SLc} \HOLConst{incomplete})) \HOLSymConst{\HOLTokenConj{}}
     \HOLBoundVar{P} \HOLConst{PLAN_PB} (\HOLConst{trap} (\HOLConst{SLc} \HOLConst{crossLD})) \HOLSymConst{\HOLTokenConj{}}
     \HOLBoundVar{P} \HOLConst{PLAN_PB} (\HOLConst{trap} (\HOLConst{SLc} \HOLConst{incomplete})) \HOLSymConst{\HOLTokenConj{}}
     \HOLBoundVar{P} \HOLConst{MOVE_TO_ORP} (\HOLConst{trap} (\HOLConst{SLc} \HOLConst{conductORP})) \HOLSymConst{\HOLTokenConj{}}
     \HOLBoundVar{P} \HOLConst{MOVE_TO_ORP} (\HOLConst{trap} (\HOLConst{SLc} \HOLConst{incomplete})) \HOLSymConst{\HOLTokenConj{}}
     \HOLBoundVar{P} \HOLConst{CONDUCT_ORP} (\HOLConst{trap} (\HOLConst{SLc} \HOLConst{moveToPB})) \HOLSymConst{\HOLTokenConj{}}
     \HOLBoundVar{P} \HOLConst{CONDUCT_ORP} (\HOLConst{trap} (\HOLConst{SLc} \HOLConst{incomplete})) \HOLSymConst{\HOLTokenConj{}}
     \HOLBoundVar{P} \HOLConst{MOVE_TO_PB} (\HOLConst{trap} (\HOLConst{SLc} \HOLConst{conductPB})) \HOLSymConst{\HOLTokenConj{}}
     \HOLBoundVar{P} \HOLConst{MOVE_TO_PB} (\HOLConst{trap} (\HOLConst{SLc} \HOLConst{incomplete})) \HOLSymConst{\HOLTokenConj{}}
     \HOLBoundVar{P} \HOLConst{CONDUCT_PB} (\HOLConst{trap} (\HOLConst{SLc} \HOLConst{completePB})) \HOLSymConst{\HOLTokenConj{}}
     \HOLBoundVar{P} \HOLConst{CONDUCT_PB} (\HOLConst{trap} (\HOLConst{SLc} \HOLConst{incomplete})) \HOLSymConst{\HOLTokenConj{}}
     \HOLBoundVar{P} \HOLConst{PLAN_PB} (\HOLConst{discard} (\HOLConst{SLc} \HOLConst{crossLD})) \HOLSymConst{\HOLTokenConj{}}
     \HOLBoundVar{P} \HOLConst{MOVE_TO_ORP} (\HOLConst{discard} (\HOLConst{SLc} \HOLConst{conductORP})) \HOLSymConst{\HOLTokenConj{}}
     \HOLBoundVar{P} \HOLConst{CONDUCT_ORP} (\HOLConst{discard} (\HOLConst{SLc} \HOLConst{moveToPB})) \HOLSymConst{\HOLTokenConj{}}
     \HOLBoundVar{P} \HOLConst{MOVE_TO_PB} (\HOLConst{discard} (\HOLConst{SLc} \HOLConst{conductPB})) \HOLSymConst{\HOLTokenConj{}}
     \HOLBoundVar{P} \HOLConst{CONDUCT_PB} (\HOLConst{discard} (\HOLConst{SLc} \HOLConst{completePB})) \HOLSymConst{\HOLTokenConj{}}
     (\HOLSymConst{\HOLTokenForall{}}\HOLBoundVar{v\sb{\mathrm{8}}} \HOLBoundVar{v\sb{\mathrm{6}}}. \HOLBoundVar{P} \HOLBoundVar{v\sb{\mathrm{8}}} (\HOLConst{discard} (\HOLConst{ESCc} \HOLBoundVar{v\sb{\mathrm{6}}}))) \HOLSymConst{\HOLTokenConj{}}
     \HOLBoundVar{P} \HOLConst{MOVE_TO_ORP} (\HOLConst{discard} (\HOLConst{SLc} \HOLConst{crossLD})) \HOLSymConst{\HOLTokenConj{}}
     \HOLBoundVar{P} \HOLConst{CONDUCT_ORP} (\HOLConst{discard} (\HOLConst{SLc} \HOLConst{crossLD})) \HOLSymConst{\HOLTokenConj{}}
     \HOLBoundVar{P} \HOLConst{MOVE_TO_PB} (\HOLConst{discard} (\HOLConst{SLc} \HOLConst{crossLD})) \HOLSymConst{\HOLTokenConj{}}
     \HOLBoundVar{P} \HOLConst{CONDUCT_PB} (\HOLConst{discard} (\HOLConst{SLc} \HOLConst{crossLD})) \HOLSymConst{\HOLTokenConj{}}
     \HOLBoundVar{P} \HOLConst{COMPLETE_PB} (\HOLConst{discard} (\HOLConst{SLc} \HOLConst{crossLD})) \HOLSymConst{\HOLTokenConj{}}
     \HOLBoundVar{P} \HOLConst{PLAN_PB} (\HOLConst{discard} (\HOLConst{SLc} \HOLConst{conductORP})) \HOLSymConst{\HOLTokenConj{}}
     \HOLBoundVar{P} \HOLConst{CONDUCT_ORP} (\HOLConst{discard} (\HOLConst{SLc} \HOLConst{conductORP})) \HOLSymConst{\HOLTokenConj{}}
     \HOLBoundVar{P} \HOLConst{MOVE_TO_PB} (\HOLConst{discard} (\HOLConst{SLc} \HOLConst{conductORP})) \HOLSymConst{\HOLTokenConj{}}
     \HOLBoundVar{P} \HOLConst{CONDUCT_PB} (\HOLConst{discard} (\HOLConst{SLc} \HOLConst{conductORP})) \HOLSymConst{\HOLTokenConj{}}
     \HOLBoundVar{P} \HOLConst{COMPLETE_PB} (\HOLConst{discard} (\HOLConst{SLc} \HOLConst{conductORP})) \HOLSymConst{\HOLTokenConj{}}
     \HOLBoundVar{P} \HOLConst{PLAN_PB} (\HOLConst{discard} (\HOLConst{SLc} \HOLConst{moveToPB})) \HOLSymConst{\HOLTokenConj{}}
     \HOLBoundVar{P} \HOLConst{MOVE_TO_ORP} (\HOLConst{discard} (\HOLConst{SLc} \HOLConst{moveToPB})) \HOLSymConst{\HOLTokenConj{}}
     \HOLBoundVar{P} \HOLConst{MOVE_TO_PB} (\HOLConst{discard} (\HOLConst{SLc} \HOLConst{moveToPB})) \HOLSymConst{\HOLTokenConj{}}
     \HOLBoundVar{P} \HOLConst{CONDUCT_PB} (\HOLConst{discard} (\HOLConst{SLc} \HOLConst{moveToPB})) \HOLSymConst{\HOLTokenConj{}}
     \HOLBoundVar{P} \HOLConst{COMPLETE_PB} (\HOLConst{discard} (\HOLConst{SLc} \HOLConst{moveToPB})) \HOLSymConst{\HOLTokenConj{}}
     \HOLBoundVar{P} \HOLConst{PLAN_PB} (\HOLConst{discard} (\HOLConst{SLc} \HOLConst{conductPB})) \HOLSymConst{\HOLTokenConj{}}
     \HOLBoundVar{P} \HOLConst{MOVE_TO_ORP} (\HOLConst{discard} (\HOLConst{SLc} \HOLConst{conductPB})) \HOLSymConst{\HOLTokenConj{}}
     \HOLBoundVar{P} \HOLConst{CONDUCT_ORP} (\HOLConst{discard} (\HOLConst{SLc} \HOLConst{conductPB})) \HOLSymConst{\HOLTokenConj{}}
     \HOLBoundVar{P} \HOLConst{CONDUCT_PB} (\HOLConst{discard} (\HOLConst{SLc} \HOLConst{conductPB})) \HOLSymConst{\HOLTokenConj{}}
     \HOLBoundVar{P} \HOLConst{COMPLETE_PB} (\HOLConst{discard} (\HOLConst{SLc} \HOLConst{conductPB})) \HOLSymConst{\HOLTokenConj{}}
     \HOLBoundVar{P} \HOLConst{PLAN_PB} (\HOLConst{discard} (\HOLConst{SLc} \HOLConst{completePB})) \HOLSymConst{\HOLTokenConj{}}
     \HOLBoundVar{P} \HOLConst{MOVE_TO_ORP} (\HOLConst{discard} (\HOLConst{SLc} \HOLConst{completePB})) \HOLSymConst{\HOLTokenConj{}}
     \HOLBoundVar{P} \HOLConst{CONDUCT_ORP} (\HOLConst{discard} (\HOLConst{SLc} \HOLConst{completePB})) \HOLSymConst{\HOLTokenConj{}}
     \HOLBoundVar{P} \HOLConst{MOVE_TO_PB} (\HOLConst{discard} (\HOLConst{SLc} \HOLConst{completePB})) \HOLSymConst{\HOLTokenConj{}}
     \HOLBoundVar{P} \HOLConst{COMPLETE_PB} (\HOLConst{discard} (\HOLConst{SLc} \HOLConst{completePB})) \HOLSymConst{\HOLTokenConj{}}
     (\HOLSymConst{\HOLTokenForall{}}\HOLBoundVar{v\sb{\mathrm{9}}}. \HOLBoundVar{P} \HOLBoundVar{v\sb{\mathrm{9}}} (\HOLConst{discard} (\HOLConst{SLc} \HOLConst{incomplete}))) \HOLSymConst{\HOLTokenConj{}}
     (\HOLSymConst{\HOLTokenForall{}}\HOLBoundVar{v\sb{\mathrm{13}}} \HOLBoundVar{v\sb{\mathrm{11}}}. \HOLBoundVar{P} \HOLBoundVar{v\sb{\mathrm{13}}} (\HOLConst{trap} (\HOLConst{ESCc} \HOLBoundVar{v\sb{\mathrm{11}}}))) \HOLSymConst{\HOLTokenConj{}}
     \HOLBoundVar{P} \HOLConst{MOVE_TO_ORP} (\HOLConst{trap} (\HOLConst{SLc} \HOLConst{crossLD})) \HOLSymConst{\HOLTokenConj{}}
     \HOLBoundVar{P} \HOLConst{CONDUCT_ORP} (\HOLConst{trap} (\HOLConst{SLc} \HOLConst{crossLD})) \HOLSymConst{\HOLTokenConj{}}
     \HOLBoundVar{P} \HOLConst{MOVE_TO_PB} (\HOLConst{trap} (\HOLConst{SLc} \HOLConst{crossLD})) \HOLSymConst{\HOLTokenConj{}}
     \HOLBoundVar{P} \HOLConst{CONDUCT_PB} (\HOLConst{trap} (\HOLConst{SLc} \HOLConst{crossLD})) \HOLSymConst{\HOLTokenConj{}}
     \HOLBoundVar{P} \HOLConst{COMPLETE_PB} (\HOLConst{trap} (\HOLConst{SLc} \HOLConst{crossLD})) \HOLSymConst{\HOLTokenConj{}}
     \HOLBoundVar{P} \HOLConst{PLAN_PB} (\HOLConst{trap} (\HOLConst{SLc} \HOLConst{conductORP})) \HOLSymConst{\HOLTokenConj{}}
     \HOLBoundVar{P} \HOLConst{CONDUCT_ORP} (\HOLConst{trap} (\HOLConst{SLc} \HOLConst{conductORP})) \HOLSymConst{\HOLTokenConj{}}
     \HOLBoundVar{P} \HOLConst{MOVE_TO_PB} (\HOLConst{trap} (\HOLConst{SLc} \HOLConst{conductORP})) \HOLSymConst{\HOLTokenConj{}}
     \HOLBoundVar{P} \HOLConst{CONDUCT_PB} (\HOLConst{trap} (\HOLConst{SLc} \HOLConst{conductORP})) \HOLSymConst{\HOLTokenConj{}}
     \HOLBoundVar{P} \HOLConst{COMPLETE_PB} (\HOLConst{trap} (\HOLConst{SLc} \HOLConst{conductORP})) \HOLSymConst{\HOLTokenConj{}}
     \HOLBoundVar{P} \HOLConst{PLAN_PB} (\HOLConst{trap} (\HOLConst{SLc} \HOLConst{moveToPB})) \HOLSymConst{\HOLTokenConj{}}
     \HOLBoundVar{P} \HOLConst{MOVE_TO_ORP} (\HOLConst{trap} (\HOLConst{SLc} \HOLConst{moveToPB})) \HOLSymConst{\HOLTokenConj{}}
     \HOLBoundVar{P} \HOLConst{MOVE_TO_PB} (\HOLConst{trap} (\HOLConst{SLc} \HOLConst{moveToPB})) \HOLSymConst{\HOLTokenConj{}}
     \HOLBoundVar{P} \HOLConst{CONDUCT_PB} (\HOLConst{trap} (\HOLConst{SLc} \HOLConst{moveToPB})) \HOLSymConst{\HOLTokenConj{}}
     \HOLBoundVar{P} \HOLConst{COMPLETE_PB} (\HOLConst{trap} (\HOLConst{SLc} \HOLConst{moveToPB})) \HOLSymConst{\HOLTokenConj{}}
     \HOLBoundVar{P} \HOLConst{PLAN_PB} (\HOLConst{trap} (\HOLConst{SLc} \HOLConst{conductPB})) \HOLSymConst{\HOLTokenConj{}}
     \HOLBoundVar{P} \HOLConst{MOVE_TO_ORP} (\HOLConst{trap} (\HOLConst{SLc} \HOLConst{conductPB})) \HOLSymConst{\HOLTokenConj{}}
     \HOLBoundVar{P} \HOLConst{CONDUCT_ORP} (\HOLConst{trap} (\HOLConst{SLc} \HOLConst{conductPB})) \HOLSymConst{\HOLTokenConj{}}
     \HOLBoundVar{P} \HOLConst{CONDUCT_PB} (\HOLConst{trap} (\HOLConst{SLc} \HOLConst{conductPB})) \HOLSymConst{\HOLTokenConj{}}
     \HOLBoundVar{P} \HOLConst{COMPLETE_PB} (\HOLConst{trap} (\HOLConst{SLc} \HOLConst{conductPB})) \HOLSymConst{\HOLTokenConj{}}
     \HOLBoundVar{P} \HOLConst{PLAN_PB} (\HOLConst{trap} (\HOLConst{SLc} \HOLConst{completePB})) \HOLSymConst{\HOLTokenConj{}}
     \HOLBoundVar{P} \HOLConst{MOVE_TO_ORP} (\HOLConst{trap} (\HOLConst{SLc} \HOLConst{completePB})) \HOLSymConst{\HOLTokenConj{}}
     \HOLBoundVar{P} \HOLConst{CONDUCT_ORP} (\HOLConst{trap} (\HOLConst{SLc} \HOLConst{completePB})) \HOLSymConst{\HOLTokenConj{}}
     \HOLBoundVar{P} \HOLConst{MOVE_TO_PB} (\HOLConst{trap} (\HOLConst{SLc} \HOLConst{completePB})) \HOLSymConst{\HOLTokenConj{}}
     \HOLBoundVar{P} \HOLConst{COMPLETE_PB} (\HOLConst{trap} (\HOLConst{SLc} \HOLConst{completePB})) \HOLSymConst{\HOLTokenConj{}}
     \HOLBoundVar{P} \HOLConst{COMPLETE_PB} (\HOLConst{trap} (\HOLConst{SLc} \HOLConst{incomplete})) \HOLSymConst{\HOLTokenConj{}}
     (\HOLSymConst{\HOLTokenForall{}}\HOLBoundVar{v\sb{\mathrm{17}}} \HOLBoundVar{v\sb{\mathrm{15}}}. \HOLBoundVar{P} \HOLBoundVar{v\sb{\mathrm{17}}} (\HOLConst{exec} (\HOLConst{ESCc} \HOLBoundVar{v\sb{\mathrm{15}}}))) \HOLSymConst{\HOLTokenConj{}}
     \HOLBoundVar{P} \HOLConst{MOVE_TO_ORP} (\HOLConst{exec} (\HOLConst{SLc} \HOLConst{crossLD})) \HOLSymConst{\HOLTokenConj{}}
     \HOLBoundVar{P} \HOLConst{CONDUCT_ORP} (\HOLConst{exec} (\HOLConst{SLc} \HOLConst{crossLD})) \HOLSymConst{\HOLTokenConj{}}
     \HOLBoundVar{P} \HOLConst{MOVE_TO_PB} (\HOLConst{exec} (\HOLConst{SLc} \HOLConst{crossLD})) \HOLSymConst{\HOLTokenConj{}}
     \HOLBoundVar{P} \HOLConst{CONDUCT_PB} (\HOLConst{exec} (\HOLConst{SLc} \HOLConst{crossLD})) \HOLSymConst{\HOLTokenConj{}}
     \HOLBoundVar{P} \HOLConst{COMPLETE_PB} (\HOLConst{exec} (\HOLConst{SLc} \HOLConst{crossLD})) \HOLSymConst{\HOLTokenConj{}}
     \HOLBoundVar{P} \HOLConst{PLAN_PB} (\HOLConst{exec} (\HOLConst{SLc} \HOLConst{conductORP})) \HOLSymConst{\HOLTokenConj{}}
     \HOLBoundVar{P} \HOLConst{CONDUCT_ORP} (\HOLConst{exec} (\HOLConst{SLc} \HOLConst{conductORP})) \HOLSymConst{\HOLTokenConj{}}
     \HOLBoundVar{P} \HOLConst{MOVE_TO_PB} (\HOLConst{exec} (\HOLConst{SLc} \HOLConst{conductORP})) \HOLSymConst{\HOLTokenConj{}}
     \HOLBoundVar{P} \HOLConst{CONDUCT_PB} (\HOLConst{exec} (\HOLConst{SLc} \HOLConst{conductORP})) \HOLSymConst{\HOLTokenConj{}}
     \HOLBoundVar{P} \HOLConst{COMPLETE_PB} (\HOLConst{exec} (\HOLConst{SLc} \HOLConst{conductORP})) \HOLSymConst{\HOLTokenConj{}}
     \HOLBoundVar{P} \HOLConst{PLAN_PB} (\HOLConst{exec} (\HOLConst{SLc} \HOLConst{moveToPB})) \HOLSymConst{\HOLTokenConj{}}
     \HOLBoundVar{P} \HOLConst{MOVE_TO_ORP} (\HOLConst{exec} (\HOLConst{SLc} \HOLConst{moveToPB})) \HOLSymConst{\HOLTokenConj{}}
     \HOLBoundVar{P} \HOLConst{MOVE_TO_PB} (\HOLConst{exec} (\HOLConst{SLc} \HOLConst{moveToPB})) \HOLSymConst{\HOLTokenConj{}}
     \HOLBoundVar{P} \HOLConst{CONDUCT_PB} (\HOLConst{exec} (\HOLConst{SLc} \HOLConst{moveToPB})) \HOLSymConst{\HOLTokenConj{}}
     \HOLBoundVar{P} \HOLConst{COMPLETE_PB} (\HOLConst{exec} (\HOLConst{SLc} \HOLConst{moveToPB})) \HOLSymConst{\HOLTokenConj{}}
     \HOLBoundVar{P} \HOLConst{PLAN_PB} (\HOLConst{exec} (\HOLConst{SLc} \HOLConst{conductPB})) \HOLSymConst{\HOLTokenConj{}}
     \HOLBoundVar{P} \HOLConst{MOVE_TO_ORP} (\HOLConst{exec} (\HOLConst{SLc} \HOLConst{conductPB})) \HOLSymConst{\HOLTokenConj{}}
     \HOLBoundVar{P} \HOLConst{CONDUCT_ORP} (\HOLConst{exec} (\HOLConst{SLc} \HOLConst{conductPB})) \HOLSymConst{\HOLTokenConj{}}
     \HOLBoundVar{P} \HOLConst{CONDUCT_PB} (\HOLConst{exec} (\HOLConst{SLc} \HOLConst{conductPB})) \HOLSymConst{\HOLTokenConj{}}
     \HOLBoundVar{P} \HOLConst{COMPLETE_PB} (\HOLConst{exec} (\HOLConst{SLc} \HOLConst{conductPB})) \HOLSymConst{\HOLTokenConj{}}
     \HOLBoundVar{P} \HOLConst{PLAN_PB} (\HOLConst{exec} (\HOLConst{SLc} \HOLConst{completePB})) \HOLSymConst{\HOLTokenConj{}}
     \HOLBoundVar{P} \HOLConst{MOVE_TO_ORP} (\HOLConst{exec} (\HOLConst{SLc} \HOLConst{completePB})) \HOLSymConst{\HOLTokenConj{}}
     \HOLBoundVar{P} \HOLConst{CONDUCT_ORP} (\HOLConst{exec} (\HOLConst{SLc} \HOLConst{completePB})) \HOLSymConst{\HOLTokenConj{}}
     \HOLBoundVar{P} \HOLConst{MOVE_TO_PB} (\HOLConst{exec} (\HOLConst{SLc} \HOLConst{completePB})) \HOLSymConst{\HOLTokenConj{}}
     \HOLBoundVar{P} \HOLConst{COMPLETE_PB} (\HOLConst{exec} (\HOLConst{SLc} \HOLConst{completePB})) \HOLSymConst{\HOLTokenConj{}}
     \HOLBoundVar{P} \HOLConst{COMPLETE_PB} (\HOLConst{exec} (\HOLConst{SLc} \HOLConst{incomplete})) \HOLSymConst{\HOLTokenImp{}}
     \HOLSymConst{\HOLTokenForall{}}\HOLBoundVar{v} \HOLBoundVar{v\sb{\mathrm{1}}}. \HOLBoundVar{P} \HOLBoundVar{v} \HOLBoundVar{v\sb{\mathrm{1}}}
\end{SaveVerbatim}
\newcommand{\HOLssmPBTheoremsPBNSXXind}{\UseVerbatim{HOLssmPBTheoremsPBNSXXind}}
\begin{SaveVerbatim}{HOLssmPBTheoremsPBOutXXdef}
\HOLTokenTurnstile{} (\HOLConst{PBOut} \HOLConst{PLAN_PB} (\HOLConst{exec} (\HOLConst{SLc} \HOLConst{crossLD})) \HOLSymConst{=} \HOLConst{MoveToORP}) \HOLSymConst{\HOLTokenConj{}}
   (\HOLConst{PBOut} \HOLConst{PLAN_PB} (\HOLConst{exec} (\HOLConst{SLc} \HOLConst{incomplete})) \HOLSymConst{=} \HOLConst{PlanPB}) \HOLSymConst{\HOLTokenConj{}}
   (\HOLConst{PBOut} \HOLConst{MOVE_TO_ORP} (\HOLConst{exec} (\HOLConst{SLc} \HOLConst{conductORP})) \HOLSymConst{=} \HOLConst{ConductORP}) \HOLSymConst{\HOLTokenConj{}}
   (\HOLConst{PBOut} \HOLConst{MOVE_TO_ORP} (\HOLConst{exec} (\HOLConst{SLc} \HOLConst{incomplete})) \HOLSymConst{=} \HOLConst{MoveToORP}) \HOLSymConst{\HOLTokenConj{}}
   (\HOLConst{PBOut} \HOLConst{CONDUCT_ORP} (\HOLConst{exec} (\HOLConst{SLc} \HOLConst{moveToPB})) \HOLSymConst{=} \HOLConst{MoveToPB}) \HOLSymConst{\HOLTokenConj{}}
   (\HOLConst{PBOut} \HOLConst{CONDUCT_ORP} (\HOLConst{exec} (\HOLConst{SLc} \HOLConst{incomplete})) \HOLSymConst{=} \HOLConst{ConductORP}) \HOLSymConst{\HOLTokenConj{}}
   (\HOLConst{PBOut} \HOLConst{MOVE_TO_PB} (\HOLConst{exec} (\HOLConst{SLc} \HOLConst{conductPB})) \HOLSymConst{=} \HOLConst{ConductPB}) \HOLSymConst{\HOLTokenConj{}}
   (\HOLConst{PBOut} \HOLConst{MOVE_TO_PB} (\HOLConst{exec} (\HOLConst{SLc} \HOLConst{incomplete})) \HOLSymConst{=} \HOLConst{MoveToPB}) \HOLSymConst{\HOLTokenConj{}}
   (\HOLConst{PBOut} \HOLConst{CONDUCT_PB} (\HOLConst{exec} (\HOLConst{SLc} \HOLConst{completePB})) \HOLSymConst{=} \HOLConst{CompletePB}) \HOLSymConst{\HOLTokenConj{}}
   (\HOLConst{PBOut} \HOLConst{CONDUCT_PB} (\HOLConst{exec} (\HOLConst{SLc} \HOLConst{incomplete})) \HOLSymConst{=} \HOLConst{ConductPB}) \HOLSymConst{\HOLTokenConj{}}
   (\HOLConst{PBOut} \HOLConst{PLAN_PB} (\HOLConst{trap} (\HOLConst{SLc} \HOLConst{crossLD})) \HOLSymConst{=} \HOLConst{PlanPB}) \HOLSymConst{\HOLTokenConj{}}
   (\HOLConst{PBOut} \HOLConst{PLAN_PB} (\HOLConst{trap} (\HOLConst{SLc} \HOLConst{incomplete})) \HOLSymConst{=} \HOLConst{PlanPB}) \HOLSymConst{\HOLTokenConj{}}
   (\HOLConst{PBOut} \HOLConst{MOVE_TO_ORP} (\HOLConst{trap} (\HOLConst{SLc} \HOLConst{conductORP})) \HOLSymConst{=} \HOLConst{MoveToORP}) \HOLSymConst{\HOLTokenConj{}}
   (\HOLConst{PBOut} \HOLConst{MOVE_TO_ORP} (\HOLConst{trap} (\HOLConst{SLc} \HOLConst{incomplete})) \HOLSymConst{=} \HOLConst{MoveToORP}) \HOLSymConst{\HOLTokenConj{}}
   (\HOLConst{PBOut} \HOLConst{CONDUCT_ORP} (\HOLConst{trap} (\HOLConst{SLc} \HOLConst{moveToPB})) \HOLSymConst{=} \HOLConst{ConductORP}) \HOLSymConst{\HOLTokenConj{}}
   (\HOLConst{PBOut} \HOLConst{CONDUCT_ORP} (\HOLConst{trap} (\HOLConst{SLc} \HOLConst{incomplete})) \HOLSymConst{=} \HOLConst{ConductORP}) \HOLSymConst{\HOLTokenConj{}}
   (\HOLConst{PBOut} \HOLConst{MOVE_TO_PB} (\HOLConst{trap} (\HOLConst{SLc} \HOLConst{conductPB})) \HOLSymConst{=} \HOLConst{MoveToPB}) \HOLSymConst{\HOLTokenConj{}}
   (\HOLConst{PBOut} \HOLConst{MOVE_TO_PB} (\HOLConst{trap} (\HOLConst{SLc} \HOLConst{incomplete})) \HOLSymConst{=} \HOLConst{MoveToPB}) \HOLSymConst{\HOLTokenConj{}}
   (\HOLConst{PBOut} \HOLConst{CONDUCT_PB} (\HOLConst{trap} (\HOLConst{SLc} \HOLConst{completePB})) \HOLSymConst{=} \HOLConst{ConductPB}) \HOLSymConst{\HOLTokenConj{}}
   (\HOLConst{PBOut} \HOLConst{CONDUCT_PB} (\HOLConst{trap} (\HOLConst{SLc} \HOLConst{incomplete})) \HOLSymConst{=} \HOLConst{ConductPB}) \HOLSymConst{\HOLTokenConj{}}
   (\HOLConst{PBOut} \HOLConst{PLAN_PB} (\HOLConst{discard} (\HOLConst{SLc} \HOLConst{crossLD})) \HOLSymConst{=} \HOLConst{unAuthenticated}) \HOLSymConst{\HOLTokenConj{}}
   (\HOLConst{PBOut} \HOLConst{MOVE_TO_ORP} (\HOLConst{discard} (\HOLConst{SLc} \HOLConst{conductORP})) \HOLSymConst{=}
    \HOLConst{unAuthenticated}) \HOLSymConst{\HOLTokenConj{}}
   (\HOLConst{PBOut} \HOLConst{CONDUCT_ORP} (\HOLConst{discard} (\HOLConst{SLc} \HOLConst{moveToPB})) \HOLSymConst{=}
    \HOLConst{unAuthenticated}) \HOLSymConst{\HOLTokenConj{}}
   (\HOLConst{PBOut} \HOLConst{MOVE_TO_PB} (\HOLConst{discard} (\HOLConst{SLc} \HOLConst{conductPB})) \HOLSymConst{=}
    \HOLConst{unAuthenticated}) \HOLSymConst{\HOLTokenConj{}}
   (\HOLConst{PBOut} \HOLConst{CONDUCT_PB} (\HOLConst{discard} (\HOLConst{SLc} \HOLConst{completePB})) \HOLSymConst{=}
    \HOLConst{unAuthenticated})
\end{SaveVerbatim}
\newcommand{\HOLssmPBTheoremsPBOutXXdef}{\UseVerbatim{HOLssmPBTheoremsPBOutXXdef}}
\begin{SaveVerbatim}{HOLssmPBTheoremsPBOutXXind}
\HOLTokenTurnstile{} \HOLSymConst{\HOLTokenForall{}}\HOLBoundVar{P}.
     \HOLBoundVar{P} \HOLConst{PLAN_PB} (\HOLConst{exec} (\HOLConst{SLc} \HOLConst{crossLD})) \HOLSymConst{\HOLTokenConj{}}
     \HOLBoundVar{P} \HOLConst{PLAN_PB} (\HOLConst{exec} (\HOLConst{SLc} \HOLConst{incomplete})) \HOLSymConst{\HOLTokenConj{}}
     \HOLBoundVar{P} \HOLConst{MOVE_TO_ORP} (\HOLConst{exec} (\HOLConst{SLc} \HOLConst{conductORP})) \HOLSymConst{\HOLTokenConj{}}
     \HOLBoundVar{P} \HOLConst{MOVE_TO_ORP} (\HOLConst{exec} (\HOLConst{SLc} \HOLConst{incomplete})) \HOLSymConst{\HOLTokenConj{}}
     \HOLBoundVar{P} \HOLConst{CONDUCT_ORP} (\HOLConst{exec} (\HOLConst{SLc} \HOLConst{moveToPB})) \HOLSymConst{\HOLTokenConj{}}
     \HOLBoundVar{P} \HOLConst{CONDUCT_ORP} (\HOLConst{exec} (\HOLConst{SLc} \HOLConst{incomplete})) \HOLSymConst{\HOLTokenConj{}}
     \HOLBoundVar{P} \HOLConst{MOVE_TO_PB} (\HOLConst{exec} (\HOLConst{SLc} \HOLConst{conductPB})) \HOLSymConst{\HOLTokenConj{}}
     \HOLBoundVar{P} \HOLConst{MOVE_TO_PB} (\HOLConst{exec} (\HOLConst{SLc} \HOLConst{incomplete})) \HOLSymConst{\HOLTokenConj{}}
     \HOLBoundVar{P} \HOLConst{CONDUCT_PB} (\HOLConst{exec} (\HOLConst{SLc} \HOLConst{completePB})) \HOLSymConst{\HOLTokenConj{}}
     \HOLBoundVar{P} \HOLConst{CONDUCT_PB} (\HOLConst{exec} (\HOLConst{SLc} \HOLConst{incomplete})) \HOLSymConst{\HOLTokenConj{}}
     \HOLBoundVar{P} \HOLConst{PLAN_PB} (\HOLConst{trap} (\HOLConst{SLc} \HOLConst{crossLD})) \HOLSymConst{\HOLTokenConj{}}
     \HOLBoundVar{P} \HOLConst{PLAN_PB} (\HOLConst{trap} (\HOLConst{SLc} \HOLConst{incomplete})) \HOLSymConst{\HOLTokenConj{}}
     \HOLBoundVar{P} \HOLConst{MOVE_TO_ORP} (\HOLConst{trap} (\HOLConst{SLc} \HOLConst{conductORP})) \HOLSymConst{\HOLTokenConj{}}
     \HOLBoundVar{P} \HOLConst{MOVE_TO_ORP} (\HOLConst{trap} (\HOLConst{SLc} \HOLConst{incomplete})) \HOLSymConst{\HOLTokenConj{}}
     \HOLBoundVar{P} \HOLConst{CONDUCT_ORP} (\HOLConst{trap} (\HOLConst{SLc} \HOLConst{moveToPB})) \HOLSymConst{\HOLTokenConj{}}
     \HOLBoundVar{P} \HOLConst{CONDUCT_ORP} (\HOLConst{trap} (\HOLConst{SLc} \HOLConst{incomplete})) \HOLSymConst{\HOLTokenConj{}}
     \HOLBoundVar{P} \HOLConst{MOVE_TO_PB} (\HOLConst{trap} (\HOLConst{SLc} \HOLConst{conductPB})) \HOLSymConst{\HOLTokenConj{}}
     \HOLBoundVar{P} \HOLConst{MOVE_TO_PB} (\HOLConst{trap} (\HOLConst{SLc} \HOLConst{incomplete})) \HOLSymConst{\HOLTokenConj{}}
     \HOLBoundVar{P} \HOLConst{CONDUCT_PB} (\HOLConst{trap} (\HOLConst{SLc} \HOLConst{completePB})) \HOLSymConst{\HOLTokenConj{}}
     \HOLBoundVar{P} \HOLConst{CONDUCT_PB} (\HOLConst{trap} (\HOLConst{SLc} \HOLConst{incomplete})) \HOLSymConst{\HOLTokenConj{}}
     \HOLBoundVar{P} \HOLConst{PLAN_PB} (\HOLConst{discard} (\HOLConst{SLc} \HOLConst{crossLD})) \HOLSymConst{\HOLTokenConj{}}
     \HOLBoundVar{P} \HOLConst{MOVE_TO_ORP} (\HOLConst{discard} (\HOLConst{SLc} \HOLConst{conductORP})) \HOLSymConst{\HOLTokenConj{}}
     \HOLBoundVar{P} \HOLConst{CONDUCT_ORP} (\HOLConst{discard} (\HOLConst{SLc} \HOLConst{moveToPB})) \HOLSymConst{\HOLTokenConj{}}
     \HOLBoundVar{P} \HOLConst{MOVE_TO_PB} (\HOLConst{discard} (\HOLConst{SLc} \HOLConst{conductPB})) \HOLSymConst{\HOLTokenConj{}}
     \HOLBoundVar{P} \HOLConst{CONDUCT_PB} (\HOLConst{discard} (\HOLConst{SLc} \HOLConst{completePB})) \HOLSymConst{\HOLTokenConj{}}
     (\HOLSymConst{\HOLTokenForall{}}\HOLBoundVar{v\sb{\mathrm{8}}} \HOLBoundVar{v\sb{\mathrm{6}}}. \HOLBoundVar{P} \HOLBoundVar{v\sb{\mathrm{8}}} (\HOLConst{discard} (\HOLConst{ESCc} \HOLBoundVar{v\sb{\mathrm{6}}}))) \HOLSymConst{\HOLTokenConj{}}
     \HOLBoundVar{P} \HOLConst{MOVE_TO_ORP} (\HOLConst{discard} (\HOLConst{SLc} \HOLConst{crossLD})) \HOLSymConst{\HOLTokenConj{}}
     \HOLBoundVar{P} \HOLConst{CONDUCT_ORP} (\HOLConst{discard} (\HOLConst{SLc} \HOLConst{crossLD})) \HOLSymConst{\HOLTokenConj{}}
     \HOLBoundVar{P} \HOLConst{MOVE_TO_PB} (\HOLConst{discard} (\HOLConst{SLc} \HOLConst{crossLD})) \HOLSymConst{\HOLTokenConj{}}
     \HOLBoundVar{P} \HOLConst{CONDUCT_PB} (\HOLConst{discard} (\HOLConst{SLc} \HOLConst{crossLD})) \HOLSymConst{\HOLTokenConj{}}
     \HOLBoundVar{P} \HOLConst{COMPLETE_PB} (\HOLConst{discard} (\HOLConst{SLc} \HOLConst{crossLD})) \HOLSymConst{\HOLTokenConj{}}
     \HOLBoundVar{P} \HOLConst{PLAN_PB} (\HOLConst{discard} (\HOLConst{SLc} \HOLConst{conductORP})) \HOLSymConst{\HOLTokenConj{}}
     \HOLBoundVar{P} \HOLConst{CONDUCT_ORP} (\HOLConst{discard} (\HOLConst{SLc} \HOLConst{conductORP})) \HOLSymConst{\HOLTokenConj{}}
     \HOLBoundVar{P} \HOLConst{MOVE_TO_PB} (\HOLConst{discard} (\HOLConst{SLc} \HOLConst{conductORP})) \HOLSymConst{\HOLTokenConj{}}
     \HOLBoundVar{P} \HOLConst{CONDUCT_PB} (\HOLConst{discard} (\HOLConst{SLc} \HOLConst{conductORP})) \HOLSymConst{\HOLTokenConj{}}
     \HOLBoundVar{P} \HOLConst{COMPLETE_PB} (\HOLConst{discard} (\HOLConst{SLc} \HOLConst{conductORP})) \HOLSymConst{\HOLTokenConj{}}
     \HOLBoundVar{P} \HOLConst{PLAN_PB} (\HOLConst{discard} (\HOLConst{SLc} \HOLConst{moveToPB})) \HOLSymConst{\HOLTokenConj{}}
     \HOLBoundVar{P} \HOLConst{MOVE_TO_ORP} (\HOLConst{discard} (\HOLConst{SLc} \HOLConst{moveToPB})) \HOLSymConst{\HOLTokenConj{}}
     \HOLBoundVar{P} \HOLConst{MOVE_TO_PB} (\HOLConst{discard} (\HOLConst{SLc} \HOLConst{moveToPB})) \HOLSymConst{\HOLTokenConj{}}
     \HOLBoundVar{P} \HOLConst{CONDUCT_PB} (\HOLConst{discard} (\HOLConst{SLc} \HOLConst{moveToPB})) \HOLSymConst{\HOLTokenConj{}}
     \HOLBoundVar{P} \HOLConst{COMPLETE_PB} (\HOLConst{discard} (\HOLConst{SLc} \HOLConst{moveToPB})) \HOLSymConst{\HOLTokenConj{}}
     \HOLBoundVar{P} \HOLConst{PLAN_PB} (\HOLConst{discard} (\HOLConst{SLc} \HOLConst{conductPB})) \HOLSymConst{\HOLTokenConj{}}
     \HOLBoundVar{P} \HOLConst{MOVE_TO_ORP} (\HOLConst{discard} (\HOLConst{SLc} \HOLConst{conductPB})) \HOLSymConst{\HOLTokenConj{}}
     \HOLBoundVar{P} \HOLConst{CONDUCT_ORP} (\HOLConst{discard} (\HOLConst{SLc} \HOLConst{conductPB})) \HOLSymConst{\HOLTokenConj{}}
     \HOLBoundVar{P} \HOLConst{CONDUCT_PB} (\HOLConst{discard} (\HOLConst{SLc} \HOLConst{conductPB})) \HOLSymConst{\HOLTokenConj{}}
     \HOLBoundVar{P} \HOLConst{COMPLETE_PB} (\HOLConst{discard} (\HOLConst{SLc} \HOLConst{conductPB})) \HOLSymConst{\HOLTokenConj{}}
     \HOLBoundVar{P} \HOLConst{PLAN_PB} (\HOLConst{discard} (\HOLConst{SLc} \HOLConst{completePB})) \HOLSymConst{\HOLTokenConj{}}
     \HOLBoundVar{P} \HOLConst{MOVE_TO_ORP} (\HOLConst{discard} (\HOLConst{SLc} \HOLConst{completePB})) \HOLSymConst{\HOLTokenConj{}}
     \HOLBoundVar{P} \HOLConst{CONDUCT_ORP} (\HOLConst{discard} (\HOLConst{SLc} \HOLConst{completePB})) \HOLSymConst{\HOLTokenConj{}}
     \HOLBoundVar{P} \HOLConst{MOVE_TO_PB} (\HOLConst{discard} (\HOLConst{SLc} \HOLConst{completePB})) \HOLSymConst{\HOLTokenConj{}}
     \HOLBoundVar{P} \HOLConst{COMPLETE_PB} (\HOLConst{discard} (\HOLConst{SLc} \HOLConst{completePB})) \HOLSymConst{\HOLTokenConj{}}
     (\HOLSymConst{\HOLTokenForall{}}\HOLBoundVar{v\sb{\mathrm{9}}}. \HOLBoundVar{P} \HOLBoundVar{v\sb{\mathrm{9}}} (\HOLConst{discard} (\HOLConst{SLc} \HOLConst{incomplete}))) \HOLSymConst{\HOLTokenConj{}}
     (\HOLSymConst{\HOLTokenForall{}}\HOLBoundVar{v\sb{\mathrm{13}}} \HOLBoundVar{v\sb{\mathrm{11}}}. \HOLBoundVar{P} \HOLBoundVar{v\sb{\mathrm{13}}} (\HOLConst{trap} (\HOLConst{ESCc} \HOLBoundVar{v\sb{\mathrm{11}}}))) \HOLSymConst{\HOLTokenConj{}}
     \HOLBoundVar{P} \HOLConst{MOVE_TO_ORP} (\HOLConst{trap} (\HOLConst{SLc} \HOLConst{crossLD})) \HOLSymConst{\HOLTokenConj{}}
     \HOLBoundVar{P} \HOLConst{CONDUCT_ORP} (\HOLConst{trap} (\HOLConst{SLc} \HOLConst{crossLD})) \HOLSymConst{\HOLTokenConj{}}
     \HOLBoundVar{P} \HOLConst{MOVE_TO_PB} (\HOLConst{trap} (\HOLConst{SLc} \HOLConst{crossLD})) \HOLSymConst{\HOLTokenConj{}}
     \HOLBoundVar{P} \HOLConst{CONDUCT_PB} (\HOLConst{trap} (\HOLConst{SLc} \HOLConst{crossLD})) \HOLSymConst{\HOLTokenConj{}}
     \HOLBoundVar{P} \HOLConst{COMPLETE_PB} (\HOLConst{trap} (\HOLConst{SLc} \HOLConst{crossLD})) \HOLSymConst{\HOLTokenConj{}}
     \HOLBoundVar{P} \HOLConst{PLAN_PB} (\HOLConst{trap} (\HOLConst{SLc} \HOLConst{conductORP})) \HOLSymConst{\HOLTokenConj{}}
     \HOLBoundVar{P} \HOLConst{CONDUCT_ORP} (\HOLConst{trap} (\HOLConst{SLc} \HOLConst{conductORP})) \HOLSymConst{\HOLTokenConj{}}
     \HOLBoundVar{P} \HOLConst{MOVE_TO_PB} (\HOLConst{trap} (\HOLConst{SLc} \HOLConst{conductORP})) \HOLSymConst{\HOLTokenConj{}}
     \HOLBoundVar{P} \HOLConst{CONDUCT_PB} (\HOLConst{trap} (\HOLConst{SLc} \HOLConst{conductORP})) \HOLSymConst{\HOLTokenConj{}}
     \HOLBoundVar{P} \HOLConst{COMPLETE_PB} (\HOLConst{trap} (\HOLConst{SLc} \HOLConst{conductORP})) \HOLSymConst{\HOLTokenConj{}}
     \HOLBoundVar{P} \HOLConst{PLAN_PB} (\HOLConst{trap} (\HOLConst{SLc} \HOLConst{moveToPB})) \HOLSymConst{\HOLTokenConj{}}
     \HOLBoundVar{P} \HOLConst{MOVE_TO_ORP} (\HOLConst{trap} (\HOLConst{SLc} \HOLConst{moveToPB})) \HOLSymConst{\HOLTokenConj{}}
     \HOLBoundVar{P} \HOLConst{MOVE_TO_PB} (\HOLConst{trap} (\HOLConst{SLc} \HOLConst{moveToPB})) \HOLSymConst{\HOLTokenConj{}}
     \HOLBoundVar{P} \HOLConst{CONDUCT_PB} (\HOLConst{trap} (\HOLConst{SLc} \HOLConst{moveToPB})) \HOLSymConst{\HOLTokenConj{}}
     \HOLBoundVar{P} \HOLConst{COMPLETE_PB} (\HOLConst{trap} (\HOLConst{SLc} \HOLConst{moveToPB})) \HOLSymConst{\HOLTokenConj{}}
     \HOLBoundVar{P} \HOLConst{PLAN_PB} (\HOLConst{trap} (\HOLConst{SLc} \HOLConst{conductPB})) \HOLSymConst{\HOLTokenConj{}}
     \HOLBoundVar{P} \HOLConst{MOVE_TO_ORP} (\HOLConst{trap} (\HOLConst{SLc} \HOLConst{conductPB})) \HOLSymConst{\HOLTokenConj{}}
     \HOLBoundVar{P} \HOLConst{CONDUCT_ORP} (\HOLConst{trap} (\HOLConst{SLc} \HOLConst{conductPB})) \HOLSymConst{\HOLTokenConj{}}
     \HOLBoundVar{P} \HOLConst{CONDUCT_PB} (\HOLConst{trap} (\HOLConst{SLc} \HOLConst{conductPB})) \HOLSymConst{\HOLTokenConj{}}
     \HOLBoundVar{P} \HOLConst{COMPLETE_PB} (\HOLConst{trap} (\HOLConst{SLc} \HOLConst{conductPB})) \HOLSymConst{\HOLTokenConj{}}
     \HOLBoundVar{P} \HOLConst{PLAN_PB} (\HOLConst{trap} (\HOLConst{SLc} \HOLConst{completePB})) \HOLSymConst{\HOLTokenConj{}}
     \HOLBoundVar{P} \HOLConst{MOVE_TO_ORP} (\HOLConst{trap} (\HOLConst{SLc} \HOLConst{completePB})) \HOLSymConst{\HOLTokenConj{}}
     \HOLBoundVar{P} \HOLConst{CONDUCT_ORP} (\HOLConst{trap} (\HOLConst{SLc} \HOLConst{completePB})) \HOLSymConst{\HOLTokenConj{}}
     \HOLBoundVar{P} \HOLConst{MOVE_TO_PB} (\HOLConst{trap} (\HOLConst{SLc} \HOLConst{completePB})) \HOLSymConst{\HOLTokenConj{}}
     \HOLBoundVar{P} \HOLConst{COMPLETE_PB} (\HOLConst{trap} (\HOLConst{SLc} \HOLConst{completePB})) \HOLSymConst{\HOLTokenConj{}}
     \HOLBoundVar{P} \HOLConst{COMPLETE_PB} (\HOLConst{trap} (\HOLConst{SLc} \HOLConst{incomplete})) \HOLSymConst{\HOLTokenConj{}}
     (\HOLSymConst{\HOLTokenForall{}}\HOLBoundVar{v\sb{\mathrm{17}}} \HOLBoundVar{v\sb{\mathrm{15}}}. \HOLBoundVar{P} \HOLBoundVar{v\sb{\mathrm{17}}} (\HOLConst{exec} (\HOLConst{ESCc} \HOLBoundVar{v\sb{\mathrm{15}}}))) \HOLSymConst{\HOLTokenConj{}}
     \HOLBoundVar{P} \HOLConst{MOVE_TO_ORP} (\HOLConst{exec} (\HOLConst{SLc} \HOLConst{crossLD})) \HOLSymConst{\HOLTokenConj{}}
     \HOLBoundVar{P} \HOLConst{CONDUCT_ORP} (\HOLConst{exec} (\HOLConst{SLc} \HOLConst{crossLD})) \HOLSymConst{\HOLTokenConj{}}
     \HOLBoundVar{P} \HOLConst{MOVE_TO_PB} (\HOLConst{exec} (\HOLConst{SLc} \HOLConst{crossLD})) \HOLSymConst{\HOLTokenConj{}}
     \HOLBoundVar{P} \HOLConst{CONDUCT_PB} (\HOLConst{exec} (\HOLConst{SLc} \HOLConst{crossLD})) \HOLSymConst{\HOLTokenConj{}}
     \HOLBoundVar{P} \HOLConst{COMPLETE_PB} (\HOLConst{exec} (\HOLConst{SLc} \HOLConst{crossLD})) \HOLSymConst{\HOLTokenConj{}}
     \HOLBoundVar{P} \HOLConst{PLAN_PB} (\HOLConst{exec} (\HOLConst{SLc} \HOLConst{conductORP})) \HOLSymConst{\HOLTokenConj{}}
     \HOLBoundVar{P} \HOLConst{CONDUCT_ORP} (\HOLConst{exec} (\HOLConst{SLc} \HOLConst{conductORP})) \HOLSymConst{\HOLTokenConj{}}
     \HOLBoundVar{P} \HOLConst{MOVE_TO_PB} (\HOLConst{exec} (\HOLConst{SLc} \HOLConst{conductORP})) \HOLSymConst{\HOLTokenConj{}}
     \HOLBoundVar{P} \HOLConst{CONDUCT_PB} (\HOLConst{exec} (\HOLConst{SLc} \HOLConst{conductORP})) \HOLSymConst{\HOLTokenConj{}}
     \HOLBoundVar{P} \HOLConst{COMPLETE_PB} (\HOLConst{exec} (\HOLConst{SLc} \HOLConst{conductORP})) \HOLSymConst{\HOLTokenConj{}}
     \HOLBoundVar{P} \HOLConst{PLAN_PB} (\HOLConst{exec} (\HOLConst{SLc} \HOLConst{moveToPB})) \HOLSymConst{\HOLTokenConj{}}
     \HOLBoundVar{P} \HOLConst{MOVE_TO_ORP} (\HOLConst{exec} (\HOLConst{SLc} \HOLConst{moveToPB})) \HOLSymConst{\HOLTokenConj{}}
     \HOLBoundVar{P} \HOLConst{MOVE_TO_PB} (\HOLConst{exec} (\HOLConst{SLc} \HOLConst{moveToPB})) \HOLSymConst{\HOLTokenConj{}}
     \HOLBoundVar{P} \HOLConst{CONDUCT_PB} (\HOLConst{exec} (\HOLConst{SLc} \HOLConst{moveToPB})) \HOLSymConst{\HOLTokenConj{}}
     \HOLBoundVar{P} \HOLConst{COMPLETE_PB} (\HOLConst{exec} (\HOLConst{SLc} \HOLConst{moveToPB})) \HOLSymConst{\HOLTokenConj{}}
     \HOLBoundVar{P} \HOLConst{PLAN_PB} (\HOLConst{exec} (\HOLConst{SLc} \HOLConst{conductPB})) \HOLSymConst{\HOLTokenConj{}}
     \HOLBoundVar{P} \HOLConst{MOVE_TO_ORP} (\HOLConst{exec} (\HOLConst{SLc} \HOLConst{conductPB})) \HOLSymConst{\HOLTokenConj{}}
     \HOLBoundVar{P} \HOLConst{CONDUCT_ORP} (\HOLConst{exec} (\HOLConst{SLc} \HOLConst{conductPB})) \HOLSymConst{\HOLTokenConj{}}
     \HOLBoundVar{P} \HOLConst{CONDUCT_PB} (\HOLConst{exec} (\HOLConst{SLc} \HOLConst{conductPB})) \HOLSymConst{\HOLTokenConj{}}
     \HOLBoundVar{P} \HOLConst{COMPLETE_PB} (\HOLConst{exec} (\HOLConst{SLc} \HOLConst{conductPB})) \HOLSymConst{\HOLTokenConj{}}
     \HOLBoundVar{P} \HOLConst{PLAN_PB} (\HOLConst{exec} (\HOLConst{SLc} \HOLConst{completePB})) \HOLSymConst{\HOLTokenConj{}}
     \HOLBoundVar{P} \HOLConst{MOVE_TO_ORP} (\HOLConst{exec} (\HOLConst{SLc} \HOLConst{completePB})) \HOLSymConst{\HOLTokenConj{}}
     \HOLBoundVar{P} \HOLConst{CONDUCT_ORP} (\HOLConst{exec} (\HOLConst{SLc} \HOLConst{completePB})) \HOLSymConst{\HOLTokenConj{}}
     \HOLBoundVar{P} \HOLConst{MOVE_TO_PB} (\HOLConst{exec} (\HOLConst{SLc} \HOLConst{completePB})) \HOLSymConst{\HOLTokenConj{}}
     \HOLBoundVar{P} \HOLConst{COMPLETE_PB} (\HOLConst{exec} (\HOLConst{SLc} \HOLConst{completePB})) \HOLSymConst{\HOLTokenConj{}}
     \HOLBoundVar{P} \HOLConst{COMPLETE_PB} (\HOLConst{exec} (\HOLConst{SLc} \HOLConst{incomplete})) \HOLSymConst{\HOLTokenImp{}}
     \HOLSymConst{\HOLTokenForall{}}\HOLBoundVar{v} \HOLBoundVar{v\sb{\mathrm{1}}}. \HOLBoundVar{P} \HOLBoundVar{v} \HOLBoundVar{v\sb{\mathrm{1}}}
\end{SaveVerbatim}
\newcommand{\HOLssmPBTheoremsPBOutXXind}{\UseVerbatim{HOLssmPBTheoremsPBOutXXind}}
\begin{SaveVerbatim}{HOLssmPBTheoremsPlatoonLeaderXXexecXXslCommandXXjustifiedXXthm}
\HOLTokenTurnstile{} \HOLSymConst{\HOLTokenForall{}}\HOLBoundVar{NS} \HOLBoundVar{Out} \HOLBoundVar{M} \HOLBoundVar{Oi} \HOLBoundVar{Os}.
     \HOLConst{TR} (\HOLBoundVar{M}\HOLSymConst{,}\HOLBoundVar{Oi}\HOLSymConst{,}\HOLBoundVar{Os}) (\HOLConst{exec} (\HOLConst{SLc} \HOLFreeVar{slCommand}))
       (\HOLConst{CFG} \HOLConst{authenticationTest} \HOLConst{ssmPBStateInterp}
          (\HOLConst{secContext} \HOLFreeVar{slCommand})
          (\HOLConst{Name} \HOLConst{PlatoonLeader} \HOLConst{says} \HOLConst{prop} (\HOLConst{SOME} (\HOLConst{SLc} \HOLFreeVar{slCommand}))\HOLSymConst{::}
               \HOLFreeVar{ins}) \HOLFreeVar{s} \HOLFreeVar{outs})
       (\HOLConst{CFG} \HOLConst{authenticationTest} \HOLConst{ssmPBStateInterp}
          (\HOLConst{secContext} \HOLFreeVar{slCommand}) \HOLFreeVar{ins}
          (\HOLBoundVar{NS} \HOLFreeVar{s} (\HOLConst{exec} (\HOLConst{SLc} \HOLFreeVar{slCommand})))
          (\HOLBoundVar{Out} \HOLFreeVar{s} (\HOLConst{exec} (\HOLConst{SLc} \HOLFreeVar{slCommand}))\HOLSymConst{::}\HOLFreeVar{outs})) \HOLSymConst{\HOLTokenEquiv{}}
     \HOLConst{authenticationTest}
       (\HOLConst{Name} \HOLConst{PlatoonLeader} \HOLConst{says} \HOLConst{prop} (\HOLConst{SOME} (\HOLConst{SLc} \HOLFreeVar{slCommand}))) \HOLSymConst{\HOLTokenConj{}}
     \HOLConst{CFGInterpret} (\HOLBoundVar{M}\HOLSymConst{,}\HOLBoundVar{Oi}\HOLSymConst{,}\HOLBoundVar{Os})
       (\HOLConst{CFG} \HOLConst{authenticationTest} \HOLConst{ssmPBStateInterp}
          (\HOLConst{secContext} \HOLFreeVar{slCommand})
          (\HOLConst{Name} \HOLConst{PlatoonLeader} \HOLConst{says} \HOLConst{prop} (\HOLConst{SOME} (\HOLConst{SLc} \HOLFreeVar{slCommand}))\HOLSymConst{::}
               \HOLFreeVar{ins}) \HOLFreeVar{s} \HOLFreeVar{outs}) \HOLSymConst{\HOLTokenConj{}}
     (\HOLBoundVar{M}\HOLSymConst{,}\HOLBoundVar{Oi}\HOLSymConst{,}\HOLBoundVar{Os}) \HOLConst{sat} \HOLConst{prop} (\HOLConst{SOME} (\HOLConst{SLc} \HOLFreeVar{slCommand}))
\end{SaveVerbatim}
\newcommand{\HOLssmPBTheoremsPlatoonLeaderXXexecXXslCommandXXjustifiedXXthm}{\UseVerbatim{HOLssmPBTheoremsPlatoonLeaderXXexecXXslCommandXXjustifiedXXthm}}
\begin{SaveVerbatim}{HOLssmPBTheoremsPlatoonLeaderXXjustifiedXXslCommandXXexecXXthm}
\HOLTokenTurnstile{} \HOLSymConst{\HOLTokenForall{}}\HOLBoundVar{NS} \HOLBoundVar{Out} \HOLBoundVar{M} \HOLBoundVar{Oi} \HOLBoundVar{Os} \HOLBoundVar{cmd} \HOLBoundVar{slCommand} \HOLBoundVar{ins} \HOLBoundVar{s} \HOLBoundVar{outs}.
     \HOLConst{authenticationTest}
       (\HOLConst{Name} \HOLConst{PlatoonLeader} \HOLConst{says} \HOLConst{prop} (\HOLConst{SOME} (\HOLConst{SLc} \HOLBoundVar{slCommand}))) \HOLSymConst{\HOLTokenConj{}}
     \HOLConst{CFGInterpret} (\HOLBoundVar{M}\HOLSymConst{,}\HOLBoundVar{Oi}\HOLSymConst{,}\HOLBoundVar{Os})
       (\HOLConst{CFG} \HOLConst{authenticationTest} \HOLConst{ssmPBStateInterp}
          (\HOLConst{secContext} \HOLBoundVar{slCommand})
          (\HOLConst{Name} \HOLConst{PlatoonLeader} \HOLConst{says} \HOLConst{prop} (\HOLConst{SOME} (\HOLConst{SLc} \HOLBoundVar{slCommand}))\HOLSymConst{::}
               \HOLBoundVar{ins}) \HOLBoundVar{s} \HOLBoundVar{outs}) \HOLSymConst{\HOLTokenImp{}}
     \HOLConst{TR} (\HOLBoundVar{M}\HOLSymConst{,}\HOLBoundVar{Oi}\HOLSymConst{,}\HOLBoundVar{Os}) (\HOLConst{exec} (\HOLConst{SLc} \HOLBoundVar{slCommand}))
       (\HOLConst{CFG} \HOLConst{authenticationTest} \HOLConst{ssmPBStateInterp}
          (\HOLConst{secContext} \HOLBoundVar{slCommand})
          (\HOLConst{Name} \HOLConst{PlatoonLeader} \HOLConst{says} \HOLConst{prop} (\HOLConst{SOME} (\HOLConst{SLc} \HOLBoundVar{slCommand}))\HOLSymConst{::}
               \HOLBoundVar{ins}) \HOLBoundVar{s} \HOLBoundVar{outs})
       (\HOLConst{CFG} \HOLConst{authenticationTest} \HOLConst{ssmPBStateInterp}
          (\HOLConst{secContext} \HOLBoundVar{slCommand}) \HOLBoundVar{ins}
          (\HOLBoundVar{NS} \HOLBoundVar{s} (\HOLConst{exec} (\HOLConst{SLc} \HOLBoundVar{slCommand})))
          (\HOLBoundVar{Out} \HOLBoundVar{s} (\HOLConst{exec} (\HOLConst{SLc} \HOLBoundVar{slCommand}))\HOLSymConst{::}\HOLBoundVar{outs}))
\end{SaveVerbatim}
\newcommand{\HOLssmPBTheoremsPlatoonLeaderXXjustifiedXXslCommandXXexecXXthm}{\UseVerbatim{HOLssmPBTheoremsPlatoonLeaderXXjustifiedXXslCommandXXexecXXthm}}
\begin{SaveVerbatim}{HOLssmPBTheoremsPlatoonLeaderXXslCommandXXlemma}
\HOLTokenTurnstile{} \HOLConst{CFGInterpret} (\HOLFreeVar{M}\HOLSymConst{,}\HOLFreeVar{Oi}\HOLSymConst{,}\HOLFreeVar{Os})
     (\HOLConst{CFG} \HOLConst{authenticationTest} \HOLConst{ssmPBStateInterp}
        (\HOLConst{secContext} \HOLFreeVar{slCommand})
        (\HOLConst{Name} \HOLConst{PlatoonLeader} \HOLConst{says} \HOLConst{prop} (\HOLConst{SOME} (\HOLConst{SLc} \HOLFreeVar{slCommand}))\HOLSymConst{::}
             \HOLFreeVar{ins}) \HOLFreeVar{s} \HOLFreeVar{outs}) \HOLSymConst{\HOLTokenImp{}}
   (\HOLFreeVar{M}\HOLSymConst{,}\HOLFreeVar{Oi}\HOLSymConst{,}\HOLFreeVar{Os}) \HOLConst{sat} \HOLConst{prop} (\HOLConst{SOME} (\HOLConst{SLc} \HOLFreeVar{slCommand}))
\end{SaveVerbatim}
\newcommand{\HOLssmPBTheoremsPlatoonLeaderXXslCommandXXlemma}{\UseVerbatim{HOLssmPBTheoremsPlatoonLeaderXXslCommandXXlemma}}
\begin{SaveVerbatim}{HOLssmPBTheoremsPlatoonLeaderXXslCommandXXverifiedXXthm}
\HOLTokenTurnstile{} \HOLSymConst{\HOLTokenForall{}}\HOLBoundVar{NS} \HOLBoundVar{Out} \HOLBoundVar{M} \HOLBoundVar{Oi} \HOLBoundVar{Os}.
     \HOLConst{TR} (\HOLBoundVar{M}\HOLSymConst{,}\HOLBoundVar{Oi}\HOLSymConst{,}\HOLBoundVar{Os}) (\HOLConst{exec} (\HOLConst{SLc} \HOLFreeVar{slCommand}))
       (\HOLConst{CFG} \HOLConst{authenticationTest} \HOLConst{ssmPBStateInterp}
          (\HOLConst{secContext} \HOLFreeVar{slCommand})
          (\HOLConst{Name} \HOLConst{PlatoonLeader} \HOLConst{says} \HOLConst{prop} (\HOLConst{SOME} (\HOLConst{SLc} \HOLFreeVar{slCommand}))\HOLSymConst{::}
               \HOLFreeVar{ins}) \HOLFreeVar{s} \HOLFreeVar{outs})
       (\HOLConst{CFG} \HOLConst{authenticationTest} \HOLConst{ssmPBStateInterp}
          (\HOLConst{secContext} \HOLFreeVar{slCommand}) \HOLFreeVar{ins}
          (\HOLBoundVar{NS} \HOLFreeVar{s} (\HOLConst{exec} (\HOLConst{SLc} \HOLFreeVar{slCommand})))
          (\HOLBoundVar{Out} \HOLFreeVar{s} (\HOLConst{exec} (\HOLConst{SLc} \HOLFreeVar{slCommand}))\HOLSymConst{::}\HOLFreeVar{outs})) \HOLSymConst{\HOLTokenImp{}}
     (\HOLBoundVar{M}\HOLSymConst{,}\HOLBoundVar{Oi}\HOLSymConst{,}\HOLBoundVar{Os}) \HOLConst{sat} \HOLConst{prop} (\HOLConst{SOME} (\HOLConst{SLc} \HOLFreeVar{slCommand}))
\end{SaveVerbatim}
\newcommand{\HOLssmPBTheoremsPlatoonLeaderXXslCommandXXverifiedXXthm}{\UseVerbatim{HOLssmPBTheoremsPlatoonLeaderXXslCommandXXverifiedXXthm}}
\newcommand{\HOLssmPBTheorems}{
\HOLThmTag{ssmPB}{authenticationTest_cmd_reject_lemma}\HOLssmPBTheoremsauthenticationTestXXcmdXXrejectXXlemma
\HOLThmTag{ssmPB}{authenticationTest_def}\HOLssmPBTheoremsauthenticationTestXXdef
\HOLThmTag{ssmPB}{authenticationTest_ind}\HOLssmPBTheoremsauthenticationTestXXind
\HOLThmTag{ssmPB}{PBNS_def}\HOLssmPBTheoremsPBNSXXdef
\HOLThmTag{ssmPB}{PBNS_ind}\HOLssmPBTheoremsPBNSXXind
\HOLThmTag{ssmPB}{PBOut_def}\HOLssmPBTheoremsPBOutXXdef
\HOLThmTag{ssmPB}{PBOut_ind}\HOLssmPBTheoremsPBOutXXind
\HOLThmTag{ssmPB}{PlatoonLeader_exec_slCommand_justified_thm}\HOLssmPBTheoremsPlatoonLeaderXXexecXXslCommandXXjustifiedXXthm
\HOLThmTag{ssmPB}{PlatoonLeader_justified_slCommand_exec_thm}\HOLssmPBTheoremsPlatoonLeaderXXjustifiedXXslCommandXXexecXXthm
\HOLThmTag{ssmPB}{PlatoonLeader_slCommand_lemma}\HOLssmPBTheoremsPlatoonLeaderXXslCommandXXlemma
\HOLThmTag{ssmPB}{PlatoonLeader_slCommand_verified_thm}\HOLssmPBTheoremsPlatoonLeaderXXslCommandXXverifiedXXthm
}


%\input{hi.tex}
\section*{Section Overview}
\label{sec:section-overview}


\section{PBTypeScript.sml}
\label{sec:pbtypescript.sml-1}

\ \\This file contains the datatype definitions used in ssmPB.\ \\
\begin{description}
\item[Datatype Definitions]\ 
  \begin{itemize}
    \item Each state machine implements a set of types that are defined in OMNIType.
    OMNIType defines the datatype \textbf{command} which includes a constructor
    and type for \textbf{SLc slCommand}.
    \textbf{SLc} is the “state level command” constructor and the type is \textbf{slCommand}.
    slCommand is a type that is to be further defined in each state machine.\ \\
    \item Each state machine will implement a definition for \textbf{slCommand}.
    In PBType, this is defined as follows:
    \begin{align*}
    %   slCommand\; =&\;\; crossLD\ \\
    %                  &|\; conductORP\ \\
    %                   &|\; moveToPB\ \\
    %                   &|\; conductPB\ \\
    %                   &|\; completePB\ \\
    %                   &|\; incomplete
      \HOLFreeVar{slCommand} =&\;\; \HOLConst{crossLD}\ \\
      &\HOLTokenBar{} \HOLConst{conductORP}\ \\
      &\HOLTokenBar{} \HOLConst{moveToPB}\ \\
      &\HOLTokenBar{} \HOLConst{conductPB}\ \\
      &\HOLTokenBar{} \HOLConst{completePB}\ \\
      &\HOLTokenBar{} \HOLConst{incomplete}
    \end{align*}
    There are 6 ssmPB commands.
    Except for \textbf{incomplete}, each command corresponds to a transition from one
    state to the next state.\textbf{incomplete} is a command that that does not
    change the state of the machine -- a “not done” command.\ \\
    \item Similarly, OMNIType defines the datatype \textbf{state} which has a “state level state”
      constructor and type \textbf{SLs slState}.  In PBType, \textbf{slState} is defined as follows:
      \begin{align*}
        % slState\; =&\;\; PLAN\_PB\ \\
        %               &|\; MOVE\_TO\_ORP\ \\
        %               &|\; CONDUCT\_ORP\ \\
        %               &|\; MOVE\_TO\_PB\ \\
        %               &|\; CONDUCT\_PB\ \\
        %               &|\; COMPLETE\_PB\ \\
        \HOLFreeVar{slState} =&\;\; \HOLConst{PLAN_PB}\ \\
        &\HOLTokenBar{} \HOLConst{MOVE_TO_ORP}\ \\
        &\HOLTokenBar{} \HOLConst{CONDUCT_ORP}\ \\
        &\HOLTokenBar{} \HOLConst{MOVE_TO_PB}\ \\
        &\HOLTokenBar{} \HOLConst{CONDUCT_PB}\ \\
        &\HOLTokenBar{} \HOLConst{COMPLETE_PB}
      \end{align*}

    \item OMNIType defines the datatype \textbf{output} with the “state level output” constructor
      and type as \textbf{SLo slOutput}.  In PBType, it is implemented as follows:
      \begin{align*}
        % slOutput\; =&\;\; PlanPB\ \\
        %               &|\; MoveToORP\ \\
        %               &|\; ConductORP\ \\
        %               &|\; MoveToPB\ \\
        %               &|\; ConductPB\ \\
        %               &|\; CompletePB\ \\
        %               &|\; unAuthenticated\ \\
        \HOLFreeVar{slOutput} =&\;\; \HOLConst{PlanPB}\ \\
        &\HOLTokenBar{} \HOLConst{MoveToORP}\ \\
        &\HOLTokenBar{} \HOLConst{ConductORP}\ \\
        &\HOLTokenBar{} \HOLConst{MoveToPB}\ \\
        &\HOLTokenBar{} \HOLConst{ConductPB}\ \\
        &\HOLTokenBar{} \HOLConst{CompletePB}\ \\
        &\HOLTokenBar{} \HOLConst{unAuthenticated}
      \end{align*}

    \item OMNITtype defines the datatype \textbf{principal} which has “state role” constructor and
      type \textbf{SR stateRole} as:
      \begin{align*}
        \HOLFreeVar{stateRole} = \HOLConst{PlatoonLeader}
      \end{align*}
  \end{itemize}
\end{description}

\section{ssmPBScript.sml}
\label{sec:ssmpbscript.sml}

\begin{enumerate}
\item \textbf{General Notes}
  \begin{enumerate}
  \item ssmPB is an acronym for “secure state machine patrol base”.  The “Script” notation
    tells HOL that this is to be implemented as a theory.  “.sml” is the standard
    file extension for poly ML type files.\ \\
  \item The importance of ssmPB is two-fold:
    \begin{itemize}
    \item \textit{It implements the top-level state machine in HOL.}
      \item \textit{It serves as a model for the implementation of subsequent state machines in the project.}\ \\
      \end{itemize}
    \item ssmPB uses ssm11 (secure state machine 1.1) as the general secure state machine which it
      parameterizes.  That is, functions and datatypes in ssmPB are used as parameters in the ssm11
      secure state machine.   ssmPB is the first state machine in the project that does this.  Thus,
      it is used as a model for subsequent state machines.  Because of the similarity of all state
      machines in the project, subsequent state machines can be a cut-n-paste adaptation of ssmPB,
      with appropriate updates of datatype names.  Encapsulation of each state machine facilitates
      the use of datatypes with the same name in different state machines.  Concerns about conflicts
      among datatype names while integrating state machines (if time allows) were discussed.  This
      should not pose a problem if time permits for the extension of the project to integration of
      state machines.\ \\
    \item iv.	Because ssmPB is a model for subsequent state machines, the details of ssmPB will
      be described in more detail than for subsequent state machine descriptions.\ \\
    \end{enumerate}
  \item \textbf{Theory File Description}
    \begin{enumerate}
    \item After the appropriate comments and declaration of ssmPB as a structure, HOL must open
      theories which it depends on.
      \begin{itemize}
      \item \textit{open HolKernel Parse boolLib bossLib;}
        \begin{itemize}
          \item \textsl{These are included in ALL HOL theory definitions.}
          \end{itemize}
          \item \textit{open TypeBase listTheory optionTheory;}
        \begin{itemize}
        \item \textsl{These are theories that provide added functionality and theories for
            use in the implementation of HOL theories in general.  They are provided with HOL.}
        \end{itemize}
        \item \textit{open acl_infRules aclDrulesTheory aclrulesTheory;}
        \begin{itemize}
        \item \textsl{These are the access-control logic files implemented by Prof.
            Shiu-Kai Chin and Lockwood Morris.}
        \end{itemize}
        \item \textit{open satListTheory ssm11Theory ssminfRules;}
        \begin{itemize}
        \item \textsl{These are additional theories and rules that are used in the implmentation
            of the state machines.  These were implemented by Prof. Shus-Kai Chin for his
            CIS 634 course at Syracuse University.}
        \end{itemize}
        \item \textit{open OMNITypeTheory PBTypeTheory;}
        \begin{itemize}
        \item \textsl{OMNIType contains datatype definitions that are relevant to ALL state
            machines in the project.  PBTypeTheory contains datatype definitions for ssmPB.}
        \end{itemize}
      \end{itemize}\ \\
    \item PBNS_def , PBOut_def, and secContext_def
      \begin{itemize}
      \item \textsl{These are the \textbf{next state} and \textbf{next output} functions, \textbf{PBNS}
        and PBOut respectively. They describe the behavior of the state machine. Each of these
        functions takes two inputs, the current state followed by a command, and returns the next
        state or the next output, respectively. For example, the transition from the \textbf{PLAN_PB}
        state to the next state \textbf{MOVE_TO_ORP} is given by the ‘state level command” (SLc)
        \textbf{crossLD}. In PBNS_def, this looks like:
        {\large
        \begin{align*}                                    
          \textbf{\HOLConst{PBNS} \HOLConst{PLAN_PB} (\HOLConst{exec} (\HOLConst{SLc} \HOLConst{crossLD})) \HOLSymConst{=} \HOLConst{MOVE_TO_ORP}}
        \end{align*}
     }%
    \item The description is similar for \textbf{PBOut}.   The state level command (SLc)
      \textbf{crossLD} will produce the next output \textbf{MoveToORP}.
      {\large
        \begin{align*}
          \textbf{\HOLConst{PBOut} \HOLConst{PLAN_PB} (\HOLConst{exec} (\HOLConst{SLc} \HOLConst{crossLD})) \HOLSymConst{=} \HOLConst{MoveToORP}}
        \end{align*}
      }%
    \item Each line in these definitions corresponds to a transition in the state machine.
      This list is complete.  That is, all possible transitions are delineated.
    \item Notice that there are three different types of commands (\textbf{exec}, \textbf{trap},
      and \textbf{discard}) that precede each SLc type command.  These are “orders” descripted in
      ssm11.  Because ssmPB is a “secure” state machine, each transition must not only be delineated
      in the appropriate next state and next output functions, the command must be given by an
      authenticated and authorized individual (referred to as a principal in the access-control
      logic).  Further explanation follows.
    \item \underline{Authentication}: Each secure state machine has a list of principals
      (defined in datatype definition \textbf{stateRole} in PBType).  In ssmPB, this list
      contains only one principal named \textbf{PlatoonLeader}.  Thus, \textbf{PlatoonLeader}
      is the only authenticable principal in the ssmPB state machine.  Other state machines
      may contain more than one authenticable principals.  Because only authenticable principals
      can issue commands to the state machine, commands issued by non-authenticable principals
      are “discarded.”  Hence, if a principal other than \textbf{PlatoonLeader} in ssmPB issues
      a command, then that command it discarded.  For example, if the platoon sergeant issues
      the SLc command \textbf{crossLD}  in ssmPB it would be represented in the next state function
      \textbf{PBNS} as:
      {\large
        \begin{align*}
          \textbf{(\HOLConst{PBNS} \HOLConst{PLAN_PB} (\HOLConst{discard} (\HOLConst{SLc} \HOLConst{crossLD})) \HOLSymConst{=} \HOLConst{PLAN_PB})}
        \end{align*}
      }%
      where the ssm11 order \textbf{discard} is used to indicate that this principal is not authenticated
      (or the request is not formatted correctly.  See below.)
    \item \underline{Authorized}: In addition to being authenticated, the principal must be
      authorized to issue any given command.  This is covered in the security context of the
      state machine.   The security context of ssmPB is defined in \textbf{secContext_def}.
      Because there is only one authenticable principal in this state machine, namely
      \textbf{PlatoonLeader}, and this principal is authorized on ALL SLc commands in ssmPB,
      there is only one statement in \textbf{secContext_def}.  \textbf{PlattonLeader} is
      authenticated via the “controls” statement, which indicates that \textbf{PlatoonLeader} “controls”
      any SLc command in ssmPB.
      {\large
        \begin{align*}
          \textbf{\HOLConst{Name} \HOLConst{PlatoonLeader} \HOLConst{controls} \HOLConst{prop} (\HOLConst{SOME} (\HOLConst{SLc} \HOLBoundVar{cmd}))}
        \end{align*}
      }%
      Additional principals could be added to the \textbf{stateRole} datatype definition in
      PBType and these principals could be given authority to execute some or all of the SLc
      commands in this state machine.  These authorizations would be listed in \textbf{secContext_def}.
      Only commands issued by a principal that is both authenticated AND authorized will be executed.
      The \textbf{exec} command precedes commands issued by principals that are authenticated and
      authorized.  For example, a command issued by \textbf{PlatoonLeader} to \textbf{crossLD}
      would be executed causing the state machine to transition from the initial state
      \textbf{PALN_PB} to the next state \textbf{MOVE_TO_ORP}.
      {\large
        \begin{align*}
          \textbf{%PBNS\ PLAN\_PB\ (exec\ (SLc\ crossLD))\;} &=\; \textbf{MOVE\_TO\_ORP}
          (\HOLConst{PBNS} \HOLConst{PLAN_PB} (\HOLConst{exec} (\HOLConst{SLc} \HOLConst{crossLD})) \HOLSymConst{=} \HOLConst{MOVE_TO_ORP})}
        \end{align*}
      }%
    \item \underline{Additional considerations}:  Commands issued by principals that are
      authenticated (i.e., listed in \textbf{stateRole}) but who are not authorized in
      \textbf{secContext_def} for a specific command are “trapped.”  Because there are
      no such additional principals in ssmPB, commands issued in ssmPB will not be trapped.
      They will be either discarded or executed.  Nevertheless, trapped commands are
      delineated in the next state and next output functions for completeness and in the
      event that additional principals are added at a later date.  The \textbf{trap} order
      represents such commands.  An example of a command issued  by an authenticated but
      unauthorized principal to \textbf{crossLD} would be implemented as:
       {\large
        \begin{align*}
          \textbf{%PBNS\ PLAN\_PB\ (trap\ (SLc\ crossLD))\;} &=\; \textbf{PLAN\_PB}
          (\HOLConst{PBNS} \HOLConst{PLAN_PB} (\HOLConst{trap} (\HOLConst{SLc} \HOLConst{crossLD})) \HOLSymConst{=} \HOLConst{PLAN_PB})}
        \end{align*}
      }%
    \item \underline{Other}: Commands that are improperly formatted are also discard in the same
      manner as are commands issued by unauthenticated principals.  \ \\
    }\ \\
  \end{itemize}
\item authenticationTest_def
  \begin{itemize}
  \item \textit{This definition defines which types of requests (i.e., some principal  issues a
      command) are authenticated and by whom, and which are not.  In the access-control logic, a
      principal is declared to HOL by preceding it with the identifier \textbf{Name}.  Thus,
      \textbf{Name PlatoonLeader} provides HOL with the name of the principal who is the subject
      of the statement.  A request is indicated in the access-control logic by the identifier
      \textbf{says}.  In ssmPB, \textbf{PlatoonLeader} is authenticated on commands.  That is,
        \begin{align*}
          \textbf{((Name\ PlatoonLeader)\ says\ (prop\ (cmd:(slCommand\ command)\ order))\;}
        \end{align*}
        Additional authenticable principals would be delineated in the same manner.
        \textbf{authenticationTest_def} fails for all other requests.  This also includes
        improperly formatted statemens.
        \begin{align*}
          \textbf{(authenticationTest\ \_\  =\ F)}
          \end{align*}
      }
    \end{itemize}
  \item ssmPBStateInterp_def
    \begin{itemize}
    \item \textit{This is a state interpretation function.  In ssmPB, it is effectively nil., \textbf{TT}
      is true in the access-control logic.   In subsequent state machines, the state interpretation
      function could be used describe differences in the behavior of the state machine for
      specific states.  For example, if one state required a check-list be completed prior to
      the execution of a transition from one state to the next, then it would be declared in
      this function.\ \\}
    \end{itemize}
    \item secContext_def
      \begin{itemize}
        
    \item \textit{In addition to listing authorities via the \textbf{controls} operator, additional
      security-related definitions would be added here.} \ \\
    \end{itemize}
    \item PlatoonLeader_slCommand_lemma
      \begin{itemize}
    \item \textit{This lemma uses a tactical proof technique (backwards proof) to prove that
      \textbf{PlatoonLeader} is authorized on any SLc command. }\ \\
    \end{itemize}
    \item PlatoonLeader_exec_slCommand_justified_thm
      \begin{itemize}
    \item \textit{This uses the lemma above to prove the biconditional that \textbf{PlattoonLeader}’s
      commands are executed if and only if the command is authenticated and authorized.
      \item WIth this theorem, it is easy to prove the following two theorems.}\ \\
      \end{itemize}
      \item PlatoonLeader_slCommand_verified_thm
        \begin{itemize}
    \item \textit{This theorem proves that if the \textbf{PlatoonLeader}’s command was executed,
      then it must have been verified.   }\ \\
    \end{itemize}
    \item PlatoonLeader_justified_slCommand_exec_thm
      \begin{itemize}
    \item \textit{This theorem proves that if the  \textbf{PlatoonLeader}’s command was authorized,
      then the command is executed.   } \ \\
    \end{itemize}
    \end{enumerate}
  \end{enumerate}

  \begin{center}
  \resizebox{\textwidth}{!}{\begin{tabular}{|m{5em}|c|m{8em}|m{9em}|m{8em}|m{8em}|m{8em}|m{9em}|}
                              \hline
    \multicolumn{8}{|c|}{ }\\
    \multicolumn{8}{|c|}{\textcolor{cyan}{Next State}, Next Output Table}\\
    \multicolumn{8}{|c|}{ }\\
    \hline\hline
    & \cellcolor{cyan}State &&&&&&\\
    \hline
     \rowcolor{lime}Commands/ input &  \cellcolor{white}next state, next output  & crossLD & conductORP & moveToPB & conductPB & completePB & incomeplete \\
     \hline
     & \cellcolor{cyan}PLAN_PB & \textcolor{cyan}{MOVE_TO_ORP}, MoveToORP & \rowcolor{lightgray}  &  &&& \cellcolor{white}\textcolor{cyan}{PLAN_PB}, PlanPB \\
    \hline
     & \cellcolor{cyan}MOVE_TO_ORP & \rowcolor{lightgray}& \cellcolor{white}\textcolor{cyan}{CONDUCT_ORP}, ConductORP & & & & \cellcolor{white}\textcolor{cyan}{MOVE_TO_ORP}, MoveToORP \\
    \hline
     & \cellcolor{cyan}CONDUCT_ORP &\rowcolor{lightgray} & & \cellcolor{white}\textcolor{cyan}{MOVE_TO_PB}, MoveToPB & & & \cellcolor{white}\textcolor{cyan}{CONDUCT_ORP}, ConductORP \\
    \hline
     &\cellcolor{cyan} MOVE_TO_PB & \rowcolor{lightgray}&&& \cellcolor{white}\textcolor{cyan}{CONDUCT_PB}, ConductPB && \cellcolor{white}\textcolor{cyan}{MOVE_TO_PB}, MoveToPB \\
    \hline
     & \cellcolor{cyan}CONDUCT_PB & \rowcolor{lightgray}&&&& \cellcolor{white}\textcolor{cyan}{COMPLETE_PB}, CompletePB & \cellcolor{white}\textcolor{cyan}{CONDUCT_PB}, ConductPB \\
    \hline
     & \cellcolor{cyan}COMPLETE_PB & \rowcolor{lightgray} &&&&& \\
    \hline
  \end{tabular}}
  \end{center}

Next table
%%%%%%%%%%%%%%%%%%%%%%%%%%%%%%%%  sublevel conductORP  
  \begin{center}
  \resizebox{\textwidth}{!}{\begin{tabular}{|m{5em}|c|m{5em}|m{8em}|m{8em}|m{8em}|m{9em}|}
                              \hline
    \multicolumn{7}{|c|}{ }\\
    \multicolumn{7}{|c|}{ConductORP \textcolor{cyan}{Next State}, Next Output Table}\\
    \multicolumn{7}{|c|}{ }\\
    \hline\hline
    & \cellcolor{cyan}State &&&&&\\
    \hline
     \rowcolor{lime}Commands/ input &  \cellcolor{white}next state, next output  & secure & actionsIn & withdraw & complete & plIncomplete  \\
     \hline
     & \cellcolor{cyan}CONDUCT_ORP & \textcolor{cyan}{SECURE}, Secure & \rowcolor{lightgray}  &  &&\cellcolor{white}\textcolor{cyan}{CONDUCT_ORP}, ConductORP  \\
    \hline
     & \cellcolor{cyan}SECURE & \rowcolor{lightgray}& \cellcolor{white}\textcolor{cyan}{ACTIONS_IN}, ActionsIn & & & \cellcolor{white}\textcolor{cyan}{SECURE}, Secure  \\
    \hline
     & \cellcolor{cyan}ACTIONS_IN &\rowcolor{lightgray} & & \cellcolor{white}\textcolor{cyan}{WITHDRAW}, Withdraw & & \cellcolor{white}\textcolor{cyan}{ACTIONS_IN}, ActionsIn  \\
    \hline
     &\cellcolor{cyan}WITHDRAW & \rowcolor{lightgray}&&& \cellcolor{white}\textcolor{cyan}{COMPLETE}, Complete & \cellcolor{white}\textcolor{cyan}{WITHDRAW}, Withdraw \\
    \hline

  \end{tabular}}
\end{center}

%%%%%%%%%%%%%%%%%%%%%%%%%%%%%%%%  sublevel conductPB  
  \begin{center}
  \resizebox{\textwidth}{!}{\begin{tabular}{|m{5em}|c|m{7em}|m{9em}|m{9em}|m{8em}|m{9em}|}
                              \hline
    \multicolumn{7}{|c|}{ }\\
    \multicolumn{7}{|c|}{ConductPB \textcolor{cyan}{Next State}, Next Output Table}\\
    \multicolumn{7}{|c|}{ }\\
    \hline\hline
    & \cellcolor{cyan}State &&&&&\\
    \hline
     \rowcolor{lime}Commands/ input &  \cellcolor{white}next state, next output  & securePB & actionsInPB & withdrawPB & completePB & plIncompletePB  \\
     \hline
     & \cellcolor{cyan}CONDUCT_PB & \textcolor{cyan}{SECURE_PB}, SecurePB & \rowcolor{lightgray}  &  &&\cellcolor{white}\textcolor{cyan}{CONDUCT_PB}, ConductPB  \\
    \hline
     & \cellcolor{cyan}SECURE_PB & \rowcolor{lightgray}& \cellcolor{white}\textcolor{cyan}{ACTIONS_IN_PB}, ActionsInPB & & & \cellcolor{white}\textcolor{cyan}{SECURE_PB}, SecurePB  \\
    \hline
     & \cellcolor{cyan}ACTIONS_IN_PB &\rowcolor{lightgray} & & \cellcolor{white}\textcolor{cyan}{WITHDRAW_PB}, WithdrawPB & & \cellcolor{white}\textcolor{cyan}{ACTIONS_IN_PB}, ActionsInPB  \\
    \hline
     &\cellcolor{cyan}WITHDRAW_PB & \rowcolor{lightgray}&&& \cellcolor{white}\textcolor{cyan}{COMPLETE_PB}, CompletePB & \cellcolor{white}\textcolor{cyan}{WITHDRAW_PB}, WithdrawPB \\
    \hline

  \end{tabular}}
\end{center}

%%%%%%%%%%%%%%%%%%%%%%%%%%%%%%%%  sublevel MoveToORP  
  \begin{center}
  \resizebox{\textwidth}{!}{\begin{tabular}{|m{5em}|c|m{7em}|m{7em}|m{9em}|m{8em}|m{9em}|}
                              \hline
    \multicolumn{7}{|c|}{ }\\
    \multicolumn{7}{|c|}{MoveToORP \textcolor{cyan}{Next State}, Next Output Table}\\
    \multicolumn{7}{|c|}{ }\\
    \hline\hline
    & \cellcolor{cyan}State &&&&&\\
    \hline
     \rowcolor{lime}Commands/ input &  \cellcolor{white}next state, next output  & pltForm & pltMove & pltSecureHalt & complete & incomplete  \\
     \hline
     & \cellcolor{cyan}PLAN_PB & \textcolor{cyan}{PLT_FORM}, PLTForm & \rowcolor{lightgray}  &  &&\cellcolor{white}\textcolor{cyan}{PLAN_PB}, PLTPlan  \\
    \hline
     & \cellcolor{cyan}PLT_FORM & \rowcolor{lightgray}& \cellcolor{white}\textcolor{cyan}{PLT_MOVE}, PLTMove & & & \cellcolor{white}\textcolor{cyan}{PLT_FORM}, PLTForm  \\
    \hline
     & \cellcolor{cyan}PLT_MOVE &\rowcolor{lightgray} & & \cellcolor{white}\textcolor{cyan}{PLT_SECURE_HALT}, PLTSecureHalt & & \cellcolor{white}\textcolor{cyan}{PLT_MOVE}, PLTMove  \\
    \hline
     &\cellcolor{cyan}PLT_SECURE_HALT & \rowcolor{lightgray}&&& \cellcolor{white}\textcolor{cyan}{COMPLETE}, Complete & \cellcolor{white}\textcolor{cyan}{PLT_SECURE_HALT}, PLTSecureHalt \\
    \hline

  \end{tabular}}
\end{center}

%%%%%%%%%%%%%%%%%%%%%%%%%%%%%%%%  sublevel MoveToPB 
  \begin{center}
  \resizebox{\textwidth}{!}{\begin{tabular}{|m{5em}|c|m{7em}|m{7em}|m{7em}|m{8em}|m{7em}|}
                              \hline
    \multicolumn{7}{|c|}{ }\\
    \multicolumn{7}{|c|}{MoveToPB \textcolor{cyan}{Next State}, Next Output Table}\\
    \multicolumn{7}{|c|}{ }\\
    \hline\hline
    & \cellcolor{cyan}State &&&&&\\
    \hline
     \rowcolor{lime}Commands/ input &  \cellcolor{white}next state, next output  & pltForm & pltMove & pltHalt & complete & incomplete  \\
     \hline
     & \cellcolor{cyan}PLAN_PB & \textcolor{cyan}{PLT_FORM}, PLTForm & \rowcolor{lightgray}  &  &&\cellcolor{white}\textcolor{cyan}{PLAN_PB}, PLTPlan  \\
    \hline
     & \cellcolor{cyan}PLT_FORM & \rowcolor{lightgray}& \cellcolor{white}\textcolor{cyan}{PLT_MOVE}, PLTMove & & & \cellcolor{white}\textcolor{cyan}{PLT_FORM}, PLTForm  \\
    \hline
     & \cellcolor{cyan}PLT_MOVE &\rowcolor{lightgray} & & \cellcolor{white}\textcolor{cyan}{PLT_HALT}, PLTSecureHalt & & \cellcolor{white}\textcolor{cyan}{PLT_MOVE}, PLTMove  \\
    \hline
     &\cellcolor{cyan}PLT_HALT & \rowcolor{lightgray}&&& \cellcolor{white}\textcolor{cyan}{COMPLETE}, Complete & \cellcolor{white}\textcolor{cyan}{PLT_HALT}, PLTHalt \\
    \hline

  \end{tabular}}
  \end{center}
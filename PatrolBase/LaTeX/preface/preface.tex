This documentation describes the research methods, results, future work, and conclusions
for the proof of concept project discussed herein.   The Introduction briefly describes
the purpose of the project.  The Methods, Assumptions, and Procedures  section follows.
It describes the basic concepts of applying the principle of complete mediation to the
patrol base operations using a hierarchy of \emph{secure} state machines.  It also describes the
technologies employed for the project.  Following the Methods, Assumptions, and Procedures
section is the Results  section.   This section is divided into three parts.  The first
part describes the folders and files and the folder structure for the project.  It is intended
to aid the reader in finding files and folders in the project. The next section details the
ACL and HOL verification of the project.  It details the datatypes, definitions, and theorems
used in verifying the hierarchy of \emph{secure} state machines.  In this section, the code and its
purpose are discussed.  Examples of code are also included.  The third section describes the
results of translating the patrol base operations into a hierarchy of \emph{secure} state machines.
The logic and rationale for the structure are described .   The Future Work and Research
section describes works that were discussed but not verified in the ACL and HOL.  It also
hints on how this project may be employed in future automations of the patrol base operations.
The Conclusion wraps everything up.\\

Other sections support this documentation.  This includes a list of figures and acronyms.
In addition, the appendix contains the full EmitTeX pretty-printed datatypes, definitions,
and theories generated in the project.  The actual HOL code (not so pretty-printed form) and
the diagrams for the hierarch of \emph{secure} state machines are also included in the appendix.
Platoon patrol base (PB) operations are an essential part of many military operations.
They provide Soldiers with a means to avoid enemy contact, rest, resupply, and conduct Squad
level missions (Ranger Handbook 2011, 132). Although some might believe that a PB begins from
an objective rally point (ORP), a PB must be planned for. Our PB software includes a state
machine model for tasks that include planning, movement, as well as conducting both an ORP and PB.\\

Any automation of the patrol base operations (for example Jesse? Professor Chin?) would require
that the design of such system be certified as secure from the start. This approach should be
applied to all systems. Yet, our aim in this project was to show that the certified security by
design approach could be applied specifically to the patrol base operations because it is described
in a manner that is similar to many other military operations. These operations will, no doubt,
be automated in the future. Thus, this project should be viewed as a primer on how to tackle such
an automation in a secure manner, from the start.\\

The certified security by design approach involves taking the design of the automated system and
using access-control logic (ACL) to prove that any action that takes place in the automated system
is correctly executed and has the appropriate authorizations and authentications of those authorities.
Computer-aided reasoning is used to verify that the system is secure. For this project, we used the
higher order logic (HOL) theorem prover. HOL is a trusted and reliable system. It is commonly said by
the HOL community that “HOL is never wrong.”\\

The access-control logic (ACL) was developed by Professor Shiu-Kai Chin and Professor Susan Older of
Syracuse University’s Department of Engineering and Computer Science. Specifics can be found in their
text book titled Access Control, Security, and Trust. The definitions and theorems from the text were
implemented in HOL by Professor Shiu-Kai Chin and Lockwood Morris. For reference a report of the datatypes,
definitions, and theorems that were implemented in HOL are provided in Appendices ?\\

Our method for understanding what represented mission failure and mission success was guided by Functional
Mission Analysis (Young,…?). We identified that our goal was to develop “A system to provide mission
assurance for execution of patrol base operations by means of logical proof in order to mitigate
enemy/civilian contact and mission failure.” Unacceptable losses would normally be at the Higher
Headquarter (HHQs) Commander’s discretion, but for the sake of our project they were the (L1) loss
of civilian life, (L2) 20\% of our Platoon being rendered Non-Mission Capable (NMC), and (L3) mission
failure. Hazards that could lead to unacceptable losses included (H1) contact with civilians, (H2)
contact with the enemy, and (H2) loss of adequate mission essential supplies. The table below
illustrated our causal links between hazards and losses.\\

\begin{figure}[h]               
  \centering
  \includegraphics[scale=0.46]{HazardChart.PNG}
  \caption{Hazard or Loss Table}
\end{figure}

Our functional control structure is the chain of command illustrated below outlines what individual,
in what unit level would be responsible for the actions of the Soldiers below them.\\

\begin{figure}[h]
  \centering
  \includegraphics[scale=0.3]{OrganizationalChart.png}
  \caption{Organizational Chart}
\end{figure}

The organizational chart showed the hierarchy of the Platoon and whom would be in control of
each element and responsible for mitigating risks caused by hazards. Our application design
recognized each leaders’ ability to control certain groups/units of Soldiers. With this in
mind, leaders within the Platoon would have to (C1) avoid populated areas, (C2) avoid trafficked
routes to and from those areas, (C3) maintain supplies within the HHQ standard operating
procedure (SOP). These constraints were built into our application with states that performed
reconnaissance, uniform and packing list checks, maintenance and cross loading of supplies.\\

% ---- this points LaTeX to PatrolBaseDoc.tex ----
% Local Variables:
% TeX-master: "../PatrolBaseDoc"
% End: